% Имя Фамилия автора(ов)
\author{Денис Пынькин, Юрий Адамов}
% Если несколько авторов
% \author{Тест Шаблонов, Тест Тестов}

% Город
\city{Минск}

% Организация
\affiliation{EPAM Systems}

% Название проекта, которому посвящён доклад (необязательно, 
% но желательно).
% \projecttitle{}
% Сайт(ы) проекта
\projecturl{\url{https://github.com/epam-llpd/linux\_courses}}
% Если несколько сайтов:
% \projecturl{\url{http://altlinux.ru}, \url{http://sisyphus.ru}}

% Тема доклада
\title{Linux-образование -- симбиоз ВУЗов, коммерческих компаний и LUG}

\maketitle

\begin{abstract}
	В докладе описываются инициативы IT-компаний,  связанные с обучением процессу  
	разработки в среде ОС Linux. Приводятся примеры успешного взаимодействия белорусских коммерческих
	компаний с ВУЗами и Минским LUG для достижения общих целей.
	В результате общих усилий создана и развивается открытая и свободная библиотека лекционных материалов по
	обучению различным аспектам работы в ОС Linux.
\end{abstract}



\section{Кризис в отрасли}

ОС Linux устойчиво набирает популярность в качестве платформы для разработки. Соответственно растет и количество 
коммерческих проектов с использованием этой операционной системы. Примерно в 2010-12 годах в стало заметно, что людей,
знакомых с полным циклом разработки для Linux-платформы, в особенности для встраиваемых и серверных ее применений,
не так уж много.

Хотя более правильным было бы сказать, что таких людей катастрофически не хватает для покрытия нужд белорусских IT компаний.

Основные проблемы обучения в ВУЗах РБ:

\begin{itemize}
	\item[--] существует направленность на изучение закрытого стека технологий на основе ОС Windows,
		при этом применение Linux зависит исключительно от отдельных лиц, работающих в учебном заведении\cite{dkp01};
	\item[--] еще одна проблема прямо заложена в учебных планах,  направленных на изучение теории,  что приводит 
		к самостоятельному освоению инструментальных средств программирования учащимися, 
		которое,  как правило,  заканчивается на минимальном уровне владения выбранным (или навязанным) IDE;
	\item[--] ``классическое'' обучение практически не затрагивает практики командной разработки с выделением 
		в отдельные процессы собственно самой разработки,  тестирования и развертывания ПО;
	\item[--] преподавателями,  в основной своей массе,  совершенно игнорируются подходы,  принятые в мире,  связанном со Свободным ПО\cite{pg01}.
\end{itemize}

Кроме того отдельно хотелось бы отметить проблему закрытости и кастовости,  распространенной в сообществе 
пользователей и разработчиков Linux,  что никак не способствует привлечению новых членов.

Все эти проблемы приводят к тому,  что разработчики,  тестировщики и,  в меньшей степени,  администраторы,  
умеющие работать и знающие ОС Linux ``самозарождаются'',  что является сравнительно медленным процессом и 
совершенно не подходит для коммерческих компаний.

\section{EPAM, начало}

Принципиальное решение о необходимости дополнительного привлечения молодых разработчиков к ОС Linux было принято 
в рамках департамента ``Low Level Programming Department'' в начале 2012 года. Учитывая большой опыт компании по 
работе с учебными заведениями в РБ и острую необходимость в увеличении количества разработчиков для встраиваемых
и серверных применений, уже осенью 2012 г., была оборудована совместная 
лаборатория Epam и БГУИР на базе кафедры ЭВМ КСиС.

Тогда же прошел первый набор (2012-2013) слушателей на курсы по изучению ОС Linux для разработчиков.

Учитывая,  что основная целевая аудитория курса -- студенты технических специальностей,  была разработана программа,
рассчитанная на людей, уже умеющих программировать на каком-либо языке,  но желающих получить навыки разработки в 
среде ОС Linux. В программу курса вошли технологии и знания,  помогающие адаптироваться к целевой платформе обучения.

В связи с ориентацией департамента на разработку серверных и встраиваемых решений,  было принято решение не затрагивать работу и разработку в графическом окружении,  а сконцентрироваться на работе в командной строке.
Сама программа разделена на 3 отдельных модуля, общей длительностью 66 часов:

\begin{itemize}
	\item[--] введение в GNU/Linux -- минимальный набор знаний об архитектуре и особенностях работы в среде ОС Linux;
	\item[--] программирование на bash -- разработчики рано или поздно сталкиваются с необходимостью разбираться в 
		чужих скриптах,  создавать свои,  а также автоматизировать свою работу;
	\item[--] инструментарий разработчика -- в этот модуль входят принципы разработки в ОС Linux, методы и навыки работы 
		с классическими инструментами: компилятор, управление сборкой, установкой и распространением приложения,
		совместная работа с исходным кодом, анализ исполняемого файла и его работы,
		а также другие средства и подходы применяющиеся при разработке.
\end{itemize}


Хотелось бы подчеркнуть, что курс читается не профессиональными лекторами, а разработчиками департамента,
постоянно применяющими многие из изучаемых технологий на практике.

После успешного окончания курса слушателям были вручены сертификаты о прохождении, а лучшим учащимся предложена работа 
в Epam LLPD в качестве разработчиков серверных и встраиваемых решений.

\section{EPAM, вторая итерация}

В 2013 году курсы были расширены, в результате чего появились еще 2 разработчика, готовых поделиться своими знаниями по 
следующим направлениям:

\begin{itemize}
	\item[--] программирование на языке Python;
	\item[--] программирование на языкe С.
\end{itemize}

Разумеется, также происходят эволюционные изменения и в модулях, созданных в предыдущем году.
Сама же библиотека слайдов, наработанная при подготовке первого года, пригодилась для быстрого создания отдельного 
внутреннего курса Epam по изучению Bash.

\section{Совместная работа}

На данный момент существует общая стратегическая задача, Linux сообщества, ВУЗов и корпоративного сектора -- это 
создание широкой и устойчивой экосистемы,  связанной с ОС Linux в среде IT-специалистов и желающих стать таковыми.

В процессе роста сообщества,  увеличивается не просто количество технических специалистов,  кроме того,  
увеличивается количество людей знающих и разделяющих ценности,  заложенные в принципах Свободного ПО.
Интерес компаний очевиден,  ведь именно из этого сообщества приходят так необходимые для коммерческой разработки профессионалы.

Учитывая специфику области, в профильных департаментах различных компаний, независимо друг от друга было принято 
решение вести максимально возможно открытую политику обучения.

В первую очередь это привело в неформальной договоренности о создании совместных материалов с открытым доступом 
для обучения.

Таким образом на github появился открытый проект, содержащий на данном этапе лекционные слайды, помогающие быстро создать
презентацию по необходимой теме. Материалы, согласно подходу, принятому при работе с открытыми и свободными проектами, 
добавляются по мере создания: \url{https://github.com/epam-llpd/linux_courses}, а по адресу \url{https://github.com/epam-llpd/linux_courses/network/members/} можно увидеть список лиц и организаций, так или иначе добавивших свой вклад в развитие.

Хотелось бы отметить,  что благодаря такому подходу у любого заинтересованного лица -- будь то представитель другой
компании,  либо студент,  есть возможность участвовать в процессе обучения,  корректировать материалы и создавать новые,
а также проводить свои собственные курсы,  используя уже созданные материалы.

На данный момент инициативу поддержали и открыли материалы по своим учебным материалам следующие компании:
\begin{itemize}
	\item[--] EPAM Systems -- курс по работе в ОС Linux для разработчиков, курс по Bash;
	\item[--] SaM Solutions -- курс по работе в ОС Linux для тестировщиков;
	\item[--] Promwad -- курс ``Программирование встраиваемых систем'' на базе ОС Linux.
\end{itemize}

В своем роде -- это уникальный для просторов РБ проект с открытым исходным кодом\cite{ps01}.

\begin{thebibliography}{9}
\bibitem{dkp01} \textit{Derechennik S.S., Kostiuk D.A., Pynkin D.A.} Free/libre software usage in the belarusian system of higher educational institutions // Друга міжнародна науково-практична конференція FOSS Lviv-2012: Збірник наукових праць/ Львів,  26-28 квітня 2012 р.
\bibitem{pg01} \textit{Д.А. Пынькин, И.И. Глецевич}. Открытый подход к обучению студентов технической
	специальности ВУЗа // 7-я конференция «СПО в высшей школе»: Тезисы докладов. \url{http://freeschool.altlinux.ru/wp-content/uploads/2012/01/pereslavl-winter-2012.pdf}
\bibitem{ps01} \textit{Д.А. Пынькин, В.В. Шахов}. Обучение Linux в корпоративном секторе. Зимняя международная конференция LVEE'2013. Тезисы докладов. \url{http://lvee.org/en/abstracts/57}

\end{thebibliography}


%%% Local Variables: 
%%% mode: latex
%%% TeX-master: "../main"
%%% End: 
