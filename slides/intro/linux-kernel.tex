\begin{frame}{Задачи ядра Linux}
	\begin{itemize}
		\item Инициализация системы
		\item Управление процессами и потоками
		\item Управление памятью
		\item Управление файлами
		\item IPC
		\item Разграничение доступа
		\item Сетевые возможности
		\item Интерфейс доступа к возможностям ядра
	\end{itemize}
\end{frame}


\begin{frame}{Ядро}

	Ядро ОС Linux является модульным. 

	\begin{block}{Модули}
		\begin{itemize}
			\item В виде отдельных файлов
			\item "Вкомпилированные" в ядро
		\end{itemize}
	\end{block}

	\bigskip

	Список загруженных модулей: {\tt /proc/modules}
\end{frame}


\begin{frame}{Параметры ядра}
	
	Полный список: {\tt Documentation/kernel-parameters.txt}

	\begin{block}{Некоторые часто применяемые параметры}
		\begin{itemize}
			\begin{columns}
			\column{0.3\textwidth}
				\item console=ttyS0,115200
				\item debug
				\item init=/sbin/init
				\item loglevel=[0-7]
				\item maxcpus=[num]
			\column{0.3\textwidth}
				\item mem=nn[KMG]
				\item noacpi
				\item noapic
				\item panic=nn (sec)
				\item resume=/dev/sda2
			\column{0.3\textwidth}
				\item ro
				\item rw
				\item root=/dev/sda1
				\item rootdelay=nn (sec)
				\item rootwait
				\item vga=<num>|ask
			\end{columns}
		\end{itemize}
	\end{block}

	Модулям можно передавать параметры используя синтаксис: {\tt module.param=value}

	Параметры переданные ядру во время загрузки: {\tt /proc/cmdline}
\end{frame}

\begin{frame}{Магия SysRq}

	{\tt CONFIG\_MAGIC\_SYSRQ=y}

	{\tt /proc/sysrq-trigger}

	\begin{block}{{\bf R}eboot {\bf E}ven {\bf I}f {\bf S}ystem {\bf U}tterly {\bf B}roken}
		{\bf Ctrl+Alt+SysRq+?}

		\begin{itemize}
			\item h -- вывести список сочетаний на консоль
			\item b -- перезагрузка
			\item o -- выключение
			\item e -- послать сигнал SIGTERM всем процессам кроме init
			\item i -- послать сигнал SIGKILL всем процессам кроме init
			\item s -- синхронизировать все ФС 
			\item u -- переподключить все ФС в режиме RO
		\end{itemize}
		
	\end{block}


\end{frame}
