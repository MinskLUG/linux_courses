\begin{frame}{Авторское право и лицензии}

	\begin{block}{Авторское право}
		 Возникает по факту создания ПО 

		\begin{itemize}
			\item Неимущественные права
			\item Имущественные права
		\end{itemize}
	\end{block}

	\pause

    \begin{block}{Лицензии}
		Лицензия -- средство передать какие-либо права на продукт либо его часть.

		Необходима для защиты авторских прав. 
		Средство для возможности законно пресечь несанкционирование копирование,  использование или распространение ПО. 
	\end{block}
\end{frame}


\begin{frame}{Лицензии: открытые и свободные}
	\begin{block}{Р.Столлман: 4 свободы}
		\begin{itemize}
			\item Свобода 0: Свобода запускать программу в любых целях.
			\item Свобода 1: Свобода изучения работы программы и адаптация её к вашим нуждам. 
				Доступ к исходным текстам является необходимым условием.
			\item Свобода 2: Свобода распространять копии,  так что вы можете помочь вашему товарищу.
			\item Свобода 3: Свобода улучшать программу и публиковать ваши улучшения,
				так что всё общество выиграет от этого.
				Доступ к исходным текстам является необходимым условием.
		\end{itemize}
	\end{block}
\end{frame}


\begin{frame}{Лицензии: permissive}
	\begin{columns}
	\column{0.3\textwidth}
		\center\includegraphics[width=2cm]{../../slides/intro/three-arrows@2x}

	\column{0.6\textwidth}

	\begin{itemize}
		\item BSD
		\item MIT
		\item Apache
	\end{itemize}
	\end{columns}

	\begin{block}{I want it simple and permissive.}
		\begin{itemize}
			\item практически не ограничивают свободу действий пользователей ПО и разработчиков, работающих с исходным кодом.
			\item По своему духу, распространение работы под пермиссивной лицензией схоже с помещением работы в общественное
				достояние, но не требует отказа от авторского права.
		\end{itemize}
	\end{block}

\end{frame}


\begin{frame}{\textcopyleft -- Copyleft}

	\begin{columns}
	\column{0.3\textwidth}
		\center\includegraphics[width=2cm]{../../slides/intro/circular@2x}

	\column{0.3\textwidth}

	\begin{block}{Strong}
	\begin{itemize}
		\item GPL v2
        \item GPL v3
		\item AGPL
	\end{itemize}
    \end{block}

	\column{0.3\textwidth}
	\begin{block}{Weak}
	\begin{itemize}
		\item LGPL
		\item MPL
		\item CDDL
	\end{itemize}
    \end{block}

	\end{columns}


	\begin{block}{I care about sharing improvements.}
	
	Авторское лево -- концепция и практика использования законов авторского права для обеспечения 
	невозможности ограничить любому человеку право использовать,  изменять и распространять как 
	исходное произведение,  так и произведения,  производные от него.
	\end{block}


	При копилефте все производные произведения должны распространяться под той же лицензией,
	что и оригинальное произведение.

\end{frame}

\begin{frame}[fragile]{Использование лицензии в проекте}
    \begin{itemize}
        \item Рекомендуется включать файл {\ttfamily LICENSE} в проект
        \item Использовать однозначную отсылку на лицензию в файлах:
            \begin{itemize}
                \item Использовать стандартный заголовок лицензии (если есть)
                \item Использовать SPDX-License-Identifier\\
                    \url{https://spdx.org/licenses/}
                    \begin{lstlisting}[language=C]
/*
 * SPDX-License-Identifier: GPL-2.0
 */
\end{lstlisting}
                \item Использовать URL для определение лицензии
            \end{itemize}
    \end{itemize}
\end{frame}

\begin{frame}[fragile]{Использование \copyright в проекте}
    \begin{block}{Формат записи}
        \begin{verbatim}
<a copyright indicator> <years applicable>, \
        <copyright holder(s)>.
        \end{verbatim}
    \end{block}
    \begin{itemize}
        \item Copyright 1998, Linus Torvalds.
        \item \copyright 2003, 2010, Free Software Foundation, Inc.
        \item Copr. 2005-2012, 2014, John Smith, Jane Doe.
    \end{itemize}
\end{frame}

\begin{frame}[fragile]{Пример}
    \begin{block}{Das U-Boot}
    \begin{lstlisting}[language=C]
/*
* (C) Copyright 2002
* Wolfgang Denk, DENX Software Engineering, wd@denx.de.
*
* (C) Copyright 2010
* Michael Zaidman, Kodak, michael.zaidman@kodak.com
* post_word_{load|store} cleanup.
*
* SPDX-License-Identifier:    GPL-2.0+
*/
\end{lstlisting}
    \end{block}
\end{frame}

\begin{frame}{Полезные ресурсы по лицензиям}
\begin{block}{LFC191}
    В Linux Foundation подготовили бесплатный курс по открытым лицензиям и их применению:\\
    \url{https://training.linuxfoundation.org/training/compliance-basics-for-developers/}
\end{block}
\begin{block}{Подбор лицензии для проекта}
    \url{https://tldrlegal.com/licenses/browse}
\end{block}
\end{frame}

