\begin{frame}{Практика: создание тестовой среды}

	\center\includegraphics[height=0.4\textheight]{../../slides/networking/net-practice.png}


	\begin{block}{Задача}
		Запустить 3 идентичные виртуальные машины.\\
		Каждой машине назначить адрес из отдельного IP диапазона.\\
		Организовать сетевую ''видимость'' между виртуальными машинами, а также хостом.		
	\end{block}

\end{frame}



\begin{frame}
	\frametitle{Подготовка дисковой подсистемы}
			\begin{itemize}
				\item Создать пустой файл размером от 1.5 GB и отобразить на устройство
					/dev/loop0 ({\tt dd, losetup})
				\item Создать группу томов на базе этого устройства ({\tt pvcreate, vgcreate})
				\item Выделить 1 GB под логический диск ({\tt lvcreate})
				%\item Скопировать образ виртуального диска в полученный логический том ({\tt dd})
				%\item Создать снимок логического тома на 100MB ({\tt lvcreate}) для каждой виртуальной машины.
			\end{itemize}
\end{frame}

\begin{frame}[fragile]{Установка системы}
        \begin{itemize}
		\item Установить centos-minimal на  машину из iso файла.
		\item Создать два снимка логического тома виртуальной машины
		\item Убедиться в наличии tap интерфейсов
	\end{itemize}

\end{frame}

\begin{frame}[fragile]{Пример запуска kvm}
		\begin{itemize}
          \item {\tt modprobe kvm-intel} {\small Включаем модуль поддержки виртуализации в ядре} 
          \item {\tt modprobe tun}  {\small Включаем поддержку tun, tap виртуальных сетевых интерфейсов}
          \item 
            \begin{lstlisting}[language=bash,basicstyle=\tiny] 
kvm -enable-kvm -cdrom centos-minimal.iso -hda /dev/loop0 -m 512M   \
    -boot order=cd -serial stdio -net nic,model=rtl8139 -net tap,ifname=tap0 
            \end{lstlisting}
              \begin{enumerate}
                \item[{\tt -enable-kvm}] Включает ядерную поддержку виртуализации
                \item[{\tt -cdrom}] Устройство или disk image, cdrom виртуальной машины
                \item[{\tt -hda}] Устройство или disk image, представляет жесткий диск VM
                \item[{\tt -serial}] Перенаправление com порта (консоль ядра)
                \item[{\tt -net nic}] Условная модель сетевой карточки
                \item[{\tt -net tap}] TAP интерфейс, на который будет приходить сеть из VM
                \item[{\tt -boot order}] cd (вначале cdrom (с), потом диск (d))
                \item[{\tt -m}] Объем памяти для VM
              \end{enumerate}
        \end{itemize}
\end{frame}        


\begin{frame}
	\frametitle{Настройка сети на хосте}
			\begin{itemize}
				\item Создать мост {\tt brctl} и назначить ему адреса из соответствующих диапазонов {\tt ifconfig/ip}
				\item Поднять виртуальные интерфейсы {\tt ifconfig/ip}
				\item Добавить виртуальные интерфейсы к мосту {\tt brctl}
			\end{itemize}
\end{frame}


\begin{frame}
	\frametitle{Настройка сети на виртуальных машинах}
			\begin{itemize}
				\item Назначить адрес устройству eth0 {\tt ifconfig/ip}
				\item Добавить адрес маршрутизатора по умолчанию {\tt route/ip}
				\item Проверить доступность виртуальных машин и хоста {\tt ping/nmap}
			\end{itemize}
\end{frame}

\begin{frame}
	\frametitle{Настройка роутинга и NAT}
			\begin{itemize}
				\item Разрешить форвардинг на хосте
				\item Настроить NAT на хосте ({\tt iptables},  правило {\tt MASQUERADE})
				\item Проверить доступность хостов из ''внешней'' сети {\tt ping/nmap}
			\end{itemize}
\end{frame}
