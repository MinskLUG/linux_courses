\begin{frame}{ssh}

	\begin{block}{ssh -- терминал}
		{\tt ssh [user@]host[:port]}\\
		{\tt ssh host [-l user] [-p port]}
		\begin{itemize}
			\item -v -- "разговорчивый" режим 
			\item -t -- насильное назначение псевдотерминала (для автоматизации)
		\end{itemize}
		Вся конфигурация пользователя: {\tt \$HOME/.ssh}
	\end{block}

	\pause

	\begin{block}{... и не только}
		\begin{itemize}
			\item -X -- "проброс" графики 
			\item -L [bindip:]port:rhost:rport -- "пробрасывание" порта с удаленной машины на локальную
			\item -R [bindip:]port:lhost:lport -- "пробрасывание" порта с локальной машины на удаленную
			\item -W host:port -- stdin/stdout с указанным хостом
			\item -D port -- динамический прокси
		\end{itemize}
	\end{block}
\end{frame}


