\begin{frame}{Полезные утилиты}
	\begin{center}
		\begin{itemize}
			\item netstat / ss
			\item nslookup / dig
			\item ping
			\item traceroute
			\item tcpdump
			\item telnet
			\item netcat
			\item nmap
		\end{itemize}
	\end{center}

\end{frame}


\begin{frame}[allowframebreaks]{Полезные утилиты: практика}

		\begin{block}{netstat}

			Узнать:
			\begin{itemize}
				\item список используемых сокетов
				\item серверных сокетов
				\item имена/pid серверов
				\item узнать номера портов
			\end{itemize}
		\end{block}
	
		\framebreak
		\begin{block}{telnet/netcat}

			\begin{itemize}
				\item Чат по протоколу TCP с соседом
				\item Чат по протоколу UDP с соседом
				\item Передать текстовый и бинарный файлы
			\end{itemize}
	
			При создании чата использовать {\tt netstat} и {\tt tcpdump}
			для получения информации о соединении.
		\end{block}

		\framebreak
		\begin{block}{nmap}
			\begin{itemize}
				\item сканирование соседа
				\item сканирование выделенных портов у соседа (поиск сервера чата) 
				\item узнать список открытых портов на всех машинах в аудитории
				\item узнать список работающих машин
			\end{itemize}
		\end{block}
	
\end{frame}

tcpdump
0. pcap файлы/libpcap
1. запуск монитора
2. запуск чата
3. монитор-фильтр-анализ

