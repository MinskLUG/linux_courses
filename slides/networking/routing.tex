\begin{frame}{Маршрутизация}
	\begin{itemize}
		\item netstat -r
		\item route
		\item ip route show
	\end{itemize}

	\begin{block}{Разрешить маршрутизацию на хосте}
		{\tt echo 1 > /proc/sys/net/ipv4/ip\_forward}\\
		{\tt sysctl -w net.ipv4.ip\_forward=1}
	\end{block}
	
	Hint: {\tt /proc/sys/net}

	Примеры:
	\begin{itemize}
		\item ip route add default via 192.168.0.1
		\item ip route add 192.168.1.0/24 dev ppp0
		\item ip route del 192.168.1.0/24 dev ppp0
	\end{itemize}

\end{frame}


\begin{frame}[fragile]{Упражнение}
    \begin{block}{Внутренняя сеть: маршрутизация}
        \begin{enumerate}
            \item Настроить маршруты по-умолчанию:
                \begin{itemize}
                    \item Через IP адрес соседа (для {\tt ethA1}) для netns 'A'
                    \item Через IP адрес интерфейса {\tt vethA} для netns 'B'
                \end{itemize}
            \item Разрешить IP forwarding в netns 'A'
            \item Запустить {\tt ping -n <IP>} в netns 'B'\\
                IP -- адрес соседа для {\tt ethA1}
            \item Запустить {\tt tcpdump -i ethA1 icmp} в netns 'A'
        \end{enumerate}
    \end{block}
\end{frame}
