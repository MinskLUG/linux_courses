\newcounter{tmpc}

\begin{frame}{Репозиторий}
	\begin{block}{Репозиторий пакетов}
		Место, где хранятся и поддерживаются пакеты, а также сопутствующая мета-информация, предназначенное для использования пакетным менеджером.
	\end{block}
\end{frame}

\begin{frame}{Репозиторий APT (Debian-based,ALTLinux)}
		Описание репозитория для APT на локальной системе хранится по путям:
		\begin{itemize}
		    \item {\tt /etc/apt/sources.list}
		    \item {\tt /etc/apt/sources.list.d/*.list}
		\end{itemize}
\end{frame}

\begin{frame}{Apt: команды}
	\begin{block}{Обновление данных о пакетах}
		{\tt apt-get update }
	\end{block}
	\begin{block}{Поиск}
		{\tt apt-cache search pkgname }
	\end{block}
	\begin{block}{Массовое обновление пакетов}
		\begin{itemize}
		    \item {\tt apt-get dist-upgrade}
		    \item {\tt apt-get upgrade}
		\end{itemize}
	\end{block}
	\begin{block}{Установка/обновление пакета}
		{\tt apt-get install pkgname }

                {\tt apt-get -f install}
	\end{block}
	\begin{block}{Удаление пакета}
		{\tt apt-get remove pkgname }
	\end{block}
\end{frame}

\begin{frame}{Репозиторий YUM/DNF (RH-based)}

		Описание репозитория для YUM/DNF на локальной системе хранится по пути\\
		{\tt /etc/yum.repos.d/*.repo}
\end{frame}

\begin{frame}[fragile]{Репозиторий в контейнере fedora}
	\begin{block}{Пример: Fedora Core}
	\begin{itemize}
	    \item Проверить список включенных репозиториев: \\
	    {\tt dnf repolist}
	    \item При необходимости отредактировать файл \\
		        {\tt /etc/yum.repos.d/mgts.repo}:
		\begin{itemize}
		    \item в описании репозотория {\tt [fedora-mgts]}
		    \item Изменить {\tt baseurl} на ближайшее зеркало
		\end{itemize}
	\end{itemize}
	\end{block}
\end{frame}


\begin{frame}{DNF: команды}
	\begin{block}{Обновление данных о пакетах}
		{\tt dnf makecache}
	\end{block}
	\begin{block}{Массовое обновление пакетов}
		{\tt dnf update }
	\end{block}
	\begin{block}{Поиск}
		{\tt dnf list pkgname }\\
		{\tt dnf search pkgname }
	\end{block}
	\begin{block}{Установка/обновление пакета}
		{\tt dnf install pkgname }
	\end{block}
	\begin{block}{Удаление пакета}
		{\tt dnf remove pkgname }
	\end{block}
\end{frame}


\begin{frame}[fragile]{Упражнение}
  \begin{enumerate}
      %\item Использовать {\tt apt-get} и {\tt yum/dnf} в 2-х разных окружениях
      \item Удалить пакет vim
      \item Установить заново пакет vim
      \item Посмотреть списки файлов для пакетов {\tt rpm, vim}
      \item Найти, к какому пакету относится команда {\tt ls, top}
      \item Найти пакет предоставляющий сервис ssh и установить его
    \end{enumerate}
\end{frame}
