\begin{frame}{Система управления пакетами: для чего это нужно}
\begin{itemize}
 \item ''DLL Hell''
 \item Dependency hell
 \item Общие задачи пакетного менеджера:
   \begin{itemize}
     \item Проверка целостности пакетов
     \item Проверка зависимостей пакетов
        \item Поддержание списка установленных пакетов
        \item Автоматическое удаление пакетов
     \item Предоставление доступа к репозиторию пакетов
     \item Разрешение зависимостей
   \end{itemize}
\end{itemize}
\end{frame}

\begin{frame}{Debian-based и RedHat-based системы управления пакетами}
\begin{center}
 \textbf{Два уровня пакетных менеджеров}
\end{center}
\begin{columns}
  \column{0.4\textwidth}
  \begin{center}
    \textbf{RedHat-based}
  \end{center}
  \begin{itemize}
    \item yum
    \item rpm
  \end{itemize}
  \column{0.4\textwidth}
  \begin{center}
    \textbf{Debian-based}
  \end{center}
  \begin{itemize}
    \item aptitude, apt, synaptic
    \item dpkg
  \end{itemize}
\end{columns}
\end{frame}

\begin{frame}{RPM: структура пакета}
	\begin{itemize}
		\item Метаданные
			\begin{itemize}
				\item Имя
				\item Версия/Релиз
				\item Группа
				\item Описание
				\item ...
			\end{itemize}
		\item Архив с файлами
			\begin{itemize}
				\item cpio
			\end{itemize}
		\item Скрипты
			\begin{itemize}
				\item Pre Install
				\item Post Install
				\item Pre Uninstall
				\item Post Uninstall \bigskip
				\item Triggers
			\end{itemize}
	\end{itemize}
\end{frame}
