



\begin{frame}
	\frametitle{Спецсимволы}

	\begin{itemize}
		\item \# -- Вся строка после \# является комментарием
		\item ; -- Разделение команд
		\item : -- NOP оператор (похож на встроенный вызов true)
		\item {\tt source} или {\bf .} -- скрипт выполняется в текущем экземпляре shell
	\end{itemize}

\end{frame}


\begin{frame}
	\frametitle{Спецсимволы}

	\begin{block}{Фигурные скобки и склеивание с помощью ``,``}
		\begin{itemize}
			\item Посмотреть на результат выполнения команды \\
				{\tt echo \{A,B,C\}:\{1,2,3\}}
				\pause
			\item Посмотреть на результат выполнения команды \\
				{\tt ls -l \{,/usr\}/\{bin,sbin\}/*sh}
		\end{itemize}
	\end{block}

	\pause

	\begin{block}{Фигурные скобки и перечисления с помощью ``..``}
		\begin{itemize}
			\item Посмотреть на результат выполнения команды \\
				{\tt echo \{a..d\}:\{-10..10\}}
		\end{itemize}
	\end{block}

\end{frame}

\begin{frame}[fragile]
	\frametitle{Начало скрипта \#!}

        Конструкция позволяет использовать скрипты, как системные команды. 
        Первая cтрока игнорируется интерпретатором, т.к. \# - комментарий

	\begin{block}{Sha-Bang, shebang, hasbang}
		\begin{lstlisting}
#!/bin/bash
		\end{lstlisting}
	\end{block}

	\begin{block}{Режим совместимости с POSIX}
		\begin{lstlisting}
#!/bin/sh
		\end{lstlisting}

	\end{block}

\end{frame}

\begin{frame}[fragile]
	\frametitle{Задание. Строка спецсимволов}
Вывести символ * 10 раз.
				\begin{lstlisting}
echo *********
				\end{lstlisting}
\end{frame}

\begin{frame}[fragile]
	\frametitle{Экранирование}

	\begin{columns}
		\column{0.5\textwidth}
		\begin{itemize}
			\item Экранирование одного символа \textbackslash 
			\item Частичное экранирование ''
			\item Полное экранирование '
		\end{itemize}
		\pause
		\column{0.5\textwidth}
		Спецзначения для echo и sed
		\begin{itemize}
			\item \textbackslash{n} -- новая строка
			\item \textbackslash{r} -- возврат каретки
			\item \textbackslash{t} -- табуляция
			\item \textbackslash{v} -- вертикальная табуляция \\
				\small\begin{lstlisting}
echo -e "test \v test \v test"
				\end{lstlisting}
			\item \textbackslash{b} -- перемещение на 1 символ назад
			\item \textbackslash{a} -- звуковой сигнал
			\item \textbackslash{0xxx} -- 8-миричное число
			\item \textbackslash{xXX} -- 16-ричное число
		\end{itemize}
	\end{columns}

\end{frame}

\begin{frame}[fragile]
	\frametitle{Задание. Обработка пробелов.}
Сравнить результаты команд для имени, которое содержит пробелы
				\begin{lstlisting}
ls a long file name with spaces
ls "a long file name with spaces" 
				\end{lstlisting}
\end{frame}

