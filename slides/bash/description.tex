\begin{frame}
  \frametitle{Bash против скриптовых языков}
  \begin{itemize}
   \item В bash трудно делать сложные структуры данных
   \item В bash нет типизации переменных
   \item Но! В bash очень просто организовывать взаимодействие внешних программ
   \item Stream programming \texttt{(a | b | c )}
  \end{itemize}
\end{frame}

\begin{frame}
  \frametitle{Когда использовать bash}
  \begin{columns}
   \column{0.5\textwidth}
    \begin{center}
     {\Large Использовать}
    \end{center}
    \begin{itemize}
      \item Прототипирование
      \item Системные скрипты
      \item Автоматизация консоли
    \end{itemize}
   \column{0.5\textwidth}
    \begin{center}
     {\Large Не использовать}
    \end{center}
    \begin{itemize}
      \item Критичные по скорости
      \item GUI
      \item Сложные структуры данных
    \end{itemize}
  \end{columns}
\end{frame}

\begin{frame}
  \frametitle{Раздвоения личности}
  \begin{itemize}
    \item  Bash vs POSIX shell
    \item  Внешние и встроенные команды
    \begin{itemize}
      \item \texttt{cd,help,type,alias,read} чисто встроенные команды
      \item \texttt{rm,ls} внешние команды (не часть shell по сути)
      \item \texttt{echo,pwd,test} встроены в shell, но есть как внешние
    \end{itemize}
  \end{itemize}
\end{frame}
