\begin{frame}[fragile]{Поиск файлов}
	\begin{block}{find}
		\begin{itemize}
			\item {\tt -type} -- тип файлового объекта
			\item {\tt -size} -- размер
			\item {\tt -maxdepth} -- глубина рекурсии
			\item {\tt -exec} -- выполнить команду
			\item {\tt -printf} -- форматированный вывод
		\end{itemize}
	\end{block}

	\begin{block}{Примеры}
		\begin{verbatim}
find /etc -type f -size +100k  -exec ls -l {} \;
		\end{verbatim}

		\begin{verbatim}
find -type d -user altlinux
		\end{verbatim}
	
	\end{block}
\end{frame}

\begin{frame}[fragile]{xargs}
	\begin{block}{xargs}
			Утилита для создания и запуска команд из стандартного потока ввода:
		\begin{verbatim}
xargs [options] command [command options]
		\end{verbatim}

		\begin{itemize}
			\item {\tt -d} -- разделитель
			\item {\tt -0} -- null-terminated строки
			\item {\tt -I text} -- подстановка
			\item {\tt -n N} -- максимальное количество аргументов
			\item {\tt -P N} -- максимальное количество процессов
		\end{itemize}

	\end{block}
\end{frame}

\begin{frame}[fragile]{xargs}
	\begin{block}{Примеры}
		\begin{verbatim}
find /etc -type f -size -100k | \
 xargs tar -czf /tmp/archive-100k.tar.gz
		\end{verbatim}

		\begin{verbatim}
find /etc -type f | xargs -I {} echo "Найден {} файл"
		\end{verbatim}

		\begin{verbatim}
find . -type f -name "*.mp3" -print0 | \
 xargs -0 -n 1 -P 0 -I mp3 avconv -i mp3 mp3.ogg
		\end{verbatim}
	
	\end{block}
\end{frame}

\begin{frame}[fragile]{Задание}
	\begin{itemize}
		\item Создать директорию {\tt /tmp/etc}
		\item С помощью {\tt find} скопировать только файлы,
		      из {\tt /etc} (не включая поддиректории) в {\tt /tmp/etc}
		\item Сжать все файлы из {\tt /tmp/etc}, используя все доступные процессоры
	\end{itemize}
\end{frame}

