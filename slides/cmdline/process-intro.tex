\begin{frame}{Процессы в UNIX}
  \begin{itemize}
    \item Создание процессов
      \begin{itemize}
        \item fork
        \item exec
      \end{itemize}
    \item Атрибуты процесса
      \begin{itemize}
        \item pid
        \item файловые дескрипторы
        \item environment
        \item рабочая директория (cwd)
        \item прочее в директории {\tt /proc/<pid>}
      \end{itemize}
  \end{itemize}
\end{frame}

\begin{frame}{Управление процессами}
  \begin{itemize}
    \item kill (killall)
    \item top
    \item ps
    \item pstree
    \item Команды управления процессами в bash: 
      \begin{itemize}
        \item {\tt jobs}, {\tt fg}, {\tt bg}, {\tt wait}
        \item Ctrl-C -- оборвать выполнение процесса (SIGINT)
        \item Ctrl-Z -- остановить выполнение команды (SIGTSTP)
        \item Ctrl-D -- завершить ввод
      \end{itemize}
  \end{itemize}
\end{frame}


\begin{frame}{Упражнения}
  \begin{block}{Посмотреть дерево процессов}
    {\tt pstree} \\
    {\tt ps -AfwwH}
  \end{block}
  \pause
  \begin{block}{Ctrl-C, Ctrl-Z}
    Запустить из терминала: emacs \\
    Ctrl-Z \\
    sleep infinity \& \\
    jobs -l \\
    bg +
  \end{block}
  \pause
  \begin{block}{fork bomb}

    {\tt ulimit -u 200} 

    {\tt bomb()\{ (bomb; bomb) \& \} }

    top

    killall bash

  \end{block}
\end{frame}

\begin{frame}[fragile]
  \frametitle{Доп. упражнение}
  \begin{block}{proc файловая система}
    Посмотреть вывод \verb+ ls -l /proc/$$/ +

    \verb+ ls -l /proc/$$/cwd; cd .. ; ls -l /proc/$$/cwd +
    
    \verb+ cat /proc/$$/limits; ulimit -u 200 ; cat /proc/$$/limits +

  \end{block}
\end{frame}


%\begin{frame}{Unix way}
%  \begin{enumerate}
%    \item Пишите программы, которые делают одну вещь и делают её хорошо.
%    \item Пишите программы, которые бы работали вместе.
%    \item Пишите программы, которые бы поддерживали текстовые потоки, поскольку это универсальный интерфейс. 
%  \end{enumerate}
%\end{frame}
%
%\begin{frame}{Unix way}
%  \begin{enumerate}
%    \item   Маленькое прекрасно.
%    \item   Пусть каждая программа делает одну вещь, но хорошо.
%    \item   Собирайте прототип как можно раньше.
%    \item   Предпочитайте переносимость эффективности.
%    \item   Храните данные в простых текстовых файлах.
%    \item   Используйте программные рычаги для достижения цели.
%    \item   Используйте сценарии командной строки для улучшения функционала и переносимости.
%    \item   Избегайте <<связывающего>> (captive) пользовательского интерфейса.
%    \item   Делайте каждую программу «фильтром».
%  \end{enumerate}
%\end{frame}
