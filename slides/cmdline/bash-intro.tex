\begin{frame}{Важные аббревиатуры внутри командной строки}
  \begin{itemize}
    \item Для директорий
      \begin{itemize}
        \item {\tt $\sim$} Домашняя директория
        \item {\tt $\sim$<username>} Домашняя директория пользователя
        \item {\tt ..} Родительская директория
        \item {\tt .} Текущая директория
      \end{itemize}
      \pause  
    \item Wildcards
      \begin{itemize}
        \item {\tt *} Любой набор символов {\tt file*txt : file1.txt filefilefiletxt}
        \item {\tt $[$<список>$]$ } символ из заданного набора
        \item {\tt ?} любой один символ
      \end{itemize}

  \end{itemize}
\end{frame}       

\begin{frame}{Горячие клавиши}
  \begin{itemize}
    \item \textbf{Tab} -- дополнение текущей команды
      \pause
    \item История команд
      \begin{itemize}
        \item Клавиши курсора -- навигация по истории
        \item Ctrl-R -- поиск в истории по фрагменту
        \item Ctrl-O (после выполнения вставить следующую команду из истории)
        \item Команда {\tt history}
      \end{itemize}
    \item Навигация

  \end{itemize}
\end{frame}

\begin{frame}{Переменные окружения}
  \begin{itemize}
    \item {\tt HOME}
    \item {\tt PWD}
    \item {\tt LANG}
    \item {\tt LD\_LIBRARY\_PATH}
    \item {\tt SHELL}
    \item {\tt TERM}
    \item {\tt DISPLAY}
  \end{itemize}

  Контроль

  \begin{itemize}
    \item export {\tt export VAR=value}
    \item declare -x
    \item echo 
  \end{itemize}

  Переменные окружения наследуются при создании нового процесса
\end{frame}

%\begin{frame}{Настройки bash и кастомизация}
%  \begin{itemize}
%    \item Login shell
%      \begin{itemize}
%        \item {\tt /etc/profile}
%        \item {\tt $\sim$/.profile }
%      \end{itemize}
%    \item Обычная интерактивная shell
%      \begin{itemize}
%        \item {\tt /etc/bash.bashrc}
%        \item {\tt $\sim$/.bashrc}
%      \end{itemize}
%  \end{itemize}
%
%  Полезные команды
%  \begin{itemize}
%    \item {\tt alias}
%    \item {\tt export PATH=}
%    \item {\tt Определение функции}
%    \item {\tt shopts}
%  \end{itemize}
%
%\end{frame}


