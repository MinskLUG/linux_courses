\begin{frame}{Навигация по файловой системе}
      \begin{itemize}
		  \item {\tt ls} -- список файлов в (текущей по умолчанию) директории (man ls)
		  \item {\tt cd} -- смена текущей директории (help cd)
		  \item {\tt pwd} -- имя текущей директории (help pwd)
      \end{itemize}
\end{frame}

\begin{frame}[fragile]{Команды для работы с файлами}
	\begin{itemize}
		\begin{columns}
		\column{0.2\textwidth}
			\item touch
			\item ln
			\item mkdir
			\item mknod
			\item mkfifo
		\column{0.2\textwidth}
			\item cp
			\item mv
			\item install
			\item rm
			\item rmdir
			\item file
		\column{0.4\textwidth}
			\begin{block}{Упражнение}
				\begin{enumerate}
					\item Создать иерархию директорий
						\begin{lstlisting}
dir1/dir1.1/dir1.1.1
dir1/dir1.2/dir1.2.1
dir1/dir1.2/dir1.2.2
						\end{lstlisting}
					\item Внутри каждой создать файл
					\item Удалить все созданное
				\end{enumerate}
			\end{block}
		\end{columns}
	\end{itemize}
\end{frame}


