\begin{frame}{VI}
	\begin{block}<1>{Два режима работы}
		\begin{itemize}
			\item все портить
			\item бибикать
		\end{itemize}
	\end{block}
	\pause
	\begin{block}<2->{Два режима работы}
		\begin{itemize}
			\item командный 
			\item текстового ввода
		\end{itemize}
	\end{block}

	\begin{block}<2->{Переход между режимами}
		\begin{itemize}
			\item из командного в текстовый: i, a, R (вставка, добавление, замена)
			\item из текстового в командный: ESC или Ctrl-[
		\end{itemize}
	\end{block}

\end{frame}

\begin{frame}{VI: command mode}

	\begin{block}{Перемещение курсора}
		\begin{itemize}
			\item h, j, k, l -- влево, вниз, вверх, вправо (1 элемент)
			\item \^{} или 0 -- в начало строки
			\item \$ -- в конец строки
			\item w, b -- вперед, назад на слово
			\item gg, G -- начало, конец текста (<num>G -- на строку <num>)
		\end{itemize}
	\end{block}

	\begin{block}{Редактирование}
		\begin{itemize}
			\item d -- удалить (d -- текущую строку, w -- слово)
			\item u -- отмена предыдущего изменения
			\item . -- повтор
		\end{itemize}
	\end{block}
\end{frame}

\begin{frame}{VI: command mode}

	\begin{block}{Работа с неименованым буфером}
		\begin{itemize}
			\item y -- удалить (d -- текущую строку, w -- слово)
			\item p -- вставить из буфера
		\end{itemize}
	\end{block}

	\begin{block}{Командная строка}
		: -- переход в режим командной строки
		\begin{itemize}
			\item :q или :q! -- выход без сохранения
			\item :w -- сохранить изменения
			\item :x -- выйти с сохранением (:wq)
		\end{itemize}
	\end{block}
\end{frame}

\begin{frame}{Задание на дом}
\begin{block}{}
vimtutor ru
\end{block}
\end{frame}


