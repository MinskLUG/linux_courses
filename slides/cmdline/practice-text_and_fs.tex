\begin{frame}[fragile]{Практика: работа с текстовыми файлами}
  \begin{enumerate}
	  \item Посмотреть вывод команды {\tt dmesg}
	  \item Вывести на экран первые 10 строк 
		  \pause
	  \item Создать директорию в {\tt /tmp} и перейти в нее
	  \item Скопировать в файл1 последние 100 строк из {\tt /var/log/messages}
	  \item Запустить отслеживание изменений в файле1 на отдельной консоли
	  \item Дописать в файл1 из файла {\tt /etc/services}) все строки,
		  содержащие слова {\tt mail} или {\tt cache}
		  \pause
	  \item Запустить отслеживание изменений одновременно в файле1 и в {\tt /var/log/messages}\\
		  \pause
			повторить предыдущий пункт, убрав все комментарии при помощи {\tt sed}\\
		  \pause
		  ... и добиться параллельного вывода результатов на экран и во временный файл
	  \item
  \end{enumerate}
\end{frame}

\begin{frame}{Практика: работа с файловыми объектами}
	\begin{enumerate}
		\item Cкопировать файл1 в файл2
			\pause
		\item Добиться того, чтобы файл2 состоял из 3 одинаковых копий файла1
		\item Создать жесткую ссылку на файл1
		\item Создать символическую ссылку на файл2
		\item Вывести на экран список всех файлов
			\pause
		\item Добавить вывод команды {\tt date} в файл1
		\item Вывести на экран список всех файлов
		\item Вывести на экран содержимое файла1 и жесткой ссылки
		\item Создать именованный канал (PIPE)
		\item Запустить одну команду {\tt cat} на чтение из PIPE
		\item Записать содержимое файла1 и файла2 в PIPE
		\item Удалить временную директорию
	\end{enumerate}
\end{frame}


