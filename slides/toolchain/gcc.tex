\begin{frame}[fragile]
	\frametitle{GCC}

	\begin{block}{GNU Compiler Collection}

        \sout{GNU C Compiler}

        Модульная структура:

		\begin{itemize}
			\item оптимизаторы 
			\item ''back ends'' -- генераторы кода под различные архитектуры
			\item ''front ends'' -- реализация конкретного языка
		\end{itemize}
	\end{block}

\end{frame}


\begin{frame}[fragile]
	\frametitle{GCC}

	\begin{block}{Языки}

		\begin{itemize}
			\item C
			\item C++
			\item Objective-C
			\item Objective-C++
			\item Java
			\item Fortran
			\item Ada
			\item Go
		\end{itemize}
	\end{block}

\end{frame}



\begin{frame}[fragile]
\frametitle{Компоненты gcc}
\begin{itemize}
  \item Препроцессор
\begin{lstlisting}[language=sh]
gcc -E
\end{lstlisting}
  \item Компилятор (+ассемблер)
\begin{lstlisting}[language=sh]
gcc -S, gcc -c
\end{lstlisting}
  \item Линкер
\begin{lstlisting}[language=sh]
gcc -o myprogram file1.o file2.o -lmylib1 -lmylib2
\end{lstlisting}
\end{itemize}
\end{frame}

\begin{frame}
\frametitle{Упражнение}
\begin{itemize}
  \item Написать программу выводящую {\tt Hello world!} на C
  \item Прогнать стадию препроцессора (helloworld.i)
  \item Скомпилировать helloworld.i в ассемблерный код
  \item Скомпилировать ассемблерный код helloworld.s в объектный код helloworld.o
  \item Слинковать helloworld.o и запустить программу
\end{itemize}
\end{frame}


\begin{frame}
\frametitle{Флаги gcc}

	\begin{block}{Позитивные и негативные формы}
		\begin{itemize}
			\item -ffoo / -fno-foo ({\tt -fno-builtin})
			\item -Wbar / -Wno-bar ({\tt -Wunused-variable})
		\end{itemize}
	\end{block}


	\begin{itemize}
	  \item Debug {\tt -g}
	  \item Предупреждения компилятора {\tt -Wall, -Werror}
	  \item Оптимизация {\tt -O -O1 -O2 -O3 -Os}
      \item Поиск файлов {\tt -I, -L}
	\end{itemize}
\end{frame}

\begin{frame}[fragile]
  \frametitle{Помощь по флагам и опциям}
\begin{lstlisting}[language=sh]
gcc --help=<smth>
\end{lstlisting}
\begin{block}{Некоторые виды опций у \texttt{--help}}
  \begin{itemize}
    \item \texttt{optimizers}
    \item \texttt{warnings}
    \item \texttt{params}
    \item \texttt{targets}
  \end{itemize}
\end{block}
\begin{lstlisting}[language=sh]
gcc -Q -O2 --help=optimizers
gcc -Q -Wall --help=warnings
\end{lstlisting}
\end{frame}

\begin{frame}[fragile]
	\frametitle{GCC environment variables}

	\begin{block}{Компилятор}
		\begin{itemize}
			%\item {\tt CC} -- (пере)определение компилятора
			%\item {\tt CFLAGS} -- флаги для компилятора
		    \item {\tt CPATH, C\_INCLUDE\_PATH} -- расположение включаемых файлов для {\tt -I}
		\end{itemize}
	\end{block}
    \begin{block}{Линкер}
      \begin{itemize}
          \item {\tt LIBRARY\_PATH} -- в режиме линкера, расположение библиотек {\tt -L} 
      \end{itemize}
    \end{block}
\end{frame}

\begin{frame}
	\frametitle{Упражнение}
	\begin{itemize}
		\item Добавить в код неиспользуемую переменную
		\item Скомпилировать объектный файл с установленной опцией {\tt -Wall}
		\item Скомпилировать объектный файл с установленными опциями {\tt -Wall -Werror}
	\end{itemize}
\end{frame}

\begin{frame}[fragile]
\frametitle{GNU Extensions}

	\begin{block}{Стандарты}
		\begin{itemize}
            \item {\tt -{}-std=} -- включить стандарт
			\item -Wpedandic, -pedantic -- предупреждения о несоответсвии
		\end{itemize}
        {\tt -Wall -pedantic -Werror} -- ''жесткое'' следование стандартам
	\end{block}

	\begin{block}{\_\_GNU\_\_}
        Использование расширения GNU предваряется макросом {\tt \_\_GNU\_\_} или {\tt \_\_GNUC\_\_}.
		\begin{itemize}
            \item pragmas -- как правило, для компиляции исходников для других компиляторов\\
                {\tt grep \_\_GNU /usr/include/ -R}
            \item атрибуты -- для переменных, функций, типов и т.д.
            \item многое другое 
		\end{itemize}
	\end{block}
\end{frame}

\begin{frame}[fragile]
\frametitle{GNU Extensions: примеры}

\begin{lstlisting}[language=C]
#pragma GCC diagnostic error "-Wuninitialized"
foo(a);         /* error */
#pragma GCC diagnostic push
#pragma GCC diagnostic ignored "-Wuninitialized"
foo(b);         /* no diagnostic */
#pragma GCC diagnostic pop
\end{lstlisting}

\begin{lstlisting}[language=C]
struct S { short f[3]; } __attribute__ ((aligned (8)));

extern void die(const char *format,  ...)
    __attribute__((noreturn, format(printf,  1,  2)));

void stub(char c) __attribute__((weak)){}
\end{lstlisting}
\end{frame}


\begin{frame}
  \frametitle{Доп. упражнение для желающих}
  \begin{itemize}
    \item Скомпилировать \texttt{helloworld.c} в ассемблерный код с опцией \texttt{-fno-builtin} и посмотреть отличия
    \item Выяснить какие флаги оптимизации отличаются при -O2 и -O3 (hint: \texttt{gcc -Q -{}-help=} и \texttt{diff}
    \item Скомпилировать \texttt{helloworld.c} с \texttt{-fdump-passes}
  \end{itemize}
\end{frame}
