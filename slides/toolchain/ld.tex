\begin{frame}
 \frametitle{Статическая и динамическая линковка}
 \begin{itemize}
   \item Статическая линковка
     \begin{columns}
       достоинства недостатки
     \end{columns}
   \item Динамическая линковка
 \end{itemize}
\end{frame}
\begin{frame}
 \frametitle{Создание статических библиотек}
\begin{lstlisting}[language=sh]
gcc -g -Wall -c  file1.c
gcc -g -Wall -c  file2.c
ar rcs libmylib.a file1.o file2.o
#ranlib libmylib.a
gcc -o file3 file3.c -L. -lmylib # Использование
\end{lstlisting}
\end{frame}

\begin{frame}
  \frametitle{Создание динамических разделяемых библиотек}
\begin{lstlisting}[language=sh]
gcc -g -fpic -c -Wall   file1.c; gcc -g -Wall -fpic -c  file2.c
gcc -g -shared -Wl,-soname,libmylib.so.0 -o libmylib.so.0.0 
cp libmylib.so.0.0 /usr/local/lib/
ldconfig 
gcc -g -o file3 file3.o -lmylib
\end{lstlisting}
\end{frame}

\begin{frame}
 \frametitle{Проблемы при линковке}
 \begin{itemize}
   \item Underlinking
   \item Overlinking
   \item Несовместимость версий (soname etc.)
 \end{itemize}
\end{frame}

\begin{frame}
  \frametitle{Поиск динамических библиотек}
  \begin{itemize}
    \item \verb+ LD_LIBRARY_PATH +
    \item {\tt ldconfig}
    \item {\tt /etc/ld.so.conf}
    \item {\tt /etc/ld.so.cache}
  \end{itemize}
\end{frame}

\begin{frame}
  \frametitle{Упражнение}
\end{frame}
