\begin{frame}
  \frametitle{gdb server: preparation}
  \begin{center}
    {\large Подготовка ядра:компиляция с debug символами}
  \end{center}
  \begin{itemize}
    \item Взять исходники ядра с kernel.org
    \item Развернуть исходники \texttt{tar -Jxvf linux-....tar.xz}
    \item В директории исходников \texttt{make defconfig}
    \item \texttt{make menuconfig}
    \item Kernel hacking->Compile-time checks and compiler options->Compile kernel with debug info
    \item \texttt{make}
  \end{itemize}
\end{frame}

\begin{frame}
  \frametitle{gdb server: запуск}
\begin{itemize}
    \item Запустить процесс ядра в kvm

      \texttt{ kvm -m 256M -S -gdb tcp::2345 -kernel linux-.../arch/x86\_64/boot/bzImage }
    \item Запустить gdb \texttt{gdb linux-.../vmlinux}
    \begin{itemize}
      \item target remote localhost:2345
      \item c
      \item Остановить с помощью ловкости рук и Ctrl-C
      \item \texttt{set arch i386:x86-64:intel}
      \item Начать пошагово выполнять код ядра
    \end{itemize}
\end{itemize}
\end{frame}
