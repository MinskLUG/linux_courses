%%
%% 

\begin{frame}
	\frametitle{Valgrind}

	\begin{block}{Что умеет Valgrind}
		\begin{itemize}
			\item Утечки памяти
			\item Использование неинициализированной памяти
			\item Чтение/запись за пределами выделенной {\tt malloc} памяти
			\item Чтение/запись в некорректные области стека
			\item Использование неправильных пар вуделения/удаления блоков памяти:\\
				{\tt malloc/new/new[]} и {\tt free/delete/delete[]}
			\item Перекрытие участков памяти при использованиии {\tt memcpy()} и др.
			\item "Злоупотребления" {\tt API pthreads}
		\end{itemize}
	\end{block}
\end{frame}

\begin{frame}
	\frametitle{Valgrind}

	\begin{block}{Valgrind Tools}
		\begin{enumerate}
			\item {\bf Memcheck} -- определяет ошибки при работе с памятью.
			\item {\bf Cachegrind} -- профайлер работы с кэшем и ветвлений.
			\item {\bf Callgrind} -- еще один кэш-профайлер.
			\item {\bf Helgrind} -- определяет ошибки в потоках.
			\item {\bf DRD} -- и еще один детектор ошибок в потоках.
			\item {\bf Massif} -- профайлер "кучи".
			\item {\bf DHAT} -- ...вы не поверите, но да, еще один.
			\item {\bf SGcheck} -- детектор работы со стеком и глобальными массивами.
			\item {\bf BBV} -- генератор блок-схем.
		\end{enumerate}
	\end{block}
\end{frame}
			
\begin{frame}[fragile]
	\frametitle{Упражнение: некорректный указатель}

	\begin{lstlisting}
#include <stdlib.h>

main(void)
{
char *array;

    array = (char *) malloc(10*sizeof(char));
    array[10] = 'x';
    free( array);
    exit(0);
}
	\end{lstlisting}

	Пример: {\tt inv\_ptr.c}

\end{frame}

\begin{frame}[fragile]
	\frametitle{Упражнение: утечки памяти}

	\begin{lstlisting}
char *array;
int i;

for( i=0; i<100; i++)
    array = (char *) malloc(10*sizeof(char));

free( array);
	\end{lstlisting}

	\begin{block}{Детализация}
		{\tt -{}-leak-check=full}
	\end{block}

	Пример: {\tt leaks.c}

\end{frame}

\begin{frame}[fragile]
	\frametitle{Упражнение: объем выделенной памяти}

	\begin{lstlisting}
array = (char *) malloc(1<<31);
array[10000] = 0;
	\end{lstlisting}

	Пример: {\tt huge\_size.c}
	
	\bigskip
	\pause

	Задание:
	\begin{enumerate}
		\item изменить объем запрошенной памяти на больший, чем доступно в системе
	\end{enumerate}

\end{frame}

\begin{frame}[fragile]
	\frametitle{Упражнение: использование переменной вместо указателя}

	\begin{lstlisting}
    int i;
    scanf("%d", i);
	\end{lstlisting}

	Пример: {\tt inv\_type.c}
\end{frame}

\begin{frame}[fragile]
	\frametitle{Упражнение: неинициализированные переменные}

	\begin{lstlisting}
    int a;
    if ( a)
        (...};
	\end{lstlisting}

	Пример: {\tt uninit\_var.c}
\end{frame}

\begin{frame}[fragile]
	\frametitle{GDB + Valgrind}

	Valgrind запускет программы на "синтетическом" процессоре, поэтому GDB не работает!

	Для отладки используется {\tt vgdb}: \\

	\begin{verbatim}
valgrind -vgdb=yes --vgdb-error=0 prog
	\end{verbatim}


	а {\tt gdb} запустить:
	\begin{verbatim}
gdb prog
(gdb) target remote | vgdb
	\end{verbatim}

	\pause
	\begin{block}{Упражнение}
		Попробовать провести отладку программы {\tt segfault.c} с помощью связки Valgrind и GDB.
	\end{block}
\end{frame}



