\begin{frame}
 \frametitle{Использование printf и логов}
 \begin{columns}
   \column{0.45\textwidth}
    \begin{center}
     Преимущества 
    \end{center}
    \begin{itemize}
     \item Работает везде и всегда
     \begin{itemize}
       \item В многопоточных программах
       \item При вызове из других программ
       \item Внутри ядра (printk)
       \item В реальном времени
       \item В разных языках
      \end{itemize}
    \end{itemize}
   \column{0.45\textwidth}
    \begin{center}
     Недостататки
    \end{center}
    \begin{itemize}
     \item Нужна перекомпиляция
     \item Нет интерактивности
     \item Нужно вычищать
     \item Трудно печатать сложные структуры
     \item Нет backtrace
    \end{itemize}
 \end{columns}
\end{frame}

\begin{frame}[fragile]
	\frametitle{Подготовка}

	\begin{block}{Заготовка {\tt trace.c}}
	
	\begin{lstlisting}
#include <stdio.h>

void second (void) {
}

void first (void) {
  second();
}

int main (void) {
  first();
  return 0;
}
	\end{lstlisting}
	\end{block}

\end{frame}


\begin{frame}[fragile]
	\frametitle{Полезные макросы}

	\begin{itemize}
		\item {\tt \_\_FILE\_\_}
		\item {\tt \_\_LINE\_\_}
		\item {\tt \_\_func\_\_}
	\end{itemize}

	\begin{block}{Задание}
		Добавить макрос печатающий на экран текущее имя файла, номер строки и имя функции.

		Вставить макрос в начало каждой функции.
	\end{block}

\end{frame}



