\subsection[intro]{Основы gdb}
\begin{frame}
  \frametitle{Базовые команды gdb}
  \begin{columns}
    \begin{column}{0.3\textwidth}
      \begin{itemize}
        \item help
        \item b (break)
        \item c (continue)
        \item s (step)
      \end{itemize}
    \end{column}
    
    \begin{column}{0.3\textwidth}
      \begin{itemize}
        \item p (print)
        \item l (list)
        \item (pt) ptype
        \item watch
      \end{itemize}
    \end{column}

    \begin{column}{0.3\textwidth}
      \begin{itemize}
        \item bt (backtrace)
        \item finish
        \item return
        \item frame
      \end{itemize}
    \end{column}
  \end{columns}
\end{frame}

\subsection[breakpoints]{Точки останова}
\begin{frame}
 \frametitle{Обращение с точками останова}
 \begin{columns}
 \column{0.5\textwidth}
 \begin{itemize}
   \item{Варианты установки}
   \begin{itemize}
     \item \texttt{break}
     \item \texttt{break <line>}
     \item \texttt{break <file>:<line>}
     \item \texttt{break <function>}
     \item \texttt{break <address>}
     \item \texttt{break <somewhere> if <condition>}
   \end{itemize}
\end{itemize}
\column{0.5\textwidth}
  \begin{itemize}   
    \item Удаление
     \begin{itemize}
       \item \texttt{clear} 
       \item \texttt{delete}
     \end{itemize}
     \item Временное отключение
     \begin{itemize}
       \item \texttt{disable}
       \item \texttt{enable}
       \item \texttt{ignore <n>}
     \end{itemize}
  \end{itemize}
\end{columns}
\end{frame}

\begin{frame}
 \frametitle{Отладка внешнего процесса}
 \begin{itemize}
   \item \texttt{gdb <file> -p pid}
   \item \texttt{attach} внутри gdb 
 \end{itemize} 
\end{frame}

\begin{frame}
  \frametitle{Упражнение}
  \begin{enumerate}
    \item Собрать программку counter и запустить ее с начальным значением 100
    \item Найти pid запущенной программки и подключиться к ней с gdb
    \item Установить breakpoint внутри цикла
    \item Продолжить выполнение
    \item Удалить breakpoint внутри цикла
    \item Установить breakpoint при i==20
    \item После остановки на breakpoint'е изменить значение howmany из gdb на 30
    \item Продолжить выполнение
  \end{enumerate}
\end{frame}

\begin{frame}
  \frametitle{Работа со стеком и core dump.}
  \begin{center}
    Упражнение
  \end{center}
  \begin{enumerate}
   \item Скомпилировать \texttt{segfault.c}
   \item Установить \texttt{ulimit -c 1000}\\
	   {\small Для некоторых дистрибутивов:\\ {\tt sysctl -w kernel.core\_pattern=/tmp/core-\%e-\%p}}
   \item Запустить segfault, дождаться segmentation fault
   \item Запустить \texttt{gdb ./segfault /tmp/core-....}
   \item Изучить значения переменных в текущем стековом фрейме \texttt{info locals}
   \item Изучить значения переменных в предыдущем стековом фрейме
  \end{enumerate}
%+core dump
\end{frame}

\begin{frame}
  \frametitle{Watchpoint}
  \begin{itemize}
    \item \texttt{watch} <переменная, выражение, *адрес>
    \item \texttt{rwatch} 
    \item \texttt{awatch}
  \end{itemize}
  \begin{center}
    Продолжение предыдущего упражнения
  \end{center}
  \begin{itemize}
    \item Запустить segfault под gdb
    \item Остановиться в main и поставить watchpoint на buf
    \item Найти место в котором портится buf
    \item Исправить программу
  \end{itemize}
\end{frame}

\begin{frame}[fragile]
  \frametitle{Отладочная информация в отдельных файлах (dbg)}
  \begin{center}
    Последовательность создания отдельного файла с отладочными символами
  \end{center}
  \begin{itemize}
    \item \verb+ objcopy --only-keep-debug <exec file> <dbg file>+
    \item \verb+ strip -g <exec file>+
    \item \verb+ objcopy  --add-gnu-debuglink=<dbg file> <exec file> + 
    \item \texttt{gdb: set debug-file-directory}
  \end{itemize} 
\pause
  \begin{center}
   Упражнение
  \end{center}
  \begin{itemize}
    \item Добавить к Makefile дополнительную цель counter.dbg, которая создаст \texttt{counter} (stripped) и \texttt{counter.dbg}
    \item Запустить gdb на файле counter и убедиться, что все работает
  \end{itemize}
\end{frame}  

\section[gdb server]{gdb server}

\begin{frame}
  \frametitle{gdb server: preparation}
  \begin{center}
    {\large Подготовка ядра:компиляция с debug символами}
  \end{center}
  \begin{itemize}
    \item Взять исходники ядра с kernel.org
    \item Развернуть исходники \texttt{tar -Jxvf linux-....tar.xz}
    \item В директории исходников \texttt{cp arch/x86/configs/x86\_64\_defconfig ./.config}
    \item \texttt{make menuconfig}
    \item Kernel hacking->Compile-time checks and compiler options->Compile kernel with debug info
    \item \texttt{make}
  \end{itemize}
\end{frame}

\begin{frame}
  \frametitle{gdb server: запуск}
\begin{itemize}
    \item Запустить процесс ядра в kvm

      \texttt{ kvm -m 256M -S -gdb tcp::2345 -kernel linux-3.19.1/arch/x86\_64/boot/bzImage }
    \item Запустить gdb \texttt{gdb linux-3.19.1/vmlinux}
    \begin{itemize}
      \item target remote localhost:2345
      \item c
      \item Остановить с помощью ловкости рук и Ctrl-C
      \item \texttt{set arch i386:x86-64:intel}
      \item Начать пошагово выполнять код ядра
    \end{itemize}
\end{itemize}
\end{frame}
  
\section[gdb disasm]{Работа с программами без исходников}
\begin{frame}[fragile]
  \frametitle{Основные команды для дизассемблирования}
  \begin{itemize}
    \item \verb+ info registers {pc, rax, ebp} +
    \item \verb+ print $pc +,etc
    \item \verb+ x/i $pc + -текущая инструкция
    \item \verb+ x/10i $pc-30 + -- предыдущие инструкции
    \item \verb+ disass + -- дизассемблирование текущей функции
    \item \verb+ si, stepi + -- шаг на инструкцию
    \item \verb+ ni, nexti + -- шаг на инструкцию с обходом
    \item \verb+ break *<address> +
  \end{itemize}
\end{frame}

\begin{frame}
  \frametitle{Команда x}
  Основные ключи
  \begin{itemize}
    \item \texttt{x/i <address>} объект по адресу как инструкция (\texttt{x/10i} 10 инструкций)
    \item \texttt{x/s <address>} как строка
    \item \texttt{x/d <address>} как десятичное число
    \item \texttt{x/a <address>} как pointer
    \item \texttt{x/x <address>} как 16-ричное число
  \end{itemize}
\end{frame}

\begin{frame}[fragile]
  \frametitle{Патчинг программы}
  \begin{verbatim}
gdb -write -q <program name>
  \end{verbatim}
\pause
\vspace{2ex}
  \begin{verbatim}
(gdb) set {unsigned char}<address>=<value>
  \end{verbatim}
\end{frame}

\begin{frame}
  \frametitle{Упражнение}
  \begin{itemize}
    \item Запустить testtimer
    \item Запустить testtimer в gdb, поставить breakpoint в функции time
    \item Найти точку возврата из этой функции в main
    \item Обойти проверку
  \end{itemize}
  \begin{tabular}{ll}
    инструкция & opcode \\
    jle        & 7e ... \\
    jle        & 0f 8e .. \\
    jg         & 7f ... \\
    jg         & 0f 8f ...\\
    setg       & 0f 9f ... \\
    setle      & 0f 9e ... \\
  \end{tabular}
\end{frame}




