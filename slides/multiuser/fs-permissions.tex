\begin{frame}{Типы разрешений для файлов}
	\begin{columns}
		\column{0.48\textwidth}
		\begin{center}
			\textbf{Разрешения для файла}
		\end{center}
		\begin{itemize}
			\item Три типа разрешений
				\begin{enumerate}
					\item чтение read(r)
					\item запись write(w)
					\item выполнение execute(x)
				\end{enumerate}
		\end{itemize}
		\column{0.48\textwidth}
		\begin{center}
			\textbf{Разрешения для директорий}
		\end{center}
		\begin{itemize}
			\item Три типа разрешений
				\begin{enumerate}
					\item поиск файлов в директории read(r) 
					\item добавление и удаление файлов write(w)
					\item заход в директорию execute(x)
				\end{enumerate}
		\end{itemize}
	\end{columns}

	\pause

	Попробовать {\tt ls -l /usr/bin}

	\pause

	Пересчет мнемонического разрешения в битовую маску 

	$r\to4, w\to2 , x\to1$ 

	rwxrw-r-x$\to$765
\end{frame}

\begin{frame}{Команды для управления пользователями и разрешениями файлов}
	\begin{columns}
		\column{0.48\textwidth}
		\begin{itemize}
			\item {\tt chown}
			\item {\tt chmod}
		\end{itemize}
		\column{0.48\textwidth}
		\begin{itemize}
			\item {\tt useradd, usermod, userdel}
			\item {\tt groupadd, groupmod, groupdel}
			\item {\tt su, sudo}
		\end{itemize}
	\end{columns}
\end{frame}

\begin{frame}
    \frametitle{}
	\begin{block}{Упражнения}
		\begin{enumerate}
			\item Создать директорию без r разрешения но с x разрешением, внутри нее создать поддиректорию с rwx разрешениями (для пользователя \defaultuser)
			\item Создать нового пользователя testuser.
			\item Скопировать {\tt /bin/bash} (под именем mysh) в домашнюю директорию пользователя \defaultuser  и поставить r-x разрешение только для other
			\item Попробовать выполнить скопированный файл от имени пользователя \defaultuser, затем от имени пользователя testuser
       \end{enumerate}
    \end{block}
\end{frame}
\begin{frame}
    \frametitle{}
	\begin{block}{Упражнения}
		\begin{enumerate}
			\item Создать новую группу testgroup
			\item Изменить группу владеющую mysh на testgroup и сделать {\tt chmod 474 mysh}
			\item Попробовать выполнить mysh от имени \defaultuser и root. 
			\item Добавить пользователя \defaultuser в группу testgroup и попробовать выполнить mysh еще раз
			\item Получить список групп которым принадлежат устройства в {\tt /dev}
		\end{enumerate}
	\end{block}
\end{frame}

\begin{frame}{SUID программы}
	\begin{block}{Попробовать}
		{\tt id}

		{\tt ls -l `which su`}
	\end{block}
	\pause
	\begin{itemize}
		\item Некоторые программы должны выполняться от имени обычного пользователя, но иметь больше привилегий
		\item Для этого у них устанавливается suid или sgid биты
		\item Установка suid (например {\tt chmod 4710 <file>})
	\end{itemize}
	\pause
	\begin{block}{Упражнение}
		\begin{itemize}
			\item Под root создать копию утилиты {\tt id} (назвать, например, {\tt id2}) в директории /usr/bin/
			\item Установить suid бит для этой утилиты
			\item Запустить {\tt id2} от имени пользователя \defaultuser
			\item То же с sgid битом
		\end{itemize}
	\end{block}
\end{frame}

\begin{frame}{Опасности SUID}
	\begin{itemize}
		\item Возможность backdoor через suid программу
			\begin{itemize}
				\item Shell игнорирует effective uid
				\item Скрипты обычно тоже игнорируют
				\item nosuid mount option
			\end{itemize}
		\item Атака через buffer overflow в существующей suid программе
			\begin{itemize}
				\item не использовать strcpy, sprintf, ... в security critical
				\item А если все же не уследили
					\begin{itemize}
						\item рандомизация стека
						\item grsecurity
						\item частично selinux
					\end{itemize}
			\end{itemize}
	\end{itemize}
\end{frame}


\begin{frame}{SUID, SGID и sticky bit для директорий}
	\begin{itemize}
		\item sgid для директорий -- все поддиректории и файлы внутри имеют тот же group id
		\item suid -- игнорируется
		\item Sticky bit (\tt{chmod +t mydir})
          \begin{itemize}
            \item Файлы из обычной директории может удалять любой пользователь с правами на запись в \emph{директорию}
            \item Файлы из директории со sticky bit может удалять только владелец директории, владелец файла или root.
          \end{itemize} 
	\end{itemize}
\end{frame}

\begin{frame}[fragile]
 \frametitle{UMASK}

	\begin{block}{umask}
		маска режима создания пользовательских файлов
	\end{block}

	Права доступа файлов, вычисляются c помощью побитовых операций:
    \begin{itemize}
      \item библиотечный вызов \tt{fopen} создает файл с разрешениями 
     \verb+ 0666 & ~umask +
      \item Системный вызов \tt{open(pathname,flags,mode)} создает файл с разрешениями \verb+ mode & ~umask +
   \end{itemize}
        

\end{frame}

\begin{frame}
	\begin{block}{Упражнение}
		\begin{enumerate}
			\item Создать от имени \tt{root} директорию \tt{/stick} с установленными битами \tt{sticky} и \tt{SGID}, а также разрешениями на чтение, запись и выполнение для всех
			\item От имени пользователя \tt{root} создать пустой файл \tt{testroot}
			\item Изменить \tt{umask} пользователя \tt{root} на 002
			\item От имени пользователя \tt{root} создать пустой файл \tt{testroot2}
			\item От имени пользователя \tt{\defaultuser} создать пустой файл \tt{testuser}
			\item Посмотреть получившиеся права и принадлежность файлов
			\item Отредактировать файл \tt{testroot2}
			\item Попробовать удалить файл \tt{testroot2}
		\end{enumerate}
	\end{block}
\end{frame}



