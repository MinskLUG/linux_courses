\begin{frame}{PAM}
	% http://www.opennet.ru/base/net/pam_linux.txt.html
	\begin{itemize}
		\item PAM это динамическая библиотека
		\item Конфигурация PAM
			\begin{itemize}
				\item {\tt /etc/pam.conf}
				\item {\tt /etc/pam.d/...}
					\begin{itemize}
						\item Сервисы
						\item system\_auth
					\end{itemize}
			\end{itemize}
	\end{itemize}

	\begin{block}{Формат записи}
		\begin{columns}
			\column{0.245\textwidth}
			\textbf{module type}
			 \begin{itemize}
				 \item auth
				 \item account
				 \item session
				 \item password
			 \end{itemize}
			 \column{0.245\textwidth}
			 \textbf{control flag}
			 \begin{itemize}
				 \item requisite
				 \item required
				 \item sufficient
				 \item optional
			 \end{itemize}
			 \column{0.245\textwidth}
			 \textbf{module name}
			 \column{0.245\textwidth}
			 \textbf{module options}
		 \end{columns}
	 \end{block}
\end{frame}

\begin{frame}{Диспетчер службы имен (NSS)}

	Важная информация для системы:
		
	\begin{itemize}
		\item Информация о пользователях (логин, группа, пароль и т.д.)
		\item Информация о сетевых ресурсах (имена хостов, протоколов, сервисов)
    \end{itemize}

	\pause

	\begin{block}{NSS}
		\begin{itemize}
			\item Конфигурация: {\tt /etc/nsswitch.conf}
			\item Динамические библиотеки сервисов: {\tt ls -1 /lib*/libnss\_*}
		\end{itemize}
	\end{block}

\end{frame}



