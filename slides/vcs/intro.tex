\begin{frame}
 \frametitle{История систем контроля версий}
 \begin{enumerate}
  \item[1972] SCCS
  \item[1982] RCS (Revision control system)
  \item[1990] CVS Concurrent version system
  \item[2000] Subversion(SVN)
  \item[2000] BitKeeper(proprietary)
  \item[2003] Monotone, darcs
  \item[2005] Git, Bazaar, Mercurial
 \end{enumerate}
\end{frame}

\begin{frame}
 \frametitle{Распределенные и централизованные системы контроля версий}
 \begin{columns}
  \column{0.5\textwidth}
   \begin{center}
     Централизованные (CVS,SVN)
   \end{center}
   \begin{itemize}
    \item Достоинства
     \begin{itemize}
       \item Линейная история изменений
       \item Не надо хранить весь репозиторий
       \item Лучшая поддержка бинарных файлов
     \end{itemize}
    \item Недостатки
     \begin{itemize}
       \item Single point of failure
       \item Трудно работать без доступа к сети
       \item Общая тормознутость
     \end{itemize}
   \end{itemize}
  \column{0.5\textwidth}
   \begin{center}
     Распределенные
   \end{center}
   \begin{itemize}
    \item Достоинства
     \begin{itemize}
       \item Не задушишь, не убьешь
       \item Часто быстрее
       \item Более гибкая работа в команде
       \item Не требуют сети для основной работы
     \end{itemize}
    \item Недостатки
     \begin{itemize}
       \item Большой репозиторий на локальной машине
       \item Невозможность замка на бинарном файле
       \item Зоопарк версий
     \end{itemize}
   \end{itemize}
 \end{columns}
\end{frame}
