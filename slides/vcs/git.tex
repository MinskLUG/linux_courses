\begin{frame}
 \frametitle{Основные команды git}
\end{frame}

\begin{frame}
	\frametitle{Создание нового репозитория git}
	\begin{itemize}
		\item git init [-{}-bare]
		\item git add file1 file2 file3
		\item git commit -m "First revision"
	\end{itemize}
\end{frame}

\begin{frame}[fragile]
	\frametitle{Создание локальной копии репозитория git}
	\begin{itemize}
		\item git clone <url> 
	\end{itemize}

	\begin{verbatim}
git clone ~/work/linux_courses 
	\end{verbatim}

	\begin{verbatim}
git clone git://github.com/d4s/linux_courses.git
	\end{verbatim}

\end{frame}

\begin{frame}[fragile]
	\frametitle{Кто автор?}
	\begin{itemize}
		\item git config [-{}-list] [-{}-add] [-{}-get] [-{}-set] $\dots$
	\end{itemize}

	\begin{verbatim}
git config --local --add user.name "Denis Pynkin"
git config --local --add user.email "denis_pynkin@epam.com"
git config --local --list
	\end{verbatim}

\end{frame}


\begin{frame}[fragile]
	\frametitle{Отслеживание множества копий репозитория git}
	\begin{itemize}
		\item git remote [<cmd> [opts]]
	\end{itemize}

	\begin{itemize}
		\item git remote add <name> <url>
		\item git remote del <name>
		\item git remote show <name>
		\item git remote update
	\end{itemize}

	\begin{verbatim}
git remote add github git://github.com/d4s/linux_courses.git 
	\end{verbatim}

	\begin{block}{Задача}
		Добавить репозиторий соседа для отслеживания.

		Hint: доcтуп через ssh (ip:/path/to/repository)
	\end{block}

\end{frame}


\begin{frame}[fragile]
	\frametitle{Checkout}

	\begin{itemize}
		\item git checkout <revision> [ -{}- path]
	\end{itemize}

	\begin{itemize}
		\item "Перемещение" по коммитам и не только
		\item Восстановление файлов
	\end{itemize}

	\begin{verbatim}
git checkout HEAD -- file
	\end{verbatim}

\end{frame}

\begin{frame}[fragile]
	\frametitle{Ежедневные команды}

	Скачать-закачать изменения:
	\begin{itemize}
		\item git pull
		\item git fetch
		\item git push [-{}-all] [-{}-tags]
	\end{itemize}

	Полезные мелочи:
	\begin{itemize}
		\item git log
		\item git status
		\item git tag <name> [-m "comment"]
		\item git diff <from> <to>
		\item git stash [pop]
	\end{itemize}

\end{frame}

\begin{frame}[fragile]
	\frametitle{Бранчи}

	Создание:

	\begin{itemize}
		\item git branch <old> <newname>
		\item git checkout <rev> -B <branchname> 
	\end{itemize}

	Удаление:

	\begin{itemize}
		\item git branch -D <name>
	\end{itemize}

	Переименование:

	\begin{itemize}
		\item git branch -M <oldname> <name>
	\end{itemize}

\end{frame}


\begin{frame}[fragile]
	\frametitle{Упражнение: бранчуемся}


	\begin{block}{Задача}
		\begin{itemize}
			\item Создать бранч {\tt my}
			\item В бранче {\tt my} изменить любой файл в директории {\tt epam/examples}
			\item Сделать коммит
			\item Обновить в локальной копии состояние репозитория соседа
			\item Создать бранч {\tt neigh}, соответствующий бранчу {\tt my} соседа
		\end{itemize}
	\end{block}

\end{frame}


\begin{frame}[fragile]
	\frametitle{Merge, rebase и cherry-pick}

	\begin{itemize}
		\item git merge [-s <strategy>] <rev>
		\item git rebase <rev> 
		\item git cherry-pick <rev> 
	\end{itemize}

	Merge:

	\begin{verbatim}
      A---B---C topic          A---B---C topic
     /                        /         \
D---E---F---G master     D---E---F---G---H master
	\end{verbatim}

	Rebase:

	\begin{verbatim}
      A---B---C topic                 A'--B'--C' topic
     /                               /         
D---E---F---G master    D---E---F---G master
	\end{verbatim}

\end{frame}


\begin{frame}[fragile]
	\frametitle{Упражнение: слияние}

		   
	\begin{block}{Задача}
		\begin{itemize}
			\item Добавить изменения соседа в свой бранч c помощью {\tt cherry-pick} 
			\item Запустить {\tt gitk}
			\item Сделать {\tt merge} с бранчем соседа
			\item Запустить {\tt gitk}
		\end{itemize}
	\end{block}
\end{frame}


\begin{frame}[fragile]
	\frametitle{Упражнение: найти ''гаденыша''}

	Понадобится скрипт {\tt cutter.sh} и любая книга в txt формате.

	\begin{verbatim}
cat book.txt | sh cutter.sh
	\end{verbatim}

	Задача: найти -- кто и в каком коммите испортил книгу фразой:
	\begin{verbatim}
Some anonymous crap from nobody :-E~~~
	\end{verbatim}
	
	\begin{block}{Вопрос}
		\Large{KAK ?}
	\end{block}
\end{frame}

\begin{frame}
	\frametitle{git bisect}

	\begin{itemize}
		\item git bisect start [<bad> [<good>...]
		\item git bisect good <last known commit>
		\item git bisect bad
		\item git bisect good
		\item git bisect skip
		\item git bisect reset
		\item git bisect log
		\item git bisect replay
			\pause
		\item git bisect run <command>
	\end{itemize}
\end{frame}

\begin{frame}[fragile]
	\frametitle{blame}

	Метод попроще:

	\begin{itemize}
		\item git blame 
	\end{itemize}

	\begin{verbatim}
git blame -L 1224, 1224 -- text.txt
	\end{verbatim}
	
\end{frame}

\begin{frame}[fragile]
	\frametitle{Взаимодействуем с upstream}

	Создать серию патчей:
	\begin{itemize}
		\item git format-patch <ref>
	\end{itemize}

	\begin{verbatim}
git format-patch HEAD~5
	\end{verbatim}

	Применить серию патчей:
	\begin{itemize}
		\item git am <mbox|mdir|files>
	\end{itemize}
\end{frame}

