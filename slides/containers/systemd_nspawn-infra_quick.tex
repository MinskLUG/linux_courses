\begin{frame}[fragile]{systemd-nspawn: старт/стоп контейнера}
    \begin{alert}{Disclaimer}:
        {\tt systemd-nspawn} используется только для упрощения создания учебного окружения!
    \end{alert}
    \begin{block}{Запуск контейнера}
        Использовать архив c ФС дострибутива Alpine \\ (Minimal root filesystem)

	\begin{verbatim}
mkdir -p /tmp/alpine
tar -C /tmp/alpine \
    -xf alpine-minirootfs-3.8.1-x86_64.tar.gz
systemd-machine-id-setup --root=/tmp/alpine/
\end{verbatim}

	\begin{verbatim}
systemd-nspawn -M alpine -n --network-bridge=lxcbr0 \
    -D /tmp/alpine --bind $HOME:$HOME
\end{verbatim}
    \end{block}
    {\tiny Note: На своих системах конфигурация может отличаться!}
\end{frame}

\begin{frame}[fragile]{Подготовка: очистка}
    \begin{block}{Настройка интерфейса}
    "Магические" команды обретут смысл позже!\\
    Просто введите их на хосте ;)
    \begin{verbatim}
echo 0 > /proc/sys/net/ipv4/ip_forward
iptables -F
iptables -F -t nat
\end{verbatim}
    \end{block}
\end{frame}

\begin{frame}{Упражнение}
    \begin{block}{Настройка интерфейса}
        \begin{enumerate}
            \item В контейнере: включить интерфейс {\tt host0}
            \item На хосте: найти IP адрес и диапазон адресов интерфейса {\tt lxcbr0}
            \item В контейнере: установить IP адрес для интерфейса {\tt host0} из найденного диапазона
            \item На хосте: пpоверить доступность контейнера утилитой {\tt ping}
        \end{enumerate}
    \end{block}
\end{frame}
