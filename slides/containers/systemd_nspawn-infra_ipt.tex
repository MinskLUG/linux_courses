\begin{frame}[fragile]{systemd-nspawn: старт/стоп контейнера}
    \begin{alert}{Disclaimer}:
        {\tt systemd-nspawn} используется только для упрощения создания учебного окружения!
    \end{alert}
    \begin{block}{Запуск контейнера}
        Использовать архив c ФС дострибутива Fedora:\\
        {\tt tar -C /tmp/ -xf LDC-fedora26-2017.tar.gz}

	\begin{lstlisting}
systemd-nspawn -M fedora26 --network-bridge=lxcbr0 -D /tmp/rootfs
[root@fedora26 ~]$ dhclient host0
\end{lstlisting}
    \end{block}
\end{frame}
%$
\begin{frame}{Iptables: упражнение}
    \begin{block}{Iptables}
        \begin{enumerate}
            \item Запустить {\tt ping -n <IP>} в контейнере\footnote{IP -- адрес хоста соседа}
            \item Запустить {\tt tcpdump -i eth0 icmp} на хосте в отдельной консоли
            \item Разрешить {\tt ip\_forwarding} на хосте
            \item Добавить правило для таблицы {\tt nat},
                включающее маскарадинг для IP-адреса, использующегося в контейнере
            \item Запустить {\tt tcpdump -i eth0 icmp} на хосте
            \item Заблокировать ''входящий'' трафик от соседа:\\
                {\tt iptables -A INPUT -i eth0 --src <IP> -p icmp -j REJECT --reject-with icmp-host-unreachable}
        \end{enumerate}
    \end{block}
\end{frame}
