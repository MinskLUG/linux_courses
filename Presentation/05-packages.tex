\documentclass[ignorenonframetext, professionalfonts, hyperref={pdftex, unicode}]{beamer}

\usetheme{Copenhagen}
\usecolortheme{wolverine}

\usepackage[orientation=landscape, size=custom, width=16, height=9.75, scale=0.5]{beamerposter}	

%Packages to be included

\usepackage{textcomp}

\usepackage[russian]{babel}
\usepackage[utf8]{inputenc}
\usepackage[T1]{fontenc}

\usepackage{beamerthemesplit}

\usepackage{ulem}

\usepackage{verbatim}

\usepackage{ucs}
\usepackage{listings}
\lstloadlanguages{C, make, bash}

\lstset{escapechar=`,
	extendedchars=false,
	language=C, 
	tabsize=2, 
	columns=fullflexible, 
%	basicstyle=\scriptsize,
	keywordstyle=\color{blue}, 
	commentstyle=\itshape\color{brown},
%	identifierstyle=\ttfamily, 
	stringstyle=\mdseries\color{green}, 
	showstringspaces=false, 
	numbers=left, 
	numberstyle=\tiny, 
	breaklines=true, 
	inputencoding=utf8x,
	keepspaces=true,
	morekeywords={u\_short, u\_char, u\_long, in\_addr}
	}

\definecolor{darkgreen}{cmyk}{0.7, 0, 1, 0.5}

\lstdefinelanguage{diff}
{
    morekeywords={+, -},
    sensitive=false,
    morecomment=[l]{//},
    morecomment=[s]{/*}{*/},
    morecomment=[l][\color{darkgreen}]{+},
    morecomment=[l][\color{red}]{-},
    morestring=[b]",
}
\newcounter{tmpc}


%%%%%%%%%%%%%%%%%%%%%%%%%%%%%%%%%%%%%%%%%%%%%%%%%
%%%%%%%%%% PDF meta data inserted here %%%%%%%%%%
%%%%%%%%%%%%%%%%%%%%%%%%%%%%%%%%%%%%%%%%%%%%%%%%%
\hypersetup{
	pdftitle={Введение в GNU/Linux},
	pdfauthor={Epam/LLPD}
}





%%%%%% Beamer Theme %%%%%%%%%%%%%

	
\title{Введение в GNU/Linux}
\author{Epam/LLPD}



%%%%%%%%%%%%%%%%%%%%%%%%%%%%%%%%%%%%%%%%%%%%%%%%%
%%%%%%%%%% Begin Document  %%%%%%%%%%%%%%%%%%%%%%
%%%%%%%%%%%%%%%%%%%%%%%%%%%%%%%%%%%%%%%%%%%%%%%%%




\begin{document}

\frame{
	\frametitle{Система управления пакетами}
	\titlepage
	\vspace{-0.5cm}
	\begin{center}
	%\frontpagelogo
	\end{center}
}
\frame{
	\tableofcontents
%	[hideallsubsections]
}

%\section{}
\begin{frame}{Система управления пакетами: для чего это нужно}
\begin{itemize}
 \item ''DLL Hell''
 \item Dependency hell
 \item Общие задачи пакетного менеджера:
   \begin{itemize}
     \item Проверка целостности пакетов
     \item Проверка зависимостей пакетов
        \item Поддержание списка установленных пакетов
        \item Автоматическое удаление пакетов
     \item Предоставление доступа к репозиторию пакетов
     \item Разрешение зависимостей
   \end{itemize}
\end{itemize}
\end{frame}

\begin{frame}{Debian-based и RedHat-based системы управления пакетами}
\begin{center}
 \textbf{Два уровня пакетных менеджеров}
\end{center}
\begin{columns}
  \column{0.4\textwidth}
  \begin{center}
    \textbf{RedHat-based}
  \end{center}
  \begin{itemize}
    \item yum
    \item rpm
  \end{itemize}
  \column{0.4\textwidth}
  \begin{center}
    \textbf{Debian-based}
  \end{center}
  \begin{itemize}
    \item aptitude, apt, synaptic
    \item dpkg
  \end{itemize}
\end{columns}
\end{frame}

\begin{frame}{RPM: структура пакета}
	\begin{itemize}
		\item Метаданные
			\begin{itemize}
				\item Имя
				\item Версия/Релиз
				\item Группа
				\item Описание
				\item ...
			\end{itemize}
		\item Архив с файлами
			\begin{itemize}
				\item cpio
			\end{itemize}
		\item Скрипты
			\begin{itemize}
				\item Pre Install
				\item Post Install
				\item Pre Uninstall
				\item Post Uninstall \bigskip
				\item Triggers
			\end{itemize}
	\end{itemize}
\end{frame}

\begin{frame}{RPM: команды}
	\begin{block}{Установка пакета}
		{\tt rpm -i [rpm-file1] ... [[url://]rpm-fileN] }
	\end{block}
	\begin{block}{Удаление пакета}
		{\tt rpm -e pkgname1 ... pkgnameN }
	\end{block}
	\begin{block}{Обновление пакета}
		{\tt rpm -U [rpm-file1] ... [[url://]rpm-fileN] }
	\end{block}
	\begin{block}{Проверка пакета}
		{\tt rpm -V pkgname1 ... pkgnameN }
	\end{block}
\end{frame}

\begin{frame}{RPM: часто используемые опции опроса}

	\begin{itemize}
		\item {\tt pkgname} -- выбор пакета, установленного в системе
		\item {\tt -a} -- все пакеты, установленные в системе
		\item {\tt -p} -- использовать файл RPM
	\end{itemize}


	\begin{itemize}
		\item {\tt -i} -- показать информацию пакета\\
			{\tt rpm -q -i glibc }
		\item {\tt -l} -- показать список файлов пакета \\
			{\tt rpm -q -l glibc }
		\item {\tt --whatprovides} -- \\
			{\tt rpm -q --whatprovides java}
		\item {\tt --whatrequires} -- \\
			{\tt rpm -q --whatrequires /bin/bash}
		\item {\tt --queryformat} -- формат вывода\\
			{\tt rpm -q --whatrequires /bin/bash --queryformat ''\%\{name\} ''}

	\end{itemize}

\end{frame}


\begin{frame}{Репозиторий}
	\begin{block}{Репозиторий пакетов}
		Место, где хранятся и поддерживаются пакеты, а также сопутствующая мета-информация, предназначенное для использования пакетным менеджером.
	\end{block}
	\begin{block}{Пример: Fedora Core}
		\begin{itemize}
			\item Packages/*.rpm
			\item RPM-GPG-KEY-*
			\item repodata
			\begin{itemize}
				\item множество сжатых и несжатых XML файлов для YUM
			\end{itemize}
		\end{itemize}

		Описание репозтория для YUM на локальной системе хранится по пути
		{\tt /etc/yum.repos.d/*.repo}
	\end{block}
		
\end{frame}

\begin{frame}{YUM: команды}
	\begin{block}{Установка/обновление пакета}
		{\tt yum install pkgname }
	\end{block}
	\begin{block}{Обновление всех пакетов}
		{\tt yum update }
	\end{block}
	\begin{block}{Удаление пакета}
		{\tt yum remove pkgname }
	\end{block}
	\begin{block}{Поиск}
		{\tt yum list pkgname }\\
		{\tt yum search pkgname }
	\end{block}
\end{frame}


\begin{frame}[fragile]{Упражнение}
  \begin{enumerate}
      \item Создать на {\tt /dev/sda} раздел размером примерно 10Gb
      \item Создать на этом разделе ext3 ФС и смонтировать раздел в {\tt /mnt/chroot}
      \item Развернуть {\tt /media/nfs/pub/CentOS/precreated/centOS.tar.gz} в {\tt /mnt/chroot}
      \item Смонтировать {\tt proc, sysfs} а также {\tt /dev} в соответствующие места {\tt /mnt/chroot}
      \item {\tt chroot /mnt/chroot}
      \item Отредактировать {\tt /etc/resolv.conf} -- скопировать туда информацию из {\tt resolv.conf} основной системы
      \item Отредактировать {\tt /etc/yum.conf} Добавить следующий раздел
\begin{minipage}{0.5\textwidth}
\begin{verbatim}
[base]
  name = CentOS 6
  baseurl = ftp://192.168.11.15/CentOS
  gpgcheck = 0
\end{verbatim}
\end{minipage}
\setcounter{tmpc}{\theenumi}
\end{enumerate}
\end{frame}
\begin{frame}{Продолжение упражнения}
  \begin{enumerate}
      \setcounter{enumi}{\thetmpc}
      \item {\tt yum update}
      \item Установить пакет vim
      \item Посмотреть списки файлов для пакетов {\tt yum, rpm}
      \item Найти пакет предоставляющий сервис ssh и установить его
      \item Удалить пакет vim
    \end{enumerate}
\end{frame}




\end{document}
