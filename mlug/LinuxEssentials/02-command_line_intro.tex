\input{../branding/common}
\title{Введение в GNU/Linux\\Командная строка}

\begin{document}
\begin{frame}
 \frametitle{}
 \titlepage
\end{frame}

\section{Принципы проектирования переносимых программ}
\mode<all>{\input{../../slides/intro/unixway}}

\section{Интерфейс командной строки}
\mode<all>{\input{../../slides/cmdline/clui_vs_gui.tex}}

\section{ Командная оболочка(shell) }
\begin{frame}
\frametitle{}
 \begin{center}
   {\Large Командная оболочка(shell) }
 \end{center}
\end{frame}

\mode<all>{\input{../../slides/cmdline/shell-intro.tex}}

\section{ Документация }
\begin{frame}
\frametitle{}
 \begin{center}
   {\Large Help }
 \end{center}
\end{frame}
\mode<all>{\input{../../slides/cmdline/help}}

\section{ Навигация по файловой системе}
\begin{frame}
\frametitle{}
 \begin{center}
   {\Large Навигация по файловой системе }
 \end{center}
\end{frame}
\mode<all>{\input{../../slides/fs/fs-structure.tex}}

\mode<all>{\input{../../slides/cmdline/fs-navigation-cmd.tex}}

\begin{frame}{Задание на дом}
\begin{block}{}
vimtutor ru
\end{block}
\end{frame}

\mode<all>\input{../branding/completion}
\end{document}
