\documentclass[ignorenonframetext, professionalfonts, hyperref={pdftex, unicode}]{beamer}

\usetheme{Epam}

\usepackage[russian]{babel}
\usepackage[utf8]{inputenc}
\usepackage[T1]{fontenc}

\usepackage{textcomp}

\usepackage{beamerthemesplit}

\usepackage{ulem}

\usepackage{verbatim}

\usepackage{ucs}


\usepackage{listings}
\lstloadlanguages{bash}

\lstset{escapechar=`,
	extendedchars=false,
	language=sh,
	frame=single,
	tabsize=2, 
	columns=fullflexible, 
%	basicstyle=\scriptsize,
	keywordstyle=\color{blue}, 
	commentstyle=\itshape\color{brown},
%	identifierstyle=\ttfamily, 
	stringstyle=\mdseries\color{green}, 
	showstringspaces=false, 
	numbers=left, 
	numberstyle=\tiny, 
	breaklines=true, 
	inputencoding=utf8,
	keepspaces=true,
	morekeywords={u\_short, u\_char, u\_long, in\_addr}
	}

\definecolor{darkgreen}{cmyk}{0.7, 0, 1, 0.5}

\lstdefinelanguage{diff}
{
    morekeywords={+, -},
    sensitive=false,
    morecomment=[l]{//},
    morecomment=[s]{/*}{*/},
    morecomment=[l][\color{darkgreen}]{+},
    morecomment=[l][\color{red}]{-},
    morestring=[b]",
}

\author[Epam]{{\bf Epam}\\Low Level Programming Department}

%\institution[EPAM]{EPAM}
%\logo{\includegraphics[width=1cm]{logo.png}}

\title{Введение в GNU/Linux\\Командная строка}

\begin{document}
\begin{frame}
 \frametitle{}
 \titlepage
\end{frame}

\section{Принципы проектирования переносимых программ}
\mode<all>{\begin{frame}{Главные ориентиры}
	\begin{itemize}
		\item кроссплатформенная переносимость
		\item открытые стандарты
	\end{itemize}
\end{frame}

\begin{frame}{Немного цитат}
Дуг Макилрой, изобретатель каналов <<pipes>>, сформулировал несколько постулатов,применимых для разработки ПО:
\pause
	\begin{itemize}
		\item пишите программы,  которые выполняют одну функцию и делают это хорошо;
			\pause
		\item пишите программы,  которые будут работать вместе;
			\pause
		\item пишите программы,  поддерживающие текстовые потоки,  поскольку они являются универсальным интерфейсом.
	\end{itemize}

\end{frame}


\begin{frame}{"Философия" UNIX}
	это {\bfseries не} философия,  а общие рекомендации по проектированию ПО,  накопленные сообществом программистов на опыте десятилетий разработок программ,  которые взаимодействуют друг с другом.
\end{frame}

\begin{frame}{1. Правило модульности}
	\begin{block}{Следует писать простые части,  связанные ясными интерфейсами.}
Единственным способом создания сложной программы,  не обреченной заранее на провал,  является сдерживание ее глобальной сложности.
	\end{block}
Т.е. построение программы из простых частей, соединенных четко определенными интерфейсами, 
так что большинство проблем являются локальными, 
и тогда можно рассчитывать на обновление одной из частей без разрушения целого.
\end{frame}

\begin{frame}{Размер кода и ошибки}
	\begin{center}
		\includegraphics[width=200px]{../../slides/intro/errors_density-graph.png}
	\end{center}
\end{frame}

\begin{frame}{2. Правило ясности}
	\begin{block}{Ясность -- лучше чем мастерство.}
		Последующее обслуживание программы -- важная и дорогостоящая часть жизненного цикла программы.
	\end{block}
	\pause
Писать программы необходимо так,  как если бы вы знали,  что последующей поддержкой будет заниматься неуравновешенный псих с топором,  знающий ваш домашний адрес!
\end{frame}

\begin{frame}{3. Правило композиции}
	\begin{block}{Следует разрабатывать программы,  которые будут взаимодействовать с другими программами.}
		Если разрабатываемые программы не способны взаимодействовать друг с другом,  то очень трудно избежать создания сложных монолитных  программ.
	\end{block}
	Методы взаимодействия могут быть сильными и слабыми -- по возможности рекомендуется использовать слабые методы и текстовые форматы передачи данных.
\end{frame}

\begin{frame}{4. Правило разделения}
	\begin{block}{Следует отделять политику от механизма и интерфейсы от основных модулей (engine).}
		Примеры политики и механизма:\\
		вид GUI и операции отрисовки, клиент (front-end) -- сервер (back-end), сценарии и библиотеки и др.
	\end{block}
	При жесткой связи политики и механизма:
	\begin{itemize}
		\item политика становится негибкой и усложняется ее изменение;
		\item изменение политики имеет строгую тенденцию к дестабилизации механизмов.
	\end{itemize}
\end{frame}

\begin{frame}{5. Правило простоты}
	\begin{block}{Необходимо проектировать простые программы и <<добавлять сложность>> только там,  где это необходимо.}
	\end{block}
	Основные причины добавления сложности:
	\begin{itemize}
		\item человеческий фактор (часто -- желание <<выпендриться>>);
		\item проектные требования,  продиктованные текущей модой,  маркетингом или «левой пяткой заказчика»;
	\end{itemize}
\end{frame}

\begin{frame}{6. Правило расчетливости}
	\begin{block}{Пишите большие программы,  только если после демонстрации становится ясно,  что ничего другого не остается.}
		Под <<большими программами>> здесь понимаются программы с большим объемом кода и значительной внутренней сложностью.
	\end{block}
\end{frame}

\begin{frame}{7. Правило прозрачности}
	\begin{block}{Для того,  чтобы упростить проверку и отладку программы,  ее конструкция должна быть обозримой.}
		Программа {\itshape прозрачна}, если при ее минимальном изучении можно понять, что она делает и как.\\
		Программа {\itshape воспринимаема},  когда она имеет средства для мониторинга и отображения внутреннего состояния.
	\end{block}
	Необходимо использовать достаточно простые форматы входных и выходных данных.\\
	Интерфейс должен быть приспособлен для использования в отладочных сценариях.
\end{frame}

\begin{frame}{8. Правило устойчивости}
	\begin{block}{Устойчивость -- следствие	прозрачности и простоты.}
		Программа является {\itshape устойчивой},  когда она выполняет свои функции в неожиданных условиях,  которые выходят за рамки предположений разработчика,  как и в нормальных условиях.\\
		Программа является {\itshape простой},  если происходящее в ней не представляется сложным для восприятия человеком.
	\end{block}
	Один из способов организации -- модульность(простые блоки,  ясные интерфейсы)\\
	Следует избегать частных случаев!
\end{frame}

\begin{frame}{Пример неусточивого ПО}
	\begin{center}
		\includegraphics[width=1\textwidth]{../../slides/intro/exploits_of_a_mom_rus.png}
	\end{center}
\end{frame}


\begin{frame}[fragile]{Пример <<простой>> программы}
	\begin{center}
		\begin{verbatim}
+++++++++++++++++++++++++++++++++++++++++++++
+++++++++++++++++++++++++++.+++++++++++++++++
++++++++++++.+++++++..+++.-------------------
---------------------------------------------
---------------.+++++++++++++++++++++++++++++
++++++++++++++++++++++++++.++++++++++++++++++
++++++.+++.------.--------.------------------
---------------------------------------------
----.-----------------------.
		\end{verbatim}
	\end{center}
\end{frame}

\begin{frame}{9. Правило представления}
	\begin{block}{Знания следует оставлять в данных,  чтобы логика программы могла быть примитивной и устойчивой.}
		Даже простую логику бывает сложно проверить,  но даже сложные структуры данных являются довольно простыми для моделирования и анализа (например диаграмма 50 узлов дерева и блок-схема 50 строк кода)
	\end{block}
	Если можно выбирать между усложнением структуры данных и усложнением кода,  то лучше выбирать первое.\\
	Примеры: ascii,  генератор html-таблицы.
\end{frame}

\begin{frame}{10. Правило наименьшего удивления}
	\begin{block}{При проектировании интерфейсов всегда следует использовать наименее неожиданные элементы.}
		Необходимо учитывать характер предполагаемой аудитории и традиции платформы.
	\end{block}
	Оборотная сторона: следует избегать создания внешне похожих вещей,  слегка отличающихся в действительности,  поскольку {\itshape кажущаяся привычность порождает ложные ожидания}.
\end{frame}

\begin{frame}{11. Правило тишины}
	\begin{block}{Если программе нечего сказать,  то пусть лучше молчит.}
		Внимание и сосредоточенность пользователя -- ценный и ограниченный ресурс,  который требуется только в случае необходимости.
	\end{block}
	Важная информация не должна смешиваться с подробными сведениями о работе программы.
\end{frame}

\begin{frame}{12. Правило восстановления}
	\begin{block}{Когда программа завершается аварийно,  это должно происходить явно (шумно) и по возможности быстро.}
		Если программа не способна справиться с ошибкой,  то необходимо завершить ее работу так,  чтобы максимально упростить диагностику.
	\end{block}
	Для сетевых служб следует следовать рекомендации Постела:\\
	<<{\itshape Будьте либеральны к тому,  что принимаете,  и консервативны к тому,  что отправляете}>>
\end{frame}

\begin{frame}{13. Правило экономии}
	\begin{block}{Время программиста дорого -- поэтому задача экономии его времени более приоритетна,  по сравнению с экономией машинного времени.}
		Компьютер железный -- ему не скучно (с) программистская мудрость
	\end{block}
	Использование высокоуровневых языков и <<обучение>> машины выполнять больше низкоуровневой работы по программированию,  что приводит к правилу 14.
\end{frame}

\begin{frame}{14. Правило генерации}
	\begin{block}{Избегайте кодирования вручную; если есть возможность -- пишите программы для создания программ.}
		Использование генераторов кода оправданно,  когда они могут повысить уровень абстракции,  
		т.е. когда язык спецификации для генератора проще,  чем сгенерированный код,  
		и код впоследствии не потребует ручной доработки.
	\end{block}
	Примеры: грамматические и лексические анализаторы,  генераторы make-файлов,  построители GUI-интерфейсов.
\end{frame}

\begin{frame}{15. Правило оптимизации}
	\begin{block}{Сначала -- опытный образец,  потом -- оптимизирование.}
		Добейтесь стабильной работы,  только потом оптимизируйте.
	\end{block}
	\begin{block}{Керниган и Плоджер:}
		90\% актуальной и реальной функциональности лучше,  чем 100\% функциональности перспективной и сомнительной
	\end{block}
	\begin{block}{Кнут:}
		преждевременная оптимизация -- корень всех зол
	\end{block}
	\begin{block}{Кент Бек (экстремальное программирование):}
		заставьте программу работать,  заставьте работать ее верно,  а затем сделайте ее быстрой
	\end{block}
\end{frame}

\begin{frame}{16. Правило разнообразия}
	\begin{block}{Не следует доверять утверждениям о <<единственно правильном пути>>.}
		Никто не обладает умом,  достаточым для оптимизации всего или для предвидения всех возможных вариантов использования создаваемой программы.
	\end{block}
\end{frame}

\begin{frame}{17. Правило расширяемости}
	\begin{block}{Разрабатывайте для будущего. Оно наступит быстрее,  чем вы думаете.}
		При проектировании протоколов или форматов файлов следует делать их самоописательными,  для того,  чтобы их можно было расширить.
	\end{block}
	{\itshape Всегда},  следует либо включать номер версии,  либо составлять формат из самодостаточных,  
	самоописательных команд так,  чтобы можно было легко добавить новые директивы,  
	а старые удалить, <<не сбивая с толку>> код чтения формата.
\end{frame}

\begin{frame}{Все правила сразу}
	\begin{center}
	{\Huge\bfseries K.I.S.S.}

	Keep It Simple,  Stupid!
	\end{center}
\end{frame}


}

\section{Интерфейс командной строки}
\mode<all>{% Тема. Командная строка. 
% Показать примеры использования. Рассказать о преимуществах и недостатках в
% сравненни с графическим "оконным" интерфейсом. 
% Ознакомить с назначениме  эмулятора терминала и об реализациях.

\begin{frame}{Примеры использования командной строки}
	\begin{columns}
	\column{0.5\textwidth}
        \begin{itemize}
            \item чаты
            \item компьютерные игры Quake, DotA
            \item операционные системы
        \end{itemize}
	\column{0.5\textwidth}
	% insert picture of Quake 
    \includegraphics[height=0.4\textheight]{../../slides/cmdline/330px-Tremulous_console.png}
	\end{columns}
\end{frame}

\begin{frame}{Преимущества командной строки}
	\begin{itemize}
		\item Работа через сеть либо RS232
		\item Быстрый доступ к командам системы
		\item Легче отладка сообществом
		\item Легкость автоматизации
	\end{itemize}
\end{frame}

\begin{frame}{Недостатки командной строки}
	\begin{itemize}
		\item Oтсутствуют возможности обнаружения (discoverabililty)
		\item Необходимость изучения синтаксиса команд и запоминания сокращений.  (синтаксис может различаться)
		\item Без автодополнения, ввод длинных и содержащих спецсимволы параметров с клавиатуры может быть затруднительным
		\item Отсутствие «аналогового» ввода.
	\end{itemize}
\end{frame}

\begin{frame}{Эмуляторы терминала в графическом режиме}
	\begin{itemize}
		\item xterm
		\item rxvt
        \item gnome-terminal
        \item konsole
        \item Yakuake (Yet Another Kuake)
	\end{itemize}
\end{frame}
\note { 
Примеры приложений которые лучше выглядят в графическом режиме браузер,
редакторы видео и графики. Поэтому пользователь при работе, как правило,
совмещает оба интерфейса: использует графическое окружениe в сочетании с
интерфейсом командной строки. 
В графическом окружении интерфейса командной строки предоставляют приложения -
эмуляторы терминала. 
реализации - для графической системы X Window xterm, rxvt. Для GNOME
gnome-terminal, для KDE konsole, Yakuake (Yet Another Kuake выезжает по нажатии
тильды ~ как Quake)  
Дополнительные замечания:
Терминал - устройство для ввода вывода информации, уже устарел.
Графические приложения можно запускать из командной строки. 
}
}

\section{ Командная оболочка(shell) }
\begin{frame}
\frametitle{}
 \begin{center}
   {\Large Командная оболочка(shell) }
 \end{center}
\end{frame}

\mode<all>{\begin{frame}[fragile]{Определение(не совсем формальное)}
	\textbf{Shell} -- приложение, обеспечивающее выполнение других приложений и их взаимодействие, а также представляющая услуги командной строки. 
	\begin{center}
	 или
	\end{center}
	\textbf{Shell} -- приложение, обеспечивающее доступ к основным функциям ядра.

	\pause
	\vspace{0.5in}
	Пример shell из Windows-world -- cmd.exe
	\vspace{0.5in}

	Минимальный дистрибутив Linux -- ядро + shell 

\end{frame}

\begin{frame}[fragile]{Основные типы shell в Unix}
  \begin{itemize}
    \item Bourne shell совместимые
      \begin{itemize}
        \item \textbf{sh} исходная bourne shell (Steve Bourne, 1978)
        \item \textbf{ksh} Korn shell (David Korn, 1983)
        \item \textbf{ash} $[$BSD$]$ Almquist shell (Kenneth Almquist,1989)  
        \item \textbf{bash} $[$GPL$]$ Bourne-again shell (Brian Fox, 1989)
        \item \textbf{zsh} $[$BSD$]$ Z shell (Paul Falstad,1990)
        \item \textbf{/bin/sh} Указывает на POSIX-совместимую shell
      \end{itemize}
  \item C shell совместимые
      \begin{itemize}
        \item \textbf{csh}  Исходная С shell (Bill Joy, 1978)
        \item \textbf{tcsh} $[$BSD$]$ TENEX C shell (Ken Greer, 1981)
       \end{itemize}
  \end{itemize}
\end{frame}

\begin{frame}[fragile]{Маленькое упражнение}
\begin{lstlisting}[language=bash]
cat /etc/shells
ls -l <filename> # для каждого элемента /etc/shells
readlink -e <filename> 
\end{lstlisting}
\end{frame}


}

\section{ Документация }
\begin{frame}
\frametitle{}
 \begin{center}
   {\Large Help }
 \end{center}
\end{frame}
\mode<all>{\begin{frame}[fragile]{Получение помощи}
  \begin{itemize}
    \pause
    \item \textbf{man} - помощь по внешним командам
    \pause
    \item \textbf{help} - помощь по внутренним командам bash (также man bash)
    \pause
    \item \textbf{info} - расширенная помощь по некоторым командам (texinfo format)
      \begin{itemize}
       \item   Попробовать {\tt info coreutils}
       \item   Справка по навигации -- нажать h
      \end{itemize}
  \end{itemize}
\end{frame}

\begin{frame}[fragile]{Основное о man}
\begin{columns}
	\column{2.2in}
		\begin{itemize}
			\item Прочитайте {\tt man man} !
			\item Apropos {\tt man -k <слово>}
			\item Разделы (sections)
				\begin{itemize}
					\item[1] Основная секция(юзерские программы)
					\item[2] Syscalls
					\item[3] С library
					\item[5] Конфигурационные файлы
					\item[8] Системные службы
				\end{itemize}
		\end{itemize}
	  \textbf{Замечание}

	  Обычно внутри страницы работает поиск с помощью '/'
	\pause 
	
	\column{1in}
		\begin{block}{Попробовать}
			\begin{lstlisting}
man -k printf
man 3 printf
man 1 printf
man -a printf
			\end{lstlisting}
		\end{block}
	\end{columns}
\end{frame}


}

\section{ Навигация по файловой системе}
\begin{frame}
\frametitle{}
 \begin{center}
   {\Large Навигация по файловой системе }
 \end{center}
\end{frame}
\mode<all>{\begin{frame}{Файловая структура}
	
	{\center "Дерево внутри дома?" (c) Шрек}
		
	\begin{columns}
	\column{0.2\textwidth}
		\includegraphics[height=0.8\textheight]{../../slides/fs/01-lhs.png}
	\column{0.7\textwidth}
		\begin{itemize}
			\item Директории
			\item Обычные файлы
			\item Симлинки
			\item Хардлинки
			\item Файлы устройств
			\item FIFO
			\item сокеты
		\end{itemize}
	\end{columns}
\end{frame}
}

\mode<all>{\begin{frame}{Навигация по файловой системе}
      \begin{itemize}
		  \item {\tt ls} -- список файлов в (текущей по умолчанию) директории (man ls)
		  \item {\tt cd} -- смена текущей директории (help cd)
		  \item {\tt pwd} -- имя текущей директории (help pwd)
      \end{itemize}
\end{frame}

\begin{frame}[fragile]{Команды для работы с файлами}
	\begin{itemize}
		\begin{columns}
		\column{0.2\textwidth}
			\item touch
			\item ln
			\item mkdir
			\item mknod
			\item mkfifo
		\column{0.2\textwidth}
			\item cp
			\item mv
			\item install
			\item rm
			\item rmdir
			\item file
		\column{0.4\textwidth}
			\begin{block}{Упражнение}
				\begin{enumerate}
					\item Создать иерархию директорий
						\begin{lstlisting}
dir1/dir1.1/dir1.1.1
dir1/dir1.2/dir1.2.1
dir1/dir1.2/dir1.2.2
						\end{lstlisting}
					\item Внутри каждой создать файл
					\item Удалить все созданное
				\end{enumerate}
			\end{block}
		\end{columns}
	\end{itemize}
\end{frame}


}


\section{ Дополнительные возможности оболочки}
\mode<all>{\begin{frame}{Важные аббревиатуры внутри командной строки}
  \begin{itemize}
    \item Для директорий
      \begin{itemize}
        \item {\tt $\sim$} Домашняя директория
        \item {\tt $\sim$<username>} Домашняя директория пользователя
        \item {\tt ..} Родительская директория
        \item {\tt .} Текущая директория
      \end{itemize}
      \pause  
    \item Wildcards
      \begin{itemize}
        \item {\tt *} Любой набор символов {\tt file*txt : file1.txt filefilefiletxt}
        \item {\tt $[$<список>$]$ } символ из заданного набора
        \item {\tt ?} любой один символ
      \end{itemize}

  \end{itemize}
\end{frame}       

\begin{frame}{Горячие клавиши}
  \begin{itemize}
    \item \textbf{Tab} -- дополнение текущей команды
      \pause
    \item История команд
      \begin{itemize}
        \item Клавиши курсора -- навигация по истории
        \item Ctrl-R -- поиск в истории по фрагменту
        \item Ctrl-O (после выполнения вставить следующую команду из истории)
        \item Команда {\tt history}
      \end{itemize}
    \item Навигация

  \end{itemize}
\end{frame}

\begin{frame}{Переменные окружения}
  \begin{itemize}
    \item {\tt HOME}
    \item {\tt PWD}
    \item {\tt LANG}
    \item {\tt LD\_LIBRARY\_PATH}
    \item {\tt SHELL}
    \item {\tt TERM}
    \item {\tt DISPLAY}
  \end{itemize}

  Контроль

  \begin{itemize}
    \item export {\tt export VAR=value}
    \item declare -x
    \item echo 
  \end{itemize}

  Переменные окружения наследуются при создании нового процесса
\end{frame}

%\begin{frame}{Настройки bash и кастомизация}
%  \begin{itemize}
%    \item Login shell
%      \begin{itemize}
%        \item {\tt /etc/profile}
%        \item {\tt $\sim$/.profile }
%      \end{itemize}
%    \item Обычная интерактивная shell
%      \begin{itemize}
%        \item {\tt /etc/bash.bashrc}
%        \item {\tt $\sim$/.bashrc}
%      \end{itemize}
%  \end{itemize}
%
%  Полезные команды
%  \begin{itemize}
%    \item {\tt alias}
%    \item {\tt export PATH=}
%    \item {\tt Определение функции}
%    \item {\tt shopts}
%  \end{itemize}
%
%\end{frame}


}


\begin{frame}{Задание на дом}
\begin{block}{}
vimtutor ru
\end{block}
\end{frame}

\end{document}

