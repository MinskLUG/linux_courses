\chapter{Основы Linux}

\begin{frame}{Основы ОС Linux}

	\begin{block}{Вопрос}
	Почему Linux является самой популярной
	свободной операционной системой?
	\end{block}

	\pause

	\begin{block}{Ответ}
	\begin{itemize}
		\item \textcopyleft -- Copyleft
		\item ``Философия'' Unix
		\item Открытые стандарты
	\end{itemize}
	\end{block}

\end{frame}


\chapter[Принципы]{Базовые принципы ОС Linux}

\subsection{GNU/Linux}

\mode<all>{\begin{frame}{Терминология}
	\begin{itemize}
		\item GNU -- GNU's Not Unix!
		\begin{itemize}
			\item 1983. Ричард Столлман. Свободное ПО.
		\end{itemize}

		\pause

		\item POSIX
		\begin{itemize}
			\item 1988. Portable Operating System Interface for Unix. 
		\end{itemize}

		\pause

		\item Linux
		\begin{itemize}
			\item 1991. Линус Торвальдс. Ядро.
		\end{itemize}

	\end{itemize}
\end{frame}
}

\subsection{Лицензии}

\mode<all>{\begin{frame}{Лицензии: открытые и свободные}

	\begin{block}{ Р.Столлман: 4 свободы}

		\begin{itemize}
			\item Свобода 0: Свобода запускать программу в любых целях.
			\item Свобода 1: Свобода изучения работы программы и адаптация её к вашим нуждам. 
				Доступ к исходным текстам является необходимым условием.
			\item Свобода 2: Свобода распространять копии,  так что вы можете помочь вашему товарищу.
			\item Свобода 3: Свобода улучшать программу и публиковать ваши улучшения,
				так что всё общество выиграет от этого.
				Доступ к исходным текстам является необходимым условием.
		\end{itemize}
	\end{block}


\end{frame}

\begin{frame}{Copyleft }

	\begin{block}{ \textcopyleft  -- ``Копилефт''}
	Авторское лево -- концепция и практика использования законов авторского права для обеспечения 
	невозможности ограничить любому человеку право использовать,  изменять и распространять как 
	исходное произведение,  так и произведения,  производные от него.
	\end{block}


	При копилефте все производные произведения должны распространяться под той же лицензией,
	что и оригинальное произведение.

\end{frame}


\begin{frame}{Лицензии}
	\begin{itemize}
		\item GPL
		\item LGPL
		\item AGPL
		\item BSD
		\item MIT
		\item Mozilla Public License
		\item Apache Software License
		\item Creative Commons *
	\end{itemize}
\end{frame}
}

\subsection{Принципы проектирования переносимых программ}

\mode<all>{\begin{frame}{Главные ориентиры}
	\begin{itemize}
		\item кроссплатформенная переносимость
		\item открытые стандарты
	\end{itemize}
\end{frame}

\begin{frame}{Немного цитат}
Дуг Макилрой, изобретатель каналов <<pipes>>, сформулировал несколько постулатов,применимых для разработки ПО:
\pause
	\begin{itemize}
		\item пишите программы,  которые выполняют одну функцию и делают это хорошо;
			\pause
		\item пишите программы,  которые будут работать вместе;
			\pause
		\item пишите программы,  поддерживающие текстовые потоки,  поскольку они являются универсальным интерфейсом.
	\end{itemize}

\end{frame}


\begin{frame}{"Философия" UNIX}
	это {\bfseries не} философия,  а общие рекомендации по проектированию ПО,  накопленные сообществом программистов на опыте десятилетий разработок программ,  которые взаимодействуют друг с другом.
\end{frame}

\begin{frame}{1. Правило модульности}
	\begin{block}{Следует писать простые части,  связанные ясными интерфейсами.}
Единственным способом создания сложной программы,  не обреченной заранее на провал,  является сдерживание ее глобальной сложности.
	\end{block}
Т.е. построение программы из простых частей, соединенных четко определенными интерфейсами, 
так что большинство проблем являются локальными, 
и тогда можно рассчитывать на обновление одной из частей без разрушения целого.
\end{frame}

\begin{frame}{Размер кода и ошибки}
	\begin{center}
		\includegraphics[width=200px]{../../slides/intro/errors_density-graph.png}
	\end{center}
\end{frame}

\begin{frame}{2. Правило ясности}
	\begin{block}{Ясность -- лучше чем мастерство.}
		Последующее обслуживание программы -- важная и дорогостоящая часть жизненного цикла программы.
	\end{block}
	\pause
Писать программы необходимо так,  как если бы вы знали,  что последующей поддержкой будет заниматься неуравновешенный псих с топором,  знающий ваш домашний адрес!
\end{frame}

\begin{frame}{3. Правило композиции}
	\begin{block}{Следует разрабатывать программы,  которые будут взаимодействовать с другими программами.}
		Если разрабатываемые программы не способны взаимодействовать друг с другом,  то очень трудно избежать создания сложных монолитных  программ.
	\end{block}
	Методы взаимодействия могут быть сильными и слабыми -- по возможности рекомендуется использовать слабые методы и текстовые форматы передачи данных.
\end{frame}

\begin{frame}{4. Правило разделения}
	\begin{block}{Следует отделять политику от механизма и интерфейсы от основных модулей (engine).}
		Примеры политики и механизма:\\
		вид GUI и операции отрисовки, клиент (front-end) -- сервер (back-end), сценарии и библиотеки и др.
	\end{block}
	При жесткой связи политики и механизма:
	\begin{itemize}
		\item политика становится негибкой и усложняется ее изменение;
		\item изменение политики имеет строгую тенденцию к дестабилизации механизмов.
	\end{itemize}
\end{frame}

\begin{frame}{5. Правило простоты}
	\begin{block}{Необходимо проектировать простые программы и <<добавлять сложность>> только там,  где это необходимо.}
	\end{block}
	Основные причины добавления сложности:
	\begin{itemize}
		\item человеческий фактор (часто -- желание <<выпендриться>>);
		\item проектные требования,  продиктованные текущей модой,  маркетингом или «левой пяткой заказчика»;
	\end{itemize}
\end{frame}

\begin{frame}{6. Правило расчетливости}
	\begin{block}{Пишите большие программы,  только если после демонстрации становится ясно,  что ничего другого не остается.}
		Под <<большими программами>> здесь понимаются программы с большим объемом кода и значительной внутренней сложностью.
	\end{block}
\end{frame}

\begin{frame}{7. Правило прозрачности}
	\begin{block}{Для того,  чтобы упростить проверку и отладку программы,  ее конструкция должна быть обозримой.}
		Программа {\itshape прозрачна}, если при ее минимальном изучении можно понять, что она делает и как.\\
		Программа {\itshape воспринимаема},  когда она имеет средства для мониторинга и отображения внутреннего состояния.
	\end{block}
	Необходимо использовать достаточно простые форматы входных и выходных данных.\\
	Интерфейс должен быть приспособлен для использования в отладочных сценариях.
\end{frame}

\begin{frame}{8. Правило устойчивости}
	\begin{block}{Устойчивость -- следствие	прозрачности и простоты.}
		Программа является {\itshape устойчивой},  когда она выполняет свои функции в неожиданных условиях,  которые выходят за рамки предположений разработчика,  как и в нормальных условиях.\\
		Программа является {\itshape простой},  если происходящее в ней не представляется сложным для восприятия человеком.
	\end{block}
	Один из способов организации -- модульность(простые блоки,  ясные интерфейсы)\\
	Следует избегать частных случаев!
\end{frame}

\begin{frame}{Пример неусточивого ПО}
	\begin{center}
		\includegraphics[width=1\textwidth]{../../slides/intro/exploits_of_a_mom_rus.png}
	\end{center}
\end{frame}


\begin{frame}[fragile]{Пример <<простой>> программы}
	\begin{center}
		\begin{verbatim}
+++++++++++++++++++++++++++++++++++++++++++++
+++++++++++++++++++++++++++.+++++++++++++++++
++++++++++++.+++++++..+++.-------------------
---------------------------------------------
---------------.+++++++++++++++++++++++++++++
++++++++++++++++++++++++++.++++++++++++++++++
++++++.+++.------.--------.------------------
---------------------------------------------
----.-----------------------.
		\end{verbatim}
	\end{center}
\end{frame}

\begin{frame}{9. Правило представления}
	\begin{block}{Знания следует оставлять в данных,  чтобы логика программы могла быть примитивной и устойчивой.}
		Даже простую логику бывает сложно проверить,  но даже сложные структуры данных являются довольно простыми для моделирования и анализа (например диаграмма 50 узлов дерева и блок-схема 50 строк кода)
	\end{block}
	Если можно выбирать между усложнением структуры данных и усложнением кода,  то лучше выбирать первое.\\
	Примеры: ascii,  генератор html-таблицы.
\end{frame}

\begin{frame}{10. Правило наименьшего удивления}
	\begin{block}{При проектировании интерфейсов всегда следует использовать наименее неожиданные элементы.}
		Необходимо учитывать характер предполагаемой аудитории и традиции платформы.
	\end{block}
	Оборотная сторона: следует избегать создания внешне похожих вещей,  слегка отличающихся в действительности,  поскольку {\itshape кажущаяся привычность порождает ложные ожидания}.
\end{frame}

\begin{frame}{11. Правило тишины}
	\begin{block}{Если программе нечего сказать,  то пусть лучше молчит.}
		Внимание и сосредоточенность пользователя -- ценный и ограниченный ресурс,  который требуется только в случае необходимости.
	\end{block}
	Важная информация не должна смешиваться с подробными сведениями о работе программы.
\end{frame}

\begin{frame}{12. Правило восстановления}
	\begin{block}{Когда программа завершается аварийно,  это должно происходить явно (шумно) и по возможности быстро.}
		Если программа не способна справиться с ошибкой,  то необходимо завершить ее работу так,  чтобы максимально упростить диагностику.
	\end{block}
	Для сетевых служб следует следовать рекомендации Постела:\\
	<<{\itshape Будьте либеральны к тому,  что принимаете,  и консервативны к тому,  что отправляете}>>
\end{frame}

\begin{frame}{13. Правило экономии}
	\begin{block}{Время программиста дорого -- поэтому задача экономии его времени более приоритетна,  по сравнению с экономией машинного времени.}
		Компьютер железный -- ему не скучно (с) программистская мудрость
	\end{block}
	Использование высокоуровневых языков и <<обучение>> машины выполнять больше низкоуровневой работы по программированию,  что приводит к правилу 14.
\end{frame}

\begin{frame}{14. Правило генерации}
	\begin{block}{Избегайте кодирования вручную; если есть возможность -- пишите программы для создания программ.}
		Использование генераторов кода оправданно,  когда они могут повысить уровень абстракции,  
		т.е. когда язык спецификации для генератора проще,  чем сгенерированный код,  
		и код впоследствии не потребует ручной доработки.
	\end{block}
	Примеры: грамматические и лексические анализаторы,  генераторы make-файлов,  построители GUI-интерфейсов.
\end{frame}

\begin{frame}{15. Правило оптимизации}
	\begin{block}{Сначала -- опытный образец,  потом -- оптимизирование.}
		Добейтесь стабильной работы,  только потом оптимизируйте.
	\end{block}
	\begin{block}{Керниган и Плоджер:}
		90\% актуальной и реальной функциональности лучше,  чем 100\% функциональности перспективной и сомнительной
	\end{block}
	\begin{block}{Кнут:}
		преждевременная оптимизация -- корень всех зол
	\end{block}
	\begin{block}{Кент Бек (экстремальное программирование):}
		заставьте программу работать,  заставьте работать ее верно,  а затем сделайте ее быстрой
	\end{block}
\end{frame}

\begin{frame}{16. Правило разнообразия}
	\begin{block}{Не следует доверять утверждениям о <<единственно правильном пути>>.}
		Никто не обладает умом,  достаточым для оптимизации всего или для предвидения всех возможных вариантов использования создаваемой программы.
	\end{block}
\end{frame}

\begin{frame}{17. Правило расширяемости}
	\begin{block}{Разрабатывайте для будущего. Оно наступит быстрее,  чем вы думаете.}
		При проектировании протоколов или форматов файлов следует делать их самоописательными,  для того,  чтобы их можно было расширить.
	\end{block}
	{\itshape Всегда},  следует либо включать номер версии,  либо составлять формат из самодостаточных,  
	самоописательных команд так,  чтобы можно было легко добавить новые директивы,  
	а старые удалить, <<не сбивая с толку>> код чтения формата.
\end{frame}

\begin{frame}{Все правила сразу}
	\begin{center}
	{\Huge\bfseries K.I.S.S.}

	Keep It Simple,  Stupid!
	\end{center}
\end{frame}


}

\section{Дистрибутивы ОС Linux}

\mode<all>{\begin{frame}{Дистрибутив ОС GNU/Linux}
	\begin{block}{ Определение}
		\only<1>{\center{\bf{?}}}
		\pause
		\only<2->{Набор программного обеспечения на базе ядра Linux, распространяющийся как единое целое.}
	\end{block}
\end{frame}


\begin{frame}{Задачи дистрибутива}
	\begin{itemize}
		\item Предоставление комплекта ПО (ядро + утилиты)
		\item Средства установки и настройки
		\item Средства обновления
	\end{itemize}
\end{frame}

\begin{frame}{Различия между дистрибутивами}

	\only<1>{\Large\center{\bf{?}}}
	\pause
	\only<2->{\Large\center{\bf{Цели!!!}}}

	\bigskip
	\normalsize

	\pause

	\begin{itemize}
		\begin{columns}
		\column{0.4\textwidth}
			\item Инсталлятор
			\item Первичные настройки
			\item Средства управления
			\item Набор ПО
		\column{0.4\textwidth}
			\item Менеджер пакетов
			\item Формат распространения ПО
			\item Пути к файлам
			\item Система сборки ПО
		\end{columns}
	\end{itemize}
\end{frame}

\begin{frame}{Дистрибутивы}
	\begin{itemize}
		\begin{columns}
		\column{0.3\textwidth}
			\item RedHat
			\item Fedora Core
			\item CentOS
			\item Scientific Linux
			\item Oracle Unbreakable Linux
		\column{0.3\textwidth}
			\item Slackware 
			\item Gentoo
			\item Arch
			\item OpenSUSE
			\item ALT Linux 
		\column{0.3\textwidth}
			\item Debian
			\item Ubuntu
			\item Mint
			\item Knoppix
			\item BackTrack
		\end{columns}
	\end{itemize}
\end{frame}
}

\section{Процесс загрузки ОС Linux}

\subsection{Этапы загрузки}

\mode<all>{\begin{frame}{Процесс загрузки GNU/Linux}
	\scriptsize
	\begin{enumerate}
		\item BIOS
		\item MBR
			\pause
		\item Загрузка загрузчика
		\begin{itemize}
		\footnotesize
			\item Stage 1 -- Первичный загрузчик
			\item Stage 1,5 -- Загрузка ядра загрузчика и драйвера ФС
			\item Stage 2 -- Чтение конфигурации
		\end{itemize}
			\pause

		\item Загрузка ядра в память
		\item Загрузка initrd в память
			\pause
		\item Передача управления ядру
		\begin{itemize}
		\footnotesize
			\item Распаковка
			\item Инициализация
		\end{itemize}

		\item Монтирование initrd
		\item Запуск программы инициализации в initrd
			\pause
		\item Нахождение и монтирование корневого раздела
			\pause
		\item Запуск программы init
		\begin{itemize}
		\footnotesize
			\item Монтирование оставшихся разделов ФС
			\item Инициализация оборудования
			\item Запуск демонов
		\end{itemize}

	\end{enumerate}
\end{frame}


\begin{frame}{Наиболее распространенные загрузчики}
	\begin{itemize}
		\item GRUB
		\item LILO
		\item syslinux (isolinux, pxelinux)
		\item u-boot
	\end{itemize}
\end{frame}
}

\subsection{Ядро Linux}

\mode<all>{\begin{frame}{Задачи ядра Linux}
	\begin{itemize}
		\item Инициализация системы
		\item Управление процессами и потоками
		\item Управление памятью
		\item Управление файлами
		\item IPC
		\item Разграничение доступа
		\item Сетевые возможности
		\item Интерфейс доступа к возможностям ядра
	\end{itemize}
\end{frame}


\begin{frame}{Ядро}

	Ядро ОС Linux является модульным. 

	\begin{block}{Модули}
		\begin{itemize}
			\item В виде отдельных файлов
			\item "Вкомпилированные" в ядро
		\end{itemize}
	\end{block}

	\bigskip

	Список загруженных модулей: {\tt /proc/modules}
\end{frame}


\begin{frame}{Параметры ядра}
	
	Полный список: {\tt Documentation/kernel-parameters.txt}

	\begin{block}{Некоторые часто применяемые параметры}
		\begin{itemize}
			\begin{columns}
			\column{0.3\textwidth}
				\item console=ttyS0,115200
				\item debug
				\item init=/sbin/init
				\item loglevel=[0-7]
				\item maxcpus=[num]
			\column{0.3\textwidth}
				\item mem=nn[KMG]
				\item noacpi
				\item noapic
				\item panic=nn (sec)
				\item resume=/dev/sda2
			\column{0.3\textwidth}
				\item ro
				\item rw
				\item root=/dev/sda1
				\item rootdelay=nn (sec)
				\item rootwait
				\item vga=<num>|ask
			\end{columns}
		\end{itemize}
	\end{block}

	Модулям можно передавать параметры используя синтаксис: {\tt module.param=value}

	Параметры переданные ядру во время загрузки: {\tt /proc/cmdline}
\end{frame}

\begin{frame}{Магия SysRq}

	{\tt CONFIG\_MAGIC\_SYSRQ=y}

	{\tt /proc/sysrq-trigger}

	\begin{block}{{\bf R}eboot {\bf E}ven {\bf I}f {\bf S}ystem {\bf U}tterly {\bf B}roken}
		{\bf Ctrl+Alt+SysRq+?}

		\begin{itemize}
			\item h -- вывести список сочетаний на консоль
			\item b -- перезагрузка
			\item o -- выключение
			\item e -- послать сигнал SIGTERM всем процессам кроме init
			\item i -- послать сигнал SIGKILL всем процессам кроме init
			\item s -- синхронизировать все ФС 
			\item u -- переподключить все ФС в режиме RO
		\end{itemize}
		
	\end{block}


\end{frame}
}

\subsection{Userspace}

\mode<all>{\begin{frame}{initrd}

	Система первичной загрузки.

	\begin{block}{Задача}
		Основная задача -- подготовить и проинициализировать
		устройство, на котором располагается корневая ФС.
	\end{block}
\end{frame}
}

\mode<all>{\begin{frame}{init}
	Менеджер управления работой системой и сервисами.
	
	\bigskip

	\center{\large PID = 1}

	\bigskip

	\begin{block}{Наиболее известные}
		\begin{itemize}
			\item SysVInit
			\item systemd
			\item upstart
		\end{itemize}
	\end{block}
\end{frame}
}

\subsection{Практика}

\mode<all>{\begin{frame}{Практическое задание}
	\begin{enumerate}
		\item Загрузить ОС по умолчанию
		\item Посмотреть используемые параметры ядра 
		\item Посмотреть список загруженных модулей
			\pause
		\item Переопределить init на sh
		\item SysRq. {\bf R}eboot {\bf E}ven {\bf I}f {\bf S}ystem {\bf U}tterly {\bf B}roken
			\pause
		\item Загрузить ядро с "урезанным" количеством памяти
		\item Отключить 1 или несколько процессоров
			\pause
		\item Посмотреть текущий runlevel
		\item Посмотреть список сервисов
	\end{enumerate}
\end{frame}
}


\chapter{Командная строка}

\section{Интерфейс командной строки}
\mode<all>{% Тема. Командная строка. 
% Показать примеры использования. Рассказать о преимуществах и недостатках в
% сравненни с графическим "оконным" интерфейсом. 
% Ознакомить с назначениме  эмулятора терминала и об реализациях.

\begin{frame}{Примеры использования командной строки}
	\begin{columns}
	\column{0.5\textwidth}
        \begin{itemize}
            \item чаты
            \item компьютерные игры Quake, DotA
            \item операционные системы
        \end{itemize}
	\column{0.5\textwidth}
	% insert picture of Quake 
    \includegraphics[height=0.4\textheight]{../../slides/cmdline/330px-Tremulous_console.png}
	\end{columns}
\end{frame}

\begin{frame}{Преимущества командной строки}
	\begin{itemize}
		\item Работа через сеть либо RS232
		\item Быстрый доступ к командам системы
		\item Легче отладка сообществом
		\item Легкость автоматизации
	\end{itemize}
\end{frame}

\begin{frame}{Недостатки командной строки}
	\begin{itemize}
		\item Oтсутствуют возможности обнаружения (discoverabililty)
		\item Необходимость изучения синтаксиса команд и запоминания сокращений.  (синтаксис может различаться)
		\item Без автодополнения, ввод длинных и содержащих спецсимволы параметров с клавиатуры может быть затруднительным
		\item Отсутствие «аналогового» ввода.
	\end{itemize}
\end{frame}

\begin{frame}{Эмуляторы терминала в графическом режиме}
	\begin{itemize}
		\item xterm
		\item rxvt
        \item gnome-terminal
        \item konsole
        \item Yakuake (Yet Another Kuake)
	\end{itemize}
\end{frame}
\note { 
Примеры приложений которые лучше выглядят в графическом режиме браузер,
редакторы видео и графики. Поэтому пользователь при работе, как правило,
совмещает оба интерфейса: использует графическое окружениe в сочетании с
интерфейсом командной строки. 
В графическом окружении интерфейса командной строки предоставляют приложения -
эмуляторы терминала. 
реализации - для графической системы X Window xterm, rxvt. Для GNOME
gnome-terminal, для KDE konsole, Yakuake (Yet Another Kuake выезжает по нажатии
тильды ~ как Quake)  
Дополнительные замечания:
Терминал - устройство для ввода вывода информации, уже устарел.
Графические приложения можно запускать из командной строки. 
}
}

\section{Командная оболочка (shell)}

\mode<all>{\begin{frame}[fragile]{Определение(не совсем формальное)}
	\textbf{Shell} -- приложение, обеспечивающее выполнение других приложений и их взаимодействие, а также представляющая услуги командной строки. 
	\begin{center}
	 или
	\end{center}
	\textbf{Shell} -- приложение, обеспечивающее доступ к основным функциям ядра.

	\pause
	\vspace{0.5in}
	Пример shell из Windows-world -- cmd.exe
	\vspace{0.5in}

	Минимальный дистрибутив Linux -- ядро + shell 

\end{frame}

\begin{frame}[fragile]{Основные типы shell в Unix}
  \begin{itemize}
    \item Bourne shell совместимые
      \begin{itemize}
        \item \textbf{sh} исходная bourne shell (Steve Bourne, 1978)
        \item \textbf{ksh} Korn shell (David Korn, 1983)
        \item \textbf{ash} $[$BSD$]$ Almquist shell (Kenneth Almquist,1989)  
        \item \textbf{bash} $[$GPL$]$ Bourne-again shell (Brian Fox, 1989)
        \item \textbf{zsh} $[$BSD$]$ Z shell (Paul Falstad,1990)
        \item \textbf{/bin/sh} Указывает на POSIX-совместимую shell
      \end{itemize}
  \item C shell совместимые
      \begin{itemize}
        \item \textbf{csh}  Исходная С shell (Bill Joy, 1978)
        \item \textbf{tcsh} $[$BSD$]$ TENEX C shell (Ken Greer, 1981)
       \end{itemize}
  \end{itemize}
\end{frame}

\begin{frame}[fragile]{Маленькое упражнение}
\begin{lstlisting}[language=bash]
cat /etc/shells
ls -l <filename> # для каждого элемента /etc/shells
readlink -e <filename> 
\end{lstlisting}
\end{frame}


}

\section{Don't panic! Получение помощи}

\mode<all>{\begin{frame}[fragile]{Получение помощи}
  \begin{itemize}
    \pause
    \item \textbf{man} - помощь по внешним командам
    \pause
    \item \textbf{help} - помощь по внутренним командам bash (также man bash)
    \pause
    \item \textbf{info} - расширенная помощь по некоторым командам (texinfo format)
      \begin{itemize}
       \item   Попробовать {\tt info coreutils}
       \item   Справка по навигации -- нажать h
      \end{itemize}
  \end{itemize}
\end{frame}

\begin{frame}[fragile]{Основное о man}
\begin{columns}
	\column{2.2in}
		\begin{itemize}
			\item Прочитайте {\tt man man} !
			\item Apropos {\tt man -k <слово>}
			\item Разделы (sections)
				\begin{itemize}
					\item[1] Основная секция(юзерские программы)
					\item[2] Syscalls
					\item[3] С library
					\item[5] Конфигурационные файлы
					\item[8] Системные службы
				\end{itemize}
		\end{itemize}
	  \textbf{Замечание}

	  Обычно внутри страницы работает поиск с помощью '/'
	\pause 
	
	\column{1in}
		\begin{block}{Попробовать}
			\begin{lstlisting}
man -k printf
man 3 printf
man 1 printf
man -a printf
			\end{lstlisting}
		\end{block}
	\end{columns}
\end{frame}


}

\section{Навигация по файловой системе}

\mode<all>{\begin{frame}{Навигация по файловой системе}
      \begin{itemize}
		  \item {\tt ls} -- список файлов в (текущей по умолчанию) директории (man ls)
		  \item {\tt cd} -- смена текущей директории (help cd)
		  \item {\tt pwd} -- имя текущей директории (help pwd)
      \end{itemize}
\end{frame}

\begin{frame}[fragile]{Команды для работы с файлами}
	\begin{itemize}
		\begin{columns}
		\column{0.2\textwidth}
			\item touch
			\item ln
			\item mkdir
			\item mknod
			\item mkfifo
		\column{0.2\textwidth}
			\item cp
			\item mv
			\item install
			\item rm
			\item rmdir
			\item file
		\column{0.4\textwidth}
			\begin{block}{Упражнение}
				\begin{enumerate}
					\item Создать иерархию директорий
						\begin{lstlisting}
dir1/dir1.1/dir1.1.1
dir1/dir1.2/dir1.2.1
dir1/dir1.2/dir1.2.2
						\end{lstlisting}
					\item Внутри каждой создать файл
					\item Удалить все созданное
				\end{enumerate}
			\end{block}
		\end{columns}
	\end{itemize}
\end{frame}


}

\mode<all>{\begin{frame}{Файловая структура}
	
	{\center "Дерево внутри дома?" (c) Шрек}
		
	\begin{columns}
	\column{0.2\textwidth}
		\includegraphics[height=0.8\textheight]{../../slides/fs/01-lhs.png}
	\column{0.7\textwidth}
		\begin{itemize}
			\item Директории
			\item Обычные файлы
			\item Симлинки
			\item Хардлинки
			\item Файлы устройств
			\item FIFO
			\item сокеты
		\end{itemize}
	\end{columns}
\end{frame}
}

\section{Дополнительные возможности оболочки}
\mode<all>{\begin{frame}{Важные аббревиатуры внутри командной строки}
  \begin{itemize}
    \item Для директорий
      \begin{itemize}
        \item {\tt $\sim$} Домашняя директория
        \item {\tt $\sim$<username>} Домашняя директория пользователя
        \item {\tt ..} Родительская директория
        \item {\tt .} Текущая директория
      \end{itemize}
      \pause  
    \item Wildcards
      \begin{itemize}
        \item {\tt *} Любой набор символов {\tt file*txt : file1.txt filefilefiletxt}
        \item {\tt $[$<список>$]$ } символ из заданного набора
        \item {\tt ?} любой один символ
      \end{itemize}

  \end{itemize}
\end{frame}       

\begin{frame}{Горячие клавиши}
  \begin{itemize}
    \item \textbf{Tab} -- дополнение текущей команды
      \pause
    \item История команд
      \begin{itemize}
        \item Клавиши курсора -- навигация по истории
        \item Ctrl-R -- поиск в истории по фрагменту
        \item Ctrl-O (после выполнения вставить следующую команду из истории)
        \item Команда {\tt history}
      \end{itemize}
    \item Навигация

  \end{itemize}
\end{frame}

\begin{frame}{Переменные окружения}
  \begin{itemize}
    \item {\tt HOME}
    \item {\tt PWD}
    \item {\tt LANG}
    \item {\tt LD\_LIBRARY\_PATH}
    \item {\tt SHELL}
    \item {\tt TERM}
    \item {\tt DISPLAY}
  \end{itemize}

  Контроль

  \begin{itemize}
    \item export {\tt export VAR=value}
    \item declare -x
    \item echo 
  \end{itemize}

  Переменные окружения наследуются при создании нового процесса
\end{frame}

%\begin{frame}{Настройки bash и кастомизация}
%  \begin{itemize}
%    \item Login shell
%      \begin{itemize}
%        \item {\tt /etc/profile}
%        \item {\tt $\sim$/.profile }
%      \end{itemize}
%    \item Обычная интерактивная shell
%      \begin{itemize}
%        \item {\tt /etc/bash.bashrc}
%        \item {\tt $\sim$/.bashrc}
%      \end{itemize}
%  \end{itemize}
%
%  Полезные команды
%  \begin{itemize}
%    \item {\tt alias}
%    \item {\tt export PATH=}
%    \item {\tt Определение функции}
%    \item {\tt shopts}
%  \end{itemize}
%
%\end{frame}


}

\section{Процессы}
\mode<all>{\begin{frame}{Процессы в UNIX}
  \begin{itemize}
    \item Создание процессов
      \begin{itemize}
        \item fork
        \item exec
      \end{itemize}
    \item Атрибуты процесса
      \begin{itemize}
        \item pid 
        \item файловые дескрипторы
        \item environment
        \item Рабочая директория (cwd)
        \item прочее в директории {\tt /proc/<pid>}
      \end{itemize}
  \end{itemize}
\end{frame}

\begin{frame}{Управление процессами}
  \begin{itemize}
    \item kill (killall)
    \item top
    \item pstree
    \item Команды управления процессами в bash: 
      \begin{itemize}
        \item {\tt jobs}, {\tt fg, \tt bg}
        \item Ctrl-C -- оборвать выполнение процесса (SIGINT)
        \item Ctrl-Z -- остановить выполнение команды (SIGTSTP)
        \item Ctrl-D -- завершить ввод
      \end{itemize}
  \end{itemize}
\end{frame}


\begin{frame}{Упражнения}
  \begin{block}{Посмотреть вывод pstree}
    {\tt pstree}
  \end{block}
  \pause
  \begin{block}{Ctrl-C, Ctrl-Z}
    В графическом режиме запустить из терминала emacs

    Ctrl-Z

    jobs -l

    bg +
  \end{block}
  \pause
  \begin{block}{fork bomb}

    {\tt ulimit -u 200} 

    {\tt bomb()\{ (bomb; bomb) \& \} }

    top

    killall bash

  \end{block}
\end{frame}


\begin{frame}{Unix way}
  \begin{enumerate}
    \item Пишите программы, которые делают одну вещь и делают её хорошо.
    \item Пишите программы, которые бы работали вместе.
    \item Пишите программы, которые бы поддерживали текстовые потоки, поскольку это универсальный интерфейс. 
  \end{enumerate}
\end{frame}

\begin{frame}{Unix way}
  \begin{enumerate}
    \item   Маленькое прекрасно.
    \item   Пусть каждая программа делает одну вещь, но хорошо.
    \item   Собирайте прототип как можно раньше.
    \item   Предпочитайте переносимость эффективности.
    \item   Храните данные в простых текстовых файлах.
    \item   Используйте программные рычаги для достижения цели.
    \item   Используйте сценарии командной строки для улучшения функционала и переносимости.
    \item   Избегайте <<связывающего>> (captive) пользовательского интерфейса.
    \item   Делайте каждую программу «фильтром».
  \end{enumerate}
\end{frame}

\begin{frame}{Конвееры}
%  \textbf{Цель} -- максимальная модульность: большое количество простых приложений, взаимодействующих друг с другом для решения задач
  \only<1>{
  \begin{center}
    \includegraphics[width=1.2in]{../../slides/cmdline/process}
  \end{center}
  }
  \only<2>{
    \begin{center}
      \includegraphics[width=3.6in]{../../slides/cmdline/processes}
    \end{center}
  }
  \begin{itemize}
    \item <1-> Каждое приложение открывает 3 стандартных файловых дескриптора stdin (fd 0), stdout(fd 1), stderr (fd 2)
    \item <2-> Приложения могут работать как фильтр из STDIN в STDOUT, можно объединять несколько приложений в конвейер
    \item <2-> Синтаксис {\tt <app1> | <app2>}
  \end{itemize}
\end{frame}
}

\section{Перенаправление ввода-вывода}
\mode<all>{

\begin{frame}{Конвееры}
%  \textbf{Цель} -- максимальная модульность: большое количество простых приложений, взаимодействующих друг с другом для решения задач
  \only<1>{
  \begin{center}
    \includegraphics[width=1.2in]{../../slides/cmdline/process}
  \end{center}
  }
  \only<2>{
    \begin{center}
      \includegraphics[width=3.6in]{../../slides/cmdline/processes}
    \end{center}
  }
  \begin{itemize}
    \item <1-> Каждое приложение открывает 3 стандартных файловых дескриптора stdin (fd 0), stdout(fd 1), stderr (fd 2)
    \item <2-> Приложения могут работать как фильтр из STDIN в STDOUT, можно объединять несколько приложений в конвейер
    \item <2-> Синтаксис {\tt <app1> | <app2>}
  \end{itemize}
\end{frame}
}
\mode<all>{\begin{frame}{Перенаправления в файл}

\begin{itemize}
  \item Перенаправление stdout 
    \begin{itemize}
      \item С созданием нового файла

        {\tt command > file}\\
		Например {\tt cat file1 file2 > file3}
      \item С дополнением существующего

		  {\tt command >\phantom{}>  file}
    \end{itemize}
    \pause
  \item Перенаправления stdin

    {\tt command < file}
    \pause
  \item Перенаправления stderr

    {\tt command1 2>\&1 | command2}

   {\tt command 1>file 2>\&1}

   {\tt command 2>file 1>\&2}
\end{itemize}

\end{frame}


}
\mode<all>{\begin{frame}[fragile]{Мультистрочный ввод (Here документ)}

	\begin{columns}
	\column{0.45\textwidth}
	{\center Не экранированный}
\begin{verbatim}
program <<LABEL
Тут
    много
        строк
LABEL
\end{verbatim}
	\column{0.45\textwidth}
	{\center Экранированный}
\begin{verbatim}
program <<"LABEL"
Тут
    переменная
        $VARIABLE
LABEL
\end{verbatim}
	\end{columns}
	\pause
	\begin{block}{Пример}
	Передадим несколько строк в COM-порт 1
\begin{lstlisting}[language=bash]
cat >/dev/ttyS0 <<E_O_F
ATZ
ATDT 8w0170123456
E_O_F
\end{lstlisting}
	\end{block}
\end{frame}

\begin{frame}[fragile]{Here string}

\begin{verbatim}
program <<<string
\end{verbatim}

	\begin{block}{Пример}
	Инициализация переменных из строки:
\begin{lstlisting}[language=bash]
read A B C <<<"First Second Third"

# echo $A $B $C
First Second Third
\end{lstlisting}
	\end{block}
\end{frame}
}

\section{Полезные команды}

\mode<all>{\begin{frame}{Дополнительный набор команд}
  \begin{itemize}
    \item {\tt cat} - Вывод файла в stdout, соединение нескольких файлов в stdout
    \item {\tt wc} - подсчет статистики символов в файле или в stdin 
    \item {\tt sort} - сортировка строк файла
    \item {\tt uniq} - объединение одинаковых строк в одну
    \item {\tt tr} - замена набора символов
    \item {\tt less} - программа-пейджер
    \item {\tt grep} - поиск строк, соответствующих регулярному выражению
    \item {\tt cut} - выделение полей из строк stdin
    \item {\tt awk} - небольшой язык программирования (также полезен для выделения полей)
  \end{itemize}
\end{frame}

\begin{frame}[fragile]{Некоторые примеры использования}
\begin{lstlisting}[language=bash]
cat /proc/1/environ | tr '\0' '\n' | less
ls  | wc -l # подсчет числа файлов
man uniq | tr  '[:space:]' '\n' | sort | uniq -c | sort -n | less # подсчет количества слов в тексте man uniq
history | wc -l # подсчет ранее введенных команд
cat /etc/udev/rules.d/* | wc -l
ls -s *.jpg | awk 'BEGIN{s=0};/^[ ]*[0-9]/{s+=`\$1`};END{print s}' 
\end{lstlisting}
  \pause
  \begin{block}{Упражнение}
    Посчитать статистику использования команд в history
  \end{block}
\end{frame}

\begin{frame}{Дополнительный набор команд для работы с текстом}
	\begin{itemize}
	  \item {\tt head} -- вывести первые строки
	  \item {\tt tail} -- вывести последние строки
		\begin{itemize}
			\item {\tt -f} -- отслеживать добавление данных в файл 
		\end{itemize}
	  \item {\tt tee} -- копировать стандартный вывод в файл
	  \item {\tt grep} -- печать текста, соответствующего шаблону
		\begin{itemize}
			\item {\tt -i}	
			\item {\tt -v}
			\item {\tt -o}
		\end{itemize}
	\end{itemize}
\end{frame}

}

\subsection{Архиваторы}
\mode<all>{\begin{frame}{Архивация}
	\begin{block}{Архивация: tar}
		\begin{itemize}
			\item {\tt -c} -- создать архив
			\item {\tt -x} -- извлечь из архива
				\begin{itemize}
					\item {\tt -C} -- перейти в директорию
				\end{itemize}
			\item {\tt -f} -- запись в файл
		\end{itemize}
	\end{block}

	\begin{block}{Сжатие: gzip, bzip, xz}
		\begin{itemize}
			\item {\tt -[1-9]} -- изменить уровень сжатия
			\item {\tt -d} -- распаковать
			\item {\tt -c} -- вывод на консоль
		\end{itemize}
	\end{block}
\end{frame}

\begin{frame}[fragile]{Архивация: примеры}

	Создать сжатый архив:
	\begin{verbatim}
tar -czf archive.tar.gz *
	\end{verbatim}
	\pause
	Распаковать сжатый архив в директорию {\tt /tmp}:
	\begin{verbatim}
tar -C /tmp/ -xzf archive.tar.gz 
	\end{verbatim}
	\pause
	Создать сжатый архив:
	\begin{verbatim}
tar -czf archive.tar.gz *
	\end{verbatim}
	\pause
	Создать копию текущей директории в директории {\tt /tmp/copy/}:
	\begin{verbatim}
tar -c * | tar -C /tmp/copy -x
	\end{verbatim}
	\pause
	Создать копию текущей директории на другом хосте:
	\begin{verbatim}
HostDest: netcat -l 2222 | gzip -dc | tar -C /tmp/copy/ -x
HostSrc:  tar -c * | gzip -9 | netcat HostDest 2222
	\end{verbatim}
\end{frame}

\begin{frame}[fragile]{Поиск файлов}
	\begin{block}{find}
		\begin{itemize}
			\item {\tt -type} -- тип файлового объекта
			\item {\tt -size} -- размер
			\item {\tt -maxdepth} -- глубина рекурсии
			\item {\tt -exec} -- выполнить команду
			\item {\tt -printf} -- форматированный вывод
		\end{itemize}
	\end{block}

	\begin{block}{Примеры}
		\begin{verbatim}
find /etc -type f -size +100k  -exec ls -l {} \;
		\end{verbatim}

		\begin{verbatim}
find -type d -user altlinux
		\end{verbatim}
	
	\end{block}
\end{frame}

\begin{frame}[fragile]{xargs}
	\begin{block}{xargs}
			Утилита для создания и запуска команд из стандартного потока ввода:
		\begin{verbatim}
xargs [options] command [command options]
		\end{verbatim}
	
	\end{block}

	\begin{block}{Примеры}
		\begin{verbatim}
find /etc -type f -size -100k | xargs tar -czf /tmp/archive-100k.tar.gz
		\end{verbatim}

		\begin{verbatim}
find /etc -type f | xargs -I {} echo "Найден {} файл"
		\end{verbatim}
	
	\end{block}
\end{frame}


}

\subsection{find и xargs}
\mode<all>{\begin{frame}[fragile]{Поиск файлов}
	\begin{block}{find}
		\begin{itemize}
			\item {\tt -type} -- тип файлового объекта
			\item {\tt -size} -- размер
			\item {\tt -maxdepth} -- глубина рекурсии
			\item {\tt -exec} -- выполнить команду
			\item {\tt -printf} -- форматированный вывод
		\end{itemize}
	\end{block}

	\begin{block}{Примеры}
		\begin{verbatim}
find /etc -type f -size +100k  -exec ls -l {} \;
		\end{verbatim}

		\begin{verbatim}
find -type d -user altlinux
		\end{verbatim}
	
	\end{block}
\end{frame}

\begin{frame}[fragile]{xargs}
	\begin{block}{xargs}
			Утилита для создания и запуска команд из стандартного потока ввода:
		\begin{verbatim}
xargs [options] command [command options]
		\end{verbatim}

		\begin{itemize}
			\item {\tt -d} -- разделитель
			\item {\tt -0} -- null-terminated строки
			\item {\tt -I text} -- подстановка
			\item {\tt -n N} -- максимальное количество аргументов
			\item {\tt -P N} -- максимальное количество процессов
		\end{itemize}

	\end{block}
\end{frame}

\begin{frame}[fragile]{xargs}
	\begin{block}{Примеры}
		\begin{verbatim}
find /etc -type f -size -100k | \
 xargs tar -czf /tmp/archive-100k.tar.gz
		\end{verbatim}

		\begin{verbatim}
find /etc -type f | xargs -I {} echo "Найден {} файл"
		\end{verbatim}

		\begin{verbatim}
find . -type f -name "*.mp3" -print0 | \
 xargs -0 -n 1 -P 0 -I mp3 avconv -i mp3 mp3.ogg
		\end{verbatim}
	
	\end{block}
\end{frame}

\begin{frame}[fragile]{Задание}
	\begin{itemize}
		\item Создать директорию {\tt /tmp/etc}
		\item С помощью {\tt find} скопировать только файлы,
		      из {\tt /etc} (не включая поддиректории) в {\tt /tmp/etc}
		\item Сжать все файлы из {\tt /tmp/etc}, используя все доступные процессоры
	\end{itemize}
\end{frame}

}

\subsection{Редакторы}
\mode<all>{\begin{frame}{Редакторы}
	\begin{itemize}
		\item Интерактивные
			\begin{itemize}
				\item vi
					\begin{itemize}
						\item Есть почти везде
					\end{itemize}
				\item vim
				\item emacs
			\end{itemize}
		\item Поточные
			\begin{itemize}
				\item {\tt ed}
				\item {\tt sed}
				\item {\tt awk}
			\end{itemize}
	\end{itemize}
\end{frame}

\begin{frame}[fragile]{Метасимволы}
	\begin{block}{grep, sed, awk}
	\end{block}
	\begin{itemize}
		\item {\tt .} -- любой символ за исключением пустой строки
		\item {\tt *} -- любоe количество символов, которые стоят перед {\tt *}
		\item {\tt \^{}} -- начало строки
		\item {\tt \$} -- конец строки
		\item {\tt [...]} -- любой символ из заключенных в скобки
	\end{itemize}
\end{frame}

\begin{frame}[fragile]{sed}
	\begin{block}{Сценарии}
		{\tt [ addr [ ,  addr ] ] cmd [ args ]}
	\end{block}

	\tiny
	\begin{block}{Команды}
		\begin{itemize}
		  \item {\tt a, i} -- добавить строку после (перед) текущей
			  \begin{verbatim} who | sed -e 'a Text' \end{verbatim}
		  \item {\tt c} -- удалить строку и заменить на текст
			  \begin{verbatim} who | sed -e "/$USER/ c Юзверь" \end{verbatim}
		  \item {\tt d} -- удалить строку
			  \begin{verbatim} who | sed -e '2,4 d' \end{verbatim}
			  \begin{verbatim} who | sed -e '/pts/ d' \end{verbatim}
		  \item {\tt s} -- замена по регулярному выражению
			  \begin{verbatim} who | sed -e "s/$USER/Юзверь/g" \end{verbatim}
		\end{itemize}
	\end{block}
	\pause
	\begin{block}{Задача}
		С помощью {\tt find} найти все вложенные директории в {\tt /etc} и 
		''переделать'' их в windows-style
	\end{block}
\end{frame}


}

\chapter{Система управления пакетами}
\section{Система управления пакетами}
\mode<all>{\begin{frame}
	\frametitle{И еще раз про "DLL hell"}
	
	\begin{block}{Устанавливаем программу}
	А что же с библиотеками?
	\end{block}

	\pause

	\begin{columns}
		\column{0.5\textwidth}
		\begin{block}{"В системе все есть!"}
		\begin{itemize}
			\item Oh, really???
			\item И нужной версии?
			\item А API и ABI точно не менялись?
			\item А если библиотек несколько версий?
			\item А если нужны дополнительные программы?
		\end{itemize}
		\end{block}
		\pause
		\column{0.5\textwidth}
		\begin{block}{"Всё своё, ношу с собой!"}
		\begin{itemize}
			\item А как насчет объема?
			\item Использование памяти.
			\item А что насчет лицензий?
			\item И все-таки порядок загрузки...
			\item Не спасает от проблем с 3rd-party ПО.
		\end{itemize}
		\end{block}
	\end{columns}
\end{frame}

\begin{frame}
	\frametitle{Хаос}

	\begin{center}
		"Даешь каждой платформе и языку собственную систему управления пакетами!"
	\end{center}

	\begin{block}{Увы, мы не в идеальном мире}
		\begin{itemize}
			\item Дистрибутивы: rpm\{4,5\}, deb, portage, pacman... и куча модификаций...
			\item Дополнительный софт: {\tt ./configure; make; make install}
			\item Java: {\tt ivy, ant, maven, gradle}
			\item Ruby: gem
			\item Perl: CPAN
			\item Python: pip + PyPi
		\end{itemize}
	\end{block}

\end{frame}



}

\mode<all>{\begin{frame}{Система управления пакетами: для чего это нужно}
\begin{itemize}
 \item ''DLL Hell''
 \item Dependency hell
 \item Общие задачи пакетного менеджера:
   \begin{itemize}
     \item Проверка целостности пакетов
     \item Проверка зависимостей пакетов
        \item Поддержание списка установленных пакетов
        \item Автоматическое удаление пакетов
     \item Предоставление доступа к репозиторию пакетов
     \item Разрешение зависимостей
   \end{itemize}
\end{itemize}
\end{frame}

\begin{frame}{Debian-based и RedHat-based системы управления пакетами}
\begin{center}
 \textbf{Два уровня пакетных менеджеров}
\end{center}
\begin{columns}
  \column{0.4\textwidth}
  \begin{center}
    \textbf{RedHat-based}
  \end{center}
  \begin{itemize}
    \item yum
    \item rpm
  \end{itemize}
  \column{0.4\textwidth}
  \begin{center}
    \textbf{Debian-based}
  \end{center}
  \begin{itemize}
    \item aptitude, apt, synaptic
    \item dpkg
  \end{itemize}
\end{columns}
\end{frame}

\begin{frame}{RPM: структура пакета}
	\begin{itemize}
		\item Метаданные
			\begin{itemize}
				\item Имя
				\item Версия/Релиз
				\item Группа
				\item Описание
				\item ...
			\end{itemize}
		\item Архив с файлами
			\begin{itemize}
				\item cpio
			\end{itemize}
		\item Скрипты
			\begin{itemize}
				\item Pre Install
				\item Post Install
				\item Pre Uninstall
				\item Post Uninstall \bigskip
				\item Triggers
			\end{itemize}
	\end{itemize}
\end{frame}
}

\mode<all>{\begin{frame}{RPM: команды}
	\begin{block}{Установка пакета}
		{\tt rpm -i [rpm-file1] ... [[url://]rpm-fileN] }
	\end{block}
	\begin{block}{Удаление пакета}
		{\tt rpm -e pkgname1 ... pkgnameN }
	\end{block}
	\begin{block}{Обновление пакета}
		{\tt rpm -U [rpm-file1] ... [[url://]rpm-fileN] }
	\end{block}
	\begin{block}{Проверка пакета}
		{\tt rpm -V pkgname1 ... pkgnameN }
	\end{block}
\end{frame}

\begin{frame}{RPM -q: часто используемые опции опроса}

	\begin{itemize}
		\item {\tt pkgname} -- выбор пакета, установленного в системе
		\item {\tt -a} -- все пакеты, установленные в системе
		\item {\tt -p} -- использовать файл RPM
	\end{itemize}


	\begin{itemize}
		\item {\tt -i} -- показать информацию пакета\\
			{\tt rpm -q -i glibc }
		\item {\tt -l} -- показать список файлов пакета \\
			{\tt rpm -q -l glibc }
		\item {\tt -{}-whatprovides} -- \\
			{\tt rpm -q --whatprovides java}
		\item {\tt -{}-whatrequires} -- \\
			{\tt rpm -q --whatrequires /bin/bash}
		\item {\tt -{}-queryformat} -- формат вывода\\
			{\tt rpm -q -{}-whatrequires /bin/bash -{}-queryformat ''\%\{name\} ''}

	\end{itemize}

\end{frame}


}

\mode<all>{\newcounter{tmpc}

\begin{frame}{Репозиторий}
	\begin{block}{Репозиторий пакетов}
		Место, где хранятся и поддерживаются пакеты, а также сопутствующая мета-информация, предназначенное для использования пакетным менеджером.
	\end{block}
	\begin{block}{Пример: Fedora Core}
		\begin{itemize}
			\item Packages/*.rpm
			\item RPM-GPG-KEY-*
			\item repodata
			\begin{itemize}
				\item множество сжатых и несжатых XML файлов для YUM
			\end{itemize}
		\end{itemize}

		Описание репозтория для YUM на локальной системе хранится по пути
		{\tt /etc/yum.repos.d/*.repo}
	\end{block}
		
\end{frame}

\begin{frame}{Apt: команды}
	\begin{block}{Установка/обновление пакета}
		{\tt apt-get install pkgname }

                {\tt apt-get -f install}
	\end{block}
	\begin{block}{Обновление данных о пакетах}
		{\tt apt-get update }
	\end{block}
	\begin{block}{Удаление пакета}
		{\tt apt-get remove pkgname }
	\end{block}
	\begin{block}{Поиск}
		{\tt apt-cache search pkgname }
	\end{block}
\end{frame}

\begin{frame}{YUM: команды}
	\begin{block}{Установка/обновление пакета}
		{\tt yum install pkgname }
	\end{block}
	\begin{block}{Обновление всех пакетов}
		{\tt yum update }
	\end{block}
	\begin{block}{Удаление пакета}
		{\tt yum remove pkgname }
	\end{block}
	\begin{block}{Поиск}
		{\tt yum list pkgname }\\
		{\tt yum search pkgname }
	\end{block}
\end{frame}


\begin{frame}[fragile]{Упражнение}
%  \begin{enumerate}
%      \item Создать на {\tt /dev/sda} раздел размером примерно 10Gb
%      \item Создать на этом разделе ext3 ФС и смонтировать раздел в {\tt /mnt/chroot}
%      \item Развернуть {\tt /media/nfs/pub/CentOS/precreated/centOS.tar.gz} в {\tt /mnt/chroot}
%      \item Смонтировать {\tt proc, sysfs} а также {\tt /dev} в соответствующие места {\tt /mnt/chroot}
%      \item {\tt chroot /mnt/chroot}
%      \item Отредактировать {\tt /etc/resolv.conf} -- скопировать туда информацию из {\tt resolv.conf} основной системы
%      \item Отредактировать {\tt /etc/yum.conf} Добавить следующий раздел
%\begin{minipage}{0.5\textwidth}
%\begin{verbatim}
%[base]
%  name = CentOS 6
%  baseurl = ftp://192.168.11.15/CentOS
%  gpgcheck = 0
%\end{verbatim}
%\end{minipage}
%\setcounter{tmpc}{\theenumi}
%\end{enumerate}
%\end{frame}
%\begin{frame}{Продолжение упражнения}
  \begin{enumerate}
      %\setcounter{enumi}{\thetmpc}
      \item {\tt apt-get update}
      \item Удалить пакет vim
      \item Установить заново пакет vim
      \item Посмотреть списки файлов для пакетов {\tt rpm, vim}
      \item Найти, к какому пакету относится команда {\tt ls, top}
      \item Найти пакет предоставляющий сервис ssh и установить его
    \end{enumerate}
\end{frame}


}

\chapter{Пользователи и привилегии}
\newcommand{\defaultuser}{USER}
\section{Многопользовательская модель UNIX}
\mode<all>{\begin{frame}{Многопользовательская модель}   
 \begin{itemize}
   \item Linux -- многопользовательская система
   \item Привилегии пользователей
     \begin{itemize}
       \item root
       \item other users
      \end{itemize}
     \end{itemize}
\end{frame}

%\section{Механизмы разделения привилегий}
%\subsection{Классический UNIX}

\begin{frame}{Пользователи, группы и файлы}
\begin{itemize}
  \item Каждый пользователь принадлежит одной или нескольким \textbf{группам}
  \item Каждый файл и директория принадлежит
    \begin{itemize}
      \item Одному пользователю 
      \item Одной группе
    \end{itemize}
  \pause
  \item  Разрешения что либо делать с файлом определяются по отношению к
    \begin{enumerate}
      \item Пользователю-владельцу файла
      \item Группе владеющей файлом
      \item Всем остальным пользователям
    \end{enumerate}

\end{itemize}
\pause
\begin{columns}
  \column{0.48\textwidth}
  \begin{itemize}
    \item {\tt ls -l} 3,4 поле 
    \item {\tt groups}
   \end{itemize}
  \column{0.48\textwidth}
  \begin{block}{Попробовать}
    {\tt ls -l /usr/bin/}

    {\tt groups}

    {\tt groups root}
  \end{block}
\end{columns}
\end{frame}


}

\mode<all>{\begin{frame}{Типы разрешений для файлов}
	\begin{columns}
		\column{0.48\textwidth}
		\begin{center}
			\textbf{Разрешения для файла}
		\end{center}
		\begin{itemize}
			\item Три типа разрешений
				\begin{enumerate}
					\item чтение read(r)
					\item запись write(w)
					\item выполнение execute(x)
				\end{enumerate}
		\end{itemize}
		\column{0.48\textwidth}
		\begin{center}
			\textbf{Разрешения для директорий}
		\end{center}
		\begin{itemize}
			\item Три типа разрешений
				\begin{enumerate}
					\item поиск файлов в директории read(r) 
					\item добавление и удаление файлов write(w)
					\item заход в директорию execute(x)
				\end{enumerate}
		\end{itemize}
	\end{columns}

	\pause

	Попробовать {\tt ls -l /usr/bin}

	\pause

	Пересчет мнемонического разрешения в битовую маску 

	$r\to4, w\to2 , x\to1$ 

	rwxrw-r-x$\to$765
\end{frame}

\begin{frame}{Команды для управления пользователями и разрешениями файлов}
	\begin{columns}
		\column{0.48\textwidth}
		\begin{itemize}
			\item {\tt chown}
			\item {\tt chmod}
		\end{itemize}
		\column{0.48\textwidth}
		\begin{itemize}
			\item {\tt useradd, usermod, userdel}
			\item {\tt groupadd, groupmod, groupdel}
			\item {\tt su, sudo}
		\end{itemize}
	\end{columns}
\end{frame}

\begin{frame}
    \frametitle{}
	\begin{block}{Упражнения}
		\begin{enumerate}
			\item Создать директорию без r разрешения но с x разрешением, внутри нее создать поддиректорию с rwx разрешениями (для пользователя \defaultuser)
			\item Создать нового пользователя testuser.
			\item Скопировать {\tt /bin/bash} (под именем mysh) в домашнюю директорию пользователя \defaultuser  и поставить r-x разрешение только для other
			\item Попробовать выполнить скопированный файл от имени пользователя \defaultuser, затем от имени пользователя testuser
       \end{enumerate}
    \end{block}
\end{frame}
\begin{frame}
    \frametitle{}
	\begin{block}{Упражнения}
		\begin{enumerate}
			\item Создать новую группу testgroup
			\item Изменить группу владеющую mysh на testgroup и сделать {\tt chmod 474 mysh}
			\item Попробовать выполнить mysh от имени \defaultuser и root. 
			\item Добавить пользователя \defaultuser в группу testgroup и попробовать выполнить mysh еще раз
			\item Получить список групп которым принадлежат устройства в {\tt /dev}
		\end{enumerate}
	\end{block}
\end{frame}

\begin{frame}{SUID программы}
	\begin{block}{Попробовать}
		{\tt id}

		{\tt ls -l `which su`}
	\end{block}
	\pause
	\begin{itemize}
		\item Некоторые программы должны выполняться от имени обычного пользователя, но иметь больше привилегий
		\item Для этого у них устанавливается suid или sgid биты
		\item Установка suid (например {\tt chmod 4710 <file>})
	\end{itemize}
	\pause
	\begin{block}{Упражнение}
		\begin{itemize}
			\item Под root создать копию утилиты {\tt id} (назвать, например, {\tt id2}) в директории /usr/bin/
			\item Установить suid бит для этой утилиты
			\item Запустить {\tt id2} от имени пользователя \defaultuser
			\item То же с sgid битом
		\end{itemize}
	\end{block}
\end{frame}

\begin{frame}{Опасности SUID}
	\begin{itemize}
		\item Возможность backdoor через suid программу
			\begin{itemize}
				\item Shell игнорирует effective uid
				\item Скрипты обычно тоже игнорируют
				\item nosuid mount option
			\end{itemize}
		\item Атака через buffer overflow в существующей suid программе
			\begin{itemize}
				\item не использовать strcpy, sprintf, ... в security critical
				\item А если все же не уследили
					\begin{itemize}
						\item рандомизация стека
						\item grsecurity
						\item частично selinux
					\end{itemize}
			\end{itemize}
	\end{itemize}
\end{frame}


\begin{frame}{SUID, SGID и sticky bit для директорий}
	\begin{itemize}
		\item sgid для директорий -- все поддиректории и файлы внутри имеют тот же group id
		\item suid -- игнорируется
		\item Sticky bit (\tt{chmod +t mydir})
          \begin{itemize}
            \item Файлы из обычной директории может удалять любой пользователь с правами на запись в \emph{директорию}
            \item Файлы из директории со sticky bit может удалять только владелец директории, владелец файла или root.
          \end{itemize} 
	\end{itemize}
\end{frame}

\begin{frame}[fragile]
 \frametitle{UMASK}

	\begin{block}{umask}
		маска режима создания пользовательских файлов
	\end{block}

	Права доступа файлов, вычисляются c помощью побитовых операций:
    \begin{itemize}
      \item библиотечный вызов \tt{fopen} создает файл с разрешениями 
     \verb+ 0666 & ~umask +
      \item Системный вызов \tt{open(pathname,flags,mode)} создает файл с разрешениями \verb+ mode & ~umask +
   \end{itemize}
        

\end{frame}

\begin{frame}
	\begin{block}{Упражнение}
		\begin{enumerate}
			\item Создать от имени \tt{root} директорию \tt{/stick} с установленными битами \tt{sticky} и \tt{SGID}, а также разрешениями на чтение, запись и выполнение для всех
			\item От имени пользователя \tt{root} создать пустой файл \tt{testroot}
			\item Изменить \tt{umask} пользователя \tt{root} на 002
			\item От имени пользователя \tt{root} создать пустой файл \tt{testroot2}
			\item От имени пользователя \tt{\defaultuser} создать пустой файл \tt{testuser}
			\item Посмотреть получившиеся права и принадлежность файлов
			\item Отредактировать файл \tt{testroot2}
			\item Попробовать удалить файл \tt{testroot2}
		\end{enumerate}
	\end{block}
\end{frame}



}
\section{Внутренний механизм управления пользователями}
\mode<all>{\begin{frame}{Хранение информации о пользователях в системе}

	\begin{block}{\tt /etc/group}
		{\tt  group\_name:password:GID:user\_list}
	\end{block}
	
	\pause

	\begin{block}{\tt /etc/passwd}
		{\tt account:password:UID:GID:GECOS:directory:shell}

		\begin{itemize}
			\item {\tt *} -- пароль не задан
			\item {\tt x} -- пароль задан в файле {\tt /etc/shadows}
		\end{itemize}
	\end{block}

	\pause

	\begin{block}{\tt /etc/shadow}
		\begin{enumerate}
			\begin{columns}
			\column{0.3\textwidth}

			\item login name
			\item encrypted password
			\item date of last password change

			\column{0.3\textwidth}		
			\item minimum password age 
			\item maximum password age
			\item password warning period

			\column{0.3\textwidth}
			\item password inactivity period
			\item account expiration date
			\item reserved field

			\end{columns}
		\end{enumerate}
	\end{block}

\end{frame}

\begin{frame}{Практическое задание}
    \begin{itemize}
		\item Посмотреть права доступа к файлам {\tt group}, {\tt passwd}, {\tt shadow}\\
			{\tt ls -l /etc/{group, passwd, shadow}}
		\item Добавить пользователя и группу и посмотреть изменения в перечисленных файлах
		\item Cоздать пользователя без использования системных утилит
    \end{itemize}
	\pause
	 \begin{itemize}
		\item Изменить пароль пользователю с помощью утилиты {\tt passwd}\\
			Hint: {\tt /etc/passwdqc.conf}
		\item Сбросить пароль пользователю\\
			Hint: {\tt usermod}
    \end{itemize}
\end{frame}


}

\mode<all>{\begin{frame}{PAM}
	% http://www.opennet.ru/base/net/pam_linux.txt.html
	\begin{itemize}
		\item PAM это динамическая библиотека
		\item Конфигурация PAM
			\begin{itemize}
				\item {\tt /etc/pam.conf}
				\item {\tt /etc/pam.d/...}
					\begin{itemize}
						\item Сервисы
						\item system\_auth
					\end{itemize}
			\end{itemize}
	\end{itemize}

	\begin{block}{Формат записи}
		\begin{columns}
			\column{0.245\textwidth}
			\textbf{module type}
			 \begin{itemize}
				 \item auth
				 \item account
				 \item session
				 \item password
			 \end{itemize}
			 \column{0.245\textwidth}
			 \textbf{control flag}
			 \begin{itemize}
				 \item requisite
				 \item required
				 \item sufficient
				 \item optional
			 \end{itemize}
			 \column{0.245\textwidth}
			 \textbf{module name}
			 \column{0.245\textwidth}
			 \textbf{module options}
		 \end{columns}
	 \end{block}
\end{frame}

\begin{frame}{Диспетчер службы имен (NSS)}

	Важная информация для системы:
		
	\begin{itemize}
		\item Информация о пользователях (логин, группа, пароль и т.д.)
		\item Информация о сетевых ресурсах (имена хостов, протоколов, сервисов)
    \end{itemize}

	\pause

	\begin{block}{NSS}
		\begin{itemize}
			\item Конфигурация: {\tt /etc/nsswitch.conf}
			\item Динамические библиотеки сервисов: {\tt ls -1 /lib*/libnss\_*}
		\end{itemize}
	\end{block}

\end{frame}



}
\let\defaultuser\undefined

\chapter{Дисковая подсистема}

\section{Блочные устройства}
\mode<all>{\begin{frame}{Дисковая подсистема}

	\begin{block}{Блочное устройство}
		Вид файла устройств в UNIX/Linux-системах,  обеспечивающий интерфейс к устройству,
		реальному или виртуальному, в виде файла в файловой системе.
	\end{block}

	\begin{block}{Файловая система}
		Файловая система определяет формат содержимого и способ физического хранения информации,  
		которую принято группировать в виде файлов. 
		Конкретная файловая система определяет размер имени файла (директории),  
		максимальный возможный размер файла и раздела,  набор атрибутов файла.

		Распространенные для ОС Linux: ext2, ext4, xfs, reiserfs, vfat.
	\end{block}
\end{frame}

\begin{frame}{Примеры блочных устройств}

	\begin{itemize}
		\item {\tt /dev/{\bf s}d*}
		\item {\tt /dev/{\bf h}d*}
		\item {\tt /dev/ram*}
		\item {\tt /dev/loop*}
	\end{itemize}

	\begin{block}{Практическое задание:}
		\begin{enumerate}
			\item Посмотреть список вышеперечисленных устройств
			\item Посмотреть информацию об устройствах {\tt loop0, ram, sda}\\
				Hint: {\tt fdisk -l <device>}
		\end{enumerate}
	\end{block}
\end{frame}

\begin{frame}{Структура диска}
	\begin{columns}
		\column{0.6\textwidth}
		\includegraphics[height=0.8\textheight]{../../slides/disk/04-hd-schematic.png}
		\column{0.4\textwidth}
		\includegraphics[height=0.8\textheight]{../../slides/disk/04-disk-structure.png}
	\end{columns}
\end{frame}

\begin{frame}{Отображение блочных устройств}


	\begin{block}{Device Mapper}
			{\tt /dev/mapper/*}\\
			device-mapper -- служит общим фреймворком для отображения одного блочного устройства на другое.

			Примеры: RAID, LVM, шифрованные диски и т.д.
	\end{block}

\end{frame}



}
\section{Основные команды}
\mode<all>{\begin{frame}{Полезные утилиты}
	\begin{columns}
		\column{0.25\textwidth}
		\begin{itemize}
			\item {\tt fdisk}
			\item {\tt parted}
			\item {\tt kpartx}
		\end{itemize}
		\column{0.25\textwidth}
		\begin{itemize}
			\item {\tt dd}
			\item {\tt losetup}
		\end{itemize}
		\column{0.25\textwidth}
		\begin{itemize}
			\item {\tt mkfs}
			\item {\tt fsck}
		\end{itemize}
		\column{0.25\textwidth}
		\begin{itemize}
			\item {\tt mount}
			\item {\tt umount}
			\item {\tt df}
		\end{itemize}
	\end{columns}

	\bigskip
	Понадобятся для упражнений:
	\begin{itemize}
			\item[*] {\tt chroot}
			\item[*] {\tt kvm}
	\end{itemize}
\end{frame}


\begin{frame}{Практика: отображение файла на loop-устройство}
	\begin{enumerate}
		\item Создать пустой файл размером 100MB: \\
			dd if=/dev/zero of=test bs=1M count=100
			\pause
		\item Найти неиспользуемое loop-устройство и отобразить на него файл:\\
			losetup -f \\
			losetup loop0 test
			\pause
		\item Посмотреть структуру loop-устройства, создать разделы и посмотреть результаты:\\
			fdisk -l /dev/loop0 \\
			fdisk /dev/loop0 \\
			fdisk -l /dev/loop0
			\pause
		\item Дать команду ядру перечитать разделы и создать устройства для разделов:\\
			ls -l /dev/mapper/* \\
			kpartx -a /dev/loop0 \\
			ls -l /dev/mapper/* \\
	\end{enumerate}
\end{frame}

\begin{frame}{Практика: создание файловой системы}
	\begin{enumerate}
		\item Форматируем файловую систему на устройстве: \\
			mkfs.ext2 /dev/mapper/loop0p1
			\pause
		\item и монтируем:\\
			mkdir -p /mnt/fs\\
			mount\\
			mount /dev/mapper/loop0p1 /mnt/fs\\
			mount\\
			df
			\pause
	\end{enumerate}
\end{frame}


\begin{frame}{Практика: чистимся}
	\begin{enumerate}
		\item Найти смонтированные разделы и отмонтировать их: \\
			mount \\
			umount /dev/mapper/loop0p1
			\pause
		\item Найти используемые loop-устройства\\
			losetup -a \\
			\pause
		\item Корректно удалить устройства для разделов:\\
			ls -l /dev/mapper/* \\
			kpartx -d /dev/loop0 \\
			ls -l /dev/mapper/* \\
			\pause
		\item Удалить отображение файла на loop-устройство: \\
			losetup -d /dev/loop0
	\end{enumerate}
\end{frame}


}
\section{GPT}
\mode<all>{\begin{frame}{GUID таблица разделов}

   \begin{columns}
      \column{0.5\textwidth}
      \begin{itemize}
        \item{Не менее 128 доступных разделов}
        \item{Дополнительные копии таблицы разделов}
        \item{$>1024^3$TB размер раздела}
        \item{Часть спецификации UEFI}
      \end{itemize}
      \column{0.5\textwidth}
      \includegraphics[width=0.9\textwidth]{../../slides/disk/gpt.png}
  \end{columns}
  \begin{center}
   \Large Утилиты
  \end{center}
  \begin{itemize}
    \item gdisk (аналог fdisk)
    \item cgdisk (аналог cfdisk)
    \item parted
  \end{itemize}
\end{frame}

\begin{frame}{Источники по UEFI boot}
  \begin{itemize}
    \item http://www.rodsbooks.com/efi-bootloaders/
    \item http://www.slideshare.net/mlug/arch-onmac 
  \end{itemize}
\end{frame}
}
\section{LVM}
\mode<all>{\begin{frame}{LVM -- управление логическими томами}
  \begin{center}
    \textbf{Структура LVM}
  \end{center}
  \includegraphics[width=0.7\textwidth]{../../slides/disk/LVM1-wiki.png}
\end{frame}

\begin{frame}{Преимущества LVM}
	\begin{itemize}
		\item Изменение размера
		\item Перемещение данных в активной системе
		\item Присвоение имен устройствам
		\item Чередование дисков
		\item Зеркалирование томов
		\item Снимки томов
	\end{itemize}
\end{frame}
 
\begin{frame}{LVM -- основные команды}
  \begin{itemize}
    \item Создание
      \begin{columns}
        \column{0.2\textwidth}
        \begin{itemize}
          \item pvcreate
        \end{itemize}
        \column{0.2\textwidth}
        \begin{itemize}
          \item vgcreate
        \end{itemize}
        \column{0.2\textwidth}
        \begin{itemize}
          \item lvcreate
        \end{itemize}
      \end{columns}
     \item Информация 
       \begin{columns}
         \column{0.2\textwidth}
         \begin{itemize}
           \item pvs
           \item lvs
           \item vgs
		   \item[ ]
         \end{itemize}
         \column{0.2\textwidth}
         \begin{itemize}
           \item pvscan
           \item lvscan
           \item vgscan
		   \item lvmdiskscan
         \end{itemize}
         \column{0.2\textwidth}
         \begin{itemize}
           \item pvdisplay
           \item lvdisplay
           \item vgdisplay
		   \item[ ]
         \end{itemize}
	 \end{columns}
      \item Манипулирование
        \begin{itemize}
          \item pvmove
          \item pvremove
          \item vgextend/vgreduce
          \item lvresize
         \end{itemize}
     \end{itemize}
    
\end{frame}

\begin{frame}{Упражнение: создание}
  \begin{enumerate}
    \item Создать 3 файла (200MB) и отобразить на {\tt /dev/loop[0-]}
	\item Найти устройства для работы с LVM {\tt lvmdiskscan}
	\item  {\tt pvcreate /dev/loop[0-2]}
    \item  {\tt pvscan, pvdisplay, pvs}
		\pause
    \item Создание группы томов {\tt vgcreate VG0 /dev/loop[0-2]}
    \item {\tt pvscan, vgscan, pvdisplay}
		\pause
    \item Создание логического тома {\tt lvcreate  -l 50\%VG -i 3 -n lv1 VG0}
	\item Создание файловой системы ext2 на {\tt /dev/VG0/lv1} и монтирование в {\tt /mnt/myfs}
	\end{enumerate}
\end{frame}


\begin{frame}{Упражнение: Создание снимка LVM}
  \begin{enumerate}
    \item Скопировать несколько файлов на { \tt /mnt/myfs}
    \item  {\tt lvcreate -\phantom{}-snapshot -l 10\%VG -n snap /dev/VG0/lv1}
    \item  {\tt lvdisplay, lvs, lvscan}
	\item Смонтировать снимок в {\tt /mnt/snap}
		\pause
    \item Удалить один из файлов на {\tt /mnt/snap/} или {\tt /mnt/myfs/}
	\item Отмонтировать снимок и оригинал
		\pause
	\item Объединяем снимок с оригиналом {\tt lvconvert -\phantom{}-merge VG0/snap}
	\item Монтируем {\tt /dev/VG0/lv1} в {\tt /mnt/myfs} и проверяем изменения
  \end{enumerate}
\end{frame}

\begin{frame}{Упражнение: изменение размера VG}
  \begin{enumerate}
%	\item Отмонтировать {\tt /mnt/myfs}
	\item Создать еще один файл и отобразить его на {\tt loop3}
	\item Увеличиваем размер группы {\tt vgextend VG0 /dev/loop3}
		\pause
    \item  {\tt pvscan; pvmove /dev/loop0; pvscan}
    \item  {\tt vgreduce VG0 /dev/loop0; pvscan}
    \item  {\tt pvremove /dev/loop0; pvscan}
  \end{enumerate}
\end{frame}


}

\chapter{Сетевая подсистема}

\section{Основы работы с сетевой подсистемой}

\mode<all>{\begin{frame}{Сетевая подсистема Linux}

	\begin{block}{Cетевой интерфейс}

		Сетевой интерфейс в Linux -– это абстрактный именованный объект,  используемый для передачи 
		данных через некоторую линию связи без привязки к ее (линии связи) реализации.
	\end{block}
\end{frame}

\begin{frame}{Сетевая подсистема Linux}

	\center\includegraphics[width=0.9\textwidth]{../../slides/networking/06-netstack.png}

\end{frame}


}

\subsection{Управление интерфейсами}
\mode<all>{\begin{frame}{Команды управления настройками сети}
	\begin{itemize}
	  \item ifconfig/route
	  \item iproute2
	\end{itemize}

	\begin{itemize}
		\item Список интерфейсов
			\begin{itemize}
				\item {\tt ifconfig -a}
				\item {\tt ip link show}
			\end{itemize}
		\item Включение интерфейса 
			\begin{itemize}
				\item {\tt ifconfig <iface> up}
				\item {\tt ip link set <iface> up}
			\end{itemize}
	  \item Выключение интерфейса
			\begin{itemize}
				\item {\tt ifconfig <iface> down}
				\item {\tt ip link set <iface> down}
			\end{itemize}
	  \item Назначение адреса
			\begin{itemize}
				\item {\tt ifconfig eth0 192.168.1.17 netmask 255.255.254.0 up}
				\item {\tt ip addr add 192.168.1.17/23 dev eth0}
			\end{itemize}
	\end{itemize}
\end{frame}

\begin{frame}{Конфигурационные файлы}
  \begin{itemize}
    \item {\tt /etc/resolv.conf}
    \item {\tt /etc/hosts}
	\item {\tt /etc/sysconfig/network}
    \item {\tt /etc/sysconfig/network-scripts}\\
		{\tt /etc/sysconfig/network-scripts/ifcfg-eth0}
  \end{itemize}
\end{frame}

\begin{frame}{Дополнительные интерфейсы}
	\begin{block}{Алиасы}
		\begin{itemize}
			\item ifconfig <iface>:<alias> <ip> up
			\item ifconfig <iface> add <ip> up
			\item ip addr add <ip> {\bf label} <iface>:<alias> dev <iface>
		\end{itemize}
	\end{block}
\end{frame}

\begin{frame}{''Виртуальные'' интерфейсы}
	\begin{block}{TUN/TAP}

		{\tt modprobe tun}

		\begin{itemize}
			\item Добавить -- {\tt tunctl -t <ifacename>}
			\item Удалить -- {\tt tunctl -d <ifacename>}
		\end{itemize}
	\end{block}
\end{frame}


}

\subsection{Полезные программы}
\mode<all>{\begin{frame}{Полезные утилиты}
	\begin{center}
		\begin{itemize}
			\item netstat / ss
			\item nslookup / dig
			\item ping
			\item traceroute
			\item tcpdump
			\item telnet
			\item netcat
			\item nmap
		\end{itemize}
	\end{center}

\end{frame}


\begin{frame}{Полезные утилиты: практика}

	\begin{columns}
		\column{0.5\textwidth}
		\begin{block}{netstat}

			Узнать:
			\begin{itemize}
				\item список используемых сокетов
				\item серверных сокетов
				\item имена/pid серверов
				\item узнать номера портов
			\end{itemize}
		\end{block}
	
		\pause
		\column{0.5\textwidth}
		\begin{block}{telnet/netcat}

			\begin{itemize}
				\item Чат по протоколу TCP с соседом
				\item Чат по протоколу UDP с соседом
				\item Передать текстовый и бинарный файлы
			\end{itemize}
	
			При создании чата использовать {\tt netstat} и {\tt tcpdump}
			для получения информации о соединении.
		\end{block}
	
	\end{columns}
\end{frame}

nmap
1. сканирование соседа
2. сканирование выделенных портов у соседа (поиск сервера чата) 
3. узнать список открытых портов на всех машинах в 505
4. узнать список  работающих машин

tcpdump
0. pcap файлы/libpcap
1. запуск монитора
2. запуск чата
3. монитор-фильтр-анализ

}

\section{ssh}
\mode<all>{\begin{frame}{ssh}

	\begin{block}{ssh -- терминал}
		{\tt ssh [user@]host[:port]}\\
		{\tt ssh host [-l user] [-p port]}
		\begin{itemize}
			\item -v -- "разговорчивый" режим 
			\item -t -- насильное назначение псевдотерминала (для автоматизации)
		\end{itemize}
		Вся конфигурация пользователя: {\tt \$HOME/.ssh}
	\end{block}

	\pause

	\begin{block}{... и не только}
		\begin{itemize}
			\item -X -- "проброс" графики 
			\item -L [bindip:]port:rhost:rport -- "пробрасывание" порта с удаленной машины на локальную
			\item -R [bindip:]port:lhost:lport -- "пробрасывание" порта с локальной машины на удаленную
			\item -W host:port -- stdin/stdout с указанным хостом
			\item -D port -- динамический прокси
		\end{itemize}
	\end{block}
\end{frame}


}


\section{Дополнительные типы интерфейсов}

\subsection{alias, vlan}
\mode<all>{
\begin{frame}{Дополнительные интерфейсы}
	\begin{block}{Алиасы}
		\begin{itemize}
			\item ifconfig <iface>:<alias> <ip> up
			\item ifconfig <iface> add <ip> up
			\item ip addr add <ip> {\bf label} <iface>:<alias> dev <iface>
		\end{itemize}
	\end{block}
	\pause
	\begin{block}{VLAN}
		\begin{itemize}
			\item vconfig add <iface> <id>
			\item vconfig rem <iface>{\bf.}<id>
			\item ip link add {\bf link} <iface> name <vlan\_name> {\bf type vlan id <id>}
			\item ip link delete <vlan\_name>
		\end{itemize}
	\end{block}

\end{frame}


}
\subsection{Мосты}
\mode<all>{\begin{frame}{Мосты}
	\begin{itemize}
		\item Создать -- {\tt brctl addbr <bridge>}
		\item Удалить -- {\tt brctl delbr <bridge>}
		\item Добавить интерфейс -- {\tt brctl addif <bridge> <device>}
		\item Удалить интерфейс-- {\tt brctl addif <bridge> <device>}
	\end{itemize}
\end{frame}


}
\subsection{Тоннели}
\mode<all>{
\begin{frame}{''Виртуальные'' интерфейсы}
	\begin{block}{TUN/TAP}

		{\tt modprobe tun}

		\begin{itemize}
			\item Добавить интерфейс TUN -- {\tt tunctl -n -t <ifacename>}
			\item Добавить интерфейс TAP -- {\tt tunctl -p -t <ifacename>}
			\item Удалить интерфейс -- {\tt tunctl -d <ifacename>}
		\end{itemize}
	\end{block}

	\pause

	\begin{block}{Практика}
		\begin{itemize}
			\item Создать интерфейс TAP с именем {\tt mytap}
			\item Создать мост с именем {\tt mybr}
			\item Назначить интерфейсу {\tt mytap} адрес {\tt 192.168.0.<n>/24}
			\item Добавить интерфейсы {\tt mytap} и {\tt eth0} к мосту {\tt mybr}
			\item Запустить {\tt tcpdump} на интерфейсах {\tt mybr} и {\tt mytap}
			\item Запустить {\tt ping} соседа
		\end{itemize}
	\end{block}

\end{frame}


}

\section{Маршрутизация}
\mode<all>{\begin{frame}{Маршрутизация}
	\begin{itemize}
		\item netstat -r
		\item route
		\item ip route show
	\end{itemize}

	\begin{block}{Разрешить маршрутизацию на хосте}
		{\tt echo 1 > /proc/sys/net/ipv4/ip\_forward}
	\end{block}
\end{frame}


}

\section{iptables}
\mode<all>{\begin{frame}{Iptables}

	\center\includegraphics[width=1\textwidth]{../../slides/networking/06-iptables.png}

\end{frame}

\begin{frame}{Iptables}

	\center{\bf iptables -t <table> -L}
	\center{\bf iptables -t <table> -F}
	\bigskip

	\begin{itemize}
	\begin{columns}
		\column{0.3\textwidth}

			\item filter -- файерволл
				\begin{itemize}
					\item INPUT
					\item FORWARD
					\item OUTPUT
				\end{itemize}
		\column{0.3\textwidth}
			\item nat -- преобразования адресов
				\begin{itemize}
					\item PREROUTING
					\item INPUT
					\item OUTPUT
					\item POSTROUTING
				\end{itemize}
		\column{0.3\textwidth}
			\item mangle -- специальные  изменения  пакетов (TOS, TTL, MARK)
				\begin{itemize}
					\item PREROUTING
					\item INPUT
					\item FORWARD
					\item OUTPUT
					\item POSTROUTING
				\end{itemize}
		\end{columns}
	\end{itemize}

\end{frame}


\begin{frame}{Iptables: примеры}

	\center{\bf iptables -t <table> <CRITERIA> <TARGET>}
	\small
	\begin{itemize}
		\item filter:\\
			{\tt iptables -A INPUT -s 192.168.0.1/24 -p UDP -j REJECT -{}-reject-with icmp-host-unreachable}\\
			{\tt iptables -A INPUT -d 192.168.0.1/24 -p TCP -j DROP}
		\item nat:\\
			{\tt iptables -A POSTROUTING -t nat -s 192.168.1.0/24 -j MASQUERADE}\\
			{\tt iptables -t nat -A PREROUTING -p tcp -d 192.168.251.1 
			--dport 8080 -{}-sport 1024:65535 -j DNAT -{}-to 192.168.1.200:8080}
		\item mangle:\\
			{\tt iptables -A PREROUTING -t mangle -p tcp -{}-dport 22 -j MARK -{}-set-mark 100}\\
			{\tt ip route add default dev eth0 table 1}\\
			{\tt ip rule add fwmark 100 table 1}
	\end{itemize}

\end{frame}


\begin{frame}[fragile]{Упражнение}
    \begin{block}{Внутренняя сеть: маскарадинг}
        \begin{enumerate}
            \item Добавить в netns 'A' правило для таблицы {\tt nat},
                включающее маскарадинг для IP-адреса, использующегося в netns 'B'
            \item Запустить {\tt ping -n <IP>} в netns 'B'\\
                IP -- адрес соседа для {\tt ethA0}
            \item Запустить {\tt tcpdump -i ethA0 icmp} в netns 'A'
            \item Запустить {\tt tcpdump -i eth0 icmp} на хосте (без использования netns)
            \item Запретить входящий ''{\tt ping}'' для {\tt ethA0}:\\
                {\tt iptables -A INPUT -i ethA0 -p icmp -j REJECT --reject-with icmp-host-unreachable}
        \end{enumerate}
    \end{block}
\end{frame}
}

\chapter{Самостоятельная работа}
\mode<all>{\begin{frame}{Практика: создание тестовой среды}

	\center\includegraphics[height=0.4\textheight]{../../slides/networking/net-practice.png}


	\begin{block}{Задача}
		Запустить 3 идентичные виртуальные машины.\\
		Каждой машине назначить адрес из отдельного IP диапазона.\\
		Организовать сетевую ''видимость'' между виртуальными машинами, а также хостом.		
	\end{block}

\end{frame}



\begin{frame}
	\frametitle{Подготовка дисковой подсистемы}
			\begin{itemize}
				\item Создать пустой файл размером от 1.5 GB и отобразить на устройство
					/dev/loop0 ({\tt dd, losetup})
				\item Создать группу томов на базе этого устройства ({\tt pvcreate, vgcreate})
				\item Выделить 1 GB под логический диск ({\tt lvcreate})
				%\item Скопировать образ виртуального диска в полученный логический том ({\tt dd})
				%\item Создать снимок логического тома на 100MB ({\tt lvcreate}) для каждой виртуальной машины.
			\end{itemize}
\end{frame}

\begin{frame}[fragile]{Установка системы}
        \begin{itemize}
		\item Установить centos-minimal на  машину из iso файла.
		\item Создать два снимка логического тома виртуальной машины
		\item Убедиться в наличии tap интерфейсов
	\end{itemize}

\end{frame}

\begin{frame}[fragile]{Пример запуска kvm}
		\begin{itemize}
          \item {\tt modprobe kvm-intel} {\small Включаем модуль поддержки виртуализации в ядре} 
          \item {\tt modprobe tun}  {\small Включаем поддержку tun, tap виртуальных сетевых интерфейсов}
          \item 
            \begin{lstlisting}[language=bash,basicstyle=\tiny] 
kvm -enable-kvm -cdrom centos-minimal.iso -hda /dev/loop0 -m 512M   \
    -boot order=cd -serial stdio -net nic,model=rtl8139 -net tap,ifname=tap0 
            \end{lstlisting}
              \begin{enumerate}
                \item[{\tt -enable-kvm}] Включает ядерную поддержку виртуализации
                \item[{\tt -cdrom}] Устройство или disk image, cdrom виртуальной машины
                \item[{\tt -hda}] Устройство или disk image, представляет жесткий диск VM
                \item[{\tt -serial}] Перенаправление com порта (консоль ядра)
                \item[{\tt -net nic}] Условная модель сетевой карточки
                \item[{\tt -net tap}] TAP интерфейс, на который будет приходить сеть из VM
                \item[{\tt -boot order}] cd (вначале cdrom (с), потом диск (d))
                \item[{\tt -m}] Объем памяти для VM
              \end{enumerate}
        \end{itemize}
\end{frame}        


\begin{frame}
	\frametitle{Настройка сети на хосте}
			\begin{itemize}
				\item Создать мост {\tt brctl} и назначить ему адреса из соответствующих диапазонов {\tt ifconfig/ip}
				\item Поднять виртуальные интерфейсы {\tt ifconfig/ip}
				\item Добавить виртуальные интерфейсы к мосту {\tt brctl}
			\end{itemize}
\end{frame}


\begin{frame}
	\frametitle{Настройка сети на виртуальных машинах}
			\begin{itemize}
				\item Назначить адрес устройству eth0 {\tt ifconfig/ip}
				\item Добавить адрес маршрутизатора по умолчанию {\tt route/ip}
				\item Проверить доступность виртуальных машин и хоста {\tt ping/nmap}
			\end{itemize}
\end{frame}

\begin{frame}
	\frametitle{Настройка роутинга и NAT}
			\begin{itemize}
				\item Разрешить форвардинг на хосте
				\item Настроить NAT на хосте ({\tt iptables},  правило {\tt MASQUERADE})
				\item Проверить доступность хостов из ''внешней'' сети {\tt ping/nmap}
			\end{itemize}
\end{frame}
}
