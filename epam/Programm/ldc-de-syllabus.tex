\documentclass[12pt,a4paper,oneside]{article}

%\usepackage[english]{babel}
%\usepackage[utf8]{inputenc}
%\usepackage[T1]{fontenc}

\usepackage[usenames, dvipsnames]{color}
\usepackage[usenames, dvipsnames]{xcolor}

\usepackage{listingsutf8}
\lstloadlanguages{C, make, bash}

\lstset{escapechar=`,
	language=C, 
	tabsize=2, 
	columns=fullflexible, 
	basicstyle=\ttfamily, 
	keywordstyle=\color{blue}, 
	commentstyle=\itshape\color{brown},
%	identifierstyle=\ttfamily, 
	stringstyle=\mdseries\color{OliveGreen}, 
	showstringspaces=false, 
	numbers=left, 
	numberstyle=\tiny, 
	breaklines=true, 
%	inputencoding=utf8/latin1, 
	morekeywords={u\_short, u\_char, u\_long, in\_addr}
	}

\usepackage{verbatim}
\usepackage{moreverb}

\usepackage{longtable}
\usepackage[nottoc,numbib]{tocbibind}

\usepackage{html}   %  *always* load this for LaTeX2HTML
\begin{htmlonly}
\providecommand{\lstinputlisting}[2][]{\verbatiminput{#2}}
\end{htmlonly}

\bibliographystyle{unsrt}

\begin{document}

% Формируем титульную страницу
\title{Linux Development Courses\\Development environment}
\author{Epam Low Level Linux Department}
\date{2014-2015}
\maketitle



%\renewcommand{\contentsname}{Оглавление}

\tableofcontents
\setcounter{tocdepth}{2}
\newpage

\section{Description}
%\addcontentsline{toc}{chapter}{Цель}

In this training attendees will learn how to deal with Linux command line
starting from Linux kernel booting to debug and profiling end user applications.

\subsection{Modules}
\begin{itemize}
	\item Linux Essentials (18h)
	\item Bash scripting (12h)
	\item Development tools (16h)
\end{itemize}

\subsection{Trainers}
\begin{itemize}
	\item Denis Pynkin
	\item Yuri Adamov
	\item Vikentsi Lapa
\end{itemize}

\subsection{Audience}
\begin{itemize}
	\item Students
	\item Junior Software Engineers
	\item Software Engineers
\end{itemize}

\subsubsection{Requirements}
\begin{itemize}
	\item Any programming language (C preferable)
	\item OS theory
	\item Understanding of networking (optional)
\end{itemize}

\newpage

\section{Module: Linux Essentials}

\subsection{Description}

Here attendees will learn the principals and basics of Linux system.

\subsection{Training Objectives}

By the end of the course,  attendees will be able to:
\begin{itemize}
	\item effectively work with command line in Linux environment;
	\item use command line utilities for everyday needs;
	\item understand and manipulate system boot process;
	\item manage own Linux workstation;
	\item understand principles of Open Source development.
\end{itemize}

\subsection{Agenda}

\begin{longtable}{|p{0.3\linewidth}|p{0.6\linewidth}|p{0.1\linewidth}|}
        \hline
		\textbf{Topic} & \textbf{Description} & \textbf{Hours} \\ \hline
		\endfirsthead
		\multicolumn{2}{c}%
		{\tablename\ \thetable\ -- \textit{Continue}} \\
		\hline
		\textbf{Topic} & \textbf{Description} & \textbf{Hours} \\ \hline
		\endhead
		\hline \multicolumn{2}{r}{\textit{Continued on next page}} \\
		\endfoot
		\hline
		\endlastfoot

        Intro & Licenses; ''Unixway''; Linux distributives. & 2 \\ \hline
		Linux OS boot & Boot sequence; Bootloader; Kernel parameters; initrd; init; userspace. & 2 \\ \hline
		Command line & Shells; command line basics; man; file system navigation & 2 \\ \hline
		Command line & Process management; pipes; I/O redirection; Text processing & 2 \\ \hline
		Command line & Archivers; find; xargs; stream editors & 2 \\ \hline
		User management & Multiuser model; permissions; utilities; account files; PAM & 2 \\ \hline
		Basic networking & Interfaces, aliases, bridges, TUN/TAP; routing; iptables; ssh & 2 \\ \hline
		Package management & "DLL hell"; RPM; yum & 1 \\ \hline
		Disk management & Block devices; partitioning; GPT; LVM & 3 \\ \hline
		\textbf{Total:} & & 18 \\ \hline

\end{longtable}


\section{Module: Bash scripting}

\subsection{Description}

In this module  attendees will learn the basics of usage and script programming with Bash
(environment,  variables,  arrays,  exit statuses,  tests,  loops,  branches,  functions,  arithmetics,  strings manipulation,  I/O redirection,  debugging)

\subsection{Training Objectives}

By the end of the course,  attendees will be able to:
\begin{itemize}
	\item create and run shell scripts;
	\item define custom functions and call built-in functions;
	\item use I/O redirection for files and programs;
	\item use arithmetic operations;
	\item manipulate strings;
	\item handle exceptions;
	\item debug scripts.
\end{itemize}


\subsection{Agenda}

\begin{longtable}{|p{0.3\linewidth}|p{0.6\linewidth}|p{0.1\linewidth}|}
        \hline
		\textbf{Topic} & \textbf{Description} & \textbf{Hours} \\ \hline
		\endfirsthead
		\multicolumn{2}{c}%
		{\tablename\ \thetable\ -- \textit{Continue}} \\
		\hline
		\textbf{Topic} & \textbf{Description} & \textbf{Hours} \\ \hline
		\endhead
		\hline \multicolumn{2}{r}{\textit{Continued on next page}} \\
		\endfoot
		\hline
		\endlastfoot

        Intro & Overview; Runtime environment; I/O redirection & 2 \\ \hline
		Syntax & Variables; Parameters; Tests & 2 \\ \hline
		Syntax & Arithmetic; Loops & 2 \\ \hline
		Syntax & Branches; Functions & 2 \\ \hline
		Advanced & Strings; Parameters substitution & 2 \\ \hline
		Advanced & Arrays; Signal traps; Debugging & 2 \\ \hline
		\textbf{Total:} & & 12 \\ \hline

\end{longtable}


\section{Module: Development tools}

\subsection{Description}


\subsection{Training Objectives}

By the end of the course,  attendees will be able to:
\begin{itemize}
	\item use GNU Toolchain for development;
	\item use static and dynamic linking for programs;
	\item read and create build rules with make and autotools;
	\item debug and profiling programs;
	\item use version control systems;
	\item software reference documentation generation from source code;
	\item understand the difference between run- and build-time environments;
	\item basic packaging.
\end{itemize}


\subsection{Agenda}

\begin{longtable}{|p{0.3\linewidth}|p{0.6\linewidth}|p{0.1\linewidth}|}
        \hline
		\textbf{Topic} & \textbf{Description} & \textbf{Hours} \\ \hline
		\endfirsthead
		\multicolumn{2}{c}%
		{\tablename\ \thetable\ -- \textit{Continue}} \\
		\hline
		\textbf{Topic} & \textbf{Description} & \textbf{Hours} \\ \hline
		\endhead
		\hline \multicolumn{2}{r}{\textit{Continued on next page}} \\
		\endfoot
		\hline
		\endlastfoot

        GNU Toolchain & Compiler; Linker; static and dynamic applications and libraries  & 3 \\ \hline
        Make & Make, Makefile syntax & 2 \\ \hline
        Autotools & Autotools intro; simple rules & 1 \\ \hline
        ELF analysis & structure; bintools; ELF analysis & 2 \\ \hline
        Debug & debug techniques; core dump; backtrace; debug utilities & 4 \\ \hline
        Profiling & profiling utilities; code test coverage  & 2 \\ \hline
        VCS & centralized vs distributed; SVN; GIT & 2 \\ \hline
        SW documentation & Doxygen; syntax & 1 \\ \hline
        Package management & Installation from source; RPM packaging intro. Run- and Build-time environment. & 3 \\ \hline
		\textbf{Total:} & & 20h \\ \hline

\end{longtable}


\end{document}

